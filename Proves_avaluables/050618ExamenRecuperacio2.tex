\documentclass[11pt]{article}
% SILENCE THE WARNINGS!
%\usepackage{silence}

% tufte-book ja inclou el paquet geometry, i per tant només
% cal canviar alguns paràmetres amb \geometry
% \geometry{a4paper, top=25mm, bottom=30mm, inner=20mm, outer=70mm}
% \setlength{\marginparwidth}{50mm}  % Adjust margin for sidenotes
% %\geometry{margin=3cm,headsep=0.25in}
%\geometry{showframe}% for debugging purposes -- displays the margins
% The units package provides nice, non-stacked fractions and better spacing
% for units.
%\usepackage{units}
%\usepackage{todonotes}

\usepackage[backend=bibtex,style=numeric]{biblatex}  %backend=biber is 'better'

\usepackage{framed}
\usepackage{ifthen}
\usepackage{longtable}
\usepackage{fancyvrb}
\fvset{fontsize=\normalsize}
%\usepackage{cancel}

\usepackage[utf8]{inputenc}
\usepackage[catalan]{babel}
\usepackage{lmodern}
\usepackage{amsmath,amsthm,amsfonts,amssymb,amscd}

\usepackage{multirow,booktabs}
\usepackage[dvipsnames,table]{xcolor}
%\usepackage{fullpage}
\usepackage{lastpage}
\usepackage{graphicx}
%\setkeys{Gin}{width=\linewidth,totalheight=\textheight,keepaspectratio}
\graphicspath{{../figures/}}
\usepackage{enumitem}
\usepackage{mathrsfs}
\usepackage{wrapfig}
\usepackage{setspace}
\usepackage{calc}
\usepackage{multicol}
\usepackage{gensymb}




\usepackage{cancel}
\usepackage[retainorgcmds]{IEEEtrantools}

%\newlength{\tabcont}
% \setlength{\parindent}{0.0in}
% \setlength{\parskip}{0.05in}
%\usepackage{empheq}
% es recomana que mdframed es carregui després de xcolor
\usepackage[framemethod=TikZ]{mdframed}
\mdfdefinestyle{caixa}{leftmargin=1cm,innerrightmargin=0.5cm, linecolor=blue}

\usepackage{changepage}






  
%\chemsetup[chemformula]{format=\sffamily}

%\setatomsep{2em}
%\setdoublesep{.6ex}
%\setbondstyle{semithick}
\colorlet{shadecolor}{orange!15}
\parindent 0in
\parskip 12pt


\theoremstyle{definition}
\newtheorem{defn}{Definition}
\newtheorem{reg}{Rule}
\newtheorem{exer}{Exercise}
\newtheorem{note}{Note}
%\RequirePackage{mathrsfs}
%\RequirePackage[psamsfonts]{amsfonts} %for Y&Y BSR AMS fonts
\RequirePackage{amsmath,amsfonts,amsthm,amssymb}
\RequirePackage{setspace}
\RequirePackage{fancyhdr}
\RequirePackage{lastpage}
\RequirePackage{extramarks}
%\RequirePackage{chngpage}
\RequirePackage{soul}
%\RequirePackage{graphicx,float,wrapfig}
%\RequirePackage{pgf,tikz}
%\usetikzlibrary{arrows,automata}
%\RequirePackage{pstricks}
%\RequirePackage[text]{amsthm}
%\RequirePackage{array}
%\RequirePackage{amscd}
%\RequirePackage{array}\RequirePackage{dcolumn}

\newcommand{\emx}[1]{{\em{#1}\/}}
\newcommand{\abin}{{\it ab initio}}
\newcommand{\bs}{\boldsymbol}
%\newcommand{\citepnum}{\citep}
\newcommand{\dGo}{\ensuremath{\Delta G_0}}
\newcommand{\dG}[2]{\ensuremath{\Delta G_{\rm #1}^{\rm #2}}}
\newcommand{\dX}[3]{\ensuremath{\Delta #1_{\rm #2}^{\rm #3}}}
\newcommand{\ddgo}[1]{\ensuremath{\Delta \Delta G_{\rm solv}^{\rm #1}}}
\newcommand{\ddgstarcat}{\ensuremath{\Delta \Delta g^{\ddagger}_{\rm cat}}}
\newcommand{\ddgstar}{\ensuremath{\Delta \dgstar}}
\newcommand{\ddgt}[2]{\ensuremath{\Delta \Delta G_{\rm solv}^{\rm #1, \rm #2}}}
\newcommand{\ddsstarprime}{\ensuremath{(\Delta \dsstar)'}}
\newcommand{\deltaepsel}{\ensuremath{\Delta \varepsilon_{\rm el}}}
\newcommand{\deltaeps}{\ensuremath{\Delta \varepsilon}}
\newcommand{\dgab}[2]{\ensuremath{\Delta g_{\rm #1}^{\rm #2}}}
\newcommand{\dga}[1]{\ensuremath{\Delta g_{\rm #1}}}
\newcommand{\dgb}[1]{\ensuremath{\Delta g^{\rm #1}}}
\newcommand{\dgcage}{\ensuremath{\Delta g_{\rm cage}}}
\newcommand{\dgcat}{\ensuremath{\Delta g_{\rm cat}}}
\newcommand{\dgsoltsatsa}{\ensuremath{\dgsol (\rm TSA)_{\rm TSA}}}
\newcommand{\dgsoltstsa}{\ensuremath{\dgsol (\rm TS)_{\rm TSA}}}
\newcommand{\dgsoltsts}{\ensuremath{\dgsol (\rm TS)_{\rm TS}}}
\newcommand{\dgsol}{\ensuremath{\Delta G_{\rm sol}}}
\newcommand{\dgstarcage}{\ensuremath{\dgstar_{\rm cage}}}
\newcommand{\dgstarcat}{\ensuremath{\dgstar_{\rm cat}}}
\newcommand{\dgstarw}{\ensuremath{\dgstar_{\rm w}}}
\newcommand{\dgstar}{\ensuremath{\Delta g^{\ddagger}}}
\newcommand{\dgw}{\ensuremath{\Delta g_{\rm w}}}
\newcommand{\dg}[2]{\ensuremath{\Delta g_{\rm #1}^{\rm #2}}}
\newcommand{\dino}{\texttt{DINO}}
\newcommand{\dsstarcageprime}{\ensuremath{(\dsstarcage)'}}
\newcommand{\dsstarcage}{\ensuremath{\dsstar_{\rm cage}}}
\newcommand{\dsstarcatprime}{\ensuremath{(\dsstarcat)'}}
\newcommand{\dsstarcat}{\ensuremath{\dsstar_{\rm cat}}}
\newcommand{\dsstarwprime}{\ensuremath{(\dsstarw)'}}
\newcommand{\dsstarw}{\ensuremath{\dsstar_{\rm w}}}
\newcommand{\dsstar}{\ensuremath{\Delta S^{\ddagger}}}
\newcommand{\eg}{{\it e.g.}}
\newcommand{\etal}{{\it et al.}}
\newcommand{\gamess}{\texttt{GAMESS}}
\newcommand{\gauss}{\texttt{GAUSSIAN} 98}     
\newcommand{\golpe}{\texttt{GOLPE}}                                             
\newcommand{\grid}{\texttt{GRID}}
\newcommand{\ie}{{\it i.e.}}
\newcommand{\ith}{{\it i}$^{\rm th}$\ }
\newcommand{\kbt}{\ensuremath{k_{\rm B} T}}
\newcommand{\kb}{\ensuremath{k_{\rm B}}} 
\newcommand{\kcage}{\ensuremath{k_{\rm cage}}}
\newcommand{\kcatkm}{\ensuremath{k_{\rm cat}/K_{\rm M}}}
\newcommand{\kcat}{\ensuremath{k_{\rm cat}}}
\newcommand{\km}{\ensuremath{{\rm\, kcal \, mol}^-1}}
\newcommand{\knon}{\ensuremath{k_{\rm non}}}
\newcommand{\kw}{\ensuremath{k_{\rm w}}}
\newcommand{\mepsim}{\texttt{MEPSIM}}
\newcommand{\mgp}[1]{\marginpar{\scriptsize{#1}}}
\newcommand{\mipsim}{\texttt{MIPSIM}}
\newcommand{\mola}{\texttt{MOLARIS}}
\newcommand{\msms}{\texttt{MSMS}}
\newcommand{\pdras}{p21$^{\rm ras}$}
\newcommand{\rgran}{\ensuremath{\mathbb{R}}}
\newcommand{\rx}[2]{\ensuremath{#1_{\rm #2}}}
\newcommand{\vs}{{\it vs.}}
\newcommand{\z}[1]{\ensuremath{\mathbf{#1}}} 
\newcommand{\composed}[2]{#1\mathbin\circ #2}
\newcommand{\wrt}[1]{{\mbox{\scriptsize w.r.t. \( #1 \)} }}
\newcommand{\polyspace}{\mathcal{P}}
\newcommand{\matspace}{\mathcal{M}}
\newcommand{\C}{\mathbb{C}}
\newcommand{\N}{\mathbb{N}}
\newcommand{\Q}{\mathbb{Q}}
\newcommand{\Z}{\mathbb{Z}}
\renewcommand{\Re}{\mathbb{R}}
\newcommand{\rtres}{\ensuremath{\Re^3}}
\newcommand{\union}{\cup}
\newcommand{\dotprod}{\cdot}
%\newcommand*\pkg[1]{\textsf{#1}}

\newcommand{\trans}[1]{{#1}^{\ensuremath{\mathsf{T}}}} % transpose
\newcommand{\nbyn}[1]{\ensuremath{#1 \! \times \! #1 }}
\newcommand{\nbym}[2]{#1 \! \times \! #2 }       % Use in math mode.
\newcommand{\cat}[2]{#1\!\mathbin{\raise.6ex\hbox{\( {}^\frown \)}}\!#2}
\newcommand{\generalmatrix}[3]{ %arg1: low-case letter, arg2: rows, arg3: cols
               \left(
                  \begin{array}{cccc}
                    #1_{1,1}  &#1_{1,2}  &\ldots  &#1_{1,#2}  \\
                    #1_{2,1}  &#1_{2,2}  &\ldots  &#1_{2,#2}  \\
                              &\vdots                         \\
                    #1_{#3,1} &#1_{#3,2} &\ldots  &#1_{#3,#2}
                  \end{array}
               \right)  }
\newcommand{\colvec}[1]{\begin{pmatrix} #1 \end{pmatrix}}
\newcommand{\pr}[1]{\ensuremath{\mathrm{Pr}(#1)}}
\newcommand{\rep}[2]{ {\rm Rep}_{#2}(#1) }
\newcommand{\mapsunder}[1]{\stackrel{#1}{\longmapsto}}
\newcommand{\map}[3]{\mbox{$#1\colon #2\to #3$}}
\newcommand{\identity}{\mbox{id}}
\newcommand{\stdbasis}{{\cal E}} 
\newcommand{\sequence}[1]{ \langle#1\rangle } 
\newcommand{\spacer}{\rule[-3mm]{0mm}{8mm}}
\newcommand{\email}[1]{\url{#1}}
\newcommand{\zero}{\vec{0}}
\newcommand{\proj}[2]{\mbox{proj}_{#2}({#1}) }
%\AtBeginDocument{\newlength{\heightofcdot}
%\newlength{\widthofcdot}
%\settoheight{\heightofcdot}{$\cdot$}
%\settowidth{\widthofcdot}{$\cdot$}
%\newsavebox{\dotprodcircle}       
%\savebox{\dotprodcircle}{\includegraphics{dotprod.1}} 
%\newcommand{\dotprod}{\mathbin{\raisebox{.5\heightofcdot}{%
%          \makebox[\widthofcdot]{$\smash{\usebox{\dotprodcircle}}$}}}}}
\newcommand{\spanof}[1]{\relax [#1\relax ]} % no optional argument!
\newcommand{\set}[1]{\mbox{$\{#1\}$}} \newcommand{\suchthat}{\bigm|}
\newcommand{\deter}[1]{ \mathchoice{\left|#1\right|}{|#1|}{|#1|}{|#1|} }
\newcommand{\secuence}[1]{ \langle#1\rangle }  
\newcommand{\basis}[2]{\secuence{\vec{#1}_1,\ldots,\vec{#1}_{#2}}}



%--------linsys
%  Use as \begin{linsys}{3}
%           x &+ &3y &+ &a &= &7 \\
%           x &- &3y &+ &a &= &7
%         \end{linsys}
% Remark: TeXbook pp. 167-170 says to put a medmuskip around a +; and that's
% 4/18-ths of an em.  Why does 2/18-ths of an em work?  I don't know, but
% comparing to a regular displayed equation suggests it is right.
% (darseneau says LaTeX puts in half an \arraycolsep.)
\newenvironment{linsys}[2][m]{%
\setlength{\arraycolsep}{.1111em} % p. 170 TeXbook; a medmuskip
\begin{array}[#1]{@{}*{#2}{rc}r@{}}
}{%
\end{array}}


%\newtheorem{teorema}{Teorema}
%\newtheorem{exercici}{Exercici}
%\newtheorem{definicio}{Definici\'o}
%\newtheorem{theorem}{Theorem}
\newtheorem{exercise}{Exercise}
%\newtheorem{definition}{Definition}



\parskip 4mm


\usepackage{makeidx}
\makeindex




%\setcounter{section}{-1}

\theoremstyle{definition}
\newtheorem{thm}{Theorem}
\newtheorem{dfn}{Definition}
\newtheorem{lem}{Lemma}
\newtheorem{prp}{Proposition}





%%%%%%%%%%%%%%%%%%%
% ANGLÈS
%%%%%%%%%%%%%%%%%%%

% \newcommand{\problemName}{}%
% \newcounter{problemCounter}%
% \newenvironment{problem}[1][Problem \arabic{problemCounter}]%
% 	{\stepcounter{problemCounter}%
% 		\renewcommand{\problemName}{#1}%
% 		\section*{\problemName}%
% 		\nobreak\extramarks{\problemName}{\problemName continued on next page\ldots}\nobreak%
% 		\nobreak\extramarks{\problemName (continued)}{\problemName continued on next page\ldots}\nobreak}%
% 	{\nobreak\extramarks{\problemName (continued)}{\problemName continued on next page\ldots}\nobreak%
% 		\nobreak\extramarks{\problemName}{}\nobreak}%

\newenvironment{example}{ % 
	\definecolor{shadecolor}{rgb}{0.8,1.0,0.8} %
	\begin{shaded} %
	\textcolor{OliveGreen}{\bf Example\\}%
} % 
{ %	
	\end{shaded}
} %


\newenvironment{introduction}{ % 
	\definecolor{shadecolor}{rgb}{1.0,1.0,0.8} %
	\begin{shaded} %
	% \textcolor{BrickRed}{\bf Introduction\\}%
} % 
{ %	
	\end{shaded}
} %


%%%%%%%%%%%%%%%%%%%
% CATALÀ
%%%%%%%%%%%%%%%%%%%
\newtheorem{teorema}{theorem}
\newenvironment{definicio}{ % 
	\definecolor{shadecolor}{rgb}{0.9,1.0,0.8} %
	\begin{shaded} %
	\textcolor{OliveGreen}{\bf Definicio\\}%
} % 
{ %	
	\end{shaded}
} %

%veure http://en.wikibooks.org/wiki/LaTeX/Advanced_Topics

\newcommand{\doccmd}[1]{\texttt{\textbackslash#1}}% command name -- adds backslash automatically
\newcommand{\docopt}[1]{\ensuremath{\langle}\textrm{\textit{#1}}\ensuremath{\rangle}}% optional command argument
\newcommand{\docarg}[1]{\textrm{\textit{#1}}}% (required) command argument
\newenvironment{docspec}{\begin{quote}\noindent}{\end{quote}}% command specification environment
\newcommand{\docenv}[1]{\textsf{#1}}% environment name
\newcommand{\docpkg}[1]{\texttt{#1}}% package name
\newcommand{\doccls}[1]{\texttt{#1}}% document class name
\newcommand{\docclsopt}[1]{\texttt{#1}}% document class option name
\newcommand{\logos}{%
\begin{figure}
\includegraphics{FCTE}
\end{figure}
}

% margins
% \topmargin=-0.45in      %
% \evensidemargin=0in     %
% \oddsidemargin=0in      %
% \textwidth=6in        %
% \textheight=8.5in       %
% \headsep=0.25in         %

% header and footer
\pagestyle{fancy}       %
\chead{}                %
\makeatletter
\fancyfoot[R]{%
   % We want italics
   \itshape
   % The chapter number only if it's greater than 0
   \ifnum\value{chapter}>0 \@chapapp\ \thechapter. \fi
   % The chapter title
   \leftmark}
\makeatother

%\lfoot{\includegraphics[trim=-5cm 0 0 -3cm,width=0.4\textwidth]{FCTE}}      
\lfoot{\raisebox{-0.5cm}[0pt][0pt]{\includegraphics[width=3cm]{FCTE}}} 

\cfoot{}        %
\renewcommand\headrulewidth{0.4pt}   %
\renewcommand\footrulewidth{0.4pt}   %

% Essential Formatting
   
%\usepackage{epsfig,amsmath,amsthm,amssymb}
\usepackage[TYPE]{../Examen/urmathtest_cat}[2001/05/12]
%\usepackage[answersheet]{urmathtest}[2001/05/12]
%\usepackage[answers]{urmathtest}[2001/05/12]
\usepackage{graphicx}

%% For use with pdflatex
%\pdfpagewidth\paperwidth
%\pdfpageheight\paperheight

% Basic User Defs

%\def\ds{\displaystyle}

%\newcommand{\ansbox}[1]
%{\work{
%  \pos\hfill \framebox[#1][l]{SCORE:\rule[-.3in]{0in}{.7in}}
%}{}}
%
%\newcommand{\ansrectangle}
%{\work{
%  \pos\hfill \framebox[6in][l]{SCORE:\rule[-.3in]{0in}{.7in}}
%}{}}

% Beginning of the Document
\cfoot{\bf Examen de Recuperació Química GEA-17UV}

\begin{document}
\examtitle{Enginyeria de l'Automoció}{Examen de Recuperació Química GEA-17UV}{5 de Juny de 2018}
\studentinfo
\instructions{
  \textbf{Professor: Jordi Villà i Freixa}
  

  \begin{itemize}
  \item
    \textbf{No es permet l'ús d'ordinador. Només calculadora, apunts de classe i full d'exercicis resolts del campus virtual}
  \item
%   \textbf{Please show all your work if you need to.
%           You may use back pages if necessary.
%           You may not receive full credit for
%           a correct answer if there is no work shown.}
   \textbf{Desenvolupa tot el teu argumentari de forma clara.
           No usis més espai del proveït.}
  \item
    \textbf{Només es tindran en compte les puntuacions dels dos exercicis (d'entre els tres) que més et millorin la nota \textit{global} de l'assignatura (tenint en compte els pesos dels dos parcials i el final establertes al pla de treball).}
      \item
    \textbf{Tens 2 hores per completar l'examen.}
  \end{itemize}
}
\finishfirstpage

% Problems Start Here % ----------------------------------------------------- %
%%%%%%%%%%%%%%%%%%%%%%%%%%%%%%%%%%%%%%%%
\problem{100}
{
Exercici de recuperació del primer parcial. El pes de cada pregunta es mostra entre parèntesi.
\begin{description}
\item[(30\%)] Fins quin volum cal diluir 5.00 ml de \ch{HCl} a una concentració 6 M per tal d'obtenir una dissolució 0.001 M?
\item[(30\%)] Un pneumàtic de cotxe es va inflar a una pressió de 23 lb in$^{-2}$ un dia d'hivern a -10$^{\degree}$C. Quina pressió, calculada en atm, es va mesurar l'estiu següent (assumint que el pneumàtic no  va perdre aire entre hivern i estiu) quan la temperatura era de 35$^{\degree}$C? (1 atm = 14.696 lb in$^{-2}$)
\item[(40\%)] La pressió de vapor del benzè pur a 20$^{\degree}$C és de 75 Torr, i la del metilbenzè pur és 25 Torr a la mateixa temperatura. Quina és la concentració del vapor en equlibri amb una barreja de benzè i metilbenzè en la qual la fracció molar del primer és 0.75? (assumeix un comportament ideal dels líquids)
\end{description}
}
{
\vfill
\newpage (segueix) \newpage
}
{
\begin{description}
\item[30\%]  5.00 ml de \ch{HCl} a una concentració 6 M contenen 
\[n=5.00 \cancel{ml}  \cdot \frac{6 \mathrm{\, mols \, \ch{HCl}} }{1000 \cancel{ml}}=0.03 \mathrm{\, mols \, \ch{HCl}} \]
Si volem fer una dissolució de concentració 0.001 M, haurem de tenir un volum de
\[V=0.03 \cancel{\mathrm{mols \, \ch{HCl}}} \cdot \frac{1 l}{0.001 \cancel{\mathrm{mols \, \ch{HCl}}}}=30l\]
\item[30\%] Assumin que el volum és constant. En aquest cas, 
\[\frac{P_1}{T_1}=\frac{P_2}{T_2}\]
i, per tant, si
\[23 \cancel{\mathrm{\, lb \, in^{-2}}} \cdot \frac{1 \mathrm{\, atm}}{14.696 \cancel{\mathrm{\, lb \, in^{-2}}}}= 1.56\mathrm{\, atm}\]
ens queda
\[\frac{1.56\mathrm{\, atm}}{(273-10)\mathrm{K}}=\frac{P_2}{(273+35)\mathrm{K}}\]
\[P_2 = 1.83 \mathrm{\, atm}\]
\item[40\%] pàgina 166 atkins
\end{description}
}

%%%%%%%%%%%%%%%%%%%%%%%%%%%%%%%%%%%%%%%%
%%%%%%%%%%%%%%%%%%%%%%%%%%%%%%%%%%%%%%%%
\problem{100}
{Exercici de recuperació del segon parcial. El pes de cada pregunta es mostra entre parèntesi.
\begin{description}
\item[(30\%)] Calcula el treball realitzat i la calor absorbida/emesa en dur 1 mol d'Ar, de forma isotèrmica i reversible a 20$^{\degree}$C, des d'un volum de 5 dm$^3$ fins a un volum de 10 dm$^3$.
\item[(30\%)] L'òxid nitrós, \ch{N2O}, també anomenat gas hilarant i que s'empra en anestèsia, es forma a partir de la descomposició del nitrat d'amoni segons la reacció
\[\ch{NH4NO3 -> H2O + N2O}\]
Quin volum de \ch{N2O} produirem a 25$^{\circ}$C i 1 atm a partir de 7.5 g de nitrat d'amoni?
\item[(40\%)] La reacció \ch{H2_{(g)} + I2_{(g)} <=> 2 HI_{(g)}} té, a 448$^{\degree}$C, una constant d'equilibri de 50.53. Si posem 0.001 mols de gas \ch{H2}, 0.001 mols de gas \ch{I2} i 0.002 mols de \ch{HI} en un recipient de 5 l, es formarà més \ch{HI}?
\end{description}
}
{
%\vspace{22cm}
\vfill
\newpage (segueix) \newpage
}
{
\begin{description}
\item[30\%] Si fem un procés isotèrmic, $\Delta U=0$. Per tant, $q=-w$. Només cal calcular el treball i ja tenim la calor.
En un procés reversible, la $P$ canvia de forma infinitessimal i, per tant, $w$ es calcula fent (per a un gas ideal):
\[w=-\int_{V_1}^{V_2} P dV=-nRT \int_{V_1}^{V_2} \frac{dV}{V} = nRT \ln \frac{V_1}{V_2} = 1 \mathrm{mol} \cdot 8.314 \frac{\mathrm{J}}{\mathrm{mol \, K}} \cdot (273+20)\mathrm{K} \cdot \ln 2 = 1688.5\mathrm{J}=-q\]

\item[30\%] A partir de la reacció igualada
\[\ch{NH4NO3 -> 2 H2O + N2O}\]
fem un factor de conversió per trobar els mols obtinguts de \ch{N2O}
\[7.5 \cancel{\mathrm{g \, \ch{NH4NO3}}} \cdot 
\frac{1 \mathrm{\, mols \, \ch{NH3NO3}}}{(2\cdot 14+3\cdot 16 + 4\cdot 1)\cancel{\mathrm{g \, \ch{NH4NO3}}}}
=0.094 \mathrm{\, mols \, \ch{NH4NO3}}=0.095 \mathrm{\, mols \, \ch{N2O}}
\]
Que, segons la llei dels gasos ideals, ocupen 
\[V=\frac{nRT}{P}=\frac{0.095 \mathrm{\, mols \, \ch{NH3NO3}} \cdot 0.082 \mathrm{atm \, l \, mol^{-1} \, K^{-1}}(273+25)\mathrm{K}}{1 \mathrm{\, atm}}= 2.29 \mathrm{l}\]
\end{description}
}

%%%%%%%%%%%%%%%%%%%%%%%%%%%%%%%%%%%%%%%%
\problem{100}
{Exercici de recuperació de l'examen final. El pes de cada pregunta es mostra entre parèntesi.
\begin{description}

\item[(30\%)] Donades les dues primeres reaccions, determina a) l'entalpia de la tercera i b) les entalpies de formació de \ch{HI_{(g)}} i \ch{H2O_{(g)}}:
\[\ch{H2_{(g)} + I2_{(s)} -> 2 HI_{(g)}} \quad \Delta H^{\circ} = 52.96 \, kJ \, mol^{-1}\]
\[\ch{2 H2_{(g)} + O2_{(g)} -> 2 H2O_{(g)}} \quad \Delta H^{\circ} = -483.64 \, kJ \, mol^{-1}\]
\[\ch{4 HI_{(g)} + O2_{(s)} -> 2 I2_{(s)} + 2 H2O_{(g)}}\]

\item[(30\%)] Troba la constant d'equilibri de la pila de Daniell (que explota el potencial elèctric de la reacció \ch{Cu^{2+}_{(aq)} + Zn_{(s)} -> Cu_{(s)} + Zn^{2+}_{(aq)}}), sabent que 
\[\ch{Cu^{2+}_{(aq)} + 2 e- -> Cu_{(s)}} \quad \varepsilon^0 = +0.337 \, \mathrm{V}\]
\[\ch{Zn^{2+}_{(aq)} + 2 e- -> Zn_{(s)}} \quad \varepsilon^0 = -0.763 \, \mathrm{V}\]
\item[(40\%)] Quin és el pH d'una dissolució 0.01 M d'àcid acètic si la seva $K_a$ és de $1.76 \times 10^{-5}$?.

\end{description}
}
{
\vfill 
\newpage (segueix)

}
{
\begin{description}
\item[30\%] Per obtenir la tercera reacció, veiem que ens cal restar dues vegades la primera de la segona. Fem la mateixa operació amb les entalpies de reacció i obtenim:
\[\Delta H^{\circ}= (-483.64 - 2\cdot 52.96) \mathrm{\, kJ \, mol^{-1}}=-589.56 \mathrm{\, kJ \, mol^{-1}}\]
Pel que fa a les entalpies de formació de \ch{HI_{(g)}} i \ch{H2O_{(g)}}, només cal recordar que és la calor necessària per generar 1 mol de substància a partir dels seus components en l'estat normal. En aquest cas, només ens caldrà dividir per 2 les dades que ens proporcionen:
\[\Delta H_f^{\circ} \, \ch{HI_{(g)}} = 52.96/2 \mathrm{\, kJ \, mol^{-1}} = 26.48 \mathrm{\, kJ \, mol^{-1}}\]
\[\Delta H_f^{\circ} \, \ch{H2O_{(g)}} = -483.64/2 \mathrm{\, kJ \, mol^{-1}} = -241.82 \mathrm{\, kJ \, mol^{-1}}\]
\item[30\%] El voltatge d'una cel·la es calcularà fent
\[\Delta \varepsilon=\Delta \varepsilon^{\circ}-\frac{0.059}{n} \log \frac{[C]^c[D]^d}{[A]^a[B]^b}\] 
i, en l'equilibri, $\Delta \varepsilon=0$. Per tant, es pot calcular fàcilment la $K_{eq}$ fent
\[\Delta \varepsilon^{\circ}=\frac{0.059}{n} \ln K_{eq}\]
\[(0.337-(-0.763))=\frac{0.059}{2} \ln K_{eq}\]
\[K_{eq}=e^{\frac{2\cdot 1.1}{0.059}}=1.56\cdot 10^{16}\]
la qual cosa vol dir que la reacció és pràcticament total i que es consumeix pràcticament tot el \ch{Cu^{2+}_{(aq)}}.
\item[40\%] Comencem per escriure la reacció d'equilibri i pensem en com es genera ió acetat i protons a partir de la concentració que ens diuen d'àcid acètic
\begin{center}
\begin{tabular}{lccccc}
&\ch{CH3COOH} & \ch{<=>} & \ch{CH3COO-} & $+$ & \ch{H+} \\
inicial & $0.01$ & & $0$ &  & $10^{-7}$ \\
equilibri & $0.01-x$ & & $x$ && $x+10^{-7}$ 
\end{tabular}
\end{center}
Per tant, sabent el valor de la $K_a$ podem calcular $x$:
\[K_a = \frac{[\ch{CH3COO-}][\ch{H+}]}{[\ch{CH3COOH}]}=\frac{(x+10^{-7})\cdot x}{0.01-x}=1.76 \cdot 10^{-5}\]
En primera aproximació podem considerar que $10^{-7}$ serà molt més petit que $x$ per facilitar l'operació. Llavors
\[\frac{x^2}{0.01-x} \approx 1.76 \cdot 10^{-5}\]
\[x^2+1.76 \cdot 10^{-5} x - 1.76 \cdot 10^{-7}=0\]
\[x\approx 4.3\cdot 10^{-4}\]
on veiem que l'aproximació era prou bona. A partir d'aquesta dada, el $pH$ serà:
\[pH=-\log [\ch{H+}] \approx 3.4\]
\end{description}
}



% Problems End Here % ------------------------------------------------------- %

\problemsdone
\end{document}

% End of the Document
