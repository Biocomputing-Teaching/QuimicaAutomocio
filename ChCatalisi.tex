\chapter{Catàlisi}

\section{Metalls i catàlisi}

\begin{tcolorbox}[colback=green!5,colframe=green!40!black,title= Reciclatge de Platinum Group Metals (PGM)]
Automotive catalysts were introduced in the 1970’s
to reduce harmful atmospheric emissions. Today,
three-way-catalysts are able to decrease the emission
of carbon monoxide, hydrocarbons and nitrogen oxides.
Those compounds are captured through the catalytic
properties of precious metals like platinum, rhodium and
palladium. Due to this application and their monetary
value, PGM have become an important part of industrial
processes. Automotive catalysts are generally composed
of ceramics. These ceramics are loaded with precious
metals which are located on the wash coat (Mohallem et
al, 2011). About 50-60% of the precious metals contained
in catalysts are recycled worldwide (Hagelüken,
2012). To achieve recycling rates of up to 98%, milling,
sampling and refining has to be done with modern technologies.
In 2002 about 260t of PGM were refined from
spent catalytic converters (Hagelüken et al, 2003).\linkurl{http://www.herzog-maschinenfabrik.de/fileadmin/content/downloads/en/application_notes/HZ_Application_Note_08_en.pdf}
\end{tcolorbox}

\section{Els elements no metàl·lics i les seves propietats}
\section{Metalls de transició}
\section{Estat Sòlid}
