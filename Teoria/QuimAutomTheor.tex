\documentclass[a4paper,notitlepage,nobib]{tufte-book}
%\usepackage{booksprint}
\usepackage{tuftefoot}

% SILENCE THE WARNINGS!
%\usepackage{silence}

% tufte-book ja inclou el paquet geometry, i per tant només
% cal canviar alguns paràmetres amb \geometry
% \geometry{a4paper, top=25mm, bottom=30mm, inner=20mm, outer=70mm}
% \setlength{\marginparwidth}{50mm}  % Adjust margin for sidenotes
% %\geometry{margin=3cm,headsep=0.25in}
%\geometry{showframe}% for debugging purposes -- displays the margins
% The units package provides nice, non-stacked fractions and better spacing
% for units.
%\usepackage{units}
%\usepackage{todonotes}

\usepackage[backend=bibtex,style=numeric]{biblatex}  %backend=biber is 'better'

\usepackage{framed}
\usepackage{ifthen}
\usepackage{longtable}
\usepackage{fancyvrb}
\fvset{fontsize=\normalsize}
%\usepackage{cancel}

\usepackage[utf8]{inputenc}
\usepackage[catalan]{babel}
\usepackage{lmodern}
\usepackage{amsmath,amsthm,amsfonts,amssymb,amscd}

\usepackage{multirow,booktabs}
\usepackage[dvipsnames,table]{xcolor}
%\usepackage{fullpage}
\usepackage{lastpage}
\usepackage{graphicx}
%\setkeys{Gin}{width=\linewidth,totalheight=\textheight,keepaspectratio}
\graphicspath{{../figures/}}
\usepackage{enumitem}
\usepackage{mathrsfs}
\usepackage{wrapfig}
\usepackage{setspace}
\usepackage{calc}
\usepackage{multicol}
\usepackage{gensymb}




\usepackage{cancel}
\usepackage[retainorgcmds]{IEEEtrantools}

%\newlength{\tabcont}
% \setlength{\parindent}{0.0in}
% \setlength{\parskip}{0.05in}
%\usepackage{empheq}
% es recomana que mdframed es carregui després de xcolor
\usepackage[framemethod=TikZ]{mdframed}
\mdfdefinestyle{caixa}{leftmargin=1cm,innerrightmargin=0.5cm, linecolor=blue}

\usepackage{changepage}






  
%\chemsetup[chemformula]{format=\sffamily}

%\setatomsep{2em}
%\setdoublesep{.6ex}
%\setbondstyle{semithick}
\colorlet{shadecolor}{orange!15}
\parindent 0in
\parskip 12pt


\theoremstyle{definition}
\newtheorem{defn}{Definition}
\newtheorem{reg}{Rule}
\newtheorem{exer}{Exercise}
\newtheorem{note}{Note}
%\RequirePackage{mathrsfs}
%\RequirePackage[psamsfonts]{amsfonts} %for Y&Y BSR AMS fonts
\RequirePackage{amsmath,amsfonts,amsthm,amssymb}
\RequirePackage{setspace}
\RequirePackage{fancyhdr}
\RequirePackage{lastpage}
\RequirePackage{extramarks}
%\RequirePackage{chngpage}
\RequirePackage{soul}
%\RequirePackage{graphicx,float,wrapfig}
%\RequirePackage{pgf,tikz}
%\usetikzlibrary{arrows,automata}
%\RequirePackage{pstricks}
%\RequirePackage[text]{amsthm}
%\RequirePackage{array}
%\RequirePackage{amscd}
%\RequirePackage{array}\RequirePackage{dcolumn}

\newcommand{\emx}[1]{{\em{#1}\/}}
\newcommand{\abin}{{\it ab initio}}
\newcommand{\bs}{\boldsymbol}
%\newcommand{\citepnum}{\citep}
\newcommand{\dGo}{\ensuremath{\Delta G_0}}
\newcommand{\dG}[2]{\ensuremath{\Delta G_{\rm #1}^{\rm #2}}}
\newcommand{\dX}[3]{\ensuremath{\Delta #1_{\rm #2}^{\rm #3}}}
\newcommand{\ddgo}[1]{\ensuremath{\Delta \Delta G_{\rm solv}^{\rm #1}}}
\newcommand{\ddgstarcat}{\ensuremath{\Delta \Delta g^{\ddagger}_{\rm cat}}}
\newcommand{\ddgstar}{\ensuremath{\Delta \dgstar}}
\newcommand{\ddgt}[2]{\ensuremath{\Delta \Delta G_{\rm solv}^{\rm #1, \rm #2}}}
\newcommand{\ddsstarprime}{\ensuremath{(\Delta \dsstar)'}}
\newcommand{\deltaepsel}{\ensuremath{\Delta \varepsilon_{\rm el}}}
\newcommand{\deltaeps}{\ensuremath{\Delta \varepsilon}}
\newcommand{\dgab}[2]{\ensuremath{\Delta g_{\rm #1}^{\rm #2}}}
\newcommand{\dga}[1]{\ensuremath{\Delta g_{\rm #1}}}
\newcommand{\dgb}[1]{\ensuremath{\Delta g^{\rm #1}}}
\newcommand{\dgcage}{\ensuremath{\Delta g_{\rm cage}}}
\newcommand{\dgcat}{\ensuremath{\Delta g_{\rm cat}}}
\newcommand{\dgsoltsatsa}{\ensuremath{\dgsol (\rm TSA)_{\rm TSA}}}
\newcommand{\dgsoltstsa}{\ensuremath{\dgsol (\rm TS)_{\rm TSA}}}
\newcommand{\dgsoltsts}{\ensuremath{\dgsol (\rm TS)_{\rm TS}}}
\newcommand{\dgsol}{\ensuremath{\Delta G_{\rm sol}}}
\newcommand{\dgstarcage}{\ensuremath{\dgstar_{\rm cage}}}
\newcommand{\dgstarcat}{\ensuremath{\dgstar_{\rm cat}}}
\newcommand{\dgstarw}{\ensuremath{\dgstar_{\rm w}}}
\newcommand{\dgstar}{\ensuremath{\Delta g^{\ddagger}}}
\newcommand{\dgw}{\ensuremath{\Delta g_{\rm w}}}
\newcommand{\dg}[2]{\ensuremath{\Delta g_{\rm #1}^{\rm #2}}}
\newcommand{\dino}{\texttt{DINO}}
\newcommand{\dsstarcageprime}{\ensuremath{(\dsstarcage)'}}
\newcommand{\dsstarcage}{\ensuremath{\dsstar_{\rm cage}}}
\newcommand{\dsstarcatprime}{\ensuremath{(\dsstarcat)'}}
\newcommand{\dsstarcat}{\ensuremath{\dsstar_{\rm cat}}}
\newcommand{\dsstarwprime}{\ensuremath{(\dsstarw)'}}
\newcommand{\dsstarw}{\ensuremath{\dsstar_{\rm w}}}
\newcommand{\dsstar}{\ensuremath{\Delta S^{\ddagger}}}
\newcommand{\eg}{{\it e.g.}}
\newcommand{\etal}{{\it et al.}}
\newcommand{\gamess}{\texttt{GAMESS}}
\newcommand{\gauss}{\texttt{GAUSSIAN} 98}     
\newcommand{\golpe}{\texttt{GOLPE}}                                             
\newcommand{\grid}{\texttt{GRID}}
\newcommand{\ie}{{\it i.e.}}
\newcommand{\ith}{{\it i}$^{\rm th}$\ }
\newcommand{\kbt}{\ensuremath{k_{\rm B} T}}
\newcommand{\kb}{\ensuremath{k_{\rm B}}} 
\newcommand{\kcage}{\ensuremath{k_{\rm cage}}}
\newcommand{\kcatkm}{\ensuremath{k_{\rm cat}/K_{\rm M}}}
\newcommand{\kcat}{\ensuremath{k_{\rm cat}}}
\newcommand{\km}{\ensuremath{{\rm\, kcal \, mol}^-1}}
\newcommand{\knon}{\ensuremath{k_{\rm non}}}
\newcommand{\kw}{\ensuremath{k_{\rm w}}}
\newcommand{\mepsim}{\texttt{MEPSIM}}
\newcommand{\mgp}[1]{\marginpar{\scriptsize{#1}}}
\newcommand{\mipsim}{\texttt{MIPSIM}}
\newcommand{\mola}{\texttt{MOLARIS}}
\newcommand{\msms}{\texttt{MSMS}}
\newcommand{\pdras}{p21$^{\rm ras}$}
\newcommand{\rgran}{\ensuremath{\mathbb{R}}}
\newcommand{\rx}[2]{\ensuremath{#1_{\rm #2}}}
\newcommand{\vs}{{\it vs.}}
\newcommand{\z}[1]{\ensuremath{\mathbf{#1}}} 
\newcommand{\composed}[2]{#1\mathbin\circ #2}
\newcommand{\wrt}[1]{{\mbox{\scriptsize w.r.t. \( #1 \)} }}
\newcommand{\polyspace}{\mathcal{P}}
\newcommand{\matspace}{\mathcal{M}}
\newcommand{\C}{\mathbb{C}}
\newcommand{\N}{\mathbb{N}}
\newcommand{\Q}{\mathbb{Q}}
\newcommand{\Z}{\mathbb{Z}}
\renewcommand{\Re}{\mathbb{R}}
\newcommand{\rtres}{\ensuremath{\Re^3}}
\newcommand{\union}{\cup}
\newcommand{\dotprod}{\cdot}
%\newcommand*\pkg[1]{\textsf{#1}}

\newcommand{\trans}[1]{{#1}^{\ensuremath{\mathsf{T}}}} % transpose
\newcommand{\nbyn}[1]{\ensuremath{#1 \! \times \! #1 }}
\newcommand{\nbym}[2]{#1 \! \times \! #2 }       % Use in math mode.
\newcommand{\cat}[2]{#1\!\mathbin{\raise.6ex\hbox{\( {}^\frown \)}}\!#2}
\newcommand{\generalmatrix}[3]{ %arg1: low-case letter, arg2: rows, arg3: cols
               \left(
                  \begin{array}{cccc}
                    #1_{1,1}  &#1_{1,2}  &\ldots  &#1_{1,#2}  \\
                    #1_{2,1}  &#1_{2,2}  &\ldots  &#1_{2,#2}  \\
                              &\vdots                         \\
                    #1_{#3,1} &#1_{#3,2} &\ldots  &#1_{#3,#2}
                  \end{array}
               \right)  }
\newcommand{\colvec}[1]{\begin{pmatrix} #1 \end{pmatrix}}
\newcommand{\pr}[1]{\ensuremath{\mathrm{Pr}(#1)}}
\newcommand{\rep}[2]{ {\rm Rep}_{#2}(#1) }
\newcommand{\mapsunder}[1]{\stackrel{#1}{\longmapsto}}
\newcommand{\map}[3]{\mbox{$#1\colon #2\to #3$}}
\newcommand{\identity}{\mbox{id}}
\newcommand{\stdbasis}{{\cal E}} 
\newcommand{\sequence}[1]{ \langle#1\rangle } 
\newcommand{\spacer}{\rule[-3mm]{0mm}{8mm}}
\newcommand{\email}[1]{\url{#1}}
\newcommand{\zero}{\vec{0}}
\newcommand{\proj}[2]{\mbox{proj}_{#2}({#1}) }
%\AtBeginDocument{\newlength{\heightofcdot}
%\newlength{\widthofcdot}
%\settoheight{\heightofcdot}{$\cdot$}
%\settowidth{\widthofcdot}{$\cdot$}
%\newsavebox{\dotprodcircle}       
%\savebox{\dotprodcircle}{\includegraphics{dotprod.1}} 
%\newcommand{\dotprod}{\mathbin{\raisebox{.5\heightofcdot}{%
%          \makebox[\widthofcdot]{$\smash{\usebox{\dotprodcircle}}$}}}}}
\newcommand{\spanof}[1]{\relax [#1\relax ]} % no optional argument!
\newcommand{\set}[1]{\mbox{$\{#1\}$}} \newcommand{\suchthat}{\bigm|}
\newcommand{\deter}[1]{ \mathchoice{\left|#1\right|}{|#1|}{|#1|}{|#1|} }
\newcommand{\secuence}[1]{ \langle#1\rangle }  
\newcommand{\basis}[2]{\secuence{\vec{#1}_1,\ldots,\vec{#1}_{#2}}}



%--------linsys
%  Use as \begin{linsys}{3}
%           x &+ &3y &+ &a &= &7 \\
%           x &- &3y &+ &a &= &7
%         \end{linsys}
% Remark: TeXbook pp. 167-170 says to put a medmuskip around a +; and that's
% 4/18-ths of an em.  Why does 2/18-ths of an em work?  I don't know, but
% comparing to a regular displayed equation suggests it is right.
% (darseneau says LaTeX puts in half an \arraycolsep.)
\newenvironment{linsys}[2][m]{%
\setlength{\arraycolsep}{.1111em} % p. 170 TeXbook; a medmuskip
\begin{array}[#1]{@{}*{#2}{rc}r@{}}
}{%
\end{array}}


%\newtheorem{teorema}{Teorema}
%\newtheorem{exercici}{Exercici}
%\newtheorem{definicio}{Definici\'o}
%\newtheorem{theorem}{Theorem}
\newtheorem{exercise}{Exercise}
%\newtheorem{definition}{Definition}



\parskip 4mm


\usepackage{makeidx}
\makeindex




%\setcounter{section}{-1}

\theoremstyle{definition}
\newtheorem{thm}{Theorem}
\newtheorem{dfn}{Definition}
\newtheorem{lem}{Lemma}
\newtheorem{prp}{Proposition}





%%%%%%%%%%%%%%%%%%%
% ANGLÈS
%%%%%%%%%%%%%%%%%%%

% \newcommand{\problemName}{}%
% \newcounter{problemCounter}%
% \newenvironment{problem}[1][Problem \arabic{problemCounter}]%
% 	{\stepcounter{problemCounter}%
% 		\renewcommand{\problemName}{#1}%
% 		\section*{\problemName}%
% 		\nobreak\extramarks{\problemName}{\problemName continued on next page\ldots}\nobreak%
% 		\nobreak\extramarks{\problemName (continued)}{\problemName continued on next page\ldots}\nobreak}%
% 	{\nobreak\extramarks{\problemName (continued)}{\problemName continued on next page\ldots}\nobreak%
% 		\nobreak\extramarks{\problemName}{}\nobreak}%

\newenvironment{example}{ % 
	\definecolor{shadecolor}{rgb}{0.8,1.0,0.8} %
	\begin{shaded} %
	\textcolor{OliveGreen}{\bf Example\\}%
} % 
{ %	
	\end{shaded}
} %


\newenvironment{introduction}{ % 
	\definecolor{shadecolor}{rgb}{1.0,1.0,0.8} %
	\begin{shaded} %
	% \textcolor{BrickRed}{\bf Introduction\\}%
} % 
{ %	
	\end{shaded}
} %


%%%%%%%%%%%%%%%%%%%
% CATALÀ
%%%%%%%%%%%%%%%%%%%
\newtheorem{teorema}{theorem}
\newenvironment{definicio}{ % 
	\definecolor{shadecolor}{rgb}{0.9,1.0,0.8} %
	\begin{shaded} %
	\textcolor{OliveGreen}{\bf Definicio\\}%
} % 
{ %	
	\end{shaded}
} %

%veure http://en.wikibooks.org/wiki/LaTeX/Advanced_Topics

\newcommand{\doccmd}[1]{\texttt{\textbackslash#1}}% command name -- adds backslash automatically
\newcommand{\docopt}[1]{\ensuremath{\langle}\textrm{\textit{#1}}\ensuremath{\rangle}}% optional command argument
\newcommand{\docarg}[1]{\textrm{\textit{#1}}}% (required) command argument
\newenvironment{docspec}{\begin{quote}\noindent}{\end{quote}}% command specification environment
\newcommand{\docenv}[1]{\textsf{#1}}% environment name
\newcommand{\docpkg}[1]{\texttt{#1}}% package name
\newcommand{\doccls}[1]{\texttt{#1}}% document class name
\newcommand{\docclsopt}[1]{\texttt{#1}}% document class option name
\newcommand{\logos}{%
\begin{figure}
\includegraphics{FCTE}
\end{figure}
}

% margins
% \topmargin=-0.45in      %
% \evensidemargin=0in     %
% \oddsidemargin=0in      %
% \textwidth=6in        %
% \textheight=8.5in       %
% \headsep=0.25in         %

% header and footer
\pagestyle{fancy}       %
\chead{}                %
\makeatletter
\fancyfoot[R]{%
   % We want italics
   \itshape
   % The chapter number only if it's greater than 0
   \ifnum\value{chapter}>0 \@chapapp\ \thechapter. \fi
   % The chapter title
   \leftmark}
\makeatother

%\lfoot{\includegraphics[trim=-5cm 0 0 -3cm,width=0.4\textwidth]{FCTE}}      
\lfoot{\raisebox{-0.5cm}[0pt][0pt]{\includegraphics[width=3cm]{FCTE}}} 

\cfoot{}        %
\renewcommand\headrulewidth{0.4pt}   %
\renewcommand\footrulewidth{0.4pt}   %

\input{common_lst.tex}
\usepackage{chemfig,chemmacros,chemnum}
\chemsetup[reactions]{
    before-tag = R,
    tag-open = [ , tag-close = ]
}
\renewcommand*\printatom[1]{\ensuremath{\mathsf{#1}}}
\usepackage{chemformula}

%\usepackage{chemfig,chemmacros,chemnum}
\usepackage{chemformula}
\definecolor{mygreen}{RGB}{28,172,0} % color values Red, Green, Blue
\definecolor{mylilas}{RGB}{170,55,241}
\definecolor{LightOcean}{RGB}{81, 147, 229 }
\definecolor{DeepOcean}{RGB}{51, 131, 229}
\newtcolorbox{mybox}[1][]{%
    %float, 
    %floatplacement=t,
    %enhanced, 
    colback=LightOcean!10, 
    colframe=DeepOcean,
    % overlay unbroken and first={%
    % \ifoddpage\coordinate (X) at ([xshift=-6mm,yshift=-6mm]frame.north east);
    %      \else\coordinate (X) at ([xshift=6mm,yshift=-6mm]frame.north west);\fi
    % \node at (X) {\includegraphics[width=8mm]{FCTE}};}
%   show bounding box,
    %notitle,
    if odd page={grow to right by=\marginparsep+\marginparwidth-15mm}{grow to left by=\marginparsep+\marginparwidth-15mm},
    toggle enlargement=evenpage,
    #1
}
\newcounter{myc}
%%%%%%%%%%%%%%%%%%%%%%%%%%%%%%%%%%%%%%%%%
%%%%%%%%%%%%%%%%%%%%%%%%%%%%%%%%%%%%%%%%%
% lecturer or student text
% in principle the lecturer text includes some examples to be done in the c lass
\newboolean{LECT}
\setboolean{LECT}{false}
\setboolean{LECT}{true}
%%%%%%%%%%%%%%%%%%%%%%%%%%%%%%%%%%%%%%%%%
%%%%%%%%%%%%%%%%%%%%%%%%%%%%%%%%%%%%%%%%%

\newenvironment{lect}{ % 
	\definecolor{shadecolor}{rgb}{1.0,0.8,0.8} %
	\begin{shaded} %
	\textcolor{BrickRed}{\bf Resposta\\}%

} % 
{ %	
	\end{shaded}
} %

\newcommand{\lct}[1]{\ifthenelse{\boolean{LECT}}{\begin{lect} #1 \end{lect}}{}}


%environment for questions in exams
\newenvironment{qst}{ % 
    \addtocounter{myc}{1}
	\definecolor{shadecolor}{rgb}{0.9,1.0,0.8} %
	\begin{shaded} %
	\textcolor{OliveGreen}{\bf Qüestió \arabic{myc}\\}%
} % 
{ %	
	\end{shaded}
} %

%environment for exercises in the class notes
\newenvironment{exr}{ % 
    \addtocounter{myc}{1}
	\definecolor{shadecolor}{rgb}{0.9,1.0,0.8} %
	\begin{shaded} %
	\textcolor{OliveGreen}{\bf Exercici \arabic{myc}\\}%
} % 
{ %	
	\end{shaded}
} %

% Definició d'unitats personalitzades per al sistema imperial
\DeclareSIUnit\inch{in}
\DeclareSIUnit\foot{ft}
\DeclareSIUnit\atm{atm}
\DeclareSIUnit\oz{oz}
\DeclareSIUnit\ounce{oz}
\DeclareSIUnit\pound{lb}
\DeclareSIUnit\ton{t}

\DeclareSIUnit\cubicinch{\inch\cubed}
\DeclareSIUnit\cubicfoot{\foot\cubed}
\DeclareSIUnit\gallon{gal}
\DeclareSIUnit\psi{psi}
\DeclareSIUnit\inchHg{inHg}
\DeclareSIUnit\dyn{dynes}

\DeclareSIUnit{\degreeFahrenheit}{\unit{\degree}F}
\newcommand{\degC}{\degreeCelsius}
\newcommand{\degF}{\degreeFahrenheit}


\title[Introducció a la Química]{Introducció a la Química en enginyeria de l'Automoció}
\date{\today}
\author{Jordi Villà i Freixa}

\begin{document}
\maketitle

%\thispagestyle{empty}

% \begin{center}
% {\LARGE \bf Enginyeria de l'Automoció}\\
% {\large Introducció a la Química}\\
% Segon semestre curs 2024-2025\\
% Can Muntanyola, Granollers\\
% \textit{Jordi Vill\`a i Freixa (jordi.villa@uvic.cat)}\footnote{This work is licensed under a Creative Commons Attribution-ShareAlike 3.0 Unported License.
%   \begin{center}
%     \includegraphics[scale=0.1]{CC-BY-NC-SA.png}
%   \end{center}
%   Si no s'especifica el contrari, les figures han estat extretes de la Wikipedia o altres fonts d'Internet que permeten la seva reutilització. Les figures que inclouen compostos químics, sempre que hagi estat possible, s'han creat amb el paquet \LaTeX 
%    \textsf{chemfig}.}
% \end{center}
\logos


\tableofcontents

% nova organització per al curs de química en enginyeria de l'automoció

\chapter{Les propietats i el comportament dels gasos}

L'estudi dels gasos és fonamental per a comprendre el comportament de la matèria en estat gasós. Aquests conceptes són claus tant en la química moderna com en l'aplicació industrial. Les lleis dels gasos proporcionen una base per descriure el comportament macroscòpic dels gasos en funció de la temperatura, el volum i la pressió.



\section{Les lleis dels gasos}
En general, el volum d'un gas està determinat per la seva temperatura i la pressió que suporta. Existeix una relació matemàtica entre aquests paràmetres, que s'expressa com l'\textbf{equació d'estat}:
\begin{equation}
V = V(T, P, n),
\end{equation}
on $V$ és el volum, $T$ és la temperatura, $P$ la pressió, i $n$ el nombre de mols del material. Es tracta d'una equació que pot ser molt complexa i específica per a líquids i sòlids, però en el cas dels gasos tots ells tenen un comportament molt similar. Això és degut a que en l'estat gas, les molècules són mes independents entre elles i, per tant, la seva naturalesa molecular no afecta substancialment al comportament del tot.

\begin{mybox}[title=De partícules i mols de partícules]
    El mol és la unitat bàsica del Sistema Internacional per mesurar la quantitat de substància, i s'utilitza per comptar partícules com àtoms, molècules o ions. Un mol conté exactament \(6,022 \times 10^{23}\) entitats elementals, un valor conegut com el nombre d'Avogadro. Aquesta constant permet connectar les dimensions microscòpiques (com la massa i el nombre de partícules) amb mesures macroscòpiques utilitzades en els experiments químics. Per exemple, un mol d'àtoms de carboni-12 (que representarem per \isotope*{12,C}, a partir d'ara) té una massa de 12 grams, facilitant així la relació entre l'estructura atòmica i la pràctica de la química.
\end{mybox}
    
\subsection{Pressió i força}

Un dispositiu típic per mesurar la pressió és el baròmetre, que utilitza una columna de mercuri per determinar la pressió atmosfèrica. 

\begin{figure}[h]
    \centering
    \includegraphics[width=0.33\textwidth]{Old-barometers.jpg}
    \includegraphics[width=0.33\textwidth]{Manometre.png}
    \caption[Baròmetre i Manòmetre diferencial]{El baròmetre (esquerra) utilitza una columna de mercuri per determinar la pressió atmosfèrica. Un manòmetre diferencial (dreta) mesura la diferència entre les pressions externes i d'un determinat gas.}
    \label{fig:Manometre}
    \end{figure}

La pressió és definida com la força per unitat d'àrea que un gas exerceix sobre les parets del recipient que el conté. S'expressa comunament en unitats com pascals (\si\pascal) o atmosferes (\si\atm). Matemàticament:
\begin{equation}
\text{Pressió} = \frac{\text{Força}}{\text{Àrea}} = \frac{\text{massa} \times \text{acceleració}}{\text{Àrea}} = \frac{\text{massa} \times \text{acceleració}}{\text{Volum / alçada}} 
\end{equation}
Per tant, la pressió es calcula com:
\begin{equation}
P = \rho \cdot g \cdot h,
\end{equation}
on $\rho$ és la densitat, $g$ l'acceleració gravitatòria i $h$ l'alçada.

Calculem ara què és una atmosfera quan s'expressa en funció de força per àrea unitaria. Considerem una columna de mercuri amb una alçada de 760 mm. Sabem que la densitat del mercuri és $13.6 \cdot 10^3 \si{\kg\per\meter\tothe{3}}$ i l'acceleració gravitatòria és $9.8 \si{\meter\per\square\second}$.  Considerem un tub baromètric la superfície de secció transversal del qual és 1 \si{\square\cm}. Aleshores, la força que exerceix la columna de mercuri sobre aquesta superfície és igual a la massa del mercuri que es troba al tub, multiplicada per l'acceleració deguda a la gravetat. A la vegada, la massa del mercuri que està en el tub és el volum del mercuri multiplicat per la seva densitat a $0^{\circ}\text{C}$. Així doncs, es té:

\[
\text{força} = 
\]
\[
= \text{densitat del Hg} \times \text{alçada} \times \text{àrea} \times \text{acceleració}
\]
\[
= 13,59 \, \frac{\si\g}{\si{\cubic\cm}} \times 76,00 \, \si{\cm} \times 1,000 \, \si{\square\cm} \times 980,7 \, \frac{\si{\cm}}{\si{\s}^2}
\]
\[
= 1,013 \times 10^6 \, \si\g \cdot \frac{\si\cm}{\si{\s}^2} = 10,13 \, \si\kg \cdot \frac{\si\m}{\si{\s^2}}
\]
\[
= 10,13 \si{\newton}.
\]

Aquesta és la força que exerceix una columna de mercuri de 760 mm d'alçada i d'1 $\text{cm}^2$ de superfície de secció transversal. Per tant, és també la força per unitat de superfície (un centímetre quadrat) que correspon a la pressió d'una atmosfera. Així, es té que:

\[
1 \, \si{\atm} = 760,0 \, \si\mmHg= 760 \,\si\torr
= 1,013 \times 10^6 \, \si{\dyn\per\square\cm} = 1,013 \times 10^5 \, \si{\newton\per\square\meter}.
\]

Els gasos es comporten segons certes lleis empíriques que han estat establertes experimentalment. Aquestes lleis condueixen finalment a la formulació de la llei dels gasos ideals.

\section{Llei de Boyle}
La llei de Boyle estableix que, a temperatura constant, la pressió \( P \) d'un gas és inversament proporcional al seu volum \( V \):
\begin{equation}
    P V = \text{constant}
\end{equation}
On \( P \) s'expressa en \si{Pa} (pascals) i \( V \) en \si{m^3}.

\section{Llei de Charles}
La llei de Charles afirma que, a pressió constant, el volum d'un gas és directament proporcional a la seva temperatura absoluta \( T \):
\begin{equation}
    \frac{V}{T} = \text{constant}
\end{equation}
On \( T \) es mesura en \si{K} (kelvins).

\section{Llei d'Amonton (o de Gay-Lussac)}
La llei de Gay-Lussac és una llei dels gasos que estableix que la pressió \( P \) exercida per un gas (d'una massa donada i mantingut a volum constant) varia directament amb la temperatura absoluta del gas:
\begin{equation}
    T \propto P \quad \text{o} \quad P = \text{constant} \times T
\end{equation}
En altres paraules, si un gas ideal està confinat en un recipient amb volum constant i s'incrementa la temperatura, la pressió augmentarà proporcionalment a la temperatura.


\section{Llei dels Gasos Ideals}
Combinant les tres lleis anteriors, obtenim la llei dels gasos ideals:
\begin{equation}
    P V = n R T
\end{equation}
On:
\begin{itemize}
    \item \( P \) és la pressió en \si{Pa}
    \item \( V \) és el volum en \si{m^3}
    \item \( n \) és el nombre de mols
    \item \( R \) és la constant dels gasos, amb valor \( \SI{8.314}{J.mol^{-1}.K^{-1}} \)
    \item \( T \) és la temperatura en \si{K}
\end{itemize}

\section{Llei de Dalton}
La llei de les pressions parcials de Dalton estableix que la pressió total d'una mescla de gasos ideals és igual a la suma de les pressions parcials dels gasos individuals en la mescla. Matemàticament, es pot expressar així:

\[
P_{\text{total}} = P_1 + P_2 + P_3 + \cdots + P_n
\]

on \(P_{\text{total}}\) és la pressió total de la mescla, i \(P_1, P_2, \dots, P_n\) són les pressions parcials dels diferents gasos presents a la mescla.

La pressió parcial d'un gas és la pressió que exerciria aquest gas si ocupés tot el volum per si sol, a la mateixa temperatura.



% \chapter{Combustió}
 
\tableofcontents\newpage

\section{El motor de combustió interna}

Un motor de combustió interna (IC) és un conjunt d'elements mecànics que permeten obtenir energia mecànica a partir de l'estat tèrmic d'un fluid de treball generat en el seu propi interior mitjançant un procés de combustió.
 
Els motors de combustió interna, ja siguin alternatius o de reacció, són les principals fonts d'energia en el transport terrestre, marítim i aeri gràcies a la seva elevada potència específica. Aquests motors només competeixen amb els motors elèctrics en algunes aplicacions del transport ferroviari i, de manera creixent, en vehicles elèctrics purs o en configuració híbrida\cite{de_antonio_motores_2015}. 

En un motor de combusti\'o interna s'introdueix aire i combustible. En els motors d'encesa per espurna, la mescla d'aire i combustible es preparava antigament en el carburador i es condu\"ia al cilindre. Ara es realitza per mitj\`a d'injectors, cosa que permet un estalvi de combustible i un millor aprofitament d'aquest. En els motors d'encesa per compressi\'o (Diesel), la mescla es realitza directament dins del cilindre, on el combustible s'injecta despr\'es d'haver-hi introdu\"it i comprimit l'aire. Cada cilindre del motor t\'e una v\`alvula d'admissi\'o i una d'escapament, que s'obren i tanquen en el moment oport\'u per permetre l'entrada i sortida de gasos. Els motors típics tenen entre 3 i 12 cilindres, i la pot\`encia es pot augmentar afegint m\'es cilindres.

La paret de la cambra de combustió està formada per una camisa de ferro o alumini, i està inserida en un bloc de ferro o acer.

La mescla comprimida a la cambra de combusti\'o es transforma, per efecte de la combusti\'o, en vapor d'aigua (\ch{H2O}), di\`oxid de carboni (\ch{CO2}) i nitrogen (\ch{N2}). El nitrogen, un gas inert contingut a l'aire, no interv\'e en la combusti\'o. El vapor d'aigua produ\"it en la combusti\'o es mant\'e i es comporta com un gas permanent.

Entre els altres productes de la combusti\'o es troben altres gasos com: mon\`oxid de carboni (\ch{CO}), hidrogen (\ch{H2}), metà (\ch{CH4}) i oxigen (\ch{O2}), quan la combusti\'o \`es incompleta. La quantitat d'oxigen que participa en el proc\'es dep\`en directament de l'exc\'es d'aire introdu\"it respecte al necessari per a la combusti\'o.

En conseq\"u\`encia, el fluid de treball est\`a format inicialment per l'aire i el combustible i, despr\'es, pel conjunt de gasos produ\"its durant la combusti\'o. Com \`es evident, la seva composici\'o qu\'imica varia durant el cicle de treball.


\subsection{El motor de quatre temps}

    Un motor de quatre temps és aquell que necessita quatre recorreguts del pistó, dues voltes completes del cigonyal, per completar el seu cicle termodinàmic (veure animació a \url{https://www.grc.nasa.gov/www/k-12/airplane/engopt.html}).

    \newif\ifspark
\tikzset{tangent of circles/.style args={% https://tex.stackexchange.com/a/464143/194703
    at #1 and #2 with radii #3 and #4}{insert path={%
    let \p1=($(#2)-(#1)$),\n1={atan2(\y1,\x1)},\n2={veclen(\y1,\x1)*1pt/1cm},
    \n3={atan2(#4-#3,\n2)}
     in ($(#1)+(\n3+\n1+90:#3)$) coordinate(aux1) -- 
     ($(#2)+(\n3+\n1+90:#4)$) coordinate(aux2)}},
     pics/engine/.style={code={
  \tikzset{engine/.cd,#1}
  \draw[fill=gray!20] (0,0) -- (-0.8,-0.4) coordinate[pos=0.4] (p1)
  coordinate[pos=0.8] (p2) |- (-1,-3)[rounded corners=1mm] |- (-1.2,0) [sharp corners]
  -- (-1.2,0.7) coordinate[pos=0.2] (p3)
  coordinate[pos=0.8] (p4) -- (-0.9,0.85) -- (-0.6,0.7) -- (0,0.4) -- (0.6,0.7)
  -- (0.9,0.85)-- (1.2,0.7) -- (1.2,0)coordinate[pos=0.2] (p6)
  coordinate[pos=0.8] (p5) {[rounded corners=1mm] -- (1,0)}
  [sharp corners] -- (1,-3)
  -| (0.8,-0.4) -- cycle coordinate[pos=0.2] (p8)
  coordinate[pos=0.6] (p7);
  \draw[engine/left exhaust] (p1) to[bend right=18] (p4) -- (p3) to[bend left=18] (p2) -- cycle;
  \draw[engine/right exhaust] (p7) to[bend left=18] (p6) -- (p5) to[bend right=18] (p8) -- cycle;
  \draw[fill=gray!50] (0,-4) circle[radius=5mm];
  \pgfmathsetmacro{\pistonpos}{-4+0.4*sin(\pgfkeysvalueof{/tikz/engine/rod angle})
  +sqrt(1.5*1.5-pow(0.4*cos(\pgfkeysvalueof{/tikz/engine/rod angle}),2))}
  \path (0,-4) + (\pgfkeysvalueof{/tikz/engine/rod angle}:0.4) coordinate (p9)
   (0,\pistonpos) coordinate (p10);
  \draw[fill=gray!15] (p9) circle [radius=2mm] -- (p10) circle [radius=1mm];
  \path[tangent of circles={at p10 and p9 with radii 0.1 and 0.2}]
  (aux1) coordinate (aux3) (aux2) coordinate (aux4); 
  \path[tangent of circles={at p9 and p10 with radii 0.2 and 0.1}];
  \path[fill=gray!15] (aux1) -- (aux2) -- (aux3) -- (aux4);
  \draw  (aux1) -- (aux2)  (aux3) -- (aux4);
  \path[fill=gray!45] (p9) circle [radius=1.2mm];
  \begin{scope}
   \clip (-0.8,\pistonpos)   rectangle ++ (1.6,1);
   \draw[left color=gray!60,right color=gray!50,middle color=gray!10] (-0.8,\pistonpos) 
  rectangle ++ (2,1);
  \end{scope}
  \draw[left color=\pgfkeysvalueof{/tikz/engine/interior color}!80,
  right color=\pgfkeysvalueof{/tikz/engine/interior color}!50,
  middle color=white] 
  (-0.8,\pistonpos+1) --  (-0.8,-0.4)  -- (0,0)--  (0.8,-0.4) |- cycle;
  \draw[thin,fill=gray!30] (-0.42,-0.5) 
   ++ ({90+atan(1/2)}:0.25*\pgfkeysvalueof{/tikz/engine/left valve}) 
   -- ++ ({90+atan(1/2)}:1.9) -- ++ ({atan(1/2)}:0.1)
   -- ++ ({-90+atan(1/2)}:1.9) -- ++({atan(1/2)}:0.3)
   -- ++ ({-90+atan(1/2)}:0.1) -- ++({atan(1/2)+180}:0.7)
   -- ++ ({90+atan(1/2)}:0.1) -- cycle;
  \draw[thin,fill=gray!30] (0.42,-0.5) 
   ++ ({90-atan(1/2)}:0.25*\pgfkeysvalueof{/tikz/engine/right valve}) 
   -- ++ ({90-atan(1/2)}:1.9) -- ++ ({180-atan(1/2)}:0.1)
   -- ++ ({-90-atan(1/2)}:1.9) -- ++({180-atan(1/2)}:0.3)
   -- ++ ({-90-atan(1/2)}:0.1) -- ++({-atan(1/2)}:0.7)
   -- ++ ({90-atan(1/2)}:0.1) -- cycle;
  \draw[left color=gray!60,right color=gray!50,middle color=gray!10]
   (-0.1,-0.2) rectangle (0.1,1);   
  \ifspark
  \begin{scope}
   \clip (-1.8,-0.2) rectangle (1.8,\pistonpos+1.1);
   \path (0,-0.2) node[starburst, inner color=yellow, outer color=red,minimum size=1cm]{};
  \end{scope}
  \fi 
 }},engine/.cd,left valve/.initial=1,right valve/.initial=1,
 left exhaust/.style={fill=gray!50},
 right exhaust/.style={fill=gray!50},
 rod angle/.initial=30,interior color/.initial=white,
 spark/.is if=spark}
 \begin{center}
 \scalebox{0.8}{
\begin{tikzpicture}[] 
 \path (0,0) pic{engine={left valve=0,rod angle=-40,
  left exhaust/.style={fill=gray!10}}}
 (3.2,0) pic{engine={rod angle=-170,interior color=yellow}}
 (6.4,0) pic{engine={rod angle=105,interior color=orange,spark}}
 (9.6,0) pic{engine={rod angle=-80,interior color=red}}
 (12.8,0) pic{engine={right valve=0,rod angle=-170,interior color=purple,
    right exhaust/.style={fill=purple!30}}};
\end{tikzpicture}
 }
\end{center}

\begin{itemize}

\item{Primer pas o admissió}
En aquesta etapa, quan el pistó baixa des del Punt Mort superior (PMS o, en anglès, top dead center, TDC) al Punt Mort Inferior (PMI o, en anglès bottom dead center, BDC), permet que el nou combustible entri per la vàlvula d'injecció. Mentre s'obre aquesta vàlvula, la d'escapament es manté tancada.

\item{Segon pas o compressió}
Al final de l'execució anterior, el gas dins del cilindre es comprimeix per mitjà del moviment ascendent del pistó, de manera que la vàlvula d'injecció es tanca per la pressió.

\item{Tercer pas o explosió/expansió}
Després del temps de compressió, quan el pistó torna a la posició superior, s'obté la pressió màxima dins del cilindre. En el nostre cas, tenim un motor dièsel, per la qual cosa el combustible s'injecta polvoritzat i es crema per mitjà de la pressió i la temperatura dins del cilindre. Aleshores, l'expansió del gas fa que el pistó es mogui de nou cap avall; és en aquest moment quan es crea el treball de tot el procés. El treball d'expansió obtingut és aproximadament cinc vegades el treball de compressió necessari.

\item Quart pas o escapament
En aquest últim pas, el moviment superior del pistó fa que els gasos de combustió surtin a través de la vàlvula d'escapament. Quan el pistó està a la part superior, la vàlvula d'escapament es tanca i la injecció s'obre perquè tot el procés es torni a iniciar.
\end{itemize}

El cigonyal completa dues voltes (720 graus) per cada cicle de quatre temps. Així, el motor de quatre temps necessita dues voltes completes del cigonyal per completar el seu cicle termodinàmic.

Molts dels comportaments del motor es poden descriure mitjan\c{c}ant els conceptes de les lleis dels gasos. Per exemple, segons la llei de Boyle, quan augmenta el volum de la cambra de combusti\'o durant l'aspiraci\'o, la pressi\'o disminueix i permet que l'aire entri al cilindre. Durant la compressi\'o, el gas s'escalfa i augmenta la pressi\'o. L'expansi\'o dels gasos calents, descrita per la llei de Charles, \`es el mecanisme pel qual es captura l'energia de la combusti\'o i es converteix en energia mec\`anica per impulsar el vehicle\cite{bowers_understanding_2014}.

    \subsection{Fases del Cicle Otto ideal}

    La Figure \ref{fig:otto} mostra els processos termodinàmics que es donen en el cicle Otto\cite{morales_caracterizacion_nodate}:
\begin{enumerate}
    \item 0-1 Aspiraci\'o (proc\'es isoc\`oric): 
    La v\'alvula d'admissi\'o s'obre i s'aspira una c\`arrega d'aire i combustible a una pressi\'o te\`oricament igual a l'atmosf\`erica, provocant el descens del pist\'o. La v\'alvula d'escapament roman tancada. L'injector de fuel  genera un aerosol de combustible, en forma d'una fina boira de gotes minúscules, que es barreja amb l'aire aspirat.
    
    \item 1-2 Compressi\'o (proc\'es adiab\`atic):
    No existeix intercanvi de calor entre el gas i les parets del cilindre. La v\'alvula d'admissi\'o i la d'escapament estan tancades i el pist\'o comen\c{c}a a pujar, comprimint la mescla que es vaporitza.
    
    \item 2-3 Combusti\'o (proc\'es isoc\`oric):
    Ambdues v\'alvules romanen tancades. Quan el pist\'o arriba a la part superior del seu recorregut, el gas comprimit s'inflama per l'espurna de la bugia. La combusti\'o de tota la massa gasosa \`es instant\`ania, per la qual cosa el volum no variar\`a i la pressi\'o augmentar\`a r\`apidament. Això és degut a que la reacció genera molts més mols de gas que els inicials, i la temperatura augmenta enormement degut a la reacció química.
    
    \item 3-4 Expansi\'o (proc\'es adiab\`atic): 
    El gas inflamat empeny el pist\'o. Durant l'expansi\'o, no hi ha intercanvi de calor i, en augmentar el volum, la pressi\'o tamb\'e augmenta.
    
    \item 4-1 Escapament (proc\'es isoc\`oric)
    Quan el pist\'o es troba en l'extrem inferior del seu recorregut, la v\'alvula d'admissi\'o roman tancada i s'obre la d'escapament, disminuint r\`apidament la pressi\'o sense variar el volum interior. Despr\'es, mantenint la pressi\'o igual a l'atmosf\`erica, el volum disminueix.
\end{enumerate}
    
    
        \begin{figure}
            \centering
            \scalebox{0.8}{
    \begin{tikzpicture}[annotate/.style 2 args={postaction={decorate,decoration={markings,
        mark=at position 0 with {\node[circle,inner sep=1.5pt,draw,fill=white,#1]{};},
        mark=at position 0.5 with {\arrow[>=stealth,line width=1.5pt]{>};
        \node at (0,0.4) {#2};}}}}]
         \draw[stealth-stealth] (0,5) node[below left]{$p$} |- (5,0) node[below left]{$V$};
         \begin{scope}[thick]
          \draw[annotate={label=below right:1,alias=1}{$\dbar Q=0$}] plot[variable=\x,domain=4:1.5] (\x,{5/(\x+3)});
          \draw[annotate={label=below left:2,alias=2}{}] (1.5,5/4.5) -- (1.5,15/4.5);
          \draw[annotate={label=above left:3,alias=3}{$\dbar Q=0$}] plot[variable=\x,domain=1.5:4] (\x,{15/(\x+3)});
          \draw[annotate={label=above right:4,alias=4}{}] (4,15/7) -- (1);  
         \end{scope} 
         \path (2) -- (3) coordinate[pos=0.5] (23) (1) -- (4) coordinate[pos=0.5] (14);
         \draw[stealth-] ([xshift=-2mm]23) -- ++ (-1,0) node[midway,above]{$\Delta Q_h$};
         \draw[-stealth] ([xshift=2mm]14) -- ++ (1,0) node[midway,above]{$\Delta Q_c$};
         \draw[dashed] (1) -- (1|-0,0) node[below] {$V_1$};
         \draw[dashed] (2) -- (2|-0,0) node[below] {$V_2$};
        \end{tikzpicture}
        }
        \includegraphics[scale=0.9]{../figures/Otto-real.png}
        \caption{La termodinàmica del cicle d'Otto. A l'esquerra, la situació ideal, on els processos d'expansió i compressió són adiabàtics, mentre que els de combustió i escapament són isocòrics. A la dreta, un esquema del cicle real.}
        \label{fig:otto}
    \end{figure}

    \subsection{Cicle Otto Real}

    El procés Otto real (Figura \ref{fig:otto}) s'allunya \href{http://tesla.us.es/wiki/index.php/Ciclo_Otto}{de forma significativa} de l'ideal. 



    \begin{enumerate}
        \item 0-1 Aspiraci\'o: 
        La pressi\'o del gas durant l'aspiraci\'o \'es inferior a la pressi\'o atmosf\`erica, per tant, el tancament de la v\'alvula d'admissi\'o es produeix despr\'es que el pist\'o arriba a l'extrem inferior de la seva carrera. Aix\`o prolonga el per\'iode d'admissi\'o i permet l'entrada de la m\`axima quantitat de mescla d'aire i combustible al cilindre.
        
        \item 1-2 Compressi\'o:
        El gas cedeix calor al cilindre, cosa que fa que es refredi i adquireixi menys pressi\'o.
        
        \item 2-3 Combusti\'o:
        La combusti\'o no \`es instant\`ania i el volum de la mescla varia mentre es propaga la inflamaci\'o. Per obtenir un m\`axim treball, \'es essencial triar el moment adequat per a l'encesa. La xispa ha de saltar abans que el pist\'o hagi finalitzat la carrera de compressi\'o, cosa que augmenta considerablement la pressi\'o assolida despr\'es de la combusti\'o i, per tant, el treball guanyat.
        
        \item 3-4 Expansi\'o: 
        L'augment de temperatura dins del cilindre durant la combusti\'o fa que, durant l'expansi\'o, els gasos cedeixin calor al cilindre i es refredin, resultant en una pressi\'o menor. Per tant, es tracta d'un procés no adiabàtic.
        
        \item 4-1 Escapament:
        En realitat, l'escapament no es produeix instant\`aniament. Els gasos encara tenen una pressi\'o superior a l'atmosf\`erica en aquest per\'iode. Per aix\`o, la v\'alvula d'escapament s'obre abans que el pist\'o arribi a l'extrem inferior del seu recorregut, permetent que la pressi\'o del gas disminueixi a mesura que el pist\'o acaba la seva carrera descendent. Quan el pist\'o realitza la seva carrera ascendent, troba davant seu gasos ja gaireb\'e totalment expandits. A m\'es, la v\'alvula d'admissi\'o s'obre abans que el pist\'o arribi a l'extrem superior del seu recorregut, generant una certa depressi\'o en el cilindre que afavoreix una aspiraci\'o m\'es en\`ergica.
    \end{enumerate}



\subsection{Cicle Diesel}

El motor Diesel \`es un motor de combusti\'o interna basat en el cicle Otto, per\`o amb la difer\`encia que el combustible s'injecta despr\'es de la compressi\'o de l'aire. 

Durant l'aspiraci\'o, entra nom\'es aire en el cilindre. En la compressi\'o, l'aire s'escalfa i, quan el pist\'o arriba al punt mort superior, s'injecta el di\`esel. Un motor diesel presenta uns factors de compressió molt més elevats que un motor Otto, i per tant, la temperatura de l'aire comprimit és molt més alta. Això permet que el dièsel s'encengui per la pressió i la temperatura de l'aire comprimit, sense necessitat d'una espurna. Finalment, l'escapament funciona de manera similar al motor d'encesa per espurna. 

Aquest motor permet una major efici\`encia t\'ermica i t\'e avantatges econ\`omics en diverses aplicacions. Tot i aix\`o, presenta dificultats t\`ecniques en sistemes d'injecci\'o i combusti\'o. Per garantir una combusti\'o neta i eficient, el proc\'es es realitza en mi\lgem isegons. Els motors Diesel usen ratios combustible/aire molt més baixos, amb la qual cosa la combustió és més completa.





    \section{Reaccions de combustió}

    Per definici\'o, una reacci\'o de combusti\'o \`es qualsevol reacci\'o entre un material i un oxidant [t\'ipicament \ch{O2 (g)}] que allibera energia en forma de calor. Les reaccions qu\'imiques alteren els tipus d'enlla\c{c}os i les posicions relatives dels \`atoms dins de les mol\'ecules. Els materials inicials s'anomenen reactius, i els materials finals despr\'es de la reordenaci\'o s'anomenen productes. En una reacci\'o de combusti\'o, el material inicial no oxidant s'anomena combustible i pot ser una varietat de compostos qu\'imics\cite{bowers_understanding_2014}.

Normalment, la combusti\'o es presenta en qu\'imica general i org\`anica com la reacci\'o dels combustibles hidrocarbonats amb l'oxigen per produir di\`oxid de carboni i aigua:
\begin{equation}
\ch{CH4 + 2 O2 -> CO2 + 2 H2O}
\end{equation}

No obstant aix\`o, els combustibles org\`anics contenen m\'es elements que nom\'es carboni i hidrogen, i produeixen altres gasos a banda del di\`oxid de carboni i l'aigua. Els motors de combusti\'o interna tamb\'e generen combustibles hidrocarbonats no cremats i els anomenats NOx t\'ermics, gasos amb la f\'ormula \ch{NO_x} que es formen quan el nitrogen atmosf\`eric es torna molt calent i reacciona amb l'oxigen atmosf\`eric. Aquests gasos contribueixen a les emissions dels motors i es redueixen mitjan\c{c}ant tecnologies d'emissions que es tractaran més endavant.


\subsection{Desti\lgem ació del petroli}

La majoria de motors de combustió interna de gasolina i dièsel estan dissenyats per utilitzar fraccions específiques d'hidrocarburs obtingudes del petroli cru. El petroli és una barreja complexa de compostos orgànics provinents de la descomposició de microorganismes marins enterrats. Només els components més lleugers i volàtils són adequats com a combustible per a vehicles\cite{bowers_understanding_2014}.

Aquests components se separen del petroli mitjançant desti\lgem ació, un procés on el líquid s'escalfa fins a bullir, i els vapors es refreden i es condensen en un recipient. En la desti\lgem ació industrial, això es fa en una torre de desti\lgem ació, un cilindre metà\lgem ic on els diferents components del petroli es condensen a diferents alçades segons el seu punt d'ebullició. Els compostos més lleugers surten per la part superior com a vapor, mentre que els més pesats es condensen més avall (Figura \ref{fig:torredestillacio}).

\begin{figure}
    \centering
    \includegraphics[width=\textwidth]{TorreDestillacio.jpg}
    \caption{Desti\lgem ació fraccionada del petroli\cite{noauthor_38_2015}.}
    \label{fig:torredestillacio}
\end{figure}


El dièsel es desti\lgem a entre 200°C i 350°C i conté hidrocarburs amb entre 8 i 21 àtoms de carboni. La gasolina, més volàtil, conté alcans lieals (parafines), alcans cíclics (naftalens) i alquens (olefines) de 4 a 12 carbonis i es desti\lgem a a temperatures més baixes, pel fet de ser més volàtil. Tant la gasolina com el dièsel inclouen additius químics per millorar la seva estabilitat i resistència a la compressió. Aquests additius solen ser substàncies orgàniques contenir nitrogen, fòsfor i oxigen, i també compostos aromàtics (anells de carboni amb enllaços híbrids).

Per simplificar, l'anàlisi de la combustió es centrarà en la gasolina i un dels seus principals components, l'octà, tot i que el mateix principi s'aplica a altres combustibles.

\subsection{L'índex d'octà}

    Què significa el número d'octà de la benzina o per què alguns cotxes necessiten gasolina premium? El número d'octà mesura la resistència del combustible a la ignició espontània quan es comprimeix.

    L'octà, o n-octà, és un hidrocarbur de la família dels alquans amb fórmula molecular \ch{C8H18}. És un líquid incolor, inodor i inflamable. És un component important de la gasolina, ja que té una estructura lineal que li permet tenir una alta resistència a la detonació. Això fa que sigui un combustible ideal per a motors d'alta compressió.

En un motor de combustió interna, el combustible ha de cremar quan s'encén la bugia. Si la compressió fa que es detoni abans d'horaes poden danyar components com vàlvules i pistons. Això es coneix com a picat de biela o preignició.

El número d'octà es determina en un laboratori, cremant el combustible en un motor amb ràtio de compressió variable fins que es detecta el picat. A partir d'això, es compara amb una barreja de \href{https://www.ebi.ac.uk/chebi/searchId.do?printerFriendlyView=true&chebiId=62805&structureView=applet}{isooctà} i heptà amb la mateixa resistència a la detonació. El número d'octà indica el percentatge d'isooctà en aquesta barreja equivalent. Per exemple, un combustible amb un número d'octà de 90 té la mateixa resistència a la preignició que una barreja del 90\% d'isooctà i 10\% d'heptà (veure Taula \ref{tab:octa}).

És important saber que aquest número no indica la quantitat real d'octà en la benzina. Hi ha altres compostos amb més resistència a la detonació que poden donar valors superiors a 100. En resum, com més alta sigui la ràtio de compressió del motor, més alt ha de ser el número d'octà per evitar problemes de preignició.  
\
\begin{table}[h!]
    \centering
    \caption{Taula de compostos amb les seves fórmules condensades i índex d'octà (Adaptada de \cite{noauthor_38_2015}). }
    \renewcommand{\arraystretch}{1.5}
    \scriptsize
    \begin{tabular}{p{1cm}cc|p{1cm}cc}
        \toprule
        \textbf{Nom} & \textbf{Fórmula } & \textbf{Índex} & \textbf{Nom} & \textbf{Fórmula } & \textbf{Índex} \\
        \midrule
        n-heptà & \ch{CH3-(CH2)5-CH3} & 0 &
        o-xilè & \chemfig{[:-30]**6(--(-CH3)-(-CH3)--(-[,,,,,draw=none])-)} & 107 \\
        n-hexà & \ch{CH3-(CH2)4-CH3} & 25 & 
        etanol & \ch{CH3CH2OH} & 108 \\
        n-pentà & \ch{CH3-(CH2)3-CH3} & 62 &
        t-butil alcohol & \ch{(CH3)3COH} & 113 \\
        isooctà & \ch{(CH3)3CCH2CH(CH3)2} & 100 &
        p-xilè & \chemfig{[:-30]**6((-CH3)---(-CH3)---)} & 116 \\
        benzè & \chemfig{[:-30]**6(------)} & 106 &
        metil terc-butil èter & \ch{H3COC(CH3)3} & 116 \\
        metanol & \ch{CH3OH} & 107 &
        toluè & \chemfig{[:-60]*6(-=-=(-CH3)-=)} & 118 \\
        \bottomrule
    \end{tabular}
    \normalsize
    \label{tab:octa}
\end{table}


Molts compostos actuals tenen un índex d'octà superior a 100, cosa que els fa millors combustibles que l'isooctà pur. A més, s'han desenvolupat agents anticolp, també anomenats potenciadors d'octà. Durant molts anys, un dels més utilitzats va ser el tetraetilplom \ch{(C2H5)4Pb}, que a una concentració d'aproximadament \qty{11.36}{\gram\per\litre} augmentava l'índex d'octà en 10-15 punts. No obstant això, des de 1975, els compostos de plom han estat progressivament eliminats com a additius de la gasolina a causa de la seva elevada toxicitat.

Per substituir-los, es van desenvolupar altres potenciadors com el metil terc-butil èter (MTBE), que té un índex d'octà elevat i causa poca corrosió al motor i al sistema de combustible. Tanmateix, les fuites de gasolina amb MTBE des de dipòsits subterranis han contaminat aigües subterrànies en algunes zones, fet que ha portat a restriccions o prohibicions del seu ús. Com a alternativa, s'està promovent l'ús d'etanol, que es pot obtenir de fonts renovables com el blat de moro, la canya de sucre o les gramínies.

\section{Termodinàmica de la combustió}

Les reaccions químiques poden ser, a nivell del calor que intercanvien amb l'entorn:
\begin{description}
\item[exotèrmiques] si desprenen calor i, per tant, l'energia dels productes és més baixa que la dels reactius; o bé
\item[endotèrmiques] si l'absorbeixen i els productes acaben tenint més energia que els reactius.
\end{description}

En moltes ocasions mirem d'obtenir treball a partir de la calor produïda en una reacció, com succeix, per exemple, en un procés de combustió o en les reaccions electroquímiques que fan funcionar motors de combustió o elèctrics. La calor generada per la combustió d'una quantitat de combustible s'anomena calor de combustió, i es mesura en \si{\joule\per\mole}. Aquesta quantitat varia segons el combustible i la seva composició.

La termodinàmica estudia les relacions entre energia, calor i treball.
En aquest capítol treballarem al voltant de la termoquímica, la termodinàmica associada a les reaccions químiques, no només a la combustió

\begin{mybox}[title={Sistemes, estats i funcions d'estat}]
    Anomenem \emph{sistema} a aquella part de l'\emph{univers} que volem tractar en algun càlcul o experiment. 
Per exemple, un sistema pot ser un cilindre en un motor de combustió o bé una bateria elèctrica.
\begin{center}
\includegraphics[scale=0.5]{SistEntornUnivers.png}
\end{center}
 
L'\textit{estat del sistema} es caracteritza  per unes determinades \textit{variables d'estat} ($P$, $V$, $T$, $E$, $H$,...), magnituds físiques macroscòpiques mesurables. La termodinàmica estudia els \textit{estats d'equilibri} dels sistemes, en els quals les variables d'estat són idèntiques en totes les seves parts i invariables en el temps:
\begin{enumerate}
\item En els \textit{canvis d'estat} d'un sistema, les variables d'estat només depenen de l'estat inicial i final, i no del camí utilitzat. Així, per exemple, el treball $w$ no és funció d'estat, mentre que l'energia $E$ sí que ho és.
\item En fixar els valors d'algunes d'elles, una equació d'estat determina automàticament el valor de les altres. Així, per exemple, en un gas ideal, si coneixem $P$, $V$ i $T$, podem determinar $E$, $H$, $S$, etc.
\end{enumerate}

Els canvis d'estat poden ser 
\begin{description}
\item[reversibles] quan les funcions d'estat varian de manera infinitessimal, mantenint el sistema constantment en l'equilibri (l'expansió d'un gas contra una pressió que difereix només $dP$ de la pressió interna, per exemple);
\item[irreversibles] en qualsevol altre situació (un procés de combustió, l'expansió d'un gas contra el buit, etc).
\end{description}
\end{mybox}

\subsection{Treball}

El treball realitzat per una força en desplaçar un cos entre dues posicions es calcula segons:
\[
w=\int_{x_1}^{x_2} \mathbf{F} \cdot \mathbf{dx}
\]
Tenint en compte que $P=\frac{F}{A}$, és fàcil veure que, en el cas d'un pistó que exerceixi una pressió externa sobre un gas 
\begin{center}
\includegraphics[scale=0.6]{pisto_dw.png}
\end{center}
tenim
\[
dw=-F_{ext}dx = -P_{ext} A dx = -P_{ext} dV
\]
i, per tant,
\[
w=-\int_{V_1}^{V_2} P_{ext} dV
\]
\begin{EXMP}[Treball en una expansió isobàrica]

Considerem un gas ideal que s'expandeix isobàricament (a pressió constant) des d'un volum inicial $V_1$ fins a un volum final $V_2$. El treball realitzat pel gas durant aquesta expansió es pot calcular com:

\[
w = -P_{\text{ext}} \Delta V = -P_{\text{ext}} (V_2 - V_1)
\]

On $P_{\text{ext}}$ és la pressió externa constant. Si la pressió està en \si{\pascal} i el volum en \si{\meter\cubed}, el treball es mesura en \si{\joule}.

Per exemple, si un gas s'expandeix des de \qty{1}{\meter\cubed} fins a \qty{2}{\meter\cubed} a una pressió constant de \qty{100}{\kilo\pascal}, el treball realitzat pel gas és:

\[
w = -\qty{100}{\kilo\pascal} \times (\qty{2}{\meter\cubed} - \qty{1}{\meter\cubed}) = -\qty{100}{\kilo\pascal} \times \qty{1}{\meter\cubed} = -\qty{100}{\kilo\joule}
\]

El signe negatiu indica que el treball és realitzat pel sistema (el gas) sobre l'entorn.
\end{EXMP}



\subsection{Calor}

La calor $q$ és una magnitud macroscòpica que representa l'efecte d'infinitud de treballs microscòpics deguts als moviments de les partícules d'un sistema.
Com el treball, no és una funció d'estat, ja que depèn del camí que utilitzem per transferir-lo.
La calor es medeix en calories o Joules.\marginnote{Definim com caloria la quantitat de calor necessària per escalfar 1 gr d'aigua \qty{1}{\degC}. Per tant, la capacitat calorífica de l'aigua és $C_p=\qty{1}{\cal\per\gram\per\degC}$. En realitat, això només és cert per a una temperatura donada, ja que la capacitat calorífica depèn lleugerament de la temperatura de partida. En el cas de l'aigua, la caloria es defineix per al pas de 14.5\unit{\degC} a 15.5\unit{\degC}. La quantitat de treball necessària per produir aquesta calor es va determinar per Mayer y Joule el s. XIX com \qty{1}{\cal}=\qty{4.1860}{\joule}. En química usem més sovint les Capacitats calorífiques molars, $C_m$,  que indiquen la quantitat de calor necessària per escalfar un mol d'una substància 1\unit{\degC}.}

La quantitat de calor necessària per incrementar la temperatura un determinat valor d'\qty{1}{\mole} de substància és
\[
q=nC_m\Delta T
\]
Si aquesta expressió la usem per explicar un procés infinitessimal obtenim
\[
C_m=\frac{1}{n}\frac{dq}{dT}
\]
I com que la capacitat calorífica es pot obtenir a $V=\text{cnt}$ o a $P=\text{cnt}$, podem calcular
\[
q_v=\int_{T_1}^{T_2} n C_{v,m} dT
\]
i
\[
q_p=\int_{T_1}^{T_2} n C_{p,m} dT
\]

\begin{EXMP}[Calor en un procés isocòric]

Considerem ara un gas ideal que s'escalfa isocòricament (a volum constant) des d'una temperatura inicial $T_1$ fins a una temperatura final $T_2$. La calor transferida al gas durant aquest procés es pot calcular com:

\[
q_v = n C_{v,m} \Delta T = n C_{v,m} (T_2 - T_1)
\]

On $n$ és el nombre de mols de gas, $C_{v,m}$ és la capacitat calorífica molar a volum constant, i $\Delta T$ és el canvi de temperatura. Si la capacitat calorífica està en \si{\joule\per\mole\per\kelvin} i la temperatura en \si{\kelvin}, la calor es mesura en \si{\joule}.
\end{EXMP}

\subsection{Primera llei de la termodinàmica}

La primera llei de la termodinàmica estableix que l'energia no es pot crear ni destruir, sinó que es pot transformar d'una forma a una altra. Això es pot expressar com:
\begin{equation}
    \Delta U = q + w
\end{equation}
on $\Delta U$ és la variació d'{\bf energia interna}, $q$ és la calor transferida al sistema i $w$ és el treball realitzat sobre el sistema. $U$ és una {\bf funció d'estat}, ja que el seu increment $\Delta U$ depèn només de l'estat inicial i final, no del camí seguit per arribar-hi. És per això que l'escrivim en majúscules, a diferència de la calor i el treball, que són funcions de camí.

Imaginem una reacció que es dona a pressió constant. En aquest cas, la calor transferida al sistema és la calor de combustió, i el treball realitzat és el treball de compressió. Així, la primera llei de la termodinàmica es pot reescriure com:
\begin{equation}
    \Delta U = q - P \Delta V
\end{equation}
on $P$ és la pressió i $\Delta V$ és el canvi de volum.
En forma integral, si la pressió no fos constant, això es pot expressar com:
\begin{equation}
    \Delta U = q - \int_{V_1}^{V_2} P \diff V
\end{equation}

Si la reacció estudiada fos a volum constant, és a dir, en un recipient tancat, el treball de compressió seria zero i la primera llei es reduiria a:
\begin{equation}
    \Delta U = q_v
\end{equation}

Per tant, per mesurar la calor de combustió d'un combustible, es pot utilitzar un calorímetre a volum constant, on tota l'energia alliberada per la reacció es converteix en calor. Les reaccions que desprenen calor s'anomenen \textbf{exotèrmiques}, mentre que les que l'absorbeixen s'anomenen \textbf{endotèrmiques}. Si el sistema absorbeix calor, la variació d'energia interna serà positiva, i si la despren, serà negativa.

Normalment, però, les reaccions químiques succeeixen a pressió constant, i per tant, la calor de combustió es mesura a pressió constant. Això es pot fer amb un calorímetre a pressió constant, on la calor de combustió es converteix en treball de compressió. En aquest cas, ens convé més usar una altra funció d'estat, l'{\bf entalpia}, que es defineix com:
\begin{equation}
    H = U + PV
\end{equation}
i la calor de combustió es pot expressar com:
\begin{equation}
    \Delta H = \Delta U + \Delta (PV) = q+ w + \Delta (PV) \end{equation}
    cal notar que a pressió constant, $w = -P \Delta V$, i $\Delta (PV)=P\Delta V$. Per tant,
    \begin{equation}
    \Delta H = q_p
\end{equation}

Novament, per a un procés exotèrmic a pressió constant, la variació d'entalpia serà negativa, ja que el sistema allibera calor. Això és el que succeeix en una reacció de combustió.

Cal notar que $\Delta H$ i $\Delta U$ són funcions d'estat, però no són iguals, ja que $H$ inclou el treball de compressió. No obstant això, en processos en solució, el treball de compressió és negligible i $\Delta U \approx \Delta H$.

\subsection{Increment d'entalpia estàndard}

L'increment d'entalpia estàndard d'una reacció, $\Delta H^\circ$, és la variació d'entalpia que es produeix quan els reactius en els seus estats estàndard es converteixen en productes en els seus estats estàndard. Els estats estàndard es defineixen a una pressió d'1 bar i una temperatura específica, generalment 298.15 K (25°C). Aquesta magnitud és molt útil per calcular la calor alliberada o absorbida en una reacció química, ja que permet comparar diferents reaccions en condicions similars. Per exemple, l'increment d'entalpia estàndard de la combustió del metà és:
\begin{equation}
\ch{CH4(g) + 2 O2(g) -> CO2(g) + 2 H2O(l)} \quad \Delta H^\circ = -890.3 \, \si{\kilo\joule\per\mole}
\end{equation} 

Aquesta equació indica que la combustió d'un mol de metà allibera 890.3 kJ d'energia en forma de calor. Els valors d'increment d'entalpia estàndard per a moltes reaccions es poden trobar en taules termodinàmiques\cite{lide_crc_2005} (algunes estàn recollides en la \href{https://biocomputing-teaching.github.io/WebQuimicaAutomocio/pdf/TaulaUnitats.pdf}{taula de paràmetres termodinàmics} del curs).

\subsection{Llei de Hess}

La llei de Hess estableix que el canvi d'entalpia d'una reacció química és independent del camí seguit per arribar als productes finals, depenent només dels estats inicial i final. Això permet calcular l'entalpia de reaccions complexes a partir de reaccions més senzilles. Per exemple, considerem la combustió del propà (\ch{C3H8}):
\begin{equation}
\ch{C3H8(g) + 5 O2(g) -> 3 CO2(g) + 4 H2O(l)}
\end{equation}

Podem descompondre aquesta reacció en passos més simples basats en les entalpies de formació (veure \href{https://biocomputing-teaching.github.io/WebQuimicaAutomocio/pdf/TaulaUnitats.pdf}{taula de paràmetres termodinàmics}):
\begin{align}
\ch{C3H8(g) -> 3 C(s) + 4 H2(g)} & \quad \Delta H_1^\circ = 104.7 \, \si{\kilo\joule\per\mole} \\
\ch{C(s) + O2(g) -> CO2(g)} & \quad \Delta H_2^\circ = -393.5 \, \si{\kilo\joule\per\mole} \\
\ch{H2(g) + 1/2 O2(g) -> H2O(l)} & \quad \Delta H_3^\circ = -285.8 \, \si{\kilo\joule\per\mole}
\end{align}

L'entalpia total de la reacció de combustió es pot calcular, aleshores, sumant les entalpies dels passos individuals:
\begin{equation}
\Delta H^\circ = \Delta H_1^\circ + 3 \Delta H_2^\circ + 4 \Delta H_3^\circ
\end{equation}

Substituint els valors:
\begin{equation*}
\Delta H^\circ = 104.7 + 3(-393.5) + 4(-285.8) = 104.7 - 1180.5 - 1143.2 = -2219 \, \si{\kilo\joule\per\mole}
\end{equation*}

Així, la llei de Hess ens permet determinar l'entalpia de reaccions complexes utilitzant dades d'entalpia de reaccions més simples.

L'{\bf entalpia estàndard de formació}, $\Delta H_f^\circ$ és la variació d'entalpia que es produeix quan un mol d'una substància es forma a partir dels seus elements en els seus estats estàndard. L'entalpia d'una reacció es pot calcular a partir de les entalpies de formació estàndard dels reactius i productes utilitzant la següent fórmula:

\begin{equation}
\Delta H^\circ_{\text{reacció}} = \sum \Delta H^\circ_f (\text{productes}) - \sum \Delta H^\circ_f (\text{reactius})
\end{equation}

La Figura \ref{Fig:combustioGlucosa}, per exemple, mostra el cicle termodinàmic que ens permet calcular l'entalpia de combustió de la glucosa a partir d'entalpies de formació tabulades.

\begin{figure}[htbp]
    \centering
        \includegraphics[width=0.9\textwidth]{Hess1.png}
    \caption{Cicle termodinàmic de la combustió de la glucosa.
    La fletxa verda etiquetada \(\Delta H^\circ_{\text{comb}}\) representa la reacció de combustió. Alternativament, podríem primer convertir els reactius en els elements mitjançant la inversió de les equacions que defineixen les seves entalpies estàndard de formació (fletxa ascendent, etiquetada com \(\Delta H^\circ_1\) i \(\Delta H^\circ_2\)).      
    A continuació, podríem convertir els elements en els productes mitjançant les equacions que defineixen les seves entalpies estàndard de formació (fletxes descendents, etiquetades com \(\Delta H^\circ_3\) i \(\Delta H^\circ_4\)).  
        Com que l'entalpia és una funció d'estat, es compleix que:  
    $\Delta H^\circ_{\text{comb}} = \Delta H^\circ_1 + \Delta H^\circ_2 + \Delta H^\circ_3 + \Delta H^\circ_4$. Adaptat de \cite{noauthor_78_2015}}
    \label{Fig:combustioGlucosa}
\end{figure} 

\subsection{Capacitat calorífica}   

Com hem vist més amunt, la capacitat calorífica es pot expressar com:
\begin{equation}
    C = \frac{q}{\Delta T}
\end{equation}

La capacitat calorífica a pressió constant es denota com $C_p$ i a volum constant com $C_v$. La diferència entre ambdues és el treball de compressió, i es pot expressar, en el cas dels gasos, com:
\begin{equation}
    C_p - C_v = R
\end{equation}

En el cas de líquids i sòlids, les dues capacitats calorífiques són pràcticament iguals, ja que el treball de compressió és negligible. En el cas dels gasos, la capacitat calorífica a pressió constant és lleugerament més gran que a volum constant, ja que el treball de compressió és positiu. 

\begin{table}[h!]
    \caption{Capacitats calorífiques (\si{\cal\per\mole\per\kelvin}) de diverses substàncies a 298 K i a pressió constant\cite{mahan_quimica_1997}.}
    \centering
    \renewcommand{\arraystretch}{1.5}
    \begin{tabular}{ccc|ccc}
        \toprule
        Substància & Fórmula & $C_p$  & Substància & Fórmula & $C_p$  \\
        \midrule
        Monòxid de carboni & \ch{CO} & 6.97 & Metà & \ch{CH4} & 8.53 \\
        Oxigen & \ch{O2} & 7.05 & Nitrogen & \ch{N2} & 6.97 \\
        Diòxid de carboni & \ch{CO2} & 8.96 & Hidrogen & \ch{H2} & 6.88 \\
        Aigua (vapor) & \ch{H2O(g)} & 8.02 & Etanol & \ch{C2H5OH} & 26.9 \\
        Propà & \ch{C3H8} & 17.6 & Butà & \ch{C4H10} & 23.5 \\
        \bottomrule
    \end{tabular}

    \label{tab:capacitats_calorifiques}
\end{table} 

\begin{mybox}[title=Relació entre la capacitat calorífica a pressió constant i a volum constant] 
Per deduir aquesta relació, considerem la primera llei de la termodinàmica:
\[
    \Delta U = q - P \Delta V
\]

A volum constant, el treball de compressió és zero ($\Delta V = 0$), i per tant. 
\[
    \Delta U = q_v = C_v \Delta T
\]

A pressió constant, la calor afegida al sistema es descompon en l'increment d'energia interna i el treball de compressió. 
\[
    q_p = \Delta U + P \Delta V = C_p \Delta T
\]

Utilitzant l'equació d'estat dels gasos ideals, $P \Delta V = nR \Delta T$, i per a $n = 1$ podem escriure:
\[
    C_p \Delta T = C_v \Delta T + R \Delta T
\]
 i, per tant:
\[
    C_p - C_v = R
\]

\end{mybox}

La capacitat calorífica es pot expressar en forma diferencial com:

\begin{equation}
    C_v = \left( \frac{\partial U}{\partial T} \right)_V = \left( \frac{\partial q_v}{\partial T} \right)_V
\end{equation}

\begin{equation}
    C_p = \left( \frac{\partial H}{\partial T} \right)_P = \left( \frac{\partial q_p}{\partial T} \right)_P
\end{equation}

\subsection{Dependència de l'entalpia amb la temperatura}

L'entalpia d'una substància depèn de la temperatura. Imaginem que volem calcular l'entalpia d'una reacció a una temperatura $T_2$ a partir de l'entalpia a una temperatura $T_1$. Com que l'entalpia és una funció d'estat, podem calcular la variació d'entalpia entre $T_1$ i $T_2$ seguint aquest camí:

\begin{center}
    \schemestart
      \ch{a A + b B}
      \arrow{->[\state{H}^{}_1$(T_2)$]}[,1.5]
      \ch{c C + d D}
      \arrow{<-[\state{H}^{prod}$(T_2\to T_1)$]}[-90,2]
      \ch{c C + d D}
      \arrow{<-[\state{H}^{}_2$(T_1)$]}[180,1.5]
      \ch{a A + b B}  
      \arrow(@c4--@c1){->[\state{H}^{react}$(T_1\to T_2)$]}
    \schemestop
    \end{center}

Podem obtenir la variació d'entalpia com (recordem que la variació d'entalpia és precisament la variació de calor a pressió constant):

\begin{equation}
    \Delta H_2 = \Delta H_1 + \int_{T_1}^{T_2} C_p(\text{productes})\, dT - \int_{T_1}^{T_2} C_p(\text{reactius})\, dT.
\end{equation}

Si definim la diferència de calor específica:
\begin{equation}
    \Delta C_p = C_p(\text{productes}) - C_p(\text{reactius}),
\end{equation}
les integrals es poden combinar en:
\begin{equation}
    \Delta H_2 = \Delta H_1 + \int_{T_1}^{T_2} \Delta C_p dT.
\end{equation}

En el cas que $\Delta C_p$ sigui constant, la integral es resol com:
\begin{equation}
    \Delta H_2 = \Delta H_1 + \Delta C_p (T_2 - T_1).
    \label{eq:entalpia_temperatura}
\end{equation}

\begin{EXMP}[Càlcul de l'entalpia a una altra temperatura]
    Per exemple, donada la reacció:
    \begin{center}
    \ch{CO + $\frac{1}{2}$ O2 -> CO2}
    \end{center}
amb $\Delta H_{298} = \qty{-67.640}{\cal}$, calculem $\Delta H^\circ$ a \qty{398}{\kelvin}. De la Taula \ref{tab:capacitats_calorifiques}, tenim:   
\begin{align*}
    C_p(\ch{CO}) &= \qty{6.97}{\cal\per\mole\per\kelvin},\\
    C_p(\ch{O2}) &= \qty{7.05}{\cal\per\mole\per\kelvin},\\
    C_p(\ch{CO2}) &= \qty{8.96}{\cal\per\mole\per\kelvin}
\end{align*}
Substituïm a la fórmula \ref{eq:entalpia_temperatura}:
\begin{align*}
    \Delta C_p &= 8.96 - 6.97 - \frac{7.05}{2} = \qty{-1.53}{\cal\per\mole\per\kelvin},\\
    \Delta H_{398} &= \Delta H_{298} - \Delta C_p (\qty{398}{\kelvin} - \qty{298}{\kelvin})\\
    &= \qty{-67.640}{\cal} - (\qty{-1.53}{\cal\per\mole\per\kelvin} \times \qty{100}{\kelvin})\\
    &= \qty{-67.793}{\cal} = \qty{-283.6}{\kilo\joule\per\mole}.    
\end{align*}
\end{EXMP}

\section{Turbocompressors}

En qualsevol reacció química, la combustió implica un reactiu limitant. La cambra de combustió actua com un reactor on el combustible es barreja amb l'oxidant (\ch{O2}) i s'encén mitjançant una descàrrega elèctrica que supera la barrera d'energia d'activació. L'entrada de combustible és altament controlable, però el reactiu limitant sol ser l'\ch{O2}. Millorar l'aportació d'\ch{O2} augmenta la potència i l'eficiència del motor.

Un motor de quatre temps amb aspiració natural depèn del buit generat pel moviment descendent del pistó per captar \ch{O2}. Idealment, un cicle complet del pistó absorbeix un volum de gas igual a la cilindrada del motor. No obstant això, les pèrdues per fricció redueixen l'entrada real, definint l'\href{https://x-engineer.org/calculate-volumetric-efficiency/}{eficiència volumètrica}, $\eta_V$ (aire real que entra al cilindre entre volum d'aquest cilindre), que sempre és inferior a 1. Això, combinat amb el \qty{21}{\percent} de \ch{O2} en l'aire, limita l'eficiència de la combustió.

 

\begin{mybox}[title=Turbocompressors front compressors volumètrics]
    Els turbocompressors i els compressors volumètrics, actuant com a compressors centrífugs, forcen aire addicional al motor, superant les limitacions d'eficiència volumètrica i augmentant l'energia alliberada per cicle. Un compressor centrífug, mitjançant impe\lgem idors, imparteix energia cinètica a l'aire d'entrada. Les pales rotatives de l'impe\lgem idor empenyen el gas cap a l'exterior, augmentant l'energia cinètica i provocant un flux en espiral. L'augment de velocitat segueix l'equació de la dinàmica de fluids d'Euler:

\begin{equation}
    W_s = u_{\text{out}} C_{\theta,\text{out}} - u_{\text{in}} C_{\theta,\text{in}}
\end{equation}

on $W_s$ és la potència d'entrada de l'eix, $u$ representa la velocitat de la punta de les pales, i $C_{\theta}$ és la velocitat tangencial del gas a l'entrada i sortida de l'impe\lgem idor.
Això és essencialment una aplicació inversa del principi de Bernoulli, que estableix que un augment en la velocitat d'un gas es produeix simultàniament amb una disminució de la seva pressió. Tant els compressors volumètrics com els turbocompressors operen sota aquests principis, però es diferencien en la força motriu que fa girar les pales de l'impel·lidor. Els compressors volumètrics són accionats pel sistema de politges del motor (prenent part de la potència per generar-ne més), mentre que els turbocompressors són accionats per la calor i el flux dels gasos d'escapament (recuperant energia que, en cas contrari, es perdria). Per tant, els turbocompressors es poden considerar "sistemes de recuperació d'energia" i proporcionen millores significatives en l'eficiència del motor.
\end{mybox}

Analitzem el procés de combustió en un motor convencional i en un motor turboalimentat per entendre millor el paper dels sistemes d'admissió forçada. Si assumim que tota l'energia del motor prové de la combustió de l'octà pur, la reacció termoquímica equilibrada a \qty{298}{\kelvin} és:

    \ch{C8H18 + $\frac{25}{2}$ O2 -> 8 CO2 + 9 H2O} \quad $\Delta_cH^0$ = \qty{-5470}{\kilo\joule\per\mole}

Considerem un motor bòxer de \qty{2.5}{\liter} d'un Subaru Outback del 2005 amb una eficiència volumètrica del \qty{80}{\percent}, un valor típic en motors atmosfèrics. Si la bomba de combustible proporciona prou octà segons la relació estequiomètrica, per determinar l'energia generada en un cicle del motor, cal calcular la quantitat d'\ch{O2} disponible en aquest cicle:

\begin{equation}
    n_{\ch{O2}} = \frac{P_{\ch{O2}} V\eta_V}{RT} 
\end{equation}

Si considerem que tot passa a 17 graus de temperatura externa i pressió atmosfèrica:

\begin{equation}
    n_{\ch{O2}} = \frac{0.21 \times \qty{2.5}{\liter} \times 0.80}{0.0821 \times \qty{290}{\kelvin}} = \qty{0.017}{\mole}
\end{equation}

L'energia generada en un cicle serà:

\begin{equation}
    \Delta_cH^0 = \qty{-5470}{\kilo\joule\per\mole} \times \frac{1}{\frac{25}{2}} \times \qty{0.017}{\mole} = \qty{-7.4}{\kilo\joule\per\cycle}
\end{equation}

ASi considerem el Subaru Outback 2.5 XT del 2005, amb el mateix motor equipat amb un turbocompressor que genera \(\qty{13.5}{\psi}\) de sobrealimentació, això augmenta la pressió d'admissió de \(\qty{1}{\atm}\) (\(\qty{14.7}{\psi}\)) a \(\qty{1.92}{\atm}\) (\(\qty{28.2}{\psi}\)). Assumint les mateixes condicions i simplificacions, calculem l'energia generada:

\begin{equation}
    n_{\ch{O2}} = \frac{0.21 \times \qty{2.5}{\liter} \times 0.80 \times 1.92}{0.0821 \times \qty{290}{\kelvin}} = \qty{0.034}{\mole}
\end{equation}

\begin{equation}
    \Delta_cH^0 = \qty{-5470}{\kilo\joule\per\mole} \times \frac{1}{\frac{25}{2}} \times \qty{0.034}{\mole} = \qty{-14}{\kilo\joule\per\cycle}
\end{equation}

Això representa un augment d'aproximadament el \(\qty{92}{\percent}\) en l'energia per cicle, coherent amb l'increment de pressió. No obstant això, l'augment real de potència no es duplica. Segons Subaru, el motor atmosfèric produeix \(\qty{168}{\hp}\) a \(\qty{5500}{\rpm}\), mentre que el turboalimentat genera \(\qty{250}{\hp}\) a \(\qty{6000}{\rpm}\). Convertim aquestes potències en energia per cicle:

\begin{equation}
    \qty{168}{\hp} \times \qty{0.00134}{\kilo\joule\per\hp\per\second} \times \frac{1}{\qty{5500}{\rpm}} \times \qty{60}{\second\per\minute} \times 2 = \qty{2.7}{\kilo\joule\per\cycle}
\end{equation}

\begin{equation}
    \qty{250}{\hp} \times \qty{0.00134}{\kilo\joule\per\hp\per\second} \times \frac{1}{\qty{6000}{\rpm}} \times \qty{60}{\second\per\minute} \times 2 = \qty{3.7}{\kilo\joule\per\cycle}
\end{equation}

Aquestes xifres són menors que les prediccions ideals, ja que el model assumeix una conversió del \(\qty{100}{\percent}\) de calor en treball. L'eficiència real d'un motor de combustió típic és inferior. També cal considerar l'augment de temperatura de l'aire d'admissió causat pel turbocompressor, que redueix la densitat de l'aire i el nombre de mols d'\(\ch{O2}\) disponibles.


Per mitigar aquest problema, s'usa un intercanviador de calor, com un intercooler aire-aire, que refreda l'aire d'admissió amb l'aire extern, millorant la densitat del gas i l'eficiència del motor. El Subaru de \qty{250}{\hp} inclou un intercooler amb una presa d'aire funcional, cosa que implica que el guany real degut al turbocompressor és inferior al \qty{37}{\percent} anunciat per Subaru.

\begin{mybox}[title=Sistemes d'injecció metanol/aigua]
    Per reduir les temperatures del gas d'entrada en sistemes d'inducció forçada, es pot utilitzar el refredament evaporatiu d'un fluid injectat directament al corrent de gas. Injectar metanol o una barreja d'aigua/metanol abans o després del cos de l'accelerador, o directament a la cambra de combustió, ajuda a refredar els gasos d'entrada. L'entalpia de vaporizació de l'aigua és de \qty{40.68}{\kilo\joule\per\mole} i la del metanol \qty{35.3}{\kilo\joule\per\mole}, eliminant així calor durant la vaporizació. 

Aquests compostos tenen punts d'ebullició moderats, permetent-los ser líquids a temperatures ambientals i vaporitzar-se fàcilment en l'aire d'entrada dels vehicles d'inducció forçada (metanol: \qty{148}{\degreeFahrenheit}, aigua: \qty{212}{\degreeFahrenheit}). Les velocitats d'evaporació són altes gràcies a sistemes d'injecció que generen gotes petites, augmentant la superfície de contacte, com en els injectors de combustible.

Els gasos d'entrada més freds augmenten la densitat d'oxigen a les cambres de combustió, permetent cremar més combustible i augmentant la potència del motor. Aquest refredament evaporatiu també pot prevenir la predetonació en motors d'inducció forçada o d'alta compressió, permetent optimitzar el temps d'encesa o utilitzar combustibles de menor octanatge. Refredar els gasos de combustió redueix també la temperatura dels gasos d'escapament, minimitzant la producció de NOx tèrmics (els NOx generats quan el nitrogen atmosfèric contacta amb superfícies molt calentes com col·lectors d'escapament i convertidors catalítics). 

Aquests sistemes d'injecció poden utilitzar-se amb o sense interrefredadors, i existeixen sistemes comercials disponibles per a automòbils\cite{bowers_understanding_2014}.
\end{mybox}

A més, els turbocompressors (i en menor mesura els compressors mecànics) poden millorar l'eficiència del combustible en els vehicles. Per exemple, si es considera que \qty{150}{hp} és una potència adequada per a un automòbil i una empresa fabrica un motor de \qty{2.0}{L} sense aspiració forçada amb aquesta potència, també es podria aconseguir la mateixa potència amb un motor més petit, per exemple de \qty{1.5}{L}, i un turbocompressor. Tots dos motors consumirien la mateixa quantitat de combustible durant l'acceleració, però a velocitat de creuer constant, el motor de \qty{1.5}{L} hauria de consumir menys combustible que la versió de \qty{2.0}{L}. Tot i això, el règim de revolucions per minut en una velocitat de creuer específica depèn de la transmissió i altres factors, fent que el motor turbo de \qty{1.5}{L} probablement funcioni a un règim més alt que el motor de \qty{2.0}{L}, reduint el guany d'eficiència previst.

\begin{mybox}[title=Injecció d'òxid nitrós]
    Un altre mètode per augmentar l'eficiència del motor és la injecció de òxid nitrós (N\textsubscript{2}O), que actua com un oxidant alternatiu a la cambra de combustió. Quan una mol de \ch{N2O} es descompon en un cilindre calent del motor, es genera una mol de \ch{N2} i mig mol d'\ch{O2}, proporcionant una atmosfera amb un \qty{33}{\percent} d'oxigen, molt superior al \qty{21}{\percent} de l'aire atmosfèric. A més d'aquest enriquiment en oxigen, l'evaporació del \ch{N2O} líquid en el sistema d'admissió refreda substancialment els gasos d'entrada, augmentant-ne la densitat i millorant l'eficiència volumètrica del motor.

Els principals inconvenients d'aquest sistema són que només funciona mentre hi hagi \ch{N2O} en els dipòsits a bord (a diferència dels turbocompressors i compressors mecànics, que operen contínuament), la pressió del dipòsit requereix un control acurat i les peces del motor pateixen tensions més altes\cite{bowers_understanding_2014}.
\end{mybox}

\section{Biofuels}

Podem també utilitzar les entalpies de formació i combustió per entendre per què els cotxes tenen un menor rendiment de combustible amb combustibles que inclouen etanol. Com que la gasolina és una mescla complexa de compostos orgànics (com s'ha esmentat anteriorment), suposarem que la gasolina és predominantment octà per a aquest exercici (basat en \cite{bowers_understanding_2014}). Usant entalpies de formació, és fàcil veure que la combustió d'1 mol d'octà genera \qty{5470}{\kilo\joule} d'energia. Utilitzant la densitat típica de la gasolina (\qty{0.74}{\kilo\gram\per\liter}) i un simple factor de conversió, podem determinar la densitat energètica de la gasolina pura d'octà:
 
\begin{equation*}
\qty{5470}{\kilo\joule\per\mole} \times \frac{1 \text{ mol } \ch{C8H18}}{\qty{114.224}{\gram}} \times \frac{\qty{1000}{\gram}}{\qty{1}{\kilo\gram}} \times \frac{\qty{0.740}{\kilo\gram}}{\qty{1}{\liter}} = \qty{35400}{\kilo\joule\per\liter}
\end{equation*}

Ara, podem utilitzar els conceptes de la llei de Hess i les equacions termoquímiques per determinar l'energia alliberada en cremar 1 mol d'etanol i després usar la densitat de massa i la massa molecular per determinar la seva densitat energètica.

Les entalpies estàndard de formació per al \ch{CO2} (g), l'etanol (l) i l'aigua líquida són, respectivament:
\begin{eqnarray*}
\Delta_f H^0 (\ch{CO2}) &=& -393.5\ \si{\kilo\joule\per\mole}\\
\Delta_f H^0 (\ch{C2H5OH}) &=& -277.7\ \si{\kilo\joule\per\mole}\\
\Delta_f H^0 (\ch{H2O}) &=& -285.5\ \si{\kilo\joule\per\mole}
\end{eqnarray*}

Calcularem l'energia calorífica obtinguda en la combustió d'1 \si{\litre} d'alcohol etílic amb densitat \qty{790}{\kilo\gram\per\meter\cubed} en condicions estàndard.

\begin{equation*}
\qty{1367}{\kilo\joule\per\mole} \times \frac{1 \text{ mol } \ch{C2H5OH}}{\qty{46.068}{\gram}} \times \frac{\qty{1000}{\gram}}{\qty{1}{\kilo\gram}} \times \frac{\qty{0.790}{\kilo\gram}}{\qty{1}{\liter}} = \qty{23400}{\kilo\joule\per\liter}
\end{equation*}

Es pot notar que hi ha substancialment menys energia per litre d'etanol que per litre d'octà. Ara suposem que el teu trajecte diari per l'autopista requereix \qty{200000}{\kilo\joule} d'energia. Aquesta energia s'utilitza per accelerar el vehicle fins a la velocitat de creuer i inclou les pèrdues d'energia per ineficiències del motor i la transmissió, forces de fregament per mantenir la velocitat i l'energia convertida per l'alternador.
Si el teu cotxe utilitzés gasolina composta per un 100\% d'octà, el càlcul següent mostra que necessites cremar \qty{5.65}{\liter} de gasolina per fer aquest trajecte:

\begin{equation*}
\frac{\qty{200000}{\kilo\joule}}{\qty{35400}{\kilo\joule\per\liter}} = \qty{5.65}{\liter}
\end{equation*}

Ara calculem el volum de combustible necessari per al mateix trajecte si el dipòsit conté un 90\% en volum d'octà i un 10\% d'etanol. L'energia per litre del combustible barrejat és la mitjana ponderada de les densitats energètiques de l'etanol i l'octà:

\begin{equation*}
\frac{\qty{200000}{\kilo\joule}}{(0.9 \times \qty{35400}{\kilo\joule} + 0.1 \times \qty{23400}{\kilo\joule})/\text{litre}} = \qty{5.83}{\liter}
\end{equation*}

La situació és encara pitjor per a un vehicle E-85 que crema un 85\% d'etanol, on càlculs similars mostren que cal cremar \qty{7.95}{\liter} per cobrir la mateixa distància.

\subsection{L'etanol i el nostre futur energètic}

Malgrat aquests resultats termoquímics, moltes benzineres venen combustibles enriquits amb etanol i existeixen subvencions per a les plantes de producció d'etanol. Això indica que incloure etanol en els combustibles té avantatges. Un és que l'etanol crema a una temperatura més baixa que la gasolina o el dièsel, cosa que redueix la producció de sutge i gasos NOx (NO i \ch{NO2}), que es formen principalment per reaccions entre el \ch{N2} i l'oxigen a temperatures molt altes.

L'etanol té un índex d'octà de 113 i, per tant, afegir etanol permet ajustar químicament aquest índex. A més, la disminució de l'eficiència del combustible no és tan severa en la pràctica, ja que l'etanol crema més eficientment en un motor perquè és un combustible oxigenat. Els combustibles oxigenats contenen oxigen en la seva estructura, ajudant a assegurar una combustió completa.

Finalment, diluir la gasolina amb etanol produït localment redueix la dependència de les importacions de combustible i millora la seguretat energètica. A més, l'etanol derivat de cultius és un combustible renovable que absorbeix gairebé tot el \ch{CO2} que allibera durant la combustió en el següent cicle de creixement, fent-lo un combustible de \ch{CO2} net zero. Tot i així, el balanç entre seguretat energètica i seguretat alimentària és subtil i cal tenir-lo present!

Així, per determinar si l'etanol (\ch{C2H5OH}) proporciona realment algun avantatge energètic substancial o una reducció en la producció de gasos d'efecte hivernacle, cal una anàlisi completa del cicle de vida. Les anàlisis del cicle de vida són processos complexos que requereixen tenir en compte totes les entrades i sortides d'energia, així com els efectes econòmics d'un combustible des de la seva producció fins a la seva eliminació. Sovint impliquen moltes hipòtesis fonamentades que poden influir en els resultats.  

La majoria dels models mostren que la matèria primera de l'etanol (blat de moro, canya de sucre, etc.) és un factor important per determinar si l'etanol derivat de cultius suposa un avantatge energètic. La selecció de la matèria primera també té un impacte addicional en el subministrament global d'aliments. La modelització de l'economia energètica està més enllà de l'abast d'aquest curs, però és evident que l'ús d'etanol en combustibles és un tema controvertit que posa de manifest la interacció entre la ciència, la societat, els valors, la política i l'economia (llegiu, pere exemple, \cite{baird_environmental_2012}).  

\subsection{Biodièsel: Convertint residus en energia}

Els motors dièsel són més eficients que els de gasolina i poden funcionar amb olis vegetals purs o modificats químicament. Una de les barreres principals és la viscositat: molts olis vegetals tenen punts de gelificació alts, impedint el seu ús en climes freds sense preescalfadors.  

El combustible dièsel està compost per hidrocarburs de cadena llarga (10-20 àtoms de carboni), similars als àcids grassos presents en els olis vegetals. Aquesta similitud permet que els motors dièsel funcionin amb olis vegetals purs, tot i que hi ha limitacions.  
\begin{itemize}
    \item \textbf{Viscositat i fluïdesa:} A temperatures baixes, l'oli vegetal pot gelificar, impedint el seu flux cap al motor.
    \item \textbf{Impureses:} Cal filtrar l'oli per eliminar sediments que podrien obstruir els injectors i generar dipòsits de carboni.
    \item \textbf{Reactivitat química:} Els olis vegetals s'oxiden més fàcilment que els combustibles derivats del petroli, reduint la seva estabilitat i afectant la lubricació del motor.
\end{itemize}

Molts vehicles amb oli vegetal usen dos dipòsits: un amb oli vegetal i un altre amb dièsel per facilitar l'arrencada i evitar dipòsits de carboni.

Una altra possibilitat per utilitzar oli vegetal nou o usat és convertir-lo en biodièsel, un procés químic que produeix un combustible força similar al dièsel derivat del petroli en molts aspectes. La majoria dels olis vegetals contenen principalment polímers biològics amb cadenes de carboni relativament curtes (10–25 àtoms de carboni), anomenats àcids grassos, units a una estructura química curta anomenada glicerina o glicerol. Aquests triglicèrids són els reactius crítics en l'oli vegetal que es convertiran en combustible. Els olis vegetals sovint contenen una certa quantitat d'\textit{àcids grassos lliures}, que són cadenes individuals amb un grup funcional àcid carboxílic intacte.

Si s'utilitza oli vegetal usat per produir biodièsel, el primer pas és filtrar-lo i aplicar altres mètodes de pretractament per eliminar restes d'aliments i altres sòlids. Normalment, aquests pretractaments són físics més que químics i, per tant, no es discutiran aquí. Un cop l'oli està net, es pot prendre una petita mostra, escalfar-la i titrar-la amb una base forta com l'hidròxid de sodi per determinar-ne el contingut d'àcids grassos lliures\cite{noauthor_115_2023}.

El següent pas és dur a terme una reacció química anomenada \textit{transesterificació}, que separa les molècules d'àcids grassos dels triglicèrids, donant com a resultat èsters d'àcids grassos i glicerol\cite{noauthor_82_nodate}:

\begin{align*}
\ch{R1COOCH2-CH(OOCR2)-CH2OOCR3} + 3 \ch{ROH} \rightarrow & \ch{R1COOR} + \\&\ch{R2COOR} + \\&\ch{R3COOR} + \\&\ch{CH2OH-CHOH-CH2OH}
\end{align*}

\begin{center}
    \includegraphics[width=0.8\textwidth]{transesterificacio.jpg}
\end{center}

Aquest procés es realitza generalment afegint un alcohol a l'oli. El metanol (\ch{CH3OH}) és l'alcohol més utilitzat per la seva abundància i baix cost, encara que qualsevol alcohol de cadena curta pot dur a terme la reacció de transesterificació. Aquesta reacció és sovint lenta a causa de la resistència a la desprotonació de l'alcohol, i per això s'afegeix una base forta (com \ch{NaOH}), que té dues funcions: neutralitzar els àcids grassos lliures i desprotonar l'alcohol per formar un anió alcoxi reactiu.

La barreja d'oli vegetal, alcohol i base es remena i es deixa reaccionar. El resultat són dues capes orgàniques separades: una amb els èsters dels àcids grassos i una altra amb glicerina contaminada amb impureses. La capa d'èsters d'àcids grassos es retira i està pràcticament llesta per a l'ús. En algunes estratègies de producció de biodièsel, aquesta capa es renta amb aigua per eliminar impureses solubles en aigua, incloent-hi la base i l'alcohol residuals. L'alcohol també es pot recuperar desti\lgem ant la capa no rentada d'èsters d'àcids grassos. Un cop separades la capa aquosa i la dels èsters d'àcids grassos, aquests últims es sequen i s'emmagatzemen per a ser utilitzats en vehicles dièsel.

El biodièsel presenta diversos avantatges i inconvenients en comparació amb el dièsel derivat del petroli. Un avantatge important és la seva manca de sofre. El biodièsel és essencialment un combustible lliure de sofre, mentre que el dièsel de petroli conté diverses quantitats de sofre. L'absència de sofre en el biodièsel implica que no contribueix significativament a la formació de gasos \ch{SO_x}, responsables de la pluja àcida. A més, estudis han demostrat que el biodièsel pur genera nivells més baixos d'hidrocarburs policíclics aromàtics (PAH), que es consideren cancerígens, així com nivells baixos d'hidrocarburs incombustos, monòxid de carboni (\ch{CO}), diòxid de carboni (\ch{CO2}) i òxids de nitrogen (\ch{NO_x}).

Tanmateix, el biodièsel conté diferents tipus de molècules en comparació amb el dièsel derivat del petroli i, per tant, pot accelerar la degradació de materials orgànics com mànegues i juntes de goma. Alguns dels beneficis del biodièsel també es poden obtenir utilitzant mescles de biodièsel i dièsel de petroli. Hi ha diverses categories estàndard de mescla, i el B-20 (una mescla amb un 20\% de biodièsel) ha demostrat reduir les emissions contaminants en comparació amb el dièsel pur.

Com en el cas de l'etanol, determinar si el biodièsel és realment un benefici energètic requereix una anàlisi completa del cicle de vida. No obstant això, convertir un flux de residus com l'oli vegetal usat en combustible pot ser una avantatge energètica i és un gran projecte per fer-hi recerca.


\section{Tecnologia de Reducció Catalítica Selectiva (SCR)}

Un sistema SCR injecta un producte (Diesel Exhaust Fluid (DEF), conegut comercialment com AdBlue\copyright) en els gasos d'escapament calents, on es descompon en amoníac, que després reacciona a la superfície del catalitzador per produir nitrogen i vapor d'aigua.
L'amoníac (\ch{NH3}) s'utilitza com a agent reductor; no obstant això, a causa de la seva agressivitat i toxicitat, no s'aplica directament, sinó en forma d'una substància innòcua que allibera \ch{NH3} en el corrent de gasos d'escapament. Això permet que el motor funcioni en condicions òptimes, minimitzant així el consum de combustible i, per tant, les emissions de \ch{CO}, així com la descàrrega de tots els contaminants excepte \ch{NO_x}.

Després que els gasos de combustió hagin sortit del motor, primer passen a través d'un catalitzador d'oxidació previ, on els hidrocarburs, el monòxid de carboni i les partícules no cremades s'oxiden tan completament com sigui possible. El \ch{NO} s'oxida parcialment a \ch{NO2}, ja que la reducció posterior es produeix més ràpidament amb una proporció de mescla \ch{NO} : \ch{NO2} de 1:1. A continuació, una bomba, controlada per una unitat de monitoratge, injecta AdBlue des d'un tanc separat en el corrent de gasos d'escapament calents, on es produeix la hidròlisi a \ch{NH3} i \ch{CO2}. En la reducció catalítica selectiva pròpiament dita, l'\ch{NH3} reacciona amb la barreja \ch{NO}/\ch{NO2} per formar nitrogen i aigua (vapor), que constitueixen el 80~\% de la composició natural de l'atmosfera.

AdBlue és una solució aquosa que conté un \qty{32.5}{\percent} d'urea d'alta puresa (per pes) en aigua desionitzada. L'amoníac es genera en fase gasosa mitjançant les següents reaccions\cite{selleri_overview_2021}:

\begin{align}
    \ch{(NH2)2CO(aq)} &\rightarrow \ch{(NH2)2CO(s\ or\ l)} + 6.9 \ch{H2O(g)} && \text{(evaporació de l'aigua)} \\
    \ch{(NH2)2CO(s\ o\ l)} &\rightarrow \ch{NH3(g)} + \ch{HNCO(g)} && \text{(termòlisi de la urea)} \\
    \ch{HNCO(g)} + \ch{H2O(g)} &\rightarrow \ch{NH3(g)} + \ch{CO2(g)} && \text{(hidròlisi de l'àcid isociànic)}
\end{align}

En conjunt, el procés SCR es basa en les següents reaccions globals, que tenen lloc en fase gasosa o adsorbida:

\begin{align}
    4 \ch{NH3} + 4 \ch{NO} + \ch{O2} &\rightarrow 4 \ch{N2} + 6 \ch{H2O} && \text{(SCR estàndard)} \\
    2 \ch{NH3} + \ch{NO} + \ch{NO2} &\rightarrow 2 \ch{N2} + 3 \ch{H2O} && \text{(SCR ràpid)} \\
    8 \ch{NH3} + 6 \ch{NO2} &\rightarrow 7 \ch{N2} + 12 \ch{H2O} && \text{(SCR amb NO2)}
\end{align}

Entre aquestes, la reacció més ràpida és la SCR ràpida, que ocorre quan hi ha una quantitat equimolar de \ch{NO} i \ch{NO2} en els gasos d'escapament. Per aquest motiu, s'ha introduït un catalitzador d'oxidació dièsel (DOC) per assegurar l'oxidació de \ch{NO} a \ch{NO2}. Els DOCs estan guanyant atenció en estudis recents per complir amb regulacions més estrictes, com es veu a la Figura \ref{Fig:ATS}.

\begin{figure}[h!]
    \centering
    \includegraphics[width=0.7\textwidth]{postexhaustion.png}
    \caption{Evolució de la configuració dels sistemes de postractament (ATS) per a motors Diesel (a dalt) i de benzina (a baix) en vehicles lleugers\cite{selleri_overview_2021}.}
    \label{Fig:ATS}
\end{figure}

A més de les reaccions esmentades, es poden produir reaccions no desitjades, com l'oxidació selectiva de l'amoníac a nitrogen o la seva conversió no selectiva a \ch{NOx}, reduint així l'eficiència del procés: 

\begin{align}
    4 \ch{NH3} + 3 \ch{O2} &\rightarrow 2 \ch{N2} + 6 \ch{H2O} && \text{(oxidació selectiva de l'amoníac)} \\
    4 \ch{NH3} + 5 \ch{O2} &\rightarrow 4 \ch{NO} + 6 \ch{H2O} && \text{(oxidació no selectiva a NO)} \\
    2 \ch{NH3} + 2 \ch{O2} &\rightarrow \ch{N2O} + 2 \ch{H2O} && \text{(oxidació no selectiva a \ch{N2O})}
\end{align}

Aquestes reaccions es tornen rellevants a temperatures altes (superiors a \qty{400}{\degC}) i en absència de \ch{NO2} en l'alimentació. Per contra, una alta presència de \ch{NO2} a baixes temperatures pot conduir a la formació no desitjada de nitrat d'amoni i \ch{N2O}:

\begin{align}
    2 \ch{NH3} + 2 \ch{NO2} &\rightarrow \ch{NH4NO3} + \ch{N2} + \ch{H2O} && \text{(formació de nitrat d'amoni)} \\
    2 \ch{NH3} + 2 \ch{NO2} &\rightarrow \ch{N2O} + \ch{N2} + 3 \ch{H2O} && \text{(formació d'òxid nitrós)}
\end{align}

La formació de nitrat d'amoni és crítica per sota de \qty{180}{\degC}, limitant l'eficàcia del procés SCR a temperatures baixes. No obstant això, aquest compost pot ser reduït per \ch{NO} a \qty{200}{\degC}, generant \ch{NO2} i possibilitant la reacció SCR ràpida, especialment en sistemes catalítics com els Fe-zeolites i \ch{V2O5}-\ch{WO3}/\ch{TiO2}.

Finalment, la presència de sofre en el combustible, combinada amb la funció oxidativa del DOC, pot conduir a la formació de \ch{SO3} i, posteriorment, de sulfats i àcid sulfúric:

\begin{align}
    \ch{SO3} + \ch{H2O} &\rightarrow \ch{H2SO4} && \text{(formació d'àcid sulfúric)} \\
    \ch{NH3} + \ch{SO3} + \ch{H2O} &\rightarrow \ch{(NH4)HSO4} && \text{(formació de bisulfat d'amoni)} \\
    2 \ch{NH3} + \ch{SO3} + \ch{H2O} &\rightarrow \ch{(NH4)2SO4} && \text{(formació de sulfat d'amoni)}
\end{align}

Aquests compostos poden acumular-se al catalitzador, bloquejant-ne els porus i provocant desactivació, tot i que aquest efecte és reversible amb un augment de temperatura. L'àcid sulfúric també pot provocar corrosió en els components del sistema de posttractament i la formació d'aerosols nocius.


Per saber-ne més, consulteu \cite{selleri_overview_2021}.
% \input{GEAREDOX}
% \input{GEAIntermolecularforces}
% \input{GEAThermodynamics}
% \input{GEAMaterials}
% \input{GEALight}


% antiga classificació estructurada de forma clàssica per a un curs de química
% \chapter{Estructura Atòmica de la Matèria}
La química és una ciència fonamentalment experimental\footnote{Treballarem més endavant alguns conceptes matemàtics de la mecànica quàntica aplicada a la química} que estudia la natura de les substàncies de l'Univers, així com els processos en què participen per formar-ne de noves.

Aquest capítol està basat en diverses referències.\cite{Caamano1984,Mahan1977,Yen2008}

\section{Classificació de la matèria}

\begin{itemize}
\item Una \emph{mescla heterogènia} o, senzillament, mescla, és la matèria a la què, a simple vista o mitjançant un microscopi ordinari, s'hi poden distingir diferents \emph{components}. Exemples de mescles heterogènies són el granit (quars, feldespat i mica). En una mescla els components mantenen les seves propietats característiques. Les propietats de la mescla són combinació de les dels components. Són heterogènies a la subdivisió. Els components es poden barrejar en qualsevol proporció. Una anàlisi profunda de les \emph{dispersions col·loidals} les mostra com a mescla (Figura \ref{fig:Colloid} i Taula \ref{tab:colloid}). \footnote{Un dels principals problemes en química analítica és la preparació de les mostres. Hi ha moltes fonts d'informació sobre com solucionar el problema d'homogeneïtzar les mostres que s'han de dur a analitzar d'una mescla. Hi ha un munt de recursos accessibles a internet, però pots començar per \linkurl{https://www.epa.gov/sites/production/files/2015-05/documents/402-b-04-001b-12-final.pdf}.}
\begin{figure}[h]
\centering
\includegraphics[scale=0.8]{figures/Colloid.png}
\caption[Dissolucions, suspensions i col·loides]{(a) Una dissolució és una mescla homogènia, com l'aigua salada de l'aquari. (b) En un col·loide les partícules són més grans, però es mantenen disperses i no precipiten (llet). (c) Una suspensió és una mescla heterogènia amb particules que poden acabar precipitant. (adaptat de \cite{Colloids2016})}
\label{fig:Colloid}
\end{figure}

\begin{table}[h!]
  \begin{center}
    \caption{Diferents tipus de dispersions}
    \label{tab:colloid}
    \begin{tabular}{l|l|l|l}
      & \multicolumn{3}{c}{Dispersant}\\ 
      \hline
      Dispers & Sòlid & Líquid & Gas\\
      \hline
      Sòlid & Alguns al·liatges; gemmes acolorides & Gels o suspensions (pintures) & Aerosols (fum) \\
      Líquid & Geles (gelatina) & Emulsions & Aerosols (boira) \\
      Gas & Escuma aïllant & Escuma (cervesa) & \\
      \hline
    \end{tabular}
  \end{center}
\end{table}
\item La \emph{matèria homogènia} és aquella que és idèntica quant a composició i propietats sigui quina sigui la porció que n'agafem. L'aigua de mar, la sal o un lingot d'or en són exemples:
\begin{itemize}
\item L'aigua de mar és una \emph{dissolució}, en tant que formada per dos o més components. Les propietats d'una dissolució poden ser radicalment diferents de les dels seus components: aigua i sal no són conductors de l'electricitat, però la seva dissolució ho és. Les dissolucions són homogènies a la subdivisió, però heterogènies al canvi d'estat. Els components no es poden barrejar en qualsevol proporció (per exemple, la solubilitat depèn de la temperatura). Les propietats depenen de la concentració, com mostra la Figura \ref{fig:watsuc2} (temperatura d'ebullició o fusió de l'aigua amb sals o anticongelants).
\begin{figure}[h]
\centering
\includegraphics[scale=0.8]{figures/watsuc2.png}
\caption{La temperatura d'ebullició d'una dissolució aquosa es modifica en funció de la concentració de solut.}
\label{fig:watsuc2}
\end{figure}
\item La sal i l'or són \emph{substàncies pures}, en tant que formades per un sol component i amb propietats físiques i químiques característiques d'aquests (densitat, temperatura de fusió o d'ebullició, solbilitat en un dissolvent donat, índex de refracció, viscositat, etc).
\end{itemize}
\end{itemize}


La Figura \ref{fig:SeparacioMescles} mostra un resum esquemàtic dels processos per reduir la complexitat d'una determinada mescla.

\begin{figure}[h]
\centering
\includegraphics[scale=0.35]{figures/Mixtures.png}
\includegraphics[scale=0.50]{figures/SeparacioMescles.png}
\caption{Classificació de la matèria (adaptat de \cite{Caamano1984})}
\label{fig:SeparacioMescles}
\end{figure}

Algunes tècniques de separació:
\begin{itemize}
\item mètodes simples de separació de mescles: filtració, decantació, centrifugació, cristal·lització, destilació, etc;
\item mètodes per separar mescles de múltiples components: 
\begin{itemize}
\item cristal·lització fraccionada (dos sòlids amb solubilitats diferents);
\item destil·lació fraccionada (dos líquids de punts d'ebullició semblants amb columnes de fraccionament, veure Figura \ref{fig:Crude_Oil_Distillation_Unit});
\item cromatografia (d'adsorció, de repartiment, d'intercanvi iònic, de filtració sobre gels); sobre paper, de columna, sobre capa fina...; gas-sòlid, líquid-líquid, gas-líquid.
\end{itemize}  
\begin{figure}[h]
\centering
\includegraphics[scale=0.8]{figures/Crude_Oil_Distillation_Unit.png}
\caption{Esquema de la columna de destilació del petroli.}
\label{fig:Crude_Oil_Distillation_Unit}
\end{figure}
\end{itemize}

Per a caracteritzar la puresa d'una substància, analitzem les seves propietats característiques mitjançant tècniques molt diverses, entre elles l'anàlisi dels seus espectres:
\begin{itemize}
\item espectroscopia infraroja (IR), 
\item espectroscopia visible i ultraviolada (UV)
\item espectroscopia de resonància magnpetica nuclear (NMR)
\item difracció de raigs X
\item espectroscopia de masses
\item etc
\end{itemize}

\subsection{Elements i compostos}

Els \emph{compostos} es poden descomposar en substàncies més simples. En canvi, els \emph{elements} o substàncies simples són indivisibles. Addicionalment, un compost es pot sintetitzar a partir de substàncies més simples, cosa que no passa amb els elements (Figura \ref{fig:CompostosElements}). 

Robert Boyle va donar la primera definició d'Element l'any 1661. Lavoisier va donar una llista d'elements coneguts el 1789 però hi va incloure la llum i la calor. També va formular la llei de la conservació de la massa. El 19800, Alessandro Volta va inventar la pila elèctrica i va permetre separar nous elements (electròlits) per part de Humphry Davy el 1807: el sodi i el potassi. El mateix 1807, Dalton publica la seva teoria atòmica.

Es poden combinar els elements de qualsevol manera per formar compostos? Ho veurem a la secció següent.

\begin{figure}[h]
\centering
\includegraphics[scale=0.8]{figures/CompostosElements.png}
\caption{Relació entre compostos i elements.}
\label{fig:CompostosElements}
\end{figure}


\section{La naturalesa atòmica de la matèria}

Durant molts anys l'observació de fenòmens químics senzills va servir per determinar les lleis fonamentals de la química que la van distingir de l'alquímia.

\begin{mdframed}[backgroundcolor=gray!30,frametitle=Llei de les proporcions definides]
En un compost donat, els elements constituients es combinen sempre en les mateixes proporcions pondarebles, sigui quin sigui l'origen i el mode de preparació dels compostos.
\end{mdframed}

Veurem que a cada element se li pot assignar un pes determinat i això fa que la seva combinació dongui un pes determinat i una fórmula determinada als compostos.
Per exemple, si pensem en l'òxid nítric, \ch{NO} i mirem d'augmentar el mínim possible (un àtom d'oxigen o un de nitrogen) arribem a nous compostos molt diferents de l'inicial i entre ells, \ch{N2O} i \ch{NO2}. 

De fet, la llei és força simple i poc precisa, ja que els elements poden tenir diferents isòtops.
Els isòtops d'un element tenen diferents masses atòmiques (sempre tindran el mateix número de protons, però poden diferir en el número de neutrons). La massa atòmica promig es mesura ponderant les masses atòmiques de cada isòtop amb la seva abundància a la terra (veure Figura \ref{fig:PropietatsHidrogen}).
Els objectes extraterrestres poden tenir composicions isotòpiques molt diferents.
\begin{figure}[h]
\centering
\includegraphics[scale=0.35]{figures/PropietatsHidrogen.png}
\caption{Propietats físiques de l'àtom d'hidrogen (Adaptat de Connexions \linkurl{http://cnx.org/content/col10984/1.4})}
\label{fig:PropietatsHidrogen}
\end{figure}
\begin{exr}
Entra a \linkurl{https://teachchemistry.org/periodical/issues/may-2017/isotopes-calculating-average-atomic-mass} i calcula la massa atòmica promig d'alguns elements.
\end{exr}
També es viola aquesta llei en alguns sòlids iònics com l'òxid de zinc, el sulfur cuprós (entre \ch{Cu_{1.7}S} i \ch{Cu2S}) o l'òxid ferrós.

Els compostos sòlids que no estan formats per molècules definides presenten característiques diferents. Podem preparar cristalls de \ch{TiO}, però variant les condicions de preparació podem anar de \ch{Ti_{0.75}O} a \ch{TiO_{0.69}}, encara que la seva estructura espaial sigui idèntica. Es tracta de compostos no estequiomètrics i, tot i que no varïin les seves propietats estrcuturals, sí que ho fan les òptiques o la resistivitat (veure Figura \ref{fig:TiOPhaseDiagram}).
\begin{figure}[h]
\centering
\includegraphics[scale=0.35]{figures/TiOPhaseDiagram.png}
\caption{Diagrama de fase de l'òxid de titani en diverses composicions (no)estequiomètriques\cite{DeVries1954}}
\label{fig:TiOPhaseDiagram}
\end{figure}

\begin{mdframed}[backgroundcolor=gray!30,frametitle=Llei de les proporcions múltiples]
Si dos elements formen més d'un compost, els diferents pesos d'un d'ells que es combinen amb el mateix pes de l'altre, estan en una relació de números enters petits.
\end{mdframed}

Els diferents òxids del nitrogen en són un bon exemple: amb 16g d'oxigen es poden combinar 28, 14 o 7 grams de nitrogen (proporció 4:2:1)

\begin{mdframed}[backgroundcolor=gray!30,frametitle=Llei de les proporcions equivalents]
Si una determinada quantitat de l'element C reacciona amb una quantitat donada de l'element A, p$_A$, i una donada de l'element B, p$_B$, A i B reaccionen en una proporció que és múltiple simple o fracció de números sencers de la raó p$_A$/p$_B$.
\end{mdframed}

El nitrogen i l'oxigen reaccionen amb l'hidrogen per formar amoniac (\ch{NH3}) i aigua (\ch{H2O}), respectivament. 1g d'hidrogen reacciona amb 4,66g de Nitrògen per formar amoniac i amb 8 grams d'oxigen per formar aigua. Per tant, 4.66/8.00=0.583.
En el cas de la reacció \ch{N2 + O2 -> 2NO}, 28g de nitrogen reaccionen amb 32g d'oxigen. 28/32=0.875, que és 1.5 vegades 0.583 o, el que és el mateix, la fracció 3/2.



\subsection{Pesos atòmics i fòrmules moleculars (M)}

En principi, doncs, era possible relacionar els pesos de tots els elements i la manera en què es combinaven. El problema és que no es tenia cap certesa sobre cap compost inicial, ni sobre cap pes atòmic. Dalton no podia fer més que conjectures.

Gay-Lussac va mostrar el volum de la combinació de dues substàncies estava relacionat amb números sencers, com també marcava la llei de proporcions múltiples. Entre aquesta observació i l'admissió de que els elements nitrogen i oxigen podien ser poliatòmics, va dur Amedeo Avogadro (1811) es van acostar a la determinació dels pesos moleculars, però encara hi havia indeterminació. Finalment, Cannizaro va crear l'escala de pesos a partir de l'observació que la teoria atòmica de Dalton establia que el níumero d'àtoms en una molècula havia de seguir proporcions senceres. Va establir el pes de l'hidrogen com a 2g.

Per ser més precisos en la determinació de pesos atòmics es poden usar altres tècniques. Per exemple, considerant que a temperatura 0K els gasos es comporten de manera ideal i la densitat és proporcional a la pressió ($\delta=\alpha P$, on $\alpha$ seria la densitat que tindria un gas ideal a pressió 1 atm) però que a pressions més elevades cal millorar aquesta aproximació ($\delta = \alpha P + \beta P^2$), podem graficar $\delta/P$ en funció de la $P$ i obtindrem línies rectes com la de la Figura \ref{fig:DensitatIdealCO2}.
\begin{figure}[h]
\centering
\includegraphics[scale=0.3]{figures/DensitatIdealCO2.png}
\caption{Determinació de la densitat ideal del \ch{CO2} a 273,1$^º$K.}
\label{fig:DensitatIdealCO2}
\end{figure}
\begin{exr}
Si la densitat del gas oxigen és de 1.428g/l a 1atm i 273.1K i la del \ch{CO2} és 1.9635g/l a les mateixes condicions, i si el pes molecular de l'oxigen és 31,998, quin és el pes molecular del \ch{CO2}?
\end{exr}

\begin{mdframed}[backgroundcolor=gray!30,frametitle=Concepte de mol]
El número d'àtoms de carboni contingut en, exactament, 12g de \ch{C^{12}} s'anomena número d'Avogadro, \emph{N}. Un mol és la quantitat de matèria que conté el número d'Avogadro de partícules.
\end{mdframed}

\newpage

\subsection{Estequiometria (M)}

\section{Gasos}
\label{sec:gasos}

Dels gasos podem mesurar una sèrie de propietats característiques que ens permeten el seu estudi precís: \textbf{massa}, \textbf{volum}, densitat, \textbf{pressió}, \textbf{temperatura}, compressibilitat, coeficient de dilatació tèrmica, capacitat calorífica, viscositat, conductivitat tèrmica, difusivitat, etc, de les quals hem remarcat les anomenades \textbf{propietats primàries}.

La pressió es mesura en el S.I. en Pascals:
\[
1 {\rm Pa}=\frac{1 {\rm N}}{1 {\rm m}^2}
\]
però es treballa sovint en atmosferes (1 atm = 1.01325 $\times$ 10$^5$ Pa = 760 Torr = 760 mmHg).
\begin{figure}[h]
\centering
\includegraphics[scale=0.33]{figures/Manometre.png}
\caption[Manòmetre diferencial]{Un manòmetre diferencial mesura la diferència entre les pressions externes i d'un determinat gas. Cal tenir en compte la pressió atmosfèrica exterior (approx 1 atm).}
\label{fig:Manometre}
\end{figure}
Les equacions que depenen de les quatre propietats primàries s'anomenen equacions d'estat:
\[F(p,m,V,T)=0\]
En situacions normals (absència de camps elèctrics externs, per exemple) tres de les propietats són suficients per determinar la quarta (10g de N$_2$ a 30$\degree$C no podem saber en quin estat es troben -spolid, líquid o gas- ja que ens manca conèixer la pressió, i així en qualsevol situació).

D'altra banda, les propietats es poden classificar entre extensives (m, V, ...) o intensives (T, P, capacitat calorífica ...), segons depenguin de la quantitat de substància o no. La raó entre dues propietats extensives és sempre intensiva: $\delta = \frac{m}{V}$; $\nu = \frac{V}{m}$. Només necessitem dues propietats intensives per determinar l'estat d'un gas ($P$ i $T$) i, per tant, amb tres variables intensives podem construir una equació d'estat: 
\[F(p,V_{\rm m},T)=0\]

La mesura d'una propietat per mol sanomena valor molar d'aquesta variable: $V_{\rm m} = \frac{V}{n}$. 
\subsection{Llei de Boyle}

Robert Boyle (1627-1691) va notar, fent servir un manometre com el de la Figura \ref{fig:Manometre}, que existia una determinada llei de proporcionalitat entre la pressió exercida sobre un gas i el volum d'aquest
\begin{figure}[h]
\centering
\includegraphics[scale=0.10]{figures/Boyle.png}
\caption{Experiment de Boyle i llei de proporcionalitat entre la pressió exercida sobre un gas i el seu volum.}
\label{fig:Boyle}
\end{figure}
Va descobrir que el producte entre el volum i la pressió és una constant, la qual cosa duu a que sota dues condicions diferents de pressió els volums es comporten de la següent manera per al mateix gas a una temperatura donada:
\[
\frac{V_1}{V_2}=\frac{P_2}{P_1}
\]


\subsection{Llei de Charles i Gay-Lussac}

JAcques Charles (1787) i posteriorment Gay-Lussac van trobar que per a una mateixa pressió, la relació $\frac{V_{100 \degree C}}{V_{0 \degree C}}$ era identica per a tots els gasos (1.376).

Això duu a extrapolar fàcilment el comportament dels gasos i determinar el zero absolut de temperatura segons el gràfic \ref{fig:zeroabsolut}. Lord Kelvin (1848) va proposar usar el punt d'intersecció del gràfic amb la línia de les abcisses com a origen d'una nova escala de temperatura: $T/\rm{K} = t/\degree \rm{C} + 273.15$.\footnote{en realitat s'usa 273.16, que és el punt triple de l'aigua, temperatura a la qual coexisteixen en equilibri aigua, gel i vapor en un recipient tancat}
\begin{figure}[h]
\centering
\includegraphics[scale=1.0]{figures/zeroabsolut.png}
\caption{Gràfic del zero absolut a partir de la llei de Charles i Gay-Lussac.}
\label{fig:zeroabsolut}
\end{figure}

De la nova llei es desprèn que 
\[\frac{P V_{\rm m}}{T}= cnt = R\]
o bé
\[P V = n R T\]
que es coneix com a equació d'estat d'un gas ideal.

\subsection{Gas ideal}

Per tal de determinar la $R$ no podem simplement calcular el quocient $\frac{P V_{\rm m}}{T}$ per a qualsevol gas, ja que cadascun d'ells donarà un valor diferent (només és vàlida l'expressió per a un gas ideal!). Veure la Figura \ref{fig:R2}.
\begin{figure}[h]
\centering
\includegraphics[scale=1.0]{figures/R2.png}
\caption[Determinació de la constant dels gasos $R$]{R es pren com al valor límit de la fracció $\frac{P V_m}{T}$ per a tots els gasos: 
$R=\lim_{P \to 0} \frac{P V_{\rm m}}{T}= 0.08205 \frac{{\rm atm l}}{{\rm mol K}}$
}
\label{fig:R2}
\end{figure}

\begin{exr}
Calcular el volum molar d'un gas ideal a condicions normals (1 atm i 0$\degree$C).
\end{exr}

\begin{exr}
Quant gas hi ha en una mostra de volum 0.5 dm$^3$, a 80 graus Celsius i 800 Torr de pressió?.
\end{exr}

\begin{exr}
Un conductor comprova la pressió dels pneumàtics pel matí aviat, quan la temperatura és de 15$\degree$C, i és de 1.3$\times$10$^5$ Pa. Al migdia la temperatura és 15 graus més elevada. Quina és la pressió dels pneumàtics ara?.
\end{exr}

\begin{exr}
Si a CN la densitat d'un gas ideal és de 1.62 g dm$^{-1}$, quina és la seva massa molar? i quina densitat tindrà a 300 K i 2.4$\times$10$^5$ Pa?.
\end{exr}

\begin{exr}
Dalt de l'Everest, la pressió atmosfèrica és de 0,33 atm i la temperatura de 50 sota zero. Quina és la densitat de l'aire i en CN és de 1.29 g dm$^-3$?.
\end{exr}

\subsection{Teoria cinètica dels gasos (M)}

Per tal de relacionar aquestes descobertes amb l'estructura atòmica de la matèria, ens cal pensar una teoria que representi els gasos de forma extremadament simple: un \textit{model}. En el nostre cas (veure Figura \ref{fig:TeoriaCinetica}),
\begin{itemize}
\item el gas es format per partícules que es comporten com a punts de massa,
\item que llevat no col·lideixin no exerceixen força els uns sobre els altres
\end{itemize}
\begin{figure}[h]
\centering
\includegraphics[scale=0.2]{figures/TeoriaCinetica.png}
\caption{Representació del moviment de les partícules en un gas ideal.}
\label{fig:TeoriaCinetica}
\end{figure}
\begin{exr}
Pots calcular el volum ocupat per molècula en un gas ideal a CN?. Es troben dues molècules molt freqüentment en un gas a baixa pressió?
\end{exr}
Aquesta teoria, de forma relativament simple, ens permet expressar la pressió que s'exercexi sobre les parets d'un recipient per part del gas que conté segons:
\[
PV=\frac{2}{3} \left< E_c \right> = \frac{2}{3} \left< \frac{mc^2}{2} \right>
\]

D'aquí s'extreuen resultats interessants, com que l'energia cinètica translacional d'un mol de gas és \[N_0 \frac{m <c^2>}{2}=\frac{3}{2} RT\] o bé, si dividim pel número d'Avogadro a esquerra i dreta obtenim la constant dels gasos per molècula a partir de l'energia cinètica per molècula (constant de Boltzmann $k$): \[\frac{m <c^2>}{2}=\frac{3}{2} kT\].
Aquest resultat ens diu que si dos gasos tenen la mateixa $T$, les seves molècules tenen la mateixa energia cinètica promig. 
\begin{exr}
Qui es mou més ràpid, una molècula d'oxigen o una de nitrogen en dues mostres d'aquests gasos a la mateixa temperatura? Pots explicar perquè la pressió és independent de la natura de les molècules?
\end{exr}

\begin{exr}
Calcula la velocitat mitjana de les molècules d'hidrògen a 25$\degree$C.
\end{exr}

La distribució de les velocitats de les partícules d'un gas segueix la distribució de Maxwell-Boltzmann:
\[
\frac{\Delta N}{N}=4 \pi \left( \frac{m}{2 \pi kT}\right)^{3/2} \underbrace{e^{-mc^2/2kT}}_{\rm Boltzmann} c^2 \Delta c
\]
El factor de Boltzmann ens diu, en aquesta equació, que a qualsevol temperatura particular, acostuma a haver moltes menys molècules amb energies altes que amb energies baixes.
\begin{figure}[h]
\centering
\includegraphics[scale=0.5]{figures/BolzDist.png}
\includegraphics[scale=0.5]{figures/MXDIST.png}
\caption{La distribució de Maxwell per a diferents molècules i temperatures}
\label{fig:TeoriaCinetica}
\end{figure}

\subsection{Capacitat calorífica}

La capacitat calorífica d'una substància és la quantitat de calor en calories necessària per elevar 1$\degree$C la temperatura d'un gram de la substància.

De fet, això necessita precissió: no és el mateix fer aquest procés dpescalfament a volum constant que a pressió constant ($C_V$ vs $C_P$).

Si afegim calor a un gas, o bé s'expandeix (i per tant fa treball) o bé la velocitat de les seves particules augmenta.
A $V$ constant, l'escalfament produeix un increment d'energia cinètica:
\[\Delta E = \frac{3}{2} R \Delta T\]
però resulta que $\Delta E/ \Delta T$ és, justament, $C_V$ i, per tant, per a un gas monoatòmic ideal, $C_V=\frac{3}{2}R$ o, aproximadament, 3 cal/mol·grau.

En el cas de pressió constant, les partícules augmenten la seva energia cinètica i també exerceixen treball ($\Delta(PV)$):
\[\Delta(PV)=P\Delta V = P(V_2-V_1)=PV_2-PV_1\]
Per a un mol de gas, resulta que $PV=RT$ i, per tant, 
\[PV_2-PV_1=RT_2-RT_1=R\Delta T\]
Per tant, la capacitat calorífica extra pel fet de fer el procés a pressió constant és
\[\frac{\Delta (PV)}{\Delta T}=R\]
i, per tant, 
\[C_P=C_V+R \\
=\frac{3}{2} R + R= \frac{5}{2}R\]
És fàcil veure que $C_P/C_V=5/3=1.67$ i podem comparar aquests coeficients per a diversos gasos reals, per tal d'establir diferències amb el seu comportament ideal (\ref{tab:cpcv}).
\begin{table}[h!]
  \begin{center}
    \caption{Quocients de capacitat calorífica \cite{Mahan1977}}
    \label{tab:cpcv}
    \begin{tabular}{cc|cc}
      \hline
      Gas & $C_P/C_V$ & Gas & $C_P/C_V$\\
      \hline
      He & 1.66 & \ch{H2} & 1.41 \\
      Ne & 1.66 & \ch{O2} & 1.40 \\
      Ar & 1.66 & \ch{N2} & 1.40 \\
      Kr & 1.66 & \ch{CO} & 1.40 \\
      Xe & 1.66 & \ch{NO} & 1.40 \\
      Hg & 1.66 & \ch{Cl2} & 1.36 \\
      \hline
    \end{tabular}
  \end{center}
\end{table}
\begin{exr}
Perquè hi ha aquestes diferències entre la columna de l'esquerra i la de la dreta de la Taula \ref{tab:cpcv}? (Adona't que si un gas monoatpomic ideal, pel fet d'estar només augmentant la seva energia cinètica té una $C_V=\frac{3}{2}R$, es pot entendre que per a cada component necessita $\frac{3}{2}R$)
\end{exr}




\subsection{Gasos no ideals}

En gasos reals, el factor 
\[z=\frac{V_m}{V_{m,i}}=\frac{V_m}{RT/P}=\frac{PV_m}{RT}\]
no és 1, com succeiria a un gas ideal (veure Figura \ref{fig:FactorCompress}).
\begin{figure}[h]
\centering
\includegraphics[scale=0.5]{figures/FactorCompress.png}
\caption{Factor de compressibilitat per a diferents gasos a 0$\degree$C}
\label{fig:FactorCompress}
\end{figure}

Per tal de millorar l'aproximació a la realitat podem considerar diferents aproximacions. La més simple consisteix a pensar que el volum ideal explorat per les molècules és més gran que el volum que poden explorar en realitat, per un valor $b$ que prové del volum exclós per la presència d'altres molècules.
\[V_m = V_{m,i}+b=\frac{RT}{P}+b\]

Segons això,
\[z=\frac{PV_m}{RT}=1+\frac{b}{RT}P\]
que té una forma lineal. Aixó explicaria el cas de la molècula d'hidrogen.
Però què passa amb \ch{CH4} o \ch{CO}? Val la pena pensar que són molècules que es podran trobar líquides a temperatures més baixes amb major facilitat que no pas \ch{H2}. En un gas real, la pressió que exerciran les molècules sobre la paret del recipient serà més baixa que en un gas ideal:
\[P=P_i -\Delta P\]
Es pot veure que aquest increment de pressió és proporcional al quadrat de la concentració ($c=\frac{m}{V}=\frac{1}{V_m}$):
\[\Delta P \alpha c^2 \alpha \frac{1}{{V_m}^2} \]
o bé
\[\Delta P = \frac{a}{{V_m}^2}\]
i, per tant, 
\[ P_i=P+\frac{a}{{V_m}^2}\]

Si ara substituïm el volum i pressió ideals en l'equació dels gasos ideals:
\[
\left( P + \frac{a}{{V_m}^2} \right) (V_m -b)=RT
\]
o bé
\begin{equation}
\left( P + \frac{n^2 a}{V^2} \right) (V -nb)=nRT
\label{Eq:vdW}
\end{equation}

\begin{exr}
Què passa segons l'Equació \ref{Eq:vdW} si la pressió es fa propera a zero o bé la temperatura es fa molt gran per a un gas real? (veure la Figura \ref{fig:FactorCompressT}).
\end{exr}

\begin{figure}[h]
\centering
\includegraphics[scale=1.0]{figures/FactorCompressT.png}
\caption{Factor de compressibilitat per a un mateix gas a diferents temperatures}
\label{fig:FactorCompressT}
\end{figure}

Les forces de van der Waals que fan que es perdi la idealitat són degudes a tres contribucions:
\begin{enumerate}
\item Efecte d'orientació: forces dipol-dipol.
\item Efecte de distorsió: forces d'inducció.
\item Efecte de dispersió: forces de dispersió.
\end{enumerate}

\begin{exr}
Perquè \ch{CO2} i \ch{O2} tenen una desviació negativa respecte al comportament del gas ideal a pressions i temperatures moderades, mentres que l'He i el \ch{H2} presenten una deviació positiva en les mateixes condicions?
\end{exr}


%\subsection{Fenòmens de transport (M)}



\section{Sòlids}
\subsection{Propietats (M)}

Es caracteritzen per la seva rigidesa, incompressibilitat i per les seves característiques geomètriques.
La teoria atòmica ens ajuda a entendre el seu comportament basat en l'estructura d'una xarxa cristalogràfica.



\subsection{Tipus (M)}

Podem distingir:
\begin{description}
\item[sòlids cristal·lins] Exemples: clorur sòdic, sofre... 
\begin{itemize}
\item Són anisotròpics (la refracció, la conductivitat elèctrica, etc, depenen de la direcció en la qual es mesurin). 
\item Tenen punts de fusió ben definits. 
\item Es presenten a la natura sota formes polièdriques limitades per cares planes.
\end{itemize}
\item[substàncies amorfes] Exemples: vidre, plàstic... \begin{itemize}
\item No tenen caracterìstiques regulars. 
\item Són isotròpics. 
\item No estan ben establerts (són progressius) els seus punts de fusió.
\end{itemize}
\end{description}
Tots dos tipus de substàncies tenen alguna mena d'interacció local d'enllaç, però en el cas dels sòlids amorfs no existeix cap cel·la unitat.

La formació dels cristalls és depenent de la temperatura i la velocitat del procés en què es formen. Processos lents impliquen cristalls més grans, per exemple.

Els cristalls es poden estudiar per difracció de raigs X, que es basa en el fet que la longitud d'ona dels raigs X és similar a l'espaiat dels àtoms en el cristall. 

\begin{table}[h!]
  \begin{center}
    \caption[Sistemes cristal·lins i retícules de Bravais]{Sistemes cristal·lins i retícules de Bravais (veure Figura \ref{fig:Bravais}). P: centrada en les cantonades; I: centrada en el cos; F: centrada en la cara; C: amb punt central (adaptat de \cite{Yen2008}).}
    \label{tab:Bravais}
    \begin{tabular}{cccc}
      \hline
      Sistema & Cel·la unitat & retícula de Bravais \\ 
      \hline
	  Cúbic      & $a=b=c$ & P,I (Fig. \ref{fig:crystal_structure}b),F (Fig. \ref{fig:crystal_structure}a)\\
	             & $\alpha=\beta=\gamma=90\degree$ & \\
	  Tetragonal & $a=b\neq c$ & P,I \\
	             & $\alpha=\beta=\gamma=90\degree$ & \\
	  Ortoròmbic & $a\neq b\neq c$ & P,I,C,F\\
	             & $\alpha=\beta=\gamma=90\degree$ & \\
	  Romboèdric & $a=b=c$ & R(P)\\
	             & $\alpha=\beta=\gamma \neq 90\degree$ & \\
	  Hexagonal  & $a=b\neq c$ & P (Fig. \ref{fig:crystal_structure}c) \\
	             & $a=b \neq c$ & \\
	             & $\alpha = \beta = 90 \degree$ & \\
	             & $\gamma = 120\degree$ &\\
	  Monoclínic & $a \neq b \neq c$ & P,C\\
	             & $\alpha=\gamma \neq \beta$&\\
	  Triclínic  & $a \neq b \neq c$ & P\\
	             & $\alpha \neq \beta \neq \gamma$&\\
      \hline
    \end{tabular}
  \end{center}
\end{table}


\begin{figure}[h]
\centering
\includegraphics[scale=0.1]{figures/Bravais.png}
\caption{Retícules de Bravais i relació amb els 7 típus de cel·la unitat \cite{Yen2008}.}
\label{fig:Bravais}
\end{figure}
\begin{figure}[h]
\centering
\includegraphics[scale=0.4]{figures/FCC_crystal_structure.png}
\includegraphics[scale=0.12]{figures/CBC_crystal_structure.png}
\includegraphics[scale=0.32]{figures/HCP_crystal_structure.png}
\caption[Exemples d'estructures cristal·lines]{Exemples d'estructures: a) cel·la unitat cúbica centrada en la cara, amb 4 àtoms per cel·la unitat; b) cel·la unitat cúbica centrada en el cos; c) hexagonal (en la imatge, Hidrur de Crom, \ch{CrH_x})}
\label{fig:crystal_structure}
\end{figure}

\begin{exr}
La ratio d'empaquetament d'una cel·la unitat es defineix com la fracció entre el volum omplert pels àtoms que la formen i el seu volum total. Calcula el RE de la cel·la unitat cúbica centrada en la cara i de la cel·la unitat cúbica centrada en el cos (veure Figura \ref{fig:crystal_structure}).
\end{exr}

La Figura \ref{fig:bonding_motifs_PT} mostra els modes d'empaquetament dels elements de la taula periódica. A partir dels modes d'empaquetament també es poden calcular relacions diverses que ens indiquen el grau de direccionalitat dels enllaços entre els diferents àtoms, ajudant a identificar els diferents tipus de sòlids cristal·lins.
\begin{figure}[h]
\centering
\includegraphics[scale=0.13]{figures/bonding_motifs_PT.png}
\caption{Estructura cristal·lina o modes d'enllaç a la taula periòdica \cite{Yen2008}.}
\label{fig:bonding_motifs_PT}
\end{figure}

Podem distingir cinc tipus de sòlids, com es veu a la Taula \ref{tab:TipusSolids}.\footnote{veure també \linkurl{https://chemistry.tutorvista.com/inorganic-chemistry/types-of-solids.html}}
\begin{table}[h!]
  \begin{center}
    \caption{Tipus de sòlids (adaptat de \cite{Yen2008}.}
    \label{tab:TipusSolids}
    \begin{tabular}{llllr}
      \hline
      Tipus & Components & característiques & Exemples & Energia de cohesió \\ 
            &         &                  &          & (kJ mol$^{-1}$) \\ 
      \hline
Iònic   &  càrregues $+$ i $-$ & fràgils, aïllants             & NaCl     & 795 \\
        &                      & alt punt de fusió             & LiF      & 1010 \\
Covalent&  àtoms enllaçats     & durs, no conductors (si purs),& diamant  &   715 \\
        &                      & alt punt de fusió             & SiC      & 1010  \\
Metàl·lic & ions positius en     &  molt conductors            & Na       & 110  \\
          & un núvol d'electrons &                             & Fe       & 395  \\
vdW (mol·leculars) &  molècules o àtoms &  tous, baix punt de fusió,  &  Ar   &  7.6  \\
                   &                    & volàtils i aïllants         &  \ch{CH4} &  10  \\
Enllaç d'hidrogen  &  molècules amb enllaços  & baixa fussió, aïllants & \ch{H2O} &  50  \\
                   &  d'hidrogen              &                        & \ch{HF}  &  30  \\
      \hline
    \end{tabular}
  \end{center}
\end{table}

\subsubsection{Cristalls metàl·lics}
  
Per entendre l'estructura dels metalls, considerem primer que els nuclis d'un determinat element són formant un cristall de les característiques que hem explicat més amunt i que es representen a la Figura \ref{fig:bonding_motifs_PT}. Els elements metàlics tenen energies d'ionització baixes (de menys de 220 kcal mol$^{-1}$, exccepte en el cas del mercuri, on és de 240 kcal mol$^{-1}$), amb la qual cosa tenen tendència a perdre amb facilitat els electrons més externs (anomenats de valència; veurem més endavant com podem racionalitzar això). Per tant, aquests electrons estan poc atrets i, per tant, formaran enllaços covalents poc forts, com es veu a la Taula \ref{tab:DisMet}. 
Serà més forta la interacció dels electrons de valència amb molts àtoms que no pas amb només dos.
\begin{table}[h!]
  \begin{center}
    \caption{Energies de dissociació de molècules d'elements metàlics en kcal mol$^{-1}$ (adaptat de \cite{Mahan1977}.)}
    \label{tab:DisMet}
    \begin{tabular}{llll}
      \hline
\ch{Li2} & 25 & \ch{Zn2} & 5.7 \\
\ch{Na2} & 17 & \ch{Cd2} & 2.0 \\
\ch{K2}  & 12 & \ch{Hg2} & 1.4 \\
\ch{Rb2} & 11 & \ch{Pb2} & 16 \\
\ch{Cs2} & 10.4 & \ch{Bi2} & 39 \\
\ch{NaK} & 14 & \ch{NaRb} & 13\\
      \hline
    \end{tabular}
  \end{center}
\end{table}

També veurem que el seu número d'electrons de valència és menor que el número d'orbitals de valència (i per tant no es satura la valència a partir del principi d'exclusió de Pauli). Considerem que aquests nuclis tenen un electró lliure cadascun i volem entendre com es comporten si tots ells el contribueixen per crear un gas d'electrons al voltant del cristall (Figura \ref{fig:TightBoundModel}, a dalt).
A això ajuda que el número de coordinació és alt en cristalls metàl·lics (8 en el cas d'una estructura cúbica centrada com s'aprecia en la Figura \ref{fig:crystal_structure}b o bé superior en altres estructures).

\begin{figure}[h]
\centering
\includegraphics[scale=0.5]{figures/TightBoundModel1.png}
\includegraphics[scale=0.13]{figures/TightBoundModel2.png}
\caption[Tight Bound Model]{Tight Bound Model aplicat a l'estructura electrònica d'un metall. A dalt: podem considerar que cada electró està fortament lligat al seu nucli. A baix: estructura electrònica de considerar un conjunt molt gran d'àtoms d'hidrogen formant un hipotètic cristall metàl·lic; la suma dels orbitals electrònics forma un orbital molecular  amb un continu d'energies \cite{Yen2008}.}
\label{fig:TightBoundModel}
\end{figure}

Els electrons disposats d'aquesta manera ocupen orbitals atòmics que es combinen linealment seguint la teoria LCAO (la veurem més endavant) i generen el mateix nombre d'orbitals moleculars. En el cas d'un hipotètic cristall d'àtoms d'hidrogen obtindríem un diagrama energètic dels diferents orbitals fins a formar una banda com es mostra en la part inferior de la Figura \ref{fig:TightBoundModel}.

Un millor model és l'anomenat de l'electró lliure. 
En aquest model, es considera que els electrons es poden moure lliurement per l'estructura tridimensional del metall. 
Així, la quantitat d'electrons $N$ de massa $\mu$ que poden ser encabits en un nivell d'energia $E_{max}$ en un cub tridimensional de costat $a$ ve donat per:
\[
N=\frac{8 \pi a^3}{3} \left( \frac{2 \mu E_{max}}{h^2} \right)^{\frac{3}{2}}
\]
o, dit d'una altra manera, l'energia d'una determinada densitat d'electrons $\rho$ és:
\[E_{max} = \frac{h^2}{2 \mu} \left( \frac{3 \rho}{8 \pi} \right)^{\frac{2}{3}}\]

\begin{figure}[h]
\centering
\includegraphics[scale=0.08]{figures/FreeElectron.png}
\caption[Model de l'electró lliure]{En el model de l'electró lliure, la densitat d'estats d'un gas d'electrons és proporcional a l'arrel quadrada de l'energia cinètica de les partícules \cite{Yen2008}. }
\label{fig:FreeElectron}
\end{figure}

A 0K els electrons només ocupen els estats d'energia més baixos fins a l'anomenat nivell de Fermi (Figura \ref{fig:FreeElectron}). Els electrons addicionals que entrin al sistema a causa de la conducció elèctrica omplen els orbitals vacants.

Entre els orbitals ocupats i els orbitals vacants pot existir un gap que fa que a baixa temperatura aquests metalls no condueixin. Es necessita major temperatura per tal que tornin a ser conductors. 
És el que s'anomena un semiconductor. Veure Exercici \ref{Ex:Fermi}.

\begin{exr}
La funció de Fermi $f(E)$ dóna la probabilitat de que un determinat estat energètic sigui ocupat a una determinada temperatura superior a 0K:
\[f(E)=\left( 1 + \exp \left[ \frac{E-E_f}{k_B T} \right] \right)^{-1}\]
a) Dibuixa $f(E)$. b) Si el nivell de Fermi per al coure és de 7eV, raona com es distribuiran els seus electrons a 0K i a 1000K.
\label{Ex:Fermi}.
\end{exr}

\begin{figure}[h]
\centering
\includegraphics[scale=0.7]{figures/FermiBand.png}
\caption{Efecte de la temperatura en un semiconductor.}
\label{fig:FermiBand}
\end{figure}

\subsubsection{Cristalls iònics}
 Cada ió està lligat per una força Coulòmbica als altres i això fa que tinguin energies de dissociació (lattice energies) molt altes. 
\begin{equation}
U=-k\frac{Q_1 Q_2}{r_0} = \frac{NMz_+z_-e^2}{4\pi \varepsilon_0 r_0}
\label{Eq:Coulomb}
\end{equation}
Les energies depenen de la càrrega.
A l'Equació \ref{Eq:Coulomb}, M és l'anmenada constant de Madelung, que és fruït de considerar la interacció amb tota la resta d'ions que ocupen determinades posicions en la retícula.
\begin{figure}[h]
\centering
\includegraphics[scale=0.8]{figures/latticeEnergy.png}
\includegraphics[scale=0.6]{figures/latticeEnergySize.png}
\caption{Energies reticulars (lattice energies) de diversos sòlids iònics.}
\label{fig:latticeEnergy}
\end{figure}
\begin{exr}
Ordena GaP, BaS, CaO, and RbCl per ordre de les seves energies de dissociació.
\end{exr}

\begin{exr}
Usant la descripció del Cicle de Born-Haber que trobaràs a la Wikipedia (\linkurl{https://ca.wikipedia.org/wiki/Cicle_de_Born-Haber}) calcula l'energia reticular del Fluorur de Liti.
\end{exr}

\subsubsection{Cristalls moleculars}

Formats per molècules covalents. 
\begin{itemize}
\item[no polars] Es mantenen units per forces de van der Waals de tipus dispersius.
\item[polars] Es mantenen units per unions de vdW dipol-dipol.
\item[ponts d'hidrogen] 
\end{itemize}  

\subsubsection{Sòlids covalents}

Tots els àtoms estan units per enllaços covalents (diamant, grafit). Veurem en seccions properes com es formen aquests enllaços.


\begin{exr}
Compara, per als diferents tipus de sòlids descrits, les següents característiques:
\begin{enumerate}
\item pressió de vapor
\item punt de fusió
\item punt d'ebullició
\item duresa
\item fragilitat
\item conducció elèctrica en estat sòlid
\item conducció elèctrica en estat líquid
\end{enumerate}
\end{exr}

\subsection{Defectes}

Les xarxes cristal·lines  incorporen gran nombre de defectes que, sovint, els donen les seves propietats més interessants. Hi ha diversos tipus de defectes en els cristalls:
\begin{description}
\item[Defectes de punt] Impliquen una sola posició (veure Figura \ref{fig:DefectesPunt}). Identifiquem els defectes de Schottky com aquells on apareixen vacants catió-anió en parelles. En un defecte de Frenkel, en canvi, hi ha un desplaçament d'un catiuó cap a una posició intersticial. En els dos casos es manté la neutralitat de l'estructura (a la fluorita, per exemple, els intersticis són grans, i per tant és fpacil trobar defectes de Frenkel). Si el defecte és l'absència d'un anió podem tenir un defecte de tipus centre F.
\item[Defectes de línea] Tenen a veure amb desplaçaments o alteracions d'una fila de posicions a la xarxa. Es poden provocar dislocacions d'aresta (de l'ordre de 10$^6$ per cm$^2$ en un metall templat o 10$^{12}$ per cm$^2$ en un metall treballat en fred.
\item[Defectes de pla] Bidimensionals. Els àtoms en l'extrem dels microcristalls poden ser més reactius per estar exposats amb més facilitat.
\end{description}

\begin{figure}[h]
\centering
\includegraphics[scale=0.1]{figures/DefectesPunt.png}
\caption{Diversos tipus de defecte de punt en un cristall.}
\label{fig:DefectesPunt}
\end{figure}

\begin{exr}
El coure té una estructura cúbica de cara centrada, i l'aresta de la cel·la unitària és de 3.61${\AA}$. Pots suggerir algun tipus d'àtom que es pugui col·locar en els  intersticis de la seva xarxa sense distorsionar-la?
\end{exr}

\begin{exr}
Si la densitat del clorur sòdic sense defectes és de 2.165 g cm$^{-3}$, quina seria la densitat si tingués un ratio de 10$^{-3}$ defectes de a) Frenkel; b) Schottky. (el volum no varia amb els defectes)
\end{exr}

%\subsection{Propietats tèrmiques (M)}

\section{Líquids i dissolucions}

\subsection{Teoria cinètica (M)}

Les partícules que conformen un líquid es poden moure en el seu sí, i Robert Brown (1827) va suggerir que feien de forma aleatòria. Això era degut a la petita mida de les partícules (de l'ordre de 10${-6}$m) i el constant xoc de les partícules que el formen.

\begin{figure}[h]
\centering
\includegraphics[scale=0.5]{figures/Brownian_motion.png}
\caption[Moviment Brownià]{Representació del moviment Brownià d'una partícula petita en un fluïd. La seva mida fa que els xocs amb les molècules del fluïd faci variar la seva trajectòria de forma globalment aleatòria.}
\label{fig:Brownian_motion}
\end{figure}

El moviment Brownià es pot representar per un model estocàstic en el qual els canvis de posició des d'un instant a l'altre estan produïts per moviments aleatoris extrets d'una distribució normal amb mitjana $\mu=0.0$ i variança $\sigma^2 \times \Delta t$, o $N(0,\sigma^2 \times \Delta t)$. En altres paraules, la variança augmenta amb el temps de forma lineal amb pendent $\sigma^2$. 

\begin{exr}
Usant R, prova d'executar aquest script que mostra com simular el moviment Brownià d'una partícula en un líquid (extret de \linkurl{http://www.phytools.org/eqg/phytools/}):
\begin{lstlisting}[language=R]
t <- 0:100  # temps de simulació
sig2 <- 0.01
## primer, calcula un conjunt de desviacions aleatòries puntuals
x <- rnorm(n = length(t) - 1, sd = sqrt(sig2))
## després, acumula'n els resultats
x <- c(0, cumsum(x))
plot(t, x, type = "l", ylim = c(-2, 2))
\end{lstlisting}

Després, executa el següent script, que produeix 10000 simulacions diferents:

\begin{lstlisting}[language=R]
nsim <- 1000
## creo una matriu que hostatgi totes les simulacions
X <- matrix(0, nsim, length(t))
for (i in 1:nsim) X[i, ] <- c(0, cumsum(rnorm(n = length(t) - 1, sd = sqrt(sig2))))
plot(t, X[1, ], xlab = "temps", ylab = "desviacions", ylim = c(-2, 2), type = "l")
for (i in 1:nsim) lines(t, X[i, ])
\end{lstlisting}
\end{exr}
\begin{exr}
Per saber la variança que s'obté de la simulació podem fer
\begin{lstlisting}[language=R]
var(X[, length(t)])
\end{lstlisting}
i per mostrar l'histograma de posicions finals:
\begin{lstlisting}[language=R]
hist(X[, length(t)])
\end{lstlisting}
o bé:
\begin{lstlisting}[language=R]
plot(density(X[, length(t)]))
\end{lstlisting}
Calcula la variança de la distribució per a diferents valors del nombre de simulacions o el temps simulat.
\end{exr}

A partir de la teoria cinètico-molecular es pot veure que l'energia mitjana de les partícules en un moviment Brownià és $3/2 RT$, la mateixa que la de les molècules d'un gas a la mateixa temperatura. Només cal pensar en què passa en la interfície d'un líquid i un gas a la mateixa temperatura.


\subsection{Equilibris de fase (M)}

El pas de líquid a vapor s'anomena \emph{vaporització}, i es pot donar a la superfície del líquid (\emph{evaporització}) o en tot el seu volum (\emph{ebullició}).
Es tracta d'un procés endotèrmic. El seu procés contrari és la \emph{condensació}.

\begin{exr}
Explica, segons la teoria cinètico-molecular, la Figura \ref{fig:evap_vs_condens}. Com interpretes els termes \emph{equilibri dinàmic} i \emph{saturació}?
\end{exr}
\begin{figure}[h]
\centering
\includegraphics[scale=0.6]{figures/evap_vs_condens.png}
\caption{Esquema de la dependència de les velocitats d'evaporació i condensació respecte el temps en un líquid que s'evapora dins d'un recipient tancat.}
\label{fig:evap_vs_condens}
\end{figure}

La pressió que exerceix el vapor d'una substància a una temperatura determinada un cop és en equilibri amb la mateixa substància líquida és el que anomenem \emph{pressio de vapor} d'aquesta substància, $p_v$. La pressió de vapor augmenta en augmentar $T$, com es veu a la Figura \ref{fig:Water_vapor_pressure_graph}. Tots els líquids presenten corbes similars, amb pendent positiva.

\begin{figure}[h]
\centering
\includegraphics[scale=0.4]{figures/Water_vapor_pressure_graph.png}
\caption{Pressió de vapor de l'aigua en funció de la temperatura.}
\label{fig:Water_vapor_pressure_graph}
\end{figure}

La \emph{temperatura d'ebullició normal} és la que presenta un líquid a pressió 1 atm. La temperatura d'ebullició d'un líquid dependrà de la pressió exterior i de la natura del líquid.

\begin{exr}
És possible que un líquid arribi a estar sobreescalfat: temperatura superior a la d'ecullició per a aquella pressió però encara estat líquid, la qual cosa succeeix quan és molt pur i no hi ha partícules de pols.
Com aconseguiries que no es produeixi aquest sobreecalfament?
\end{exr} 

Hi ha una relació força directa entre la  pressió de vapor, la temperatura d'ebullició i la calor de vaporització (veure Taula \ref{tab:pv}). En general, com més intenses són les force sintermoleculars més alta és $\Delta H_v$ i $T_e$.
\begin{table}[h!]
  \begin{center}
    \caption{Pressió de vapor a 20$\degree$C, temperatura d'ebullició i calor de vaporització d'alguns líquids  (adaptat de \cite{Caamano1984}).}
    \label{tab:pv}
    \begin{tabular}{ccccc}
%    \begin{tabular}{ccS[table-format=2.4]S[table-format=4.1S[table-format=2.4]}
      \hline
      Líquid & naturalesa & $p_v$/10$^5$Pa & $T_e$/$\degree$C & $\Delta H_v$/kJ mol$^{-1}$\\
      \hline
      \ch{He} & no polar & $-$ & -268.9 & 0.1003 \\
      \ch{H2} & no polar & $-$ & -252.7 & 0.9028 \\
      \ch{CH4} & no polar & $-$ & -161.4 & 9.263 \\
      \ch{n-C4H10} & no polar & 2.03 & -1.5 & 24.24 \\
      \ch{CCl4} & no polar & 0.121 & 76.7 & 34.57 \\
      \ch{NH3} & polar & 10.1 & -33-6 & 20.15 \\
      \ch{H2O} & polar & 0.0233 & 100.0 & 40.62 \\
      \ch{CH3CH2OH} & polar & 0.0586 & 78.5 & 40.44 \\
      \ch{CH3OCH3} & polar & 5.06 & -23.7 & 22.61 \\
      \ch{CH3COCH3} & polar & 0.247 & 56.5 & 31.94 \\
      \hline
    \end{tabular}
  \end{center}
\end{table}

La Figura \ref{fig:hidrurs_boiling_point} mostra com la $T_e$ evoluciona en paral·lel a la taula periodica, i també com algunes substàncies són significativament excepcions d'aquesta norma degut a la seva capacitat d'establir ponts d'hidrogen.
\begin{figure}[h]
\centering
\includegraphics[scale=0.4]{figures/hidrurs_boiling_point.png}
\caption{Punts d'ebullició de diversos hidrurs relacionats amb la posició dels seus elements no metàl·lics a la taula periòdica.}
\label{fig:hidrurs_boiling_point}
\end{figure}

\subsection{Propietats crítiques}
\label{sec:PropietatsCritiques}

Michael Faraday va licuar gas clor el 1823, però per a altres gasos (\ch{H2}, \ch{N2} o \ch{O2}) no es va aconseguir i no va ser fins que Thomas Andrews va aconseguir liquar \ch{CO2} només si treballava a temperatures inferiors a 31$\degree$C. Així va sorgir el concepte de \emph{temperatura crítica}, $T_c$, com a propietat característica dels gasos, i que es defineix com aquella a partir de la qual no és possible liquar-los (veure Figura \ref{fig:punt_critic}).
En el punt crític, la $P_c$ és la pressió de vapor del líquid a $T_c$. És la màxima pressió de vapor del líquid, ja que a més $T$ no té sentit parlar-ne, ja que no existeix l'estat líquid. La corba de pressió front a la temperatura finalitza, doncs, en aquest punt.

\begin{figure}[h]
\centering
\includegraphics[scale=0.6]{figures/punt_critic.png}
\caption[Punt crític]{a) Cada corba isoterma representa la relació entre $P$ i $V$ a una temperatura donada. Les corbes inferiors deixen de sser hipèrboles perquè el gas es torna no ideal; b) el terme ``vapor'' es refereix a gas a una temperatura inferior al punt d'ebullició. Veure valors de punts crítics de substpancies comunes a \linkurl{http://philschatz.com/physics-book/contents/m42218.html}.\cite{Phase2018}}
\label{fig:punt_critic}
\end{figure}


\begin{figure}[h]
\centering
\includegraphics[scale=0.8]{figures/WaterPD.png}
\caption[Diagrama de fases simplificat de l'aigua]{Diagrama de fases simplificat de l'aigua. El gràfic no és a escala i tampoc conté diverses variants específiques que el farien molt complex.\cite{Phase2018}}
\label{fig:WaterPD}
\end{figure}
Al PC, la concentració molecular i tota la resta de propietats es fan iguals per al líquid i el gas.

\begin{exr}
Què ens produirà una cremada més gran: una massa $m$ d'\ch{H2O}(g) a 100 graus o la mateixa quantitat d'aigua líquida a la mateixa temperatura?
\end{exr}

\begin{exr}
En un recipient hi ha aigua líquida. Es conecta el frecipient a una bomba de buit i es va abaixant la pressió sobre el líquid. Si la temperatura és de 60 graus, a quina pressió bullirà l'aigua?
\end{exr}

\begin{exr}
Perquè a la Taula \ref{tab:pv} no apareix la $p_v$ de l'\ch{He}, \ch{H_2} i \ch{CH4}?
\end{exr}

A partir de tot el què hem treballat fins a aquest punt, queda clar que hi ha dues forces motores dels processos moleculars. D'una banda tots tendeixen a la mínima energia, però això no explicaria perquè els gasos existeixen com a tals. És necessari considerar la necessitat de tendir a un màxim desordre. El primer efecte ve determinat per l'entalpia del sistema i el segon per l'entropia. Ho treballarem més endavant amb major detall.

Ara com ara sí que podem, però determinar quatre característiques importants que tornarem a retrobar i formular:
\begin{enumerate}
\item L'equilibri en els sistemes moleculars és dinàmic, conseqüència de velocitats de reacció oposades.
\item El sistema passa espontàniament a l'estat d'equilibri.
\item Un cop assolit l'equilibri, les seves propietats són sempre les mateixes.
\item L'equilibri és fruit de dues tendències oposades: la necessitat d'assolir el mínim d'energia i la tendència al màxim caos.
\end{enumerate}

Això es pot escriure en base a l'energia lliure del procés d'equilibri. En equilibri, $\Delta G=\Delta H - T\Delta S0$, i per a tot procés espontani, $\Delta G <0$.
Més endavant anirem ampliant aquests conceptes termodinàmics, però ara com ara ens serveixen per entendre els fenòmens que hem anat explornat i que veure a continuació.

\subsection{Dissolucions}

Una dissolució és una substància complexa homogènia que, dins d'uns límits raonables, té una composició que pot variar contínuament.
Veurem més endavant que aquesta definició no és massa clara (pensem en el sabó en l'aigua).

La mesura de la concentració d'una dissolució pot donar-se en:
\begin{itemize}
\item \emph{Unitats de fracció molar.} $x_1=\frac{n_1}{n_1+n_2}$ i $x_2=\frac{n_2}{n_1+n_2}$
\item \emph{Molalitat} Número de mols de solut que hi ha en 1000g de solvent.
\item \emph{Molaritat} Número de mols per 1 litre de dissolució.
\item \emph{Normalitat} Número de pesos equivalents-gram del solut en un litre de dissolució.
\end{itemize}
\subsection{Solucions ideals i no ideals}

\begin{exr}
Raona com canvia la $p_v$ d´'una dissolució en funció de la seva concentració.
\end{exr}
Una dissolució ideal es forma sense despreniments de calor i amb una pressio de vapor que evoluciona segons la llei de Raoult:
\[
P_{dissolució}=P_{dissolvent}=P_1=P_1^0x_1 = P_1^0 \left(\frac{n_1}{n_1+n_2}\right)
\]
\begin{exr}
Determina la relació entre l'increment de pressió de vapor d'una dissolució i la fracció molar del solut.
\end{exr}
\begin{exr}
La pressió de vapor de l'aigua a 20$\degree$C és 17.54 mmHg. Quan dissolem 114g de sucrosa en 1000g d'aigua, la pressió de vapor es redueix en 0.11 mmHg. Quin és el pes molecular de la sucrosa?
\end{exr}

A partir de la llei de Raoult es pot arribar amb relativa facilitat a una expressió que relaciona la molalitat amb l'increment del punt d'ebullició:
\[
\Delta T=K_b m
\]
on $K_b$ és la constant d'elevació molal del punt d'ebullició respecte la concentració.
\begin{exr}
a) Exactament 1.00g d'urea dissolts en 75.00g d'aigua donen una dissolució que bull a 100.114$\degree$C. El pes molecular de la urea és 60.1. Quina és la $K_b$ de l'aigua?
b) Una dissolució preparada dissolent 12,00g de glucosa en 100g d'aigua bull a 100.34$\degree$C. Quin és el pes molecular de la glucosa?
\end{exr}
Quelcom similar es pot deduir per al punt de fusió: \[\Delta T = K_f m\]. Cal notar que, si en lloc de només un solut n'hi ha més d'un, hem de tenir en compte la molalitat total de la dissolució.

En el cas de tenir més d'un component volàtil a la dissolució, només hem de tenir present que la llei de Raoult s'acomplirà per a totes dues, i per tant la pressió parcial de cadascuna de les dues substàncies volàtils s'haurà de sumar:
\[
P_T=P_1+P_2=x_1P_1^0+x_2P_2^0
\]

En una dissolució ideal de dues components, el vapor sempre es troba enriquit amb aquella de les dues substàncies que sigui més volàtil.
A partir de la diferència de volatilitat de dos components en una dissolució podem analitzar la composició del vapor fent servir diagrames com el de la Figura \ref{fig:PhaseDiagram}a.
Usant aquesta mena de diagrames podem estudiar la destil·lació fraccionada d'una dissolució líquida de dues substàncies volàtils (Figura \ref{fig:PhaseDiagram}b).

\begin{figure}[h]
\centering
\includegraphics[scale=0.5]{figures/PhaseDiagram.png}
\includegraphics[scale=0.5]{figures/FractionalDistillation.png}
\caption[Diagrama de fases i destil·lació fraccionada d'una dissolució ideal]{a) La corba inferior mostra el punt d'ebullició d'una dissolució ideal per a diferents composicions. La corba superior mostra la composició del vapor, si connectem, per a una $T_{ebullició}$ d'ebullició donada, els punts de tall de les dues corbes a aquesta $T$. b) Destil·lació fraccionada d'una disslució ideal.}
\label{fig:PhaseDiagram}
\end{figure}

En una dissolució ideal no es desprèn ni absorbeix calor ($\Delta H=0$) i per tant, el procés és purament entròpic, cosa que el fa espontani (si $\Delta H=0$, $\Delta G= -T \Delta S<0$).

En dissolucions no ideals, observem una desviació respecte la llei de Raoult que pot ser positiva o negativa (veure, per exemple, \linkurl{https://www.youtube.com/watch?v=4hmrDSxEN-Q} o la Figura \ref{fig:desv_raoult}).
\begin{figure}[h]
\centering
\includegraphics[scale=1.0]{figures/desv_raoult.png}
\caption[Llei de Raoult per dissolucions ideals i no ideals]{a) en una desviació positiva respecte la llei de Raoult, els dos líquids que formen la barreja les forces d'atracció entre les dues substàncies són menors que les que es produeixin dins d'una d'elles: A i B escapen fàcilment i mostren, per tant, major pressió de vapor que l'esperada; b) en una desviació negativa, els dos líquids mostren una atracció mútua més gran que en cadascun per separat: les substàncies A i B tendeixen a marxar de la dissolució menys que en la situació ideal.}
\label{fig:desv_raoult}
\end{figure}

És interessant separar les dissolucions entre aquelles que:
\begin{itemize}
\item Desprenen calor en formar-se ($\Delta H <0$). Per exemple, si dissolem cloroform (\ch{CHCl3}) en acetona (\ch{(CH3)2CO}), es desprèn calor, en tant que es formen ponts d'hidrogen entre les molècules dels dos tipus de substància, però no dins de cadascuna d'elles (veure Figura \ref{fig:CloroformAcetona}).
\begin{figure}
  \centering
\chemfig{Cl>:[,1.2]C(<[3,1.2]Cl)(<[5,1.2]Cl)(-[0]H-[,,,,dash pattern=on 2pt off 2pt]O=C(-[1]CH_3)(-[7]CH_3))}
(\cmpd{cpmd:CloroformAcetona})\par
  \caption{La interacció de pont d'hidrogen entre el cloroform i l'acetona provoca un comportament no ideal de la llei de Raoult.}
  \label{fig:CloroformAcetona}
\end{figure}
En aquest cas, la pressió de vapor serà menor a l'esperada a partir de la llei de Raoult (Figura \ref{fig:desv_raoult}b). 
\item Absorbeixen calor en formar-se ($\Delta H > 0$). Això succeeix en els casos en els què barregem sbstàncies polars amb no polars i, per tant, eliminem moltes interaccions que altrament ja serien prou favorables. Per exemple, si barregem acetona (veure estructura a \ref{cpmd:ClorofomAcetona}) amb bisulfur de carboni (\chemfig{S=C=S}). Veure Figura \ref{fig:desv_raoult}a)
\end{itemize} 
Per tant, si la $T$ canvia en fer una dissolució de dues substàncies, la dissolució és no ideal.
Si observem amb atenció la Figura \ref{fig:desv_raoult} veiem que en el límit de dilució (dilució infinita) el corresponent dissolvent es comporta de forma propera a la ideal.

En una dissolució no ideal de dos components, a diferència del què passava amb les dissolucions idelas, no sempre el vapor es troba enriquit amb la substància més volàtil. 

\begin{exr}
Raona l'efecte que té la no idealitat de les dissolucions segons la Figura \ref{fig:desv_raoult}a a) en el seu punt d'ebullició, i b) en un procés de destil·lació fraccionada.
\end{exr}

Les dissolucions que destil·len sense canvi en la composició s'anomenen azeòtropes. Per exemple, si fem una barreja d'agua i àcid clorhídric i la fem bullir un temps suficient, la seva composició arribarà a un pes d'\ch{HCl} del 20.22\% respecte el pes total.

\begin{exr}
Un azeòtrop positiu prové d'una desviació també positiva de la llei de Raoult. a) Dibuixa la corba de Temperatura d'ebullició vs composició per a un azeòtrop positiu basant-te en les Figures \ref{fig:PhaseDiagram}a i \ref{fig:desv_raoult}a. b) Raona el resultat de fer una destil·lació a partir de diverses composicions d'aquesta mescla. c) què succeiria en un azeòtrop negatiu?
\end{exr}
\begin{figure}[h]
\centering
\includegraphics[scale=1.0]{figures/AzeotropEX.png}
\caption{Dades per al sistema etanol - acetat etílic (extret de \cite{Robin2016}).}
\label{fig:AzeotropEX}
\end{figure}
\begin{exr}
Volem separar una barreja equimolar d'etanol i acetat etílic per destil·lació en productes relativament purs. La barreja forma un azeòtrop de mínim punt d'ebullició segons la Figura \ref{fig:AzeotropEX}. No obstant, la composició de l'azeòtrop és sensible a la pressió, mostrant un increment significatiu de la fracció molar de l'etanol quan incrementa la pressió, com es mostra a la Figura. Dibuixa un esquema aproximat per a la separació de les dues components de la barreja que tregui profit d'aquest fet.
\end{exr}

\subsection{Solubilitat}

En la majoria dels casos, dues substàncies no es poden dissoldre l'una en l'altra en qualsevol proporció.
La solubilitat d'una substància en un determinat dissolvent, a una temperatura donada, és la concentració del solut en la dissolució saturada.
És una propietat fonamental per separar components d'una dissolució.
La solubilitat depèn de la natura de dissolvent i solut, així com de la $T$ i la $P$.

Per entendre l'efecte de la $T$ en la solubilitat usarem l'anomenat principi de LeChatelier, segons el qual "si s'exerceix alguna acció sobre un sistema que inicialment està en equilibri que afecti algun dels factors que l'identifiquen com a tal, el sistema es regularà ell mateix de manera que tendeixi a reduir l'efecte d'aquell canvi".
\begin{exr}
Raona perquè per a una dissolució en la qual $\Delta H_{sol} <0$, un augment de la temperatura fa que la solubilitat disminueixi, i a l'inrevés.
\end{exr}

Es pot relacionar les pressió d'un gas i la seva solubilitat a una $T$ donada en un líquid segons la llei de Henry\footnote{William Henry, 1775-1836}:
\[
C=kP
\]
on $C$ és la concentració (M) del gas en el líquid, $P$ la pressió parcial del gas i $k$ la constant de Henry, que depèn de la $T$ i la natura de gas i dissolvent (veure Taula \ref{tab:Henry}).

\begin{table}[h!]
  \begin{center}
    \caption{Constants de la llei de Henry per a diferents gasos en aigua a 20$\degree$C.}
    \label{tab:Henry}
    \begin{tabular}{cc}
      \hline
      Gas & constant de Henry / mol l$^{-1}$ atm$^{-1}$ \\
      \hline
He &	3.9\\
Ne &	4.7\\
Ar &	15\\
H2 &	8.1\\
N2 &	7.1\\
O2 &	14\\
CO2& 	392\\      
      \hline
    \end{tabular}
  \end{center}
\end{table}
% \chapter{Estructura electrònica dels àtoms i propietats periòdiques}

\section{L'estructura electrònica dels àtoms}
\subsection{Radiació d'un cos negre}
\begin{figure}[h]
\centering
\includegraphics[scale=0.5]{figures/blackbody.png}
\caption{Distribució de freqüències de radiació emeses per un cos negre}
\label{fig:blackbody}
\end{figure}
Rayleigh (Juny 1900). Radiació contínua $\lambda \nu = c$:
\[
R(\nu)=\frac{2 \pi k T}{c^2} \nu^2
\]
Planck (Octubre-Desembre 1900). Radiació en paquets $h\nu$ (\textit{quantum}):
\[
R(\nu)=\frac{c_1 \nu^3}{e^{c_2 \nu T} -1}=\frac{2\pi h \nu^3}{c^2} \frac{1}{e^{h\nu/kT}-1}
\]
\subsection{Efecte fotoelèctric i experiment de Rutherford}
\begin{figure}[h]
\centering
\includegraphics[scale=1.5]{figures/photoelect.png}
\caption{Efecte fotoelèctric}
\label{fig:photoelect}
\end{figure}
Lenard (Nobel 1905, raigs catòdics):
\begin{enumerate}
\item La freqüència llindar $\nu_0$ d'emissió depèn de cada metall
\item més llum, més electrons, però amb la mateixa energia cinètica
\item Més freqüència de radiació més energia cinètica electrons
\end{enumerate}
Einstein (1905):
\[
E_{fotó}=h \nu
\]
\[
h\nu = W + 1/2 m v^2
\]
Però per explicar-ho implicava introduir el concepte de dualitat ona-corpuscle.

Des dels experiments de Thomson amb raigs catòdics (1897) i Milikan (1909) es sabia que els àtoms estaven formats per càrregues positives i negatives, però es pensava que tenien forma esfèrica amb els electrons al seu interior.
\begin{figure}[h]
\centering
\includegraphics[scale=0.5]{figures/Rutherford.png}
\caption[Model de Rutherford]{La figura de l'esquerra mostra com haurien de travessar una placa metàl·lica partícules $\alpha$ segons el model de Thomson, A la dreta de la figura apareix l'explicació del comportament experimental real segons el model de Rutherford.}
\label{fig:Rutherford}
\end{figure}
Rutherford (1911) va mostrar que l'àtom no podia ser una esfera uniforme com la predita. Va mostrar que fent impactar partícules $\alpha$ (nuclis d'àtoms d'heli; per tant, amb càrrega +2 i massa 4) sobre una placa fina de metall es produïa ampla difracció d'un nombre petit de partícules i n'hi havia moltíssimes que travessaven la placa sense cap desviació o ben poca. Això implicava que els àtoms havien d'estar formats per una massa central altament carregada positivament i havien de tenir un volum molt més gran per tal que les partícules majoritàriament travessessin la placa (Figura \ref{fig:Rutherford})\footnote{\linkurl{https://commons.wikimedia.org/wiki/File:Geiger-Marsden_experiment_expectation_and_result.svg}}.

\subsection{Àtom de Bohr}
\begin{figure}[h]
\centering
\includegraphics[scale=0.7]{figures/Bohr.png}
\caption{Model de l'àtom de Bohr}
\label{fig:Bohr}
\end{figure}
Balmer, Rydberg (1885-1910); freqüències espectrals per a l'àtom d'hidrogen:
\begin{equation}
\frac{\nu}{c}=\frac{1}{\lambda}=R\left( \frac{1}{n_b^2}-\frac{1}{n_a^2}\right)
\label{eq:rydberg}
\end{equation}
\[
n_b=1,2,3,\dots; \; n_a=2,3,4,\dots; n_a>n_b
\]
on R=109677.6 cm$^{-1}$.

Bohr (1913):
\begin{enumerate}
\item estats estacionaris de l'àtom d'H
\item un estat estacionari no emet energia electromagnètica
\item l'emissió entre estats és igual a un fotó: $E_a-E_b=h\nu$.
\end{enumerate}
A partir de l'Eq. \ref{eq:rydberg} i aquest resultat, es pot veure que l'energia dels estats estacionaris del H ve donada per $E=-Rhc/n^2$ amb $n=1,2,3,\dots$. I va afegir dos postulats més al seu model:
\begin{enumerate}
\item l'electró de l'estat estacionari es mou en un cercle de radi determinat
\item hi ha una relació entre el radi d'aquestes òrbites i la seva energia $mvr=\frac{nh}{2\pi}$
\end{enumerate}
A partir d'aquí, va deduir una energia per a cada nivell d'energia:
\[
E=\frac{-2\pi^2 m e'^4}{h^2n^2}
\]
i $R=\frac{-2\pi^2 m e'^4}{h^3c}$. El resultat concorda amb l¡experiment i dóna els nivells correctes de les energies de l'àtom de Bohr, però els dos darrers postulats són totalment falsos i va ser el 1926 quan Schrödinger va formular la seva equació de la mecànica quàntica que superava el model de Bohr.

\subsection{Hipòtesi de de Broglie i principi d'incertesa}
El 1923, de Broglie va formular la hipòtesi de que la matèria, com la llum, també tenia naturalesa dual ona-corpuscle. 
Això explicaria el rerafons del model de Bohr: els electrons mostraven nivells d'energia quantitzats.
En el cas de la llum, Einstein havia arribat a que la relació entre la longitud d'ona i la massa d'un fotó era $\lambda=h/mc$. De Broglie va aplicar el mateix raonament a una partícula de massa $m$ i velocitat $v$: $\lambda=h/mv$.

A partir de considerar aquesta hipòtesi i la natura dual de les partícules, es pot arribar a veure que el producte de les incerteses en el càlcul de la posició i el moment lineal estan relacionades per $\Delta x \Delta p_x \geq h$, o principi d'incertesa de Heisenberg (1927).

\subsection{Mecànica quàntica}

Descrita per Heisenberg, Born i Jordan (1925) i per Schrödinger (1926). 

La mecànica clàssica és determinista, mentre que la quàntica és probabilística (pel principi d'incertesa de Heisenberg). L'Estat d'un sistema es determina per la seva funció d'estat $\Psi$, que és una funció de les coordenades de les partícules i del temps:
\begin{eqnarray}
-\frac{\hbar}{i} \frac{\partial \Psi}{\partial t}&=&-\frac{\hbar^2}{2m_1}\left(\frac{\partial^2\Psi}{\partial x_1^2}+\frac{\partial^2\Psi}{\partial y_1^2}+\frac{\partial^2\Psi}{\partial z_1^2} \right)-\\
 & & \cdots -\frac{\hbar^2}{2m_n}\left(\frac{\partial^2\Psi}{\partial x_n^2}+\frac{\partial^2\Psi}{\partial y_n^2}+\frac{\partial^2\Psi}{\partial z_n^2} \right) + V \Psi
\end{eqnarray}
on $\hbar=h/2\pi$, $i=\sqrt{-1}$, $m_1,\dots , m_n$ són les masses de les $n$ partícules de coordenades $x_i,y_i,z_i$ i $V$ és l'energia potencial del sistema.

El que ens interessa ara mateix és saber que la funció d'estat ens informa sobre l'estat del sistema.
A partir d'ella ho podem saber tot del sistema. El problema és trobar-la...

Per fer un cas senzill pensem en un sistema en el què l'energia potencial sigui independent del temps, com succeeix en un pàtom o una molècula aïllats. En aquest cas, l'equació es redueix a (per a una sola partícula)
\[
-\frac{\hbar^2}{2m}\frac{d^2 \psi(x)}{dx^2}+V(x)\psi(x)=E\psi(x)
\label{Eq:Schr1x}
\]
on $\psi$ és la funció d'ona del sistema.
\begin{figure}[h]
\centering
\includegraphics[scale=0.6]{figures/ParticulaCaixa.png}
\caption{Partícula en una caixa unidimensional de potencial $V=0$ entre $x=0$ i $x=L$ i $V=\infty$ en qualsevol altre posició}
\label{fig:ParticulaCaixa}
\end{figure}
\begin{mdframed}[backgroundcolor=gray!30,frametitle=Partícula en una caixa]
Un dels sistemes més simples per als quals l'Eq. \ref{Eq:Schr1x} es pot solucionar és el cas d'una partícula en una caixa unidimensional de parets infranquejables i impenetrables.
Considerem una partícula de massa $m$ que es mou amb una energia $E$ positiva al llarg de l'eix $X$ entre $x=0$ i $x=L$ (Figura \ref{fig:ParticulaCaixa}).
A partir de l'Eq. \ref{Eq:Schr1x} obtenim, per a aquest sistema:
\[
-\frac{}{2m}\frac{d^2 \psi}{dx^2}+V\psi(x)=E\psi
\]
Ens adonem que per a la regió $0\leq x \leq L$, on $V=0$, podem escriure:
 \[
\frac{d^2 \psi}{dx^2}=-\frac{2mE}{\hbar^2}\psi
\label{Eq:PC1}
\]
Com sabem, la segona derivada d'una funció $\psi$ ens dona informació qualitativa sobre la seva corbatura. En aquest cas veiem que quan la $\psi$ sigui negativa la seva corbatura serà positiva, i a l'inrevés. La funció $\sin(x)$ és un exemple d'aquest tipus de funció. De fet, $\psi=A \sin(bx)$ és una solució de l'Eq. \ref{Eq:PC1}. Si la substituïm a l'equació:
\begin{eqnarray*}
\psi = A \sin{bx} \\
\frac{d \psi}{d x} = bA \cos{bx} \\
\frac{d^2 \psi}{dx^2}=-b^2A \sin{bx}=-b^2 \psi
\label{Eq:PC2}
\end{eqnarray*}
Per tant, $\psi=A \sin{\left( \frac{2mE}{\hbar^2}\right)^{1/2} x}$. Fixem-nos que l'energia fins ara no està quantitzada, ja que no hem "tancat" la partícula restringint-la, encara, a cap valor, sinó que és un valor qualsevol positiu. Si ara tenim en compte que aquesta partícula no és lliure de moure's sinó que està tancada entre les parets $x=0$ i $x=L$ la situació canvia. Així, en tant que el quadrat de la funció d'ona es fa zero quan la probabilitat de trobar una partícula en un punt determinat és zero, i tenint en compte que la funció $\psi$ ha de ser contínua en tots els punts, és fàcil adonar-se que $\psi(x=0)=0$ i $\psi(x=L)=0$, que corresponen a les condicions límits del problema amb què ens enfrontem. La primera condició s'acompleix de forma automàtica si substituïm $x=0$ a l'Eq. \ref{Eq:PC2}. La segona condició, però, només s'acompleix si $\left( \frac{2mE}{\hbar^2}\right)^{1/2} L=n \pi$, amb $n=1,2,3,\ldots$. Els valors d'$E$ que compleixen aquesta condició són 
\[
E_n=\frac{n^2h^2}{8mL^2}, \; n=1,2,3,\ldots
\]
que representen els valors permesos (quantitzats) d'energia, corresponents a funcions d'ona del tipus:
\[
\psi_n=A \sin{\left( \frac{2mE_n}{\hbar^2}\right)^{1/2} x}=A \sin{\frac{n\pi x}{L}}
\]
Finalment, podem trobar $A$ tenint en compte que la probabilitat total de trobar la partícula en tot l'espai accessible $x\in [0,L]$ és igual a 1. Fent $\int_0^L \psi^2_n dx =1$ trobem que $A=\left( \frac{2}{L} \right)^{1/2}$. Per tant, finalment, els resultats de l'energia i la funció d'ona d'una partícula en una caixa són:
\begin{eqnarray}
E_n=\frac{n^2h^2}{8mL^2}\\
\psi_n= \left( \frac{2}{L} \right)^{1/2} \sin{\frac{n\pi x}{L}}
\end{eqnarray}
La Figura \ref{fig:ParticulaCaixa2} mostra la forma de la funció d'ona per als primers nivells de quantització de l'energia.\footnote{\linkurl{https://en.wikipedia.org/wiki/Particle_in_a_box}}
\end{mdframed}
\begin{figure}[h]
\centering
\includegraphics[scale=0.3]{figures/ParticulaCaixa2.png}
\caption{Funcions d'ona corresponents als primers nivells d'energia d'una partícula en una caixa unidimensional.}
\label{fig:ParticulaCaixa2}
\end{figure}
De l'exemple de la partícula en una caixa podem extreure'n conceptes generals que ens serviran més endavant:
\begin{itemize}
\item Els nivells d'energia quantitzats només apareixen si confinem la partícula entre dos extrems de potencial infinit. Sempre que tinguem moviments confinats o periòdics, apareixerà quantització, com en la rotació d'una molècula.
\item A mesura que augmenta la massa de la partícula o disminueix l'espai en el què aquesta està confinada, la distància entre les energies de quantització es fa menor.
\item El fet que la funció d'ona passi de valors positius a negatius implica que hi ha punts en els quals el seu valor és zero (i que anomenem \textit{nodes}). En aquests punts, el seu quadrat també serà zero, i per tant la probabilitat de trobar-hi la partícula serà nul·la.
\end{itemize}
\section{L'àtom d'hidrogen}

\subsection{Números quàntics}
Estudiar l'àtom d'hidrògen, l'exemple més simple possible, ens permetrà comprendre la base de l'enllaç químic entre àtoms.
L'aplicació de l'equació de Schrödinger 
\begin{equation}
H\Psi = E\Psi
\label{Eq:Schr}
\end{equation}a aquest àtom dóna resultats que estan d'acord amb les dades experimentals que se'n tenen.
L'equació de Schrödinger té la virtut de no necessitar postular els números que descriuen la quantització de l'energia, com succeïa en el model de Bohr. A partir d'aquesta equació, els números de la quantització de l'energia sorgeixen de forma natural en solucionar-la. En el cas de l'àtom d'hidroegn, els números quàntics que sorgeixen són:
\begin{description}
\item[Número quàntic principal, $n$]  Determina les energies accessibles per l'àtom d'hidrogen o per qualsevol altre àtom d'un sol electró i càrrega nuclear $Z$:
\[
E=-\frac{2\pi^2 me^4 Z^2}{n^2 h^2}
\]
$n=1,2,3\ldots$
Aquest resultat s'obté de la resolució de l'Eq. \ref{Eq:Schr} i és el mateix que va trobar Bohr en el seu model.
Cal fixar-se que l'energia en un àtom d'hidrogen o en qualsevol àtom en el qual només hi hagi un electró només depèn del número atòmic $n$.
\item[Número quàntic del moment angular, $l$] En estar relacionat amb el moment angular de l'electró, també ho està amb la seva energia cinètica i, per tant, és lògic que estigui limitat pel valor de $n$ (que expressa els nivells permesos d'energia total). $l=0, 1, \ldots, n-1$.
\item[Número quàntic magnètic, $m_l$] Pel fet que un electró amb un determinat moment angular pot ser considerat com un corrent elèctric que cirsula en un anell, pot generar un camp magnètic associat a aquest corrent. Aquest camp magnètic, pel fet d'estar associat al moment angular, estarà limitat al valor d'$l$: $m_l=-1, -l+1, \ldots, 0, 1, \ldots, l-1,l$.
\item[Número quàntic d'spin, $m_s$] Mostra la propietat magnètica intrínseca de l'electró i la possibilitat de girar sobre el seu eix en un sentit o un altre: $m_s=\{+\frac{1}{2},-\frac{1}{2}\}$.
\end{description}

\section{Orbitals moleculars}
El nivell d'energia $n$ determina les possibilitats dels altres números. Per exemple, en l'estat fonamental, l'àtom d'hidrogen pot tenir les combinacions de $\{n,l,m_l,m_s\}$ $\{1,0,0,+\frac{1}{2}\}$ i $\{1,0,0,-\frac{1}{2}\}$. De la mateixa manera, podem pensar en els estats excitats de l'àtom d'hidrogen considerant altres valors dels números quàntics, de manera que anem determinant els diversos orbitals :
\begin{table}[h!]
  \begin{center}
    \caption{Números quàntics i orbitals \cite{mahan_quimico_1977}}
    \label{tab:quant}
    \begin{tabular}{cccccc}
      \hline
      $n$ & $l$ & Orbital & $m_l$ & $m_s$ & combinacions\\
      \hline
      1 & 0 & 1s & 0 & $+\frac{1}{2},-\frac{1}{2}$ & 2 \\
      2 & 0 & 2s & 0 & $+\frac{1}{2},-\frac{1}{2}$ & 2 \\
      2 & 1 & 2p & $+1,0,-1$ & $+\frac{1}{2},-\frac{1}{2}$ & 6 \\
      3 & 0 & 3s & 0 & $+\frac{1}{2},-\frac{1}{2}$ & 2 \\
      3 & 1 & 3p & $+1,0,-1$ & $+\frac{1}{2},-\frac{1}{2}$ & 6 \\
      3 & 2 & 3d & $+2,+1,0,-1,-2$ & $+\frac{1}{2},-\frac{1}{2}$ & 10 \\
      4 & 0 & 4s & 0 & $+\frac{1}{2},-\frac{1}{2}$ & 2 \\
      4 & 1 & 4p & $+1,0,-1$ & $+\frac{1}{2},-\frac{1}{2}$ & 6 \\
      4 & 2 & 4d & $+2,+1,0,-1,-2$ & $+\frac{1}{2},-\frac{1}{2}$ & 10 \\
      4 & 3 & 4f & $+3,+2,+1,0,-1,-2,-3$ & $+\frac{1}{2},-\frac{1}{2}$ & 14 \\
      \hline
    \end{tabular}
  \end{center}
\end{table}

Per a un interessant i complet resum d'aquest capítol, ves a \linkurl{https://2012books.lardbucket.org/books/principles-of-general-chemistry-v1.0/s10-05-atomic-orbitals-and-their-ener.html}.
\begin{figure}[h]
\centering
\includegraphics[scale=0.4]{figures/SphericalCoords.png}
\caption{Coordenades esfèriques.}
\label{fig:SphericalCoords}
\end{figure}
A partir de la probabilitat de trobar un electró en un punt de l'espai, que ve donat per $\|\psi^2\|=1$ podem trobar la forma de les regions que ocuparà per a cada orbital (per analogia a les òrbites del model de Bohr). Si les expressem en coordenades esfèriques (Figura \ref{fig:SphericalCoords}) les funcions d'ona corresponents a cada orbital es poden expressar com a producte d'una part angular $\xi$ i una radial $R$ (veure Taula \ref{tab:AngRadOrb}).
\begin{equation}
\psi(r,\theta,\phi)=R_{n,l}(r)\chi_{l,m}(\theta,\phi)=R(r)\chi(\theta,\phi)
\label{Eq:psisplit}
\end{equation}
\begin{table}[h!]
  \begin{center}
    \caption{Part radial i part angular  de les funcions d'ona de l'àtom d'hidrogen \cite{mahan_quimico_1977}. $a=0$ és el radi de Bohr, $0.529 \times 10^{-10}$m.}
    \label{tab:AngRadOrb}
    \begin{tabular}{cc}
      \hline
      $\chi(\theta,\phi)$ & $R(r)$ \\
      \hline
      $\chi(s)=\left(\frac{1}{4\pi}\right)^{1/2}$ & $R(1s)=2 \left(\frac{Z}{a_0}\right)^{3/2} e^{-\sigma/2}$ \\
      \hline
$\begin{array}{rcl}
\chi (p_x)&=&\left(\frac{3}{4\pi}\right)^{1/2}\sin \theta \cos \phi \\
\chi (p_y)&=&\left(\frac{3}{4\pi}\right)^{1/2}\sin \theta \sin \phi \\
\chi (p_z)&=&\left(\frac{3}{4\pi}\right)^{1/2}\cos \theta 
\end{array}$
&
$\begin{array}{rcl}
R(2s) &=& \frac{1}{2\sqrt{2}}\left(\frac{Z}{a_0}\right)^{3/2} (2-\sigma) e^{-\sigma/2} \\
R(2p) &=& \frac{1}{2\sqrt{6}}\left(\frac{Z}{a_0}\right)^{3/2} \sigma e^{-\sigma/2} 
\end{array}$\\
      \hline
      $\begin{array}{rcl}
\chi (d_{z^2})&=&\left(\frac{5}{16\pi}\right)^{1/2} (3 cos^2 \theta -1) \\
\chi (d_{xz})&=&\left(\frac{15}{4\pi}\right)^{1/2}\sin \theta cos \theta \cos \phi \\
\chi (d_{yz})&=&\left(\frac{15}{4\pi}\right)^{1/2}\sin \theta cos \theta \sin \phi \\
\chi (d_{x^2-y^2})&=&\left(\frac{15}{4\pi}\right)^{1/2} \sin^2 \theta \cos 2\phi \\
\chi (d_{xy})&=&\left(\frac{15}{4\pi}\right)^{1/2} \sin^2 \theta \sin 2\phi \\
\end{array}$
&
$\begin{array}{rcl}
R(3s) &=& \frac{1}{9\sqrt{3}}\left(\frac{Z}{a_0}\right)^{3/2} (6-6\sigma +\sigma^2) e^{-\sigma/2} \\
R(3p) &=& \frac{1}{9\sqrt{6}}\left(\frac{Z}{a_0}\right)^{3/2} \sigma (4-\sigma) \sigma e^{-\sigma/2} \\
R(3d) &=& \frac{1}{9\sqrt{30}}\left(\frac{Z}{a_0}\right)^{3/2} \sigma^2 e^{-\sigma/2} 
\end{array}$\\
      \hline
      & $\sigma=\frac{2Zr}{na_0}; \; a_0=\frac{h^2}{4\pi^2 m e^2}$\\
      \hline
    \end{tabular}
  \end{center}
\end{table}

A partir de la Taula \ref{tab:AngRadOrb} es poden obtenir les funcions d'ona de tots els orbitals de l'àtom d'hidrogen. Per exemple, per a l'orbital 1s tenim:
\[
\psi(1s)= \frac{1}{\pi^{1/2}} \left( \frac{Z}{a_0} \right)^{3/2} e^{-Zr/a_0}
\]
i dóna una probabilitat de trobar l'electró a una distància $r$ de
\[
\psi^2(1s)= \frac{1}{\pi} \left( \frac{Z}{a_0} \right)^{3} e^{2Zr/a_0}
\]
que mostra com, en un orbital 1s, la màxima probabilitat de trobar l'electró es dóna a prop del nucli, i és independent de l'angle. 

\begin{exr}
Quants nodes té la funció $\psi(2s)$? i la $\psi(3s)$? i la $\psi(2p)$? 
\end{exr}
Com a norma, el número de nodes que podem trobar és $n-1-l$. 
En afegit, i per entendre millor la distribució electrònica, podem pensar en la densitat de probabilitat radial.
Aquesta es calcula trobant, a partir de la integració de l'expressió \ref{Eq:psisplit}, la probabilitat de trobar l'electró en un àtom hidrogenoïde (un sol electró), a una distància $r$ del nucli entre $r$ i $r+dr$, amb un angle $\theta$ entre $\theta$ i $\theta+d\theta$, i un angle $\phi$ entre $\phi$ i $\phi+d\phi$:
\[
|\psi(r,\theta,\phi)| d\tau = [R(r)]^2 [\chi(\theta,\phi)]^2 r^2 \sin \theta dr d\theta d\phi
\]
Integrant per als angles trobem la distribució radial:
\[
D(r)dr=r^2[R(r)]^2 dr \underbrace{\int_0^{\pi} \int_0^{2\pi} [\chi(\theta,\phi)]^2 \sin \theta dr d\theta d\phi}_{=1}=r^2[R(r)]^2 dr
\]
\begin{exr}
A partir de la densitat de probabilitat podem preguntar-nos coses com on és el màxim de probabilitat (solucionant  $\frac{d D(r)}{dr}=0$) o bé calculant el valor promig de la distància de l'electró al nucli segons $<r>_{n,l}=\int_0^{\infty} r D(r)dr$. Mostra que $<r>_{2s}=\frac{6a_0}{Z}$ i $<r>_{2p}=\frac{5a_0}{Z}$ (veure Figura \ref{fig:D2s2p}).\footnote{https://chemistry.stackexchange.com/questions/15208/difference-between-actual-position-of-electron-and-radial-distribution-probabili}
\end{exr}
\begin{figure}[h]
\centering
\includegraphics[scale=0.35]{figures/D2s2p.png}
\caption{Funció de distribució radial $D(r)$ per a les funcions 2s i 2p}
\label{fig:D2s2p}
\end{figure}
Per a cada combinació de números quàntics tenim resultats diferents (veure \linkurl{http://hyperphysics.phy-astr.gsu.edu/hbase/hydwf.html} i Figura \ref{fig:Hydrogen_Density_Plots}).


\begin{figure}[h]
\centering
\includegraphics[scale=0.25]{figures/Hydrogen_Density_Plots.png}
\caption{Densitat electrònica dels diferents orbitals de l'hidrogen}
\label{fig:Hydrogen_Density_Plots}
\end{figure}



%Això ens permetrà  entendre, entre altres efectes, la manera en què els complexes organometàl·lics es formen (Figura \ref{fig:coordination}).
%\begin{figure}[h]
%\centering
%\includegraphics[scale=0.35]{figures/coordination.png}
%\caption{Coordinació en complexes organometàl·lics}
%\label{fig:coordination}
%\end{figure}



\section{Propietats periòdiques}

Quan analitzem els orbitals de l'àtom d'hidrogen (o d'àtoms hidrogenoïdes, amb un sol electró), i en tant que la seva energia només depèn del número quàntic principal $n$, diem que són degenerats. En el cas d'àtoms polielectrònics, l'apantallament dels electrons interns fa que aquesta degeneració desaparegui. En realitat, el que succeeix és que l'aproximació de parlar d'orbitals atòmics com en el cas de l'hidrogen ja no és vàlida i és només una aproximació, en tant que l'equació d'Schrödinger ja no es pot resoldre en aquests àtoms.
Sigui com sigui, la Figura \ref{fig:degeneracio} mostra l'efecte en l'energia dels orbitals atòmics de tenir més d'un electró en l'àtom.
\begin{figure}[h]
\centering
\includegraphics[scale=0.35]{figures/degeneracio.png}
\caption{Degeneració dels orbitals de l'àtom d'hidrogen i d'àtoms polielectrònics}
\label{fig:degeneracio}
\end{figure}

El principi d'exclusió de Pauli determina que no hi poden haver dos electrons amb els mateixos números quàntics. Per tant, a cada orbital atòmic només hi poden haver dos electrons, amb $m_s=1/2$ i $m_s=-1/2$.
\begin{exr}
L'àtom de sodi es comporta de forma similar a l'àtom d'hidrogen pel que fa a la seva facilitat de "donar" un electró. Ho pots explicar en base a les densitats de probabilitat explicades a l'apartat anterior? Pensa en la llei de Coulomb i l'efecte pantalla dels electrons interiors.
\end{exr}
\begin{exr}
Escriu la configuració electrònica de l'argó i del potassi. Perquè després d'omplir els orbitals 3p no omplim els orbitals 3d? Com raones que els metalls de transició de les  darreres columnes de la taula periòdica tinguin típicament valències de +2?
\end{exr}
\begin{exr}
Pots explicar les dades de la Figura \ref{fig:AfinitatElectronica} en base a la configuració electrònica dels elements representats?
\end{exr}
\begin{figure}[h]
\centering
\includegraphics[scale=0.5]{figures/AfinitatElectronica.png}
\caption{Afinitat electrònica de diversos elements de la taula periòdica}
\label{fig:AfinitatElectronica}
\end{figure}
% \chapter{Enllaç Químic}

Una reacció química no és més que un procés que canvia l'ordenació d'enllaços del conjunt d'àtoms que conformen un sistema químic.
Això mostra la necessitat d'entendre aquest enllaç químic amb cert detall.
Hi ha tres conceptes clau en aquesta comprinesió del fenòmen de l'enllaç químic:
\begin{itemize}
\item D'entrada, ja el 1850 es va concebre el concepte de "valència" com la capacitat de combinació d'un element: el número d'àtoms d'Hidrogen o de Clor amb qui ho fa. 
En avançar cap a l'explicació dels sistemes químics en termes electrònics, usem el terme "electró de valència" com al número d'electrons que estan feblement units al nucli de l'àtom i que, per tant, poden formar part d'enllaços químics.
\item Els enllaços es formen perquè permeten els àtoms assolir estats de menor energia.
\item Les molècules adopten una certa geometria
\end{itemize}

Aquest capítol, basat en \cite{mahan_quimico_1977}, tracta de formular una teoria que expliqui aquests fenòmens.

\section{Parametritzant l'estructura molecular}

\subsection{Energies d'enllaç}

Es defineix com el canvi d'entalpia (calor específica a pressió constant) durant la separació d'una molècula gasosa en àtoms gasosos:\footnote{Per convertir entre unitats, pots usar \linkurl{http://www.colby.edu/chemistry/PChem/Hartree.html}}
\[
\ch{H2_{(g)} -> 2 H_{(g)}}, \; D(\ch{H-H})=\Delta H = 104 kcal/mol
\]
Aquesta $D$ també es pot avaluar en molècules poliatòmiques, i en aquest cas l'entorn molecular fa que hi pugui haver diferències entre les energies d'enllaç de dos elements determinats. No obstant això, aquestes diferenències són relativament menors:
\begin{eqnarray}
\ch{CH4_{(g)} -> CH3_{(g)} + H_{(g)}}, & D(\ch{H-CH3})=103 kcal/mol\\
\ch{CH3CH3_{(g)} -> CH3CH2_{(g)} + H_{(g)}}, & D(\ch{H-CH2CH3})=96 kcal/mol\\
\ch{(CH3)3CH_{(g)} -> (CH3)3C_{(g)}+H_{(g)}}, & D(\ch{H-C(CH3)3})=90 kcal/mol
\end{eqnarray}
\begin{figure}[h]
\centering
\includegraphics[scale=0.5]{bond-energy-table.png}
\includegraphics[scale=0.3]{bonde.png}
\caption{Algunes energies d'enllaç típiques.}
\label{fig:bond-energy-table}
\end{figure}

\begin{figure}[h]
\centering
\includegraphics[scale=0.5]{ebondh2.png}
\caption{Descripció del concepte energia d'enllaç}
\label{fig:bond-energy-table}
\end{figure}

\begin{exr}
Tenint en compte aquestes energies d'enllaç:

\begin{tabular}{cc}
& $E_b$ / kJ mol$^{-1}$ \\
\hline
C-O al monòxid de carboni & +1077 \\
C-O al diòxid de carboni & +805 \\
O-H & +464 \\
H-H & +436 \\
\hline
\end{tabular}

Calcula l'entalpia de la reacció:
\ch{CO_{(g)} + H2O_{(g)} -> CO2_{(g)} + H2_{(g)}} 
\end{exr}

Es poden determinar energies d'enllaç promig a partir d'aquestes observacions:
\begin{table}[h!]
  \begin{center}
    \caption{Energies d'enllaç promig en kcal mol$^{-1}$\cite{mahan_quimico_1977}}
    \label{tab:bonde}
    \begin{tabular}{cccc}
      \hline
      Enllaç & $\bar{D}$ & Enllaç & $\bar{D}$\\
      \hline
      \ch{C-H} & 98.7 & \ch{C-C} & 82.6 \\
      \ch{C-F} & \approx 110 & \ch{C=C} & 145.8 \\
      \ch{C-Cl} & 80 & \ch{C+C} & 199.6 \\
      \ch{C-Br} & 69 & \ch{C-O} & 85 \\
      \ch{C-I} & 55 & \ch{C=O} & 178 \\
      \ch{C-N} & 80 & \ch{O-H} & 110.6 \\
      \hline
    \end{tabular}
  \end{center}
\end{table}

\begin{exr}
Fent servir les dades de la Taula \ref{tab:bonde}, estima l'energia alliberada a pressió constant en la reacció:
\[
\ch{H2_{(g)}+Cl2_{(g)} + C_{(grafit)} -> CH3Cl_{(g)}}
\]
si la calor de vaporització del grafit a àtoms de carboni és de 170.9 kcal mol$^{-1}$.
\end{exr}

\subsection{Longituds i angles d'enllaç}

Els enllaços vibren constantment per la $T$ no nul·la de les molècules.
No obstant això, podem mesurar distàncies d'enllaç promig que es pot mesurar a partir d'estudis de raigs X o l'espectroscopia molecular (Taula \ref{tab:bonddist}).
\begin{table}[h!]
  \begin{center}
    \caption{Energies d'enllaç promig en kcal mol$^{-1}$\cite{mahan_quimico_1977}}
    \label{tab:bonddist}
    \begin{tabular}{cccc}
      \hline
      Enllaç & BD / \AA & Enllaç & BD / \AA \\
      \hline
      \ch{F2}  & 1.42 & \ch{HF} & 0.92 \\
      \ch{Cl2} & 1.99 & \ch{HCl} & 1.27 \\
      \ch{Br2} & 2.28 & \ch{HBr} & 1.41 \\
      \ch{I2}  & 2.67 & \ch{HI} & 1.61 \\
      \ch{ClF} & 1.63 & \ch{H2} & 0.74 \\
      \ch{BrCl} & 2.14 & \ch{N2} & 1.094 \\
      \ch{BrF} & 1.76 & \ch{O2} & 1.207 \\
      \ch{ICl} & 2.32 & \ch{NO} & 1.151 \\
       &  & \ch{CO} & 1.128 \\
      \hline
    \end{tabular}
  \end{center}
\end{table}
Els valors de la taula són molt constants a totes les molècules que contenen aquests enllaços, fins i tot més que no pas el que succeïa amb les energies d'enllaç promig de la Taula \ref{tab:bonde}. Quan això no es compleix és perquè hi ha enllaços diferents (dobles, triples...). Per exemple, entre l'età (\ch{C2H6}), l'etilè (\ch{C2H4}) i l'acetilè (\ch{C2H2}) les energies d'enllaç entre els àtoms de C varien de 83 a 146 i 200 kcal mol$^{-1}$, i les distàncies de 1.54 a 1.34 i 1.20, respectivament.


També es dóna constància en les geometries de les molècules, com s'aprecia a la Figura \ref{fig:molecular-geometry}.

\begin{figure}[h]
\centering
\includegraphics[scale=0.5]{molecular-geometry.png}
\caption{Estructures moleculars típiques, mostrant alguns angles d'enllaç rellevants}
\label{fig:molecular-geometry}
\end{figure}

\subsection{Espectroscopia molecular}

De la mateixa manera que en una molècula els nivells electrònics estan quantitzats (amb intèrvals energètics de l'ordre de 100 kcal mol$^{-1}$), ho estàn també els nivells vibracionals (on $E_{vib}=(v+\frac{1}{2})h\nu$, on $v$ és el nñúmero quàntic vibracional; amb intèrvals de 1-7 kcal mol$^{-1}$) i rotacionals ($\Delta E_{rot}=h \nu = \frac{\hbar}{2I}[J(J+1)-J'(J'+1)]$, on $J$ és el número quàntic rotacional i $I$ és el moment d'inèrcia de la molècula; amb intèrvals de menys de 0.03 kcal mol$^{-1}$) (veure la Figura \ref{fig:ERVlevels}).
\begin{figure}[h]
\centering
\includegraphics[scale=0.3]{EVRlevels.png}
\caption{Estructures moleculars típiques, mostrant alguns angles d'enllaç rellevants}
\label{fig:EVRlevels}
\end{figure}

L'espectroscopia molecular analitza la interacció de la llum amb les molècules i n'extreu característiques geomètriques a partir d'analitzar les frequències d'absorció/emissió.
Cada característica es pot estudiar amb una banda determinada de l'espectre lumínic (F.
\begin{figure}[h]
\centering
\includegraphics[scale=0.5]{MolSpect.png}
\caption{Espectre electromagnètic i nivells d'energia molecular que es veuen afectats.}
\label{fig:MolSpect}
\end{figure}

\section{Tipus d'enllaç}
És útil usar uns models conceptuals que expliquin l'estructura molecular. Són models extrems que no sempre es segueixen de manera exacta per les substàncies químiques. Sovint tenim barreges d'aquests models en un sistema real, però serveixen per conceptualitzar la manera en què la matèria s'organitza a nivell molecular i atòmic.

La majoria dels enllaços químics tenen propietats intermèdies entre el covalent i l'iònic però estan força aprop d'algun dels dos models.

\subsection{Enllaç covalent}

%aquesta secció està incompleta i necessita un desenvolupament a partir de l'experiència a classe

\begin{mdframed}[backgroundcolor=gray!30,frametitle=Estructura molecular i enllaç covalent]
En aquesta secció explorem el concepte d'orbital molecular i d'estructura molecular covalent. Usarem alhora els conceptes d'orbitals $\sigma$ i $\pi$ (provinents de la teoria d'orbitals moleculars) amb el model de Pauling de promoció i hibridació.
\end{mdframed}

\begin{exr}
Sabries explicar perquè la rotació en la molècula d'etilè (\ch{C2H4}) és més costosa energèticament que la rotació en la molècula d'età (\ch{C2H6})?
\end{exr}

%\subsection{Enllaç metàlic}
\subsection{Enllaç iònic}

Tot i que, com hem vist, l'estructura electrònica d'un àtom és complexa, podem pensar que des de la distància la distribució dels electrons segueix una forma propera a esfèrica. Per tant, i seguint la llei de Coulomb, aquestes esferes es comporten com si la seva càrrega estigués concentrada al seu centre, i podem considerar els ions com a càrregues puntuals.

En el cas del clorur de sodi (\ch{NaCl}), l'espectroscopia de raigs X mostra que l'estructura del compost és regular amb esferes que contenen 10 i 18 e$^{-}$ cadascuna, corresponents als ions \ch{Na+} i \ch{Cl-}, respectivament. Això demostra que els ions existeixen i que, per tant, les forces que uneixen aquests ions han de ser, per força, elèctriques.

Podem avaluar l'energia de la formació del compost iònic \ch{NaCl} a partir del cicle termodinàmic  (cicle Born-Haber):

%\schemestart
%  \ch{C\sld{} + 2 H2O\gas}
%  \arrow{->[\SI{90.1}{\kilo\joule}]}[,1.5]
%  \ch{CO2\gas{} + 2 H2\gas}
%  \arrow{<-[][*{0.north west}\SI{-393.5}{\kilo\joule}]}[-125,2]
%  \ch{C\sld{} + 2 H2\gas{} + O2\gas}
%  \arrow(@c3--@c1){->[][*{0.north east}\SI{-483.6}{\kilo\joule} ]}
%\schemestop

\begin{center}
\schemestart
  \ch{Na\gas{} + Cl\gas}
  \arrow{->[I(\ch{Na})-A(\ch{Cl})]}[,1.5]
  \ch{Na+ \gas{} + Cl- \gas}
  \arrow{->[][*{0.east}$U$]}[-90,2]
  \ch{NaCl\sld}
  \arrow{<-[][*{0.north}$\Delta H_f(\ch{NaCl})$]}[180,2]
  \ch{Na\sld{} + 1/2 Cl2\gas}
  \arrow{->[][*{0.east}$\begin{array}{c}\Delta H_{vap}(\ch{Na})\\+\frac{1}{2}D(\ch{Cl2})\end{array}$]}[90,2]
\schemestop
\end{center}
 
%\begin{mdframed}[backgroundcolor=gray!30,frametitle=Llei de les proporcions definides]
%En un compost donat, els elements constituients es combinen sempre en les mateixes proporcions pondarebles, sigui quin sigui l'origen i el mode de preparació dels compostos.
%\end{mdframed}

\begin{exr}
Se sap que una molècula gasosa de \ch{NaCl} té una distància interatòmica de 2.38\AA. Quina és l'energia potencial Coulòmbica d'un mol d'aquestes molècules?
\end{exr}

A partir del resultat de l'anterior exercici, i tenint en compte altres dades experimentals, podem veure que la formació d'un mol de molècules de \ch{NaCl} implica les següents relacions energètiques: 
\[
\begin{array}{cc}
\begin{array}{c}
\ch{Na \gas-> Na+ \gas{} + e-}\\
\ch{e- + Cl \gas{} -> Cl- \gas{}}\\
\hline
\ch{Na + Cl -> Na+ \gas{} + Cl- \gas}\\
\\
\ch{Na+ \gas{} + Cl- \gas -> NaCl\gas}\\
\hline
\ch{Na \gas{} + Cl \gas -> NaCl\gas}
\end{array}
\quad
\begin{array}{c}
\Delta E=I(\ch{Na}) = 118.4 \km \\
\Delta E=-A(\ch{Cl}) = -83.4 \km\\
\hline
\Delta E= 35\km \\
\\
\Delta E = -139.3 \km\\
\hline
\Delta E = -104.3 \km
\end{array}
\end{array}
\]
L'esquema ens mostra que, d'entrada, els ions no tendiran per ells mateixos a formar-se, i necessiten de l'energia que es desprèn en formar les interaccions Coulòmbiques entre aquests ions per tal de que sigui favorable.

No obstant, ens interessa entendre com es formen els cristalls de \ch{NaCl}. De fet, aquests cristalls tenen pressions de vapor extremadament baixes i, per tant, difícilment trobarem aquestes molècules gasoses. Per a calcular quanta energia es desprèn en formar aquests sòlids  hem de tenir en compte l'entalpia de malla $\Delta H_L$. Aquesta és, per al cas que ens ocupa,  a l'entalpia molar estàndar (1 atm i 0$\degree$C) del procés  \ch{NaCl\sld -> Na+ \gas{} + Cl- \gas}.
A $T=0K$, $\Delta H_L=U_L$, l'energia de malla, que només depèn de les interaccions Coulòmbiques dels ions. A $T$ normals, la diferència entre les dues és relativament menor.

Fem un càlcul d'aquesta energia potencial. Imaginem una disposició lineal d'ions positius i negatius amb càrregues $+z$ i $-z$, respectivament, separats per una distància $d$. L'energia potencial del primer ió seria:
\begin{eqnarray}
E_p&=&\frac{1}{4\pi \varepsilon_0} \times \left(
-\frac{z^2e^2}{d}+\frac{z^2e^2}{2d}-\frac{z^2e^2}{3d}+\frac{z^2e^2}{4d}-\cdots
\right)\\
&=&\frac{z^2e^2}{4\pi \varepsilon_0 d}\times \underbrace{(-1+\frac{1}{2}-\frac{1}{3}+\frac{1}{4}-\cdots)}_{-\ln 2}\\
&=&-\frac{z^2e^2} \ln 2
\end{eqnarray}
Quantitat que haurem de multiplicar per 2 per tal de considerar els dos costats de l'ió, així com per $N_A$ per tal d'obtenir el valor molar.
Finalment, podríem generalitzar el resultat per a qualsevol xarxa d'ions de càrregues de diferent signe $z_A$ i $z_B$, tot obtenint el resultat:
\[
E_p=-A\frac{|z_A z_B|N_A e^2}{4\pi \varepsilon_0 d}
\]
on $A$ és la constant de Madelung, que depèn de l'estructura tridimensional del cristall (per al \ch{NaCl}, $A=1,748$).

No obstant, aquesta no és l'única contribució a l'energia de malla, ja que cal incorporar el solapament que es produeix entre els orbitals dels dos ions quan s'apropen. Aquesta és proporcional al factor $\exp{-\frac{d}{d^*}}$, on $d^*$ es pren amb valor 34.5pm.

Si sumem les dues contribucions i trobant-ne el mínim, obtenim l'equació de Born-Mayer:
\begin{equation}
E_{p,min}=-\frac{N_A|z_A z_B| e^2 }{4\pi \varepsilon_0 d}\left(1-\frac{d^*}{d}\right) A
\label{eq:BornMayer}
\end{equation}

\begin{exr}
Dedueix l'equació de Born-Mayer a partir de considerar, de forma simplificada, que l'energia d'atracció Coulòmbica es pot expressar com $-\frac{Me^2}{r}$ i que la repulsió entre ions es pot expressar com $\frac{B}{r^n}$.
%U=-\frac{Me^2}{r_0}(1-1/n)
\end{exr}
\begin{exr}
L'òxid de magnesi, \ch{MgO}, té la mateixa estructura que el \ch{NaCl}. Sabent que $r(\ch{Mg^{2+}})=72pm$ i que $r(\ch{O^{2-}})=140pm$, calcula l'energia de malla d'aquest compost iònic.
%\includegraphics[scale=0.5]{res_ex1.png}
%-3.84 10^3 kj mol-1
\end{exr}

\begin{exr}
Fent servir el cicle de Born-Haber, estima l'entalpia de malla del \ch{KCl}.
%\includegraphics[scale=0.5]{res_ex2.png}
%717 kj mol-1
%exercici examen CaO (veure atkins)
\end{exr}

%
%\section{Orbitals atòmics i moleculars}
%
%\section{La química d'elements rellevants}
%\subsection{Hidrògen}
%\subsection{Nitrògen}
%\subsection{Oxigen, sofre}
%\subsection{Carboni}

% \chapter{La Reacció Química}

Aquest capítol està basat en \citep{caamano_ros_quimica_1991,yen_chemistry_2008,dickerson_principios_1993}.

\section{Termodinàmica Química}

Les reaccions químiques poden ser, a nivell del calor que intercanvien amb l'entorn:
\begin{description}
\item[exotèrmiques] si desprenen calor i, per tant, l'energia dels productes és més baixa que la dels reactius; o bé
\item[endotèrmiques] si l'absorbeixen i els productes acaben tenint més energia que els reactius.
\end{description}

En moltes ocasions mirem d'obtenir treball a partir de la calor produïda en una reacció, com succeix, per exemple, en un procés de combustió o en les reaccions electroquímiques que fan funcionar motors de combustió o elèctrics.

La termodinàmica estudia les relacions entre energia, calor i treball.
En aquest capítol treballarem al voltant de la termoquímica, la termodinàmica associada a les reaccions químiques. 


\subsection{Sistemes, estats i funcions d'estat}

Anomenem "sistema" a aquella part de l'"univers" que volem tractar en algun càlcul o experiment. 
Per exemple, un sistema pot ser un cilindre en un motor de combustió o bé una bateria elèctrica.
\begin{center}
\includegraphics[scale=1.0]{SistEntornUnivers.png}
\end{center}
La resta de l'univers, que no tractem directament, l'anomenarem "entorn".

L'\textit{estat del sistema} es caracteritza  per unes determinades \textit{variables d'estat} ($P$, $V$, $T$, $E$, $H$,...), que són magnituds físiques macroscòpiques mesurables. La termodinàmica estudia els \textit{estats d'equilibri} dels sistemes, en els quals les variables d'estat són idèntiques en totes les seves parts i invariables en el temps.
Tenen dues característiques principals:
\begin{enumerate}
\item En els \textit{canvis d'estat} d'un sistema, les variables d'estat només depenen de l'estat inicial i final, i no del camí utilitzat. Així, per exemple, el treball $W$ no és funció d'estat, mentre que l'energia $E$ sí que ho és.
\item El fet de fixar els valors d'algunes d'elles, una equació d'estat determina automàticament el valor de les altres. Així, com vam veure a la Secció \ref{sec:gasos}
\end{enumerate}

Els canvis d'estat poden ser 
\begin{description}
\item[reversibles] quan les funcions d'estat varian de manera infinitessimal, mantenint el sistema constantment en l'equilibri (l'expansió d'un gas contra una pressió que difereix només $dP$ de la pressió interna, per exemple);
\item[irreversibles] en qualsevol altre situació (un procés de combustió, l'expansió d'un gas contra el buit, etc).
\end{description}


\subsection{Treball}

El treball realitzat per una força en desplaçar un cos entre dues posicions es calcula segons:
\[
w=\int_{x_1}^{x_2} \vec{F} \cdot \vec{dx}
\]
Tenint en compte que $P=\frac{F}{A}$, és fàcil veure que, en el cas d'un pistó que exerceixi una pressió externa sobre un gas 
\begin{center}
\includegraphics[scale=0.8]{pisto_dw.png}
\end{center}
tenim
\[
dw=-F_{ext}dx = -P_{ext} A dx = -P_{ext} dV
\]
i, per tant,
\[
w=-\int_{V_1}^{V_2} P_{ext} dV
\]
\begin{exr}
Calcula el treball realitzat per comprimir un gas a pressió constant entre un volum inicial $V_1$ i un volum final $V_2$.
\begin{center}
\includegraphics[scale=0.8]{wV.png}
\end{center}
\end{exr}
\begin{exr}
Calcula el treball per dur un gas en un cilindre amb un èmbol des d'un estat de pressió 2 atm i volum 10 l fins a un estat de pressió 5 atm i volum 15 l per dos camins diferents:
\begin{enumerate}
\item Primer escalfant el gas a volum constant i després alliberant l'èmbol a pressió externa constant fins al volum desitjat.
\item Segon deixant l'èmbol lliure (pressió externa constant) mentre escalfem el gas, seguit de continuar escalfant fins que arribem a la pressió objectiu.  
\end{enumerate}
\end{exr}
\begin{exr}
\begin{itemize}
\item Calcular el treball d'expansió reversible i isotèrmic, a 25$\degree$C, de 3 mols d'un gas ideal entre 2 i 3 l de volum.
\item I si es tracta d'un procés irreversible?
\item Repeteix els dos apartats anteriors per a un procés de compressió entre 3 i 2 l del mateix gas.
\end{itemize}
\end{exr}

Dels exercicis anteriors es despèns que el treball no és una funció d'estat.

\subsection{Calor}

La calor $q$ és una magnitud macroscòpica que representa l'efecte d'infinitud de treballs microscòpics deguts als moviments de les partícules d'un sistema.
Com el treball, no és una funció d'estat, ja que depèn del camí que utilitzem per transferir-lo.
La Calor es medeix en calories o Joules.

La quantitat de calor necessària per incrementar la temperatura un determinat valor és\marginnote{Definim com caloria la quantitat de calor necessària per escalfar 1 gr d'aigua 1$\degree$C. Per tant, la capacitat calorífica de l'aigua és $C_p=1 cal g^{-1} \degree C^{-1}$. En realitat, això només és cert per a una temperatura donada, ja que la capacitat calorífica depèn lleugerament de la temperatura de partida. En el cas de l'aigua, la caloria es defineix per al pas de 14.5$\degree$C a 15.5$\degree$C. La quantitat de treball necessària per produir aquesta calor es va determinar per Mayer y Joule el s. XIX com $1cal=4.1860J$. En química usem més sovint les Capacitats calorífiques molars, $C_m$,  que indiquen la quantitat de calor necessària per escalfar un mol d'una substància 1$\degree$C.}
\[
q=nC_m\Delta T
\]
Si aquesta expressió la usem per explicar un procés infinitessimal obtenim
\[
C_m=\frac{1}{n}\frac{dq}{dT}
\]
I com que la capacitat calorífica es pot obtenir a $V=cnt$ o a $P=cnt$, podem calcular
\[
q_v=\int_{T_1}^{T_2} n C_{v,m} dT
\]
i
\[
q_p=\int_{T_1}^{T_2} n C_{p,m} dT
\]

\subsection{Primera llei de la termodinàmica: calor, treball i energia}

Podem incrementar l'energia d'un sistema afegint-hi calor o bé treball. A partir del fet que l'energia es conserva (o, el que és el mateix, que l'energia de l'univers és constant), podem veure que, si obviem l'energia potencial o cinètica global del sistema (l'anomenada energia externa) 
\[
q+w=\Delta U
\]
on $U$ és l'energia interna del sistema, que és funció d'estat!
Aquesta energia interna es pot desglossar en l'energia $U_0$ de les molècules a $T=0K$, sense cap moviment, l'energia tèrmica $U_{term}$ deguda al moviment de les molècules per la temperatura i l'energia potencial intermolecular $U_p$:
\[
U=U_0+U_{term}+U_p
\]
on $U_0$ es pot calcular a partir de càlculs moleculars i es desglossa en 
\begin{enumerate}
\item energia de traslació,
\item energia de rotació,
\item energia de rotació, i 
\item energia electrònica.
\end{enumerate}
\begin{exr}
A partir de l'expressió de l'energia cinètica mitja de les molècules d'un gas ideal, calcula l'energia de traslació que tindrà aquest gas a 298 K.
\end{exr}
\begin{exr}
Quina calor se li ha de donar a un gas ideal perquè s'expandeixi de manera reversible i isotèrmica de $V_1$ a $V_2$ ($V_2>V_1$)?
%l'energia interna només depèn de la temperatura. Per tant, aquí $\Delta U=0$. Per tant, $q=-w$ i podem integrar entre els dos volums per obtenir el resultat.
\end{exr}
En absència de treball útil (elèctric, per exemple, a partir d'una reacció química), $q_V=\Delta U$.
Si el procés té lloc a pressió constant:
\[
q_P  -P\Delta V = (\Delta U)_P
\]
Podem definir una nova funció entalpia com $H=U+pV$.
Aleshores:
\[
\Delta H = \Delta U + \Delta(PV)
\]
a pressió constant és fàcil veure que
\[
(\Delta H)_P = (\Delta U)_P + P\Delta V
\]
i que 
\[
q_P=(\Delta H)_P 
\]
Per tant:
\begin{itemize}
\item la calor transferida a volum constant és igual a l'increment d'energia interna del sistema, i
\item la calor transferida a pressió constant és igual a l'increment d'entalpia  del sistema
\end{itemize}

\paragraph{Calor transferida en una reacció química}
Es pot usar una bomba calorimètrica (Figura \ref{fig:Bomb_calorimeter_scheme}) per mesurar la calor transferida a volum constant en una reacció química.
\begin{figure}[h]
\centering
\includegraphics[scale=0.5]{Bomb_calorimeter_scheme.png}
\caption{Bomba calorimètrica.}
\label{fig:Bomb_calorimeter_scheme}
\end{figure}

\begin{exr}
La calor despresa a pressió constant en la combustió del grafit a 298 K i 1 atm és de -110.52 KJ mol$^{-1}$. Si el volum molar del grafit és de 5.3 cm$^3$ mol$^{-1}$, quina és la variació d'energia interna de la mateixa reacció?
% veure exemple 4 de la pàgina 16 del \citep{Caamano1977} III
%\includegraphics[scale=0.5]{ex_grafit.png}
\end{exr}

La calor normal de reacció, $\Delta H^{\circ}$ es defineix com l'entalpia de la reacció a 298 K entre els estats normals de reactius i els estats normals de productes.
Els estats normals es defineixen com:
\begin{itemize}
\item Per a un gas, quan t´una pressió parcial d'1 atm.
\item Per a un líquid, quan és pur en la seva forma més estable a 1 atm.
\item Per a un sòlid, la seva forma més estable a 1 atm.
\item Per a un solut, quan forma una disolució ideal en una concentració de 1 mol dm$^{-3}$.
\end{itemize}
D'aquí podem extreure les calors normals de formació $\Delta H_f^{\circ}$de les diferents substàncies com les calors normals de reacció de la seva formació a partir dels seus elements, cadascun en el seu estat normal.
A partir de les calors normals de formació podem avaluar les calors normals de qualsevol reacció.
\begin{exr}
Calcula la calor normal de la reacció 
\ch{Fe2O3_{(s)}  + 3 H2_{(g)} <-> 2 Fe_{(s)} + 3 H2O_{(aq)}}
\end{exr}

\subsection{Segona llei de la termodinàmica: Entropia}

La primera lei de la termodinàmica estableix que l'energia es conserva en tots els processos ordinaris, però no imposa la direcció de les transformacions de l'energia.
No obstant, hi ha certs processos que sabem que succeeixen de forma espontània, com el traspàs de calor d'un cos calent a un cos fred.
També sabem que no és possible fer una transformació total de l'energia tèrmica a la mecànica, i per tant les diferents formes de l'energia tenen qualitats diferents (Figura \ref{fig:Equality}).
\begin{figure}[h]
\centering
\includegraphics[scale=0.10]{Equality.png}
\caption{Qualitat de la conversió d'energia entre diferents fonts.\citep{yen_chemistry_2008}}
\label{fig:Equality}
\end{figure}

Per tal de mesurar aquest grau d'espontaneïtat dels processos es va desenvolupar el concepte d'entropia. Per a un sistema determinat que pot accedir a diversos nivells d'energia possibles, l'entropia es defineix, a nivell microscopi, com el producte de la constant de Boltzmann amb el logaritme de la probabilitat de maneres d'organitzar els estats d'acord amb un valor accessible d'energia.
\[
S=k \ln \Omega
\]
A nivell macroscòpic, la variació de l'entropia es pot avaluar com la relació entre la calor subministrada de forma reversible a un sistema dividit per la temperatura del procés:
\[
\Delta S = \frac{q_{rev}}{T}
\]
Si la calor es subministra irrevesiblement, sempre tindrem 
\[
\Delta S > \frac{q_{irev}}{T}
\]
A la natura, els processos tendeixen a passar en la direcció de maximitzar l'entropia.
Les formes d'energia que tenen menor quantitat d'entropia per unitat d'energia tendeixen a transformar-se en formes d'energia amb major valor d'aquesta quantitat

\begin{table}[h!]
  \begin{center}
    \caption{Degradació de la qualitat de l'energia en l'univers. Aquest no ''sobreviu'' per cap estabilitat inherent sinó per la successió de ''troballes'' aparentment accidentals, d'obstacles, que arresten el procés normal de degradació de l'energia (adaptat de \citep{dyson_energy_1971}).}
    \label{tab:DegradacioEnergia}
       \begin{tabular}{cc}
Forma de l'energia & Entropia per unitat d'energia\\
\hline
Gravitacional & 0 \\
Nuclear & 10$^{-6}$ \\
Calor interna de les estrelles & 10$^{-3}$ \\
Llum solar & 1 \\
Reaccions químiques & 1-10 \\
Calor de la terra & 10-100 \\
Radiació còsmica de microones & 10$^4$ \\
\hline
       \end{tabular}
   \end{center}
\end{table}

\begin{figure}[h]
\centering
\includegraphics[scale=0.8]{EsgotamentEnergia.png}
\caption{Les crisis energètiques no són conseqüència de la manca d'energia, ja que no pot ser creada ni destruïda. Provenen de la degradació de l'energia entre un estat de gran qualitat a un estat de baixa qualitat. El consum d'energia química provinent de fons fòssils es justifica per la facilitat de la seva explotació, però no per la seva qualitat en termes d'entropia (font Col·lectiu CMES).}
\label{fig:EsgotamentEnergia}
\end{figure}

\subsection{La tercera llei de la termodinàmica}

L'entropia d'un cristall perfecte a zero absolut és zero.

Aquesta llei ens ajuda a posar un valor de referència per a totes les substàncies químiques, permetent-nos construir taules de comparació entre elles. 

\begin{figure}[h]
\centering
\includegraphics[scale=0.1]{EntropyPT2.png}
\caption{Tendències del valor de l'entropia normal (en cal grad$^{-1}$ mol$^{-1}$) per a diferents elements de la taula periòdica.\citep{dickerson_principios_1993}}
\label{fig:EntropyPT2}
\end{figure}

\begin{figure}[h]
\centering
\includegraphics[scale=1]{EntropyTendency.png}
\caption{Tendències del valor de l'entropia normal (en cal $\degree ^{-1} mol^{-1}$) per a diferents propietats físiques de les substàncies.\citep{dickerson_principios_1993}}
\label{fig:EntropyTendency}
\end{figure}

\begin{exr}
\begin{center}
\includegraphics[scale=0.8]{EntropyPT.png}
\end{center}

La Figura mostra l'entropia normal $S^{\circ}_{298}$ per a elements de la taula periòdica, exclosos elements poliatòmics i que no formen sòlids.\citep{thoms_periodic_1995} Pots explicar:
\begin{enumerate}
\item perquè l'entropia augmenta en augmentar el període ($n$ més gran);
\item perquè l'entropia decreix al centre de cada període;i
\item quin és l'efecte d'un augmemnt de l'empaquetament o del grau de coordinació dels elements en l'entropia?
\end{enumerate}
\end{exr}

\section{Espontaneïtat i equilibri químic}

\subsection{Energia lliure}

Per entendre com l'increment d'entropia de l'univers com a criteri d'espontaneïtat dels processos ens pot servir per estudiar l'espontaneïtat dels processos químics cal tornar a discriminar entre el nostre sistema i l'entorn, com hem fet des de l'inici del capítol:
\begin{eqnarray*}
\Delta S_{univers} &= &\Delta S_{sistema} + \Delta S_{entorn}\\
&=&\Delta S_{sistema} + \frac{q_{entorn}}{T}\\
&\geq & 0
\end{eqnarray*}

Pel fet que
\[
\Delta H= q_{sistema} = -q_{entorn}
\]
aleshores
\[
\Delta S_{univers} = \Delta S_{sistema} - \frac{\Delta H}{T} \geq 0
\]
Multiplicat a ambdós costats per $-T$ obtenim l'expressió de l'energia lliure (o energia de Gibbs, nova funció d'estat $G=H-TS$):
\begin{equation}
-T \Delta S_{univers} = \Delta H_{sistema} -T\Delta S_{sistema}= \Delta G_{sistema} \geq 0
\label{Eq:Gibbs}
\end{equation}
El signe de $\Delta G$ determina, doncs, l'espontaneïtat d'una reacció.

Podem relacionar $\Delta G$ amb el treball útil fent unes simples transformacions. Per exemple, si s'aplica treball extern sobre el sistema de forma reversible (quan $q-T\Delta S=0$) tot el treball extern diferent de l'usat per variar el volum del sistema es transforma en variació d'energia lliure:
\[w=P\Delta V + w_{ext}\]
\[\Delta E = q-P\Delta V -\w_{ext}\]
\[\Delta H= q-w_{ext}\]
\[\delta G = q - T\Delta S -w_{ext} = -w_{ext}\]
Així, en una cel·la electroquímica, com veurem, el treball realitzat per la cèl·lula és una mesura directa del descens 
d'energia lliure dins seu. Alternativament, si apliquem potencial elèctric entre els terminals d'una cel·la d'electròlisi el treball elèctric realitzat sobre la cel·la és identic a l'augment d'energia lliure dels productes químics que conté. Per exemple, si dissociem electrolíticament l'aigua augementem l'energia lliure dels seus components hidrogen i oxigen, energia que podem recuperar després:
\[\ch{H2O_{(l)]} -> H2_{(g)} + 1/2 O2_{(g)}} \quad \Delta G^{\circ}= 56.69 kcal mol^{-1}\]
La clau rau en que aquesta trasformació posterior resulti en el mínim de calor i el màxim de treball.

\subsection{Càlcul de les energies lliures normals}

A partir de les dades tabulades en diverses fonts de l'entalpia normal de formació i l'entropia normal de cada ubstància, respectivament, podem calcular els valors de qualsevol procés químic.
\begin{exr}
Calcula l'energia lliure, l'entalpia i l'entropia normals per a la reacció \ch{3 H2_{(g)} + N2_{(g)} -> 2 NH3_{(g)}}. Què afavoreix i què desafavoreix la reacció? Succeiria igual a qualsevol temperatura?
%\includegraphics[scale=0.8]{exdG.png}
\end{exr}

\begin{exr}
Calcular els canvis d'energia lliure, entalpia i entropia per a la vaporització de l'aigua líquida. Quina influència hi té la pressió de vapor de l'aigua?
\end{exr}

\subsection{La natura de l'equilibri químic}

Les reaccions químiques són reversibles i, per aquest fet, els sistemes químics tancats produeixen un equilibri entre reactius i productes.
En aquest apartat relacionarem l'estructura atòmica de la matèria amb la tendència i l'espontaneïtat descrites en els apartats anteriors.

En la Figura \ref{fig:evap_vs_condens} vam introduir el concepte de reversibilitat d'un procés d'evaporació. Aquest fenòmen, eminentment físic, conduïa a un estat d'equilibri entre dues fases, que implicava la conservació d'aquestes però no pas de les molècules individuals que les formaven, les quals anaven fluctuant entre les dues fases.
El mateix succeeix en els processos químics. 
Posem per cas la reacció de descomposició tèrmica del carbonat de calci:
\[
\ch{CaCO3_{(s)} -> CaO_{(s)} + CO2_{(g)}}
\]
Aquesta reacció pot dur-se a terme fins a la descomposició total del \ch{CaCO3} si usem algun sistema per arrastrar el \ch{CO2} que es va formant. En canvi, sabem que si la pressió de \ch{CO2} en un recipient tancat que conté \ch{CaO} és prou alta, acabem formant de nou \ch{CaCO3}. Per tant, la reacció s'ha d'escriure, millor, fent:
\[
\ch{CaCO3_{(s)} <=> CaO_{(s)} + CO2_{(g)}}
\]
que indica que els reactius i productes assoleixen un equilibri en determinades condicions. Com vam veure a la Secció \ref{sec:PropietatsCritiques}, 
\begin{enumerate}
\item L'equilibri en els sistemes moleculars és dinàmic, conseqüència de velocitats de reacció oposades.
\item El sistema passa espontàniament a l'estat d'equilibri.
\item Un cop assolit l'equilibri, les seves propietats són sempre les mateixes.
\item L'equilibri és fruit de dues tendències oposades: la necessitat d'assolir el mínim d'energia i la tendència al màxim caos.
\end{enumerate}
Amb el què hem anat veient en les darreres seccions, algunes d'aquestes propietats adquireixen nou sentit.
\begin{enumerate}
\item Si volem comprovar que l'equuilibri és dinàmic només cal usar \ch{CO2} etiquetat radioactivament (és a dir, que contingui un isòtop radioactiu que poguem detectar, com el \ch{^{14}C}). Al cap d'una estona comprovarem que part d'aquest \ch{^{14}C} s'ha incorporat al sòlid en forma de \ch{Ca^{14}CO3}.
\item Quan parlem de tendència espontània a l'equilibri, cal no confondre amb la velocitat amb que això pot succeir. Per a la reacció que estem estudiant,
\[
\Delta G^{\circ} = 177,100 - 158 T \quad J \cdot mol^{-1}
\]
Per tant, a $T$ baixes la reacció no és espontània i sí en canvi a $T$ altes, com es mostra al gràfic:
\begin{center}
\includegraphics[scale=0.6]{CaCO3_dG.png}
\end{center}
A $T$ altes, segons es pot veure en el gràfic, la barrera d'energia lliure a superar serà més baixa que a $T$ baixes per a a questa reacció, la qual cosa implica que, a més, la reacció serà més ràpida. Però aquest és un concepte cinètic que treballarem més tard.
\item En el nostre exemple, les característiques de les molècules a reactius i productes es mantenen, complint el tercer dels enunciats de més amunt: per a cada $T$ hi ha un valor de la pressió de vapor del ch{CO2} en equilibri. Cal tenir present la molaritat de la reacció química, però (és a dir, el numero de mols que posem . Ho veurem més endavant.
\item Finalment, les dues tendències oposades (la formació de carbonat càlcic a partir de l'addició de diòxid de carboni a l'òxid de calci enfront de la descomposició tèrmica de la primera substància) s'han vist clarament en l'exemple exposat. El gràfic de l'energia lliure mostra també com, en aquest cas en que es desprèn un gas a partir d'un sòlid, el terme entròpic és clarament favorable, mentre que la calor (entalpia) necessària per fer la descomposició és elevada i contrària al procés.
\end{enumerate}

\begin{exr}
Pots racionalitzar qualitativament els quatre factors implicats en la descripció de l'equilibri químic en les reaccions:
\[ \ch{H2_{(g)} <=> 2 H_{(g)}}\]
\[ \ch{H2_{(g)} + I2_{(g)} <=> 2 HI_{(g)}}\]?
Per a les dues reaccions, calcula el valor de $\Delta G^{\circ}$ a partir de dades obtingudes a la literatura (usa els enllaços de la Secció \ref{EnllacosInteres}). 
\end{exr}

\begin{exr}
Com afectaria l'ús d'un catalitzador la corba d'energia lliure que hem dibuixat en el cas de la reacció de descomposició tèrmica del \ch{CaCO3}?
\end{exr}

\subsection{Constant equilibri}

Un cop assolit un equilibri com el de la reacció 
\[ \ch{H2_{(g)} + I2_{(g)} <=> 2 HI_{(g)}}\]
hi ha infinites possibilitats de combinacions de pressions de vapor dels tres gasos, però sempre es complirà, a una $T$ donada, la relació
\[
K=\frac{P^2_{\ch{HI}}}{P_{\ch{H2}}P_{\ch{I2}}}
\]
En general, i per a una reacció en equilibri del tipus
\[
\ch{a A + b B <=> c C + d D}
\]
es complirà, a una $T$ donada
\[
K=\frac{[C]^c[D]^d}{[A]^a[B]^b}
\]
Aquesta constant $K$ només depèn de la $T$ i de les substàncies que intervenen en la reacció. Si algun component és gasos es pot usar la seva pressió de vapor en l'expressió enlloc de la seva concentració.
\begin{exr}
Perquè podem usar la pressió de vapor enlloc de la concentració per a una substància gasosa en l'expressió de la constant d'equilibri?
Succeiria el mateix si el sistema no fos ideal? Serveix l'expressió per a qualsevol concentració de les substàncies reaccionants?
\end{exr}

\begin{exr}
Pots raonar perquè la constant d'equlibri de la descomposició tèrmica del \ch{CaCO3} és igual a $P_{\ch{CO2}}$?
\end{exr}

\begin{exr}
Com afecta la constant d'equilibri el fet d'igualar la reacció química amb coefficients que són els doble o el triple dels escrits inicialment?
\end{exr}

\begin{exr}
Quina és la relació entre la constant d'equilibri d'una reacció i la de la seva inversa?
\end{exr}

\begin{exr}
Escriu la constant d'equilibri de la reacció 
\[\ch{2 NO_{(g)} + O2_{(g)} <=> N2O4_{(g)}}\]
a partir de les de les reaccions 
\[\ch{2 NO_{(g)} + O2_{(g)} <=> 2 NO2_{(g)}}\]
i 
\[\ch{2 NO2_{(g)} + O2_{(g)} <=> N2O4_{(g)}}\]
\end{exr}



A partir de la definició de la constant, és fàcil deduir què és una constant de solubilitat d'una determinada substància, com mostra el següent exercici:

\begin{exr}
Quina és la constant de solubilitat del cromat d'argent (\ch{Ag2CrO4}) si la concentració d'una dissolució saturada d'aquesta sal té una concentració de 6.7$\times$10$^{-5}$M d'ions cromat?
\end{exr}

En general, quan estem dissolent una sal poc soluble, sempre tindrem el sòlid acompanyant els ions dissolts. Així, la concentració del sòlid es podrà considerar incorporada a la constant d'equilibri i parlarem de constants o productes de solubilitat.
Els productes de solubilitat es poden emprar per fer precipitació selectiva d'ions en una dissolució complexa. 

\begin{exr}
S'afegeix ió \ch{Ag+} a una dissolució que conté \ch{Cl-} i \ch{I-}, ambdós a una concentració de 0.01 M. Què precipita abans, \ch{AgCl} i \ch{AgI}. Quina és la concentració d'ions \ch{Ag+} quan la primera sal comença a precipitar? I quina és la concentració de l'anió del primer precipitat quan la segona sal comença a precipitar?
\end{exr}

Finalment, la relació entre la constant d'equilibri i l'energia lliure de la reacció ve donada per
\[\Delta G = -RT ln K_{eq}\]

\subsection{Efectes externs sobre equilibri}

\begin{tcolorbox}[colback=green!5,colframe=green!40!black,title=Constant d'Equilibri]
Aprèn la diferència entre quocient i constant d'equilibri amb la simulació que trobaràs a \linkurl{https://teachchemistry.org/periodical/issues/november-2017/predicting-shifts-in-equilibrium}
\end{tcolorbox}

El principi de Le Chatelier estableix que si un sistema en equilibri és pertorbat o a qualsevol tensió en qualsevol dels factors que determinen aquest equilibri, el sistema reaccionarà de tal manera que disminuirà l'efecte de la pertorbació. 
 Els efectes externs sobre un equilibri es poden entendre amb faciltat pensant en el retorn a la situació d'equilibri des de la què ens hem desplaat. Per exemple:

\begin{itemize}
\item Si augmentem o disminuïm la concentració d'una substància, l'equilibri es desplaçarà cap a formar una nova relació entre components que compleixi el valor de la $K$. La Taula \ref{tab:Eqs} permet visualitzar alguns d'aquests efectes.
\item Si augmentem el volum del sistema (en una reacció en fase gas) no necessàriament canviarem les condicions d'equilibri (dependrà de la molaritat de la reacció).
\item En una reacció endo/exotermica, un augment de la temperatura desplaça l'equilibri cap a la dreta o cap a l'esquerra, respectivament.
\end{itemize}

\begin{exr}
La constant d'equilibri de la reacció d'isomerització entre l'$n$-butà i l'isobutà és 2.5. Representa gràficament la tendència del sistema en funció de diverses concentracions inicials de cadascuna de les dues substàncies. (Pista: representa la pressió de vapor de l'$n$-butà en abcisses i la de l'isobutà en ordenades com mostra la Taula \ref{tab:Eqs}).
\end{exr}

\begin{exr}
Fes una interpretació similar per al cas de la reacció de dissociació del sulfat de bari en els seus components iònics (veure Taula \ref{tab:Eqs}).
\end{exr}

\begin{exr}
La constant d'equilibri de la dissociacio del \ch{NH4HS} sòlid en amoniac i sulfur d'hidrogen és de 0.11 atm$^2$. Si posem una mica d'aquest sòlid en un recipient tancat que conté amoniac a una pressió de 0.5 atm. Quina és la pressió final del sistema un cop assolit l'equlibri?
% pàgina 188 del mahan
\end{exr}

\begin{table}[h!]
  \begin{center}
    \caption{Representació gràfica de processos en equlibri simples. Casos genèrics com \ch{a A + b B <=> c C + d D} són més complexes de visualitzar, però la tendència que segueixen quan els posem lluny de l'equilibri també es pot entendre fàcilment valorant si $\frac{[C]^c[D]^d}{[A]^a[B]^b}<K_{eq}$ o $\frac{[C]^c[D]^d}{[A]^a[B]^b}>K_{eq}$. }
    \label{tab:Eqs}
    \begin{tabular}{c|c}
      \hline
      Tipus de reacció (Exemple) & representació gràfica de l'equilibri \\
      \hline
$\begin{array}{c}\ch{A_{(s)} <=>  A_{(g)}} \\ \ch{CaCO3_{(s)} <=>  CaO_{(s)} + CO2_{(g)}}\end{array}$ &\includegraphics[scale=0.3]{Eq1.png} \\ \hline
$\begin{array}{c}\ch{A <=> B} \\  \ch{n-butà <=> isopropà}\end{array}$ & \includegraphics[scale=0.3]{Eq2.png} \\ \hline
$\begin{array}{c}\ch{A_{(s)} <=> B + C} \\ \ch{BaSO4_{(s)} <=>  Ba^{2+}_{(aq)} + SO4^{2-}_{(aq)}}\end{array}$ & \includegraphics[scale=0.3]{Eq3.png} \\\hline
$\begin{array}{c}\ch{A <=> 2B} \\ \ch{N2O4_{(g)} <=> NO2_{(g)}}\end{array}$ & \includegraphics[scale=0.4]{Eq4.png} \\
      \hline
    \end{tabular}
  \end{center}
\end{table}

\begin{exr}
La constant d'equilibri de la reacció
\[
\ch{CO2_{(g)} + H2_{(g)} <=> CO_{(g)} + H2O_{(g)}}
\]
a 690K és 0.10. Quina és la pressió d'equilibri del sistema si barregem 0.5 mol de \ch{CO2} i 0.5 mol de \ch{H2} en un recipient de 5 l a 690K?
Si augmentéssim la T, la pressió augmentaria o disminuiria?
\end{exr}

\subsection{Equilibris no ideals}

En reaccions no ideals (molt concentrades o amb gasos a alta pressió) no podem expressar la constant d'equilibri com a producte de concentracions o pressions parcials, sinó que hem d'usar el concepte activitat. L'activitat de les substàncies dissoltes o dels gasos, en els dos exemples descrits, dependrien de la concentració o la pressió, però serien tan diferents d'aquestes com més allunyades de la idealitat estiguessin. En aquests casos, per a \ch{a A + b B <=> c C + d D}, escrivim:
\[
\frac{a_C^c a_D^d}{a_A^a a_B^b}=K_{eq}
\]

%\subsection{Cinètica química}
%
%\section{Precipitació}
%
\section{Equilibri iònic en solucions aquoses}

\subsection{Reaccions àcid-base}

Un cas de gran interès en química és la comprensió de l'equilibri iònic en dissolucions aquoses. 
Aquest cas particular ens farà comprendre processos àcid base, que formen una gran part de les casuístiques d'interès en la química.

Hi ha tres gran teories que permeten explicar el concepte àcid-base:\marginnote{D'entre els molts recursos disponibles a la xarxa, és particularment simple i ben explicat el que trobareu a \linkurl{https://www.chemguide.co.uk/physical/acidbaseeqia/theories.html}}
\begin{description}
\item[Arrhenius]
Arrhenius (1880-1890) va desenvolupar la teoria segons la qual àcids i bases es dissociaven en els seus ions segons:
\[\ch{\textit{A}H <=> \textit{A^-} + H+}\]
\[\ch{\textit{B}OH <=> \textit{B^+} + OH-}\]
En realitat, l'existència de l'ió \ch{H+} és fictícia, ja que es troba sempre solvatat amb una molècula d'aigua en forma de \ch{H3O+} o estats d'hidratació superior.

\item[Lowry-Br{\o}nsted] Això ens duu de forma natural al concepte d'àcid-base formulat per Lowry-Br{\o}nsted (1923): un àcid és una espècie química amb tendència a donar un protó, i una base a acceptar-lo.
Així, ens queda:
\[\ch{\textit{A}H + \textit{B} <=> \textit{A}^- + \textit{B}H^+}\]

\begin{exr}
Escriu la reacció àcid-base de l'ió carbonat en aigua en equlibri amb l'ió bicarbonat. Qui té el rol d'àcid i de base en la reacció directa i la inversa?
\end{exr}

Per a una reacció àcid-base d'una substpancia acídica en aigua tindríem, per exemple:
\[\ch{HSO4- + H2O <=> H3O+ + SO4^{2-}}\]
A partir d'aquesta expressió, podem escriure la constant d'equilibri, o constant de dissociació de l'àcid $K_a$, de la reacció com\marginnote{\label{foot:KaKb}Pots trobar dades de $K_a$ i $K_b$ a \linkurl{https://chem.libretexts.org/Reference/Reference_Tables/Equilibrium_Constants}}
\[K_a = \frac{[\ch{H3O+}][\ch{SO4^{2-}}]}{[\ch{HSO4-}]}\]

\item[Lewis] Finalment, també podem entendre el concepte d'àcid-base a partir de la definició de Lewis (1923). 
Segons aquesta definició, un àcid és qualsevol substpancia que pot acceptar electrons, i una base és tota substància que en pot donar.
Es tracta d'una definició més genèrica perquè no implica la presència de protons.
\end{description}

Per al que segueix usarem essencialment la definició de Lowry-Br{\o}nsted.

\subsection{L'escala de pH}

La reacció d'equilibri de la hidròlisi de l'aigua es pot escriure com
\[\ch{H20_{(l)} + H20_{(l)} <=> H3O+_{(aq)} + OH-_{(aq)}}\]
i té associada una constant d'equilibri $K_w$:
\[K_w=[\ch{H3O+}][\ch{OH-}]\]
amb un valor de 10$^{-14}$ a 25${\degree}$C si expressem la concentració dels dos ions en $M$.
En aigua pura, doncs, la concentració d'ions \ch{H3O+} i \ch{OH-} és de 10$^{-7}$ M, respectivament. 
Per tal de facilitar els càlculs treballem normalment en escala logarítmica i definim
\[pH = - \log_{10} [\ch{H3O+}]\]
Per tant, un valor de pH=7 implica que tenim una dissolució neutra pel que fa a la seva acidesa. Una concentració superior de protons (pH<7) implica una dissolució àcida i a l'inrevés.
 
\begin{exr}
Quin és el pH d'una dissolució de 0.1 M de clorur d'hidrogen? i d'una d'àcid benzoic a la mateixa concentració?
\end{exr}

És fàcil veure que $K_w=K_a K_b$, que ens diu també que, si l'àcid és fort, la seva base conjugada és feble, i a l'inrevés.

També és fàcilment deduïble l'equació de Henderson-Hasselbalch, que relaciona el pH d'una dissolució àcida amb el pK$_a$ i la concentració d'ions presents:
\[
pH = pK_a + \log_{10} \frac{[A^-]}{[AH]}
\]

\begin{exr}
Els productes de solubilitat de \ch{Fe(OH)3} i \ch{Zn(OH)2} són 4$\cdot$10$^{-38}$ i 4.5$\cdot$10$^{-17}$. A quin pH podem considerar que la precipitació de l'hidròxid de ferro és pràcticament completa mentre que l'ió \ch{Zn^{2+}} queda a una concentració de 0.5 M?
\end{exr}

%\subsection{Solucions reguladores i tampons}
%\subsection{Tractament exacte equuilibris ionització}
%\subsection{	Valoracions}
%

\section{Reaccions REDOX}

Quan observem la taula periòdica, podem apreciar que hi ha elements amb gran capacitat de donar electrons (metalls alcalins i alcalinoterris, per exemple) i anomenem electropositius. 
De la mateixa manera, anomenem electronegatius els elements que tenen gran capacitat d'acceptar electrons.

Definim l'estat d'oxidació d'un àtom com a la suma de càrregues positives i negatives que té. Això dóna idea de la càrrega total que conté.
En l'enllaç iònic que forma el \ch{NaCl}, l'estat d'oxidació del sodi és +1 i del clor -1.
En una molècula, usem els següents criteris per assignar els estats d'oxidació als diferents elements:
\begin{enumerate}
\item L'estat d'oxidació dels elements en qualsevol forma al·lotròpica en què  presentin és zero.
\item L'EO de l'oxigen és zero en tots els seus compostos, excepte en els peròxids (\ch{H2O2}, \ch{Na2O2}).
\item L'EO de l'hidrogen és +1 en tots els compostos, excepte en aquells que forma amb metalls, on és -1.
\item L'EO de la resta d'elements d'una substància s'escullen per tal que la suma de tots ells sigui zero o bé la càrrega que hagi de tenir l'ió que formen.
\end{enumerate}

Així com les reaccions àcid-base impliquen una transferència protònica (com a mínim en la definició de Lowry-Br{\o}nsted que hem estat usant) existeixen reaccions en les quals hi ha una transferència electrònica i que anomenem de Reducció-oxidació (REDOX). En totes aquestes reaccions en les quals participa el zinc hi ha el mateix procés d'oxidació (pèrdua d'electrons) d'aquest element:
\[\ch{Zn + Cu^{2+} <=> Zn^{2+} + Cu}\]
\[\ch{Zn + 1/2 O2 <=> ZnO}\]
\[\ch{Zn + Cl2 <=> ZnCl2}\]
\[\ch{Zn + 2 H^+_{(aq)} <=> Zn^{2+}_{(aq)} + H2}\]
En totes aquestes reaccions, el \ch{Zn} actua com a agent reductor, ja que amb la seva pròpia oxidació redueix l'altra substància. 

\begin{exr}
Són reaccions REDOX
\[\ch{ClO- + NO2- <=> NO3- + Cl-} \]
\[\ch{2 CCl4 + K2CrO4 <=> 2 Cl2CO + CrO2Cl2 + 2 KCl} \]?
\end{exr}


\subsection{Concepte de mitja reacció}

Pel fet que podem identificar, en una reacció REDOX, les substancies que es redueixen i les que s'oxiden, podem també separar la reacció global en els dos processos, ja que ens serà útil per comprendre que, de la mateixa manera que fem amb els elements de la reacció, també ens caldrà igualar el nombre d'electrons que s'intercanvien. Això també implica que una reacció REDOX es pot dividir físicament en dos compartiments i que els elecetrons es poden arribar a compartir amb un conductor, com s'aprecia a la Figura \ref{fig:Galvanic_cell_with_no_cation_flow}.

\begin{figure}[h]
\centering
\includegraphics[scale=0.5]{Galvanic_cell_with_no_cation_flow.png}
\caption{Una cel·la galvànica per a la reacció 
\ch{Zn_{(s)} + Cu^{2+}_{(aq)} <=> Cu_{(s)} + Zn^{(2+)}_{(aq)}}. 
La connexió es tanca mitjançant una membrana porosa als ions, però també es podria fer amb un pont salí (tub permeable que conté una dissolució d'alguna sal com \ch{KCl} (\linkurl{https://commons.wikimedia.org/wiki/File:Galvanic_cell_with_no_cation_flow.png})}
\label{fig:Galvanic_cell_with_no_cation_flow}
\end{figure}

En la reacció global representada a la figura hi ha dos processos simultanis, un a cada vas de reacció:
\[\ch{Zn <=> Zn^{(2+)}_{(aq)} + 2 e-}\]
\[\ch{2 e- + Cu^{2+}_{(aq)} <=> Cu_{(s)}}\]
El pont salí fa que es mantingui el balanç de càrregues potivies i negatives a cada vas.


\subsection{Balanç reaccions REDOX}

Separar les dues semireaccions d'una reacció REDOX ajuda a balancejar l'equació global (tenint en compte també els electrons que s'intercanvien) a més de permetre tenir mesures de la tendencia a oxidar/reduir de cada substància.
Per fer el balanç, seguim quatre passos:
\begin{enumerate}
\item Identifiquem les espècies que es redueixen o s'oxiden.
\item Escrivim les dues mitges reaccions.
\item Igualem les dues semireaccions en base als elements i les càrregues.
\item Les sumem per obtenir la reacció global.
\end{enumerate}

\begin{exr}
Iguala la reacció \ch{H2O2 + I- <=> I2 + H2O}. Pista: quan hagis d'afegir hidrogen, fes-ho en forma de protons \ch{H+}.
\end{exr}

\begin{exr}
Iguala la reacció entre en benzaldehid i l'ió \ch{Cr2O7^{2-}} per donar àcid benzoïc i ió \ch{Cr^{+3}}. Pista: on calguin oxigens, afegeix molècules d'aigua; on calguin hidrogens, afegeix protons.
\end{exr}

\begin{exr}
Iguala la reacció \ch{ClO- + CrO2- <=> CrO4- + Cl-} en una dissolució bàsica. Pista: fes com sempre però al final tingues en compte que els reactius han d'incorporar l'ió \ch{OH-}.
\end{exr}

\subsection{Cel·les galvàniques}

Tant en la cel·la galvànica de la Figura \ref{fig:Galvanic_cell_with_no_cation_flow} com en la bateria d'ió Liti de la Figura \ref{fig:LiIon3}, aprofitem el potencial REDOX de les substàncies per tal d'acumular energia química i transformar-la en elèctrica. 
Anomenarem càtode a l'enectrode on té lloc la reducció i ànode on té lloc l'oxidació.

\begin{figure}[h]
\centering
\includegraphics[scale=1]{LiIon3.png}
\caption{Bateria d'ió Liti \citep{liu_understanding_2016}.}
\label{fig:LiIon3}
\end{figure}

Podem definir el potencial estàndar d'una cel·la, $\Delta \varepsilon^{\circ}$, com al potencial pres en unes condicions determinades, que es fixen com a 1M per a tots els materials solubles, 1 atm per als gasos i, en el cas dels sòlids, la seva forma més estable a 25$^{\degree}$.
A partir del potencial podem calcular el treball elèctric fent
\[\Delta \varepsilon^{\circ} \times q = w_{elect}\]
Si la reacció és espontànea, el potencial $\Delta \varepsilon^{\circ}$ serà positiu.
Per tal de poder tabular els potencials de moltes substàncies, es va prendre la convenció d'assignar el potencial de 0 volt a la mitja reacció:
\[\ch{H2 (1 atm) <=> 2 H+ (1 M) + 2 e-}\]

\begin{exr}
La reacció que té lloc en una bateria d'ió liti com la de la imatge
\includegraphics[scale=0.5]{LiIon.png}
és \ch{LiC6 + CoO2 <=> LiCoO2 + C6}. Escriu les dues mitges reaccions i fes-hi el balanç. Calcula el potencial de cel·la a partir de la $\Delta \varepsilon^{\circ}$ del \ch{Li+} (-3.0V) i del \ch{CoO2} (+1.1V).
Quins valors obtindries per a la reacció que tindria lloc en una bateria de Li i \ch{O2} ($\Delta \varepsilon^{\circ}$ de la reacció \ch{O2_{(g)} + 2 H+ + 2 e- -> H2O2_{(aq)}} és 0.3V).
\end{exr}

\subsection{Equació de Nernst}

El voltatge real d'una cel·la depèn de la concentració.
A partir de la $\Delta \varepsilon^{\circ}$ podem verue com, per a una reacció del tipus 
\[\ch{a A + b B <=> c C + d D}\]
el voltatge de la cel·la es calcularà fent
\[\Delta \varepsilon=\Delta \varepsilon^{\circ}-\frac{0.059}{n} \log \frac{[C]^c[D]^d}{[A]^a[B]^b}\] 
És fàcil veure que, en l'equilibri, $\Delta \varepsilon=0$.

\begin{exr}
Troba constant d'equilibri de la reacció \ch{2 Fe^{2+} + 3 I-  <=> 2 Fe ^{(2+)} + I3-}.
\end{exr}

\begin{exr}
Quina és la concentració en equilibri de \ch{Fe^{2+}} si posem una barra de ferro en una dissolució 1 M d'ions \ch{Zn^{2+}}?
\end{exr}

%\subsection{Valoracions REDOX}
%\subsection{Electròlisi}
%\subsection{Aplicacions electroquímiques}

% \chapter{Química Inorgànica}

\section{Els elements metàl·lics i les seves propietats}

\section{Els elements no metàl·lics i les seves propietats}
\section{Metalls de transició}
\section{Estat Sòlid}

%\input{Termoquímica}
%\input{Cinètica Química}

%QuimInorg

% \input{Els elements metàl·lics i les seves propietats}
% \input{Els elements no metàl·lics i les seves propietats}
% \input{Estat Sòlid}

%QuimOrg

% \input{Els compostos orgànics i les seves propietats}
% \input{Estructura-activitat en química orgànica}

%QuimAn

% \input{Química àcid-base}
% \input{Instrumentació en química analítica}

%QuimSup

% \input{Col·loides}
% \input{Ciència de superfícies}

%QuimComb

% \input{Catàlisi}
% \input{Química del petroli i el gas}
% \input{Polímers}

%QuimMat

% \input{Tipus de materials}
% \input{Introducció a l'ús de materials en tecnologies avançades}

\newpage
\begin{appendix}
\section{Enllaços d'interès}
\label{sec:EnllacosInteres}

Referències generals d'interès:
\begin{description}
\item[LibreTexts] Recull de continguts oberts en química \linkurl{https://chem.libretexts.org/}
\end{description}

Institucions i fons d'informació primàries:
\begin{description}
\item[IUPAC] International Union of Pure and Applied Chemistry (IUPAC): \linkurl{https://iupac.org/}.
\item[NIST] National Institute of Standards and Technology US Dept. of Commerce \linkurl{https://webbook.nist.gov/chemistry/}. Inclou bases de dades.
\item[ChemSpider] Recurs genèric de cerca de compostos i les seves propietats \linkurl{http://www.chemspider.com/Default.aspx}.
\end{description}

Bases de dades:
\begin{description}
\item[PubChem] \linkurl{https://pubchem.ncbi.nlm.nih.gov/}. Cerca de compostos.
\item[chemexper] \linkurl{https://www.chemexper.com/}. Cerca de compostos. 
\item[NMRShiftDB] \linkurl{http://nmrshiftdb.nmr.uni-koeln.de/}. Espectres de NMR de molècules d'interès.
\item[NIST Chemical Kinetics Data] \linkurl{https://kinetics.nist.gov/kinetics/index.jsp}.
\item[CCCBDB] Computational Chemistry Comparison and Benchmark Database, NIST, \linkurl{https://cccbdb.nist.gov/}. Inclou dades termodinàmiques verificades dels compostos més comuns.
\item[IUPAC-NIST Solubility DB] \linkurl{https://srdata.nist.gov/solubility/}
\item[Spectral Database for Organic Compounds SDBS] \linkurl{http://sdbs.db.aist.go.jp}.
\item[CommonChemistry] Cerca simple de molècules o del seu \textit{CAS number} (\linkurl{http://commonchemistry.org/}).
\item[Symmetry @ Otterbein] Interessant pàgina dedicada a la simetria química \linkurl{http://symmetry.otterbein.edu/}.
\item[MatWeb] Informació sobre propietats de materials \linkurl{http://www.matweb.com/index.aspx}.
\end{description}

  \listoffigures
  \listoftables
  %\printbibliography
\end{appendix}

% The bibliography style.
\bibliographystyle{apalike}

% If you DO want the full list of references to be printed at the end.
\bibliography{QuimAutom}

% If you do NOT want the full list of references to be printed at the end.
%\nobibliography{references}

\end{document}