\begin{lstlisting}[language=Matlab, caption={Codi Matlab per dibuixar una distribució de Maxwell-Boltzmann}]
    clc; clear; close all;
    
    % Definim constants
    kB = 1.38e-23;  % Constant de Boltzmann (J/K)
    T = 300;        % Temperatura en Kelvin
    m = 4.65e-26;   % Massa de la molècula (kg) (exemple: molècula de nitrogen)
    
    % Definim el rang de velocitats
    v = linspace(0, 2000, 1000);  % Rang de velocitats (m/s)
    
    % Funció de distribució de Maxwell-Boltzmann
    f_v = ( (m / (2 * pi * kB * T))^(3/2) ) * 4 * pi * v.^2 .* exp(-m * v.^2 / (2 * kB * T));
    
    % Representació gràfica de la distribució
    figure;
    plot(v, f_v, 'b', 'LineWidth', 2);
    xlabel('Velocitat (m/s)');
    ylabel('Densitat de probabilitat f(v)');
    title('Distribució de Maxwell-Boltzmann');
    grid on;
    
    % Càlcul de la velocitat més probable, la velocitat mitjana i la velocitat quadràtica mitjana
    v_mp = sqrt(2 * kB * T / m);  % Velocitat més probable
    v_mitjana = sqrt(8 * kB * T / (pi * m));  % Velocitat mitjana
    v_rms = sqrt(3 * kB * T / m);  % Velocitat quadràtica mitjana
    
    hold on;
    xline(v_mp, '--r', 'Velocitat més probable', 'LabelHorizontalAlignment', 'right');
    xline(v_mitjana, '--g', 'Velocitat mitjana', 'LabelHorizontalAlignment', 'right');
    xline(v_rms, '--m', 'Velocitat RMS', 'LabelHorizontalAlignment', 'right');
    legend('Distribució de Maxwell-Boltzmann', 'Velocitat més probable', 'Velocitat mitjana', 'Velocitat RMS');
    hold off;
    \end{lstlisting}