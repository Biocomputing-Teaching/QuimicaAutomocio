\usepackage{chemfig,chemmacros,chemnum}
\chemsetup[reactions]{
    before-tag = R,
    tag-open = [ , tag-close = ]
}
\usepackage{chemformula}
\definecolor{mygreen}{RGB}{28,172,0} % color values Red, Green, Blue
\definecolor{mylilas}{RGB}{170,55,241}
\definecolor{LightOcean}{RGB}{81, 147, 229 }
\definecolor{DeepOcean}{RGB}{51, 131, 229}
\newtcolorbox{mybox}[1][]{%
    %float, 
    %floatplacement=t,
    breakable,
    %enhanced, 
    colback=LightOcean!10, 
    colframe=DeepOcean,
    % overlay unbroken and first={%
    % \ifoddpage\coordinate (X) at ([xshift=-6mm,yshift=-6mm]frame.north east);
    %      \else\coordinate (X) at ([xshift=6mm,yshift=-6mm]frame.north west);\fi
    % \node at (X) {\includegraphics[width=8mm]{FCTE}};}
%   show bounding box,
    %notitle,
    if odd page={grow to right by=\marginparsep+\marginparwidth-15mm}{grow to left by=\marginparsep+\marginparwidth-15mm},
    toggle enlargement=evenpage,
    #1
}
\newcounter{myc}
%%%%%%%%%%%%%%%%%%%%%%%%%%%%%%%%%%%%%%%%%
%%%%%%%%%%%%%%%%%%%%%%%%%%%%%%%%%%%%%%%%%
% lecturer or student text
% in principle the lecturer text includes some examples to be done in the c lass
\newboolean{LECT}
\setboolean{LECT}{false}
%%%%%%%%%%%%%%%%%%%%%%%%%%%%%%%%%%%%%%%%%
%%%%%%%%%%%%%%%%%%%%%%%%%%%%%%%%%%%%%%%%%

\newenvironment{lect}{ % 
	\definecolor{shadecolor}{rgb}{1.0,0.8,0.8} %
	\begin{shaded} %
	\textcolor{BrickRed}{\bf Resposta\\}%

} % 
{ %	
	\end{shaded}
} %

\newcommand{\lct}[1]{\ifthenelse{\boolean{LECT}}{\begin{lect} #1 \end{lect}}{}}


%environment for questions in exams
\newenvironment{qst}{ % 
    \addtocounter{myc}{1}
	\definecolor{shadecolor}{rgb}{0.9,1.0,0.8} %
	\begin{shaded} %
	\textcolor{OliveGreen}{\bf Qüestió \arabic{myc}\\}%
} % 
{ %	
	\end{shaded}
} %

%environment for exercises in the class notes
\newenvironment{exr}{ % 
    \addtocounter{myc}{1}
	\definecolor{shadecolor}{rgb}{1.0,1.0,1.0} %
	\begin{shaded} %
	{\bf Exercici \arabic{myc}.}%
 } % 
 { %	
 	\end{shaded}
} %

% Definició d'unitats personalitzades per al sistema imperial
\DeclareSIUnit\inch{in}
\DeclareSIUnit\foot{ft}
\DeclareSIUnit\atm{atm}
\DeclareSIUnit\oz{oz}
\DeclareSIUnit\ounce{oz}
\DeclareSIUnit\pound{lb}
\DeclareSIUnit\ton{t}

\DeclareSIUnit\cubicinch{\inch\cubed}
\DeclareSIUnit\cubicfoot{\foot\cubed}
\DeclareSIUnit\gallon{gal}
\DeclareSIUnit\psi{psi}
\DeclareSIUnit\inchHg{inHg}
\DeclareSIUnit\dyn{dynes}
\DeclareSIUnit\torr{Torr}
\DeclareSIUnit\Torr{Torr}

\DeclareSIUnit{\degreeFahrenheit}{\unit{\degree}F}
\newcommand{\degC}{\degreeCelsius}
\newcommand{\degF}{\degreeFahrenheit}


\usepackage{tikz}
\usetikzlibrary{patterns,decorations.pathmorphing}
\usetikzlibrary{arrows.meta}
\tikzset{>=latex}

\colorlet{mydarkblue}{blue!50!black}
\colorlet{myblue}{blue!30}
\colorlet{mydarkred}{red!60!black}
\colorlet{myred}{red!30}
\colorlet{mydarkgreen}{green!60!black}
\colorlet{mygreen}{green!30}
\colorlet{mydarkorange}{yellow!40!red}
\colorlet{myorange}{yellow!80!red}
\colorlet{myyellow}{yellow!80}
\colorlet{mygrey}{black!15}
\colorlet{mydarkgrey}{black!50}

\tikzstyle{piston}=[blue!40!black,top color=blue!40!black!30,bottom color=blue!40!black!30,
                    middle color=blue!50!black!15,shading angle=0]
\tikzstyle{walldark}=[blue!25!black,top color=blue!20!black!12,bottom color=blue!20!black!20,shading angle=-30]
\tikzstyle{wall}=[blue!20!black,top color=blue!35!black!6,bottom color=blue!25!black!12,shading angle=30]

% GAS MOLECULE with vector
\tikzset{
  gasparticle/.pic={
    \tikzset{/gasparticle/.cd,#1}
    \draw[-{Latex[length=4,width=3]},green!60!black,thick] (0,0) -- (\vec);
    \node[circle,fill,inner sep=1.5,ball color=black!80!blue] at (0,0) {};
  }
  /gasparticle/.search also={/tikz},
  /gasparticle/.cd,
  vec/.store in=\vec, vec={90:0.5},
}
