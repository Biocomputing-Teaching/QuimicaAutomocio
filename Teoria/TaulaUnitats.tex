\documentclass{report} 
\usepackage[utf8]{inputenc}
\usepackage[catalan]{babel}
\usepackage{siunitx}
\usepackage{chemformula}
\usepackage{biblatex}
\addbibresource{QuimAutom.bib}
% Definició d'unitats personalitzades per al sistema imperial
\DeclareSIUnit\inch{in}
\DeclareSIUnit\foot{ft}
\DeclareSIUnit\atm{atm}
\DeclareSIUnit\oz{oz}
\DeclareSIUnit\ounce{oz}
\DeclareSIUnit\pound{lb}
\DeclareSIUnit\ton{t}
\DeclareSIUnit\fah{F}
\DeclareSIUnit\cubicinch{\inch\cubed}
\DeclareSIUnit\cubicfoot{\foot\cubed}
\DeclareSIUnit\gallon{gal}
\DeclareSIUnit\psi{psi}
\DeclareSIUnit\inchHg{inHg}
\DeclareSIUnit\dyn{dynes}
\usepackage{booktabs}
\usepackage{longtable}

\begin{document}

\section*{Unitats de mesura}


\begin{longtable}{llllll}
\toprule
\bfseries Magnitud & \bfseries Unitat a SI &
\bfseries Símbol SI & \bfseries Unitat a CGS & \bfseries Símbol CGS &\bfseries Dimensió\\\midrule\endhead
Longitud & metre & \si{\meter} & centímetre& \si{\centi\meter}& \\
Volum & litre & \si\litre &&&\\
Massa & kilogram & \si{\kilo\gram}& gram& \si\gram&\\
Temperatura & kelvin & \si{\kelvin} &&& \\
mol & mol &\si{\mole}&&&\\
temps & segon & \si\second & segon & \si\second &\\
Freqüència & hertz & \si{\hertz} &&& \si{\per\second}\\
Inductància  & henry & \si\henry&&& \\
Energia & joule & \si\joule &&& \\
Força & newton &  \si\newton & dines&\si\dyn&\\
Pressió & pascal & \si\pascal &&&\\
Potencial elèctric & volt & \si\volt &&&\\
Potència & watt & \si\watt &&&\\
\bottomrule
\caption{Algunes unitats del SI rellevants per a aquest curs}
\label{tab:unitatsSI}
\end{longtable}

\begin{longtable}{ccc}
    \toprule
    \textbf{Magnitud} & \textbf{Unitat (EUA)} & \textbf{Equivalència en SI} \\
    \midrule\endhead
    Volum & \SI{1}{\cubic\inch} & \SI{16.387}{\cubic\centi\meter} \\
    Volum & \SI{1}{\cubic\foot} & \SI{28.317}{\liter} \\
    Volum & \SI{1}{\gallon} (US) & \SI{3.785}{\liter} \\
    \hline
    Pressió & \SI{1}{\psi} & \SI{6.895}{\kilo\pascal} \\
    Pressió & \SI{1}{\atm} & \SI{101.325}{\kilo\pascal} \\
    Pressió & \SI{1}{\inchHg} & \SI{3.386}{\kilo\pascal} \\
    \hline
    Temperatura & \SI{1}{\fah} & $T_C=(T_{F} - 32) \times \frac{5}{9} $ \\
    \hline
    Massa & \SI{1}{\ounce} & \SI{28.35}{\gram} \\
    Massa & \SI{1}{\pound} & \SI{0.4536}{\kilo\gram} \\
    Massa & \SI{1}{\ton} (US) & \SI{907.184}{\kilo\gram} \\
    \bottomrule
    \caption{Conversió d'unitats del sistema americà al SI}
    \label{tab:conversio}
\end{longtable}
\newpage
    \begin{longtable}{cc}
    \hline
    \textbf{Unitat de Pressió} & \textbf{Pressió (en relació a 1 atm)} \\ \midrule\endhead
    Atmosfera (atm) & 1 atm \\ 
    Pascal (Pa) & \( 101325 \, \text{Pa} \) \\ 
    Bar & \( 1.01325 \, \text{bar} \) \\ 
    Mil·límetre de mercuri (mmHg) & \( 760 \, \text{mmHg} \) \\ 
    Torra (Torr) & \( 760 \, \text{Torr} \) \\ 
    Pounds per square inch (psi) & \( 14.696 \, \text{psi} \) \\ 
    Kilopascal (kPa) & \( 101.325 \, \text{kPa} \) \\    \bottomrule
    \caption{Comparació de les unitats de pressió amb 1 atmosfera}
    \end{longtable}

    


    \begin{longtable}{cc}
        \hline
        \textbf{Valor de la constant dels gasos R} & \textbf{Unitats} \\
        \midrule\endhead
        \num{0.082} & \si{\atm\litre\per\mole\per\kelvin} \\
        \num{8.3145} & \si{m^3.Pa.K^{-1}.mol^{-1}} \\
        \num{8.3145} & \si{\joule\per\kelvin\per\mole} \\
        \num{62.363} & \si{L.Torr.K^{-1}.mol^{-1}} \\
        \num{1.9872e-3} & \si{kcal.K^{-1}.mol^{-1}} \\
        \num{8.205e-5} & \si{m^3.atm.K^{-1}.mol^{-1}} \\
        \bottomrule
        \caption{Conversió de la constant dels gasos en diferents unitats}
        \label{tab:gas_constant}
    \end{longtable}

\newpage
\section*{Dades termodinàmiques}
\begin{longtable}{lcccc}
    \toprule
    \textbf{Substància} & \multicolumn{2}{c}{\textbf{Calor de Fusió}} & \multicolumn{2}{c}{\textbf{Calor de Vaporització}} \\
    & $\Delta H_\text{fus}$ (J/g) & $\Delta H_\text{fus}$ (kJ/mol) & $\Delta H_\text{vap}$ (J/g) & $\Delta H_\text{vap}$ (kJ/mol) \\
    \midrule\endhead
    Alumini & 321 & 8.66 & 11400 & 307.6 \\
    Benzè & 127.4 & 10.0 & 390 & 30.5 \\
    Coure & 207 & 13.2 & 5069 & 322.1 \\
    Or & 67 & 13.2 & 1578 & 310.9 \\
    Ferro & 209 & 11.7 & 6340 & 354.1 \\
    Plom & 22.4 & 4.64 & 871 & 180.5 \\
    Metà & 59 & 0.946 & 537 & 8.61 \\
    Mercuri & 11.6 & 2.33 & 295 & 5.92 \\
    Metanol & 98.8 & 3.17 & 1100 & 35.2 \\
    Nitrogen & 25.5 & 0.715 & 200 & 5.60 \\
    Sodi & 113 & 2.60 & 4237 & 97.42 \\
    Aigua & 334 & 6.02 & 2260 & 40.7 \\
    \bottomrule
    \caption{Calor de Fusió i Vaporització d'algunes substàncies pures (específic $\Delta H$ en J/g i Molar $\Delta H$ en kJ/mol)}
\end{longtable}

\subsection*{Valors Clau de Termodinàmica}
La taula següent mostra els valors clau de termodinàmica per a diverses substàncies, extrets de la taula "CODATA KEY VALUES FOR THERMODYNAMICS" a \cite{lide_crc_2005}.
La taula inclou l'entalpia estàndard de formació a \qty{298.15}{\kelvin}, l'entropia a \qty{298.15}{\kelvin} i la quantitat \(H^\circ\) (\qty{298.15}{\kelvin})-\(H^\circ\) (\qty{0}{\kelvin}). Un valor de 0 a la columna \(\Delta_f H^\circ\) per a un element indica l'estat de referència per a aquest element. La pressió de l'estat estàndard és \qty{100000}{\pascal} (1 bar).


\begin{longtable}{lccc}
    \toprule
    \textbf{Substància} & \(\Delta_f H^\circ\) (298.15 K) & \(S^\circ\) (298.15 K) & \(H^\circ\) (298.15 K)–\(H^\circ\) (0) \\
    & \text{(kJ/mol)} & \text{(J/K/mol)} & \text{(kJ/mol)} \\
    \midrule\endhead
    \ch{Ar} (g) & 0 & 154.846 $\pm$ 0.003 & 6.197 $\pm$ 0.001 \\
    \ch{C} (cr, graphite) & 0 & 5.74 $\pm$ 0.10 & 1.050 $\pm$ 0.020 \\
    \ch{C} (g) & 716.68 $\pm$ 0.45 & 158.100 $\pm$ 0.003 & 6.536 $\pm$ 0.001 \\
    \ch{CO} (g) & -110.53 $\pm$ 0.17 & 197.660 $\pm$ 0.004 & 8.671 $\pm$ 0.001 \\
    \ch{CO2} (aq, undissoc.) & -413.26 $\pm$ 0.20 & 119.36 $\pm$ 0.60 & \\
    \ch{CO2} (g) & -393.51 $\pm$ 0.13 & 213.785 $\pm$ 0.010 & 9.365 $\pm$ 0.003 \\
    \ch{CO3^{2-}} (aq) & -675.23 $\pm$ 0.25 & -50.0 $\pm$ 1.0 & \\
    \ch{H2} (g) & 0 & 130.680 $\pm$ 0.003 & 8.468 $\pm$ 0.001 \\
    \ch{H2O} (g) & -241.826 $\pm$ 0.040 & 188.835 $\pm$ 0.010 & 9.905 $\pm$ 0.005 \\
    \ch{H2O} (l) & -285.830 $\pm$ 0.040 & 69.95 $\pm$ 0.03 & 13.273 $\pm$ 0.020 \\
    \ch{H2PO4^{-}} (aq) & -1302.6 $\pm$ 1.5 & 92.5 $\pm$ 1.5 & \\
    \ch{H2S} (aq, undissoc.) & -38.6 $\pm$ 1.5 & 126 $\pm$ 5 & \\
    \ch{H2S} (g) & -20.6 $\pm$ 0.5 & 205.81 $\pm$ 0.05 & 9.957 $\pm$ 0.010 \\
    \ch{N} (g) & 472.68 $\pm$ 0.40 & 153.301 $\pm$ 0.003 & 6.197 $\pm$ 0.001 \\
    \ch{NH3} (g) & -45.94 $\pm$ 0.35 & 192.77 $\pm$ 0.05 & 10.043 $\pm$ 0.010 \\
    \ch{NH4^{+}} (aq) & -133.26 $\pm$ 0.25 & 111.17 $\pm$ 0.40 & \\
    \ch{NO3^{-}} (aq) & -206.85 $\pm$ 0.40 & 146.70 $\pm$ 0.40 & \\
    \ch{N2} (g) & 0 & 191.609 $\pm$ 0.004 & 8.670 $\pm$ 0.001 \\
    \ch{S} (g) & 277.17 $\pm$ 0.15 & 167.829 $\pm$ 0.006 & 6.657 $\pm$ 0.001 \\
    \ch{SO2} (g) & -296.81 $\pm$ 0.20 & 248.223 $\pm$ 0.050 & 10.549 $\pm$ 0.010 \\
    \ch{SO4^{2-}} (aq) & -909.34 $\pm$ 0.40 & 18.50 $\pm$ 0.40 & \\
    \ch{C3H8} (g) & -104.7 $\pm$ 0.4 & 269.91 $\pm$ 0.10 & 14.66 $\pm$ 0.05 \\
    \ch{H2} (g) & 0 & 130.680 $\pm$ 0.003 & 8.468 $\pm$ 0.001 \\
    \ch{H2O} (g) & -241.826 $\pm$ 0.040 & 188.835 $\pm$ 0.010 & 9.905 $\pm$ 0.005 \\
    \ch{H2O} (l) & -285.830 $\pm$ 0.040 & 69.95 $\pm$ 0.03 & 13.273 $\pm$ 0.020 \\
    \ch{H2PO4^{-}} (aq) & -1302.6 $\pm$ 1.5 & 92.5 $\pm$ 1.5 & \\
    \ch{H2S} (aq, undissoc.) & -38.6 $\pm$ 1.5 & 126 $\pm$ 5 & \\
    \ch{H2S} (g) & -20.6 $\pm$ 0.5 & 205.81 $\pm$ 0.05 & 9.957 $\pm$ 0.010 \\
    \ch{N} (g) & 472.68 $\pm$ 0.40 & 153.301 $\pm$ 0.003 & 6.197 $\pm$ 0.001 \\
    \ch{NH3} (g) & -45.94 $\pm$ 0.35 & 192.77 $\pm$ 0.05 & 10.043 $\pm$ 0.010 \\
    \ch{NH4^{+}} (aq) & -133.26 $\pm$ 0.25 & 111.17 $\pm$ 0.40 & \\
    \ch{NO3^{-}} (aq) & -206.85 $\pm$ 0.40 & 146.70 $\pm$ 0.40 & \\
    \ch{N2} (g) & 0 & 191.609 $\pm$ 0.004 & 8.670 $\pm$ 0.001 \\
    \ch{S} (g) & 277.17 $\pm$ 0.15 & 167.829 $\pm$ 0.006 & 6.657 $\pm$ 0.001 \\
    \ch{SO2} (g) & -296.81 $\pm$ 0.20 & 248.223 $\pm$ 0.050 & 10.549 $\pm$ 0.010 \\
    \ch{SO4^{2-}} (aq) & -909.34 $\pm$ 0.40 & 18.50 $\pm$ 0.40 & \\
    \bottomrule
    \caption{Valors clau de termodinàmica per a diverses substàncies \cite{wagman_codata_1989}}
\end{longtable}

\subsection*{Calor de Combustió}

El calor de combustió d'una substància a 25°C es pot calcular a partir de les dades d'entalpia de formació (\(\Delta_f H^\circ\)). Podem escriure la reacció general de combustió com:
\[ \ch{X + O2 -> CO2 (g) + H2O (l) + Y} \]

Per a un compost que conté només carboni, hidrogen i oxigen, la reacció és simplement:

\[
\ch{C_{a}H_{b}O_{c} + $\left({a}+\frac{b}{4}-\frac{c}{2}\right)$ O2 -> a CO2 (g) + $\frac{b}{2}$ H2O (l)}
\]
i la calor estàndard de combustió \(\Delta_c H^\circ\), que es defineix com el negatiu del canvi d'entalpia per a la reacció (és a dir, el calor alliberat en el procés de combustió), es dóna per:
\[ \Delta_c H^\circ = -a \Delta_f H^\circ (CO_2, g) - \frac{b}{2} \Delta_f H^\circ (H_2O, l) + \Delta_f H^\circ (\text{C}_a \text{H}_b \text{O}_c) \]
\[ = 393.51a + 142.915b + \Delta_f H^\circ (\text{C}_a \text{H}_b \text{O}_c) \]

Aquesta equació s'aplica si els reactius comencen en els seus estats estàndard (25°C i una atmosfera de pressió) i els productes tornen a les mateixes condicions. La mateixa equació s'aplica a un compost que conté un altre element si aquest element acaba en el seu estat de referència estàndard (per exemple, nitrogen, si el producte és \ch{N2}); en general, però, els productes exactes que contenen els altres elements han de ser coneguts per calcular el calor de combustió.

La taula següent dóna la calor estàndard de combustió calculat d'aquesta manera per a algunes substàncies representatives (adaptat de la taula "Heat of Combustion" a \cite{lide_crc_2005}).

\begin{longtable}{lcc}
    \toprule
    \textbf{Fórmula Molecular} & \textbf{Nom} & \(\Delta_c H^\circ\) (kJ/mol) \\
    \midrule\endhead
    \ch{C3H8O} & 1-Propanol (l) & 2021.3 \\
    \ch{C3H8O3} & Glicerol (l) & 1655.4 \\
    \ch{C4H10O} & Èter dietílic (l) & 2723.9 \\
    \ch{C5H12O} & 1-Pentanol (l) & 3330.9 \\
    \ch{C6H6} & Fenol (s) & 3053.5 \\
    \midrule
    \textbf{Substàncies Inorgàniques} & & \\
    \ch{C} & Carboni (grafit) & 393.5 \\
    \ch{CO} & Monòxid de carboni (g) & 283.0 \\
    \ch{H2} & Hidrogen (g) & 285.8 \\
    \ch{H3N} & Amoníac (g) & 382.8 \\
    \ch{H4N2} & Hidrazina (g) & 667.1 \\
    \ch{N2O} & Òxid nitrós (g) & 82.1 \\
    \midrule
    \textbf{Compostos de Carbonil} & & \\
    \ch{CH2O} & Formaldehid (g) & 726.1 \\
    \ch{C2H2O} & Cetè (g) & 1366.8 \\
    \ch{C2H4O} & Acetaldehid (l) & 1460.4 \\
    \ch{C3H6O} & Acetona (l) & 1189.2 \\
    \ch{C3H6O} & Propanal (l) & 1822.7 \\
    \ch{C4H8O} & 2-Butanona (l) & 2444.1 \\
    \midrule
    \textbf{Hidrocarburs} & & \\
    \ch{CH4} & Metà (g) & 890.8 \\
    \ch{C2H2} & Acetilè (g) & 1301.1 \\
    \ch{C2H4} & Etilè (g) & 1411.2 \\
    \ch{C2H6} & Età (g) & 1560.7 \\
    \ch{C3H6} & Propilè (g) & 2058.0 \\
    \ch{C3H6} & Ciclopropà (g) & 2091.3 \\
    \ch{C3H8} & Propà (g) & 2219.2 \\
    \ch{C4H6} & 1,3-Butadiè (g) & 2541.5 \\
    \ch{C4H10} & Butà (g) & 2877.6 \\
    \ch{C5H12} & Pentà (l) & 3509.0 \\
    \ch{C6H6} & Benzè (l) & 3267.6 \\
    \ch{C6H12} & Ciclohexà (l) & 3919.6 \\
    \ch{C6H14} & Hexà (l) & 4163.2 \\
    \ch{C7H8} & Toluè (l) & 3910.3 \\
    \ch{C7H16} & Heptà (l) & 4817.0 \\
    \ch{C10H8} & Naftalè (s) & 5156.3 \\
    \midrule
    \textbf{Alcohols i Èters} & & \\
    \ch{CH4O} & Metanol (l) & 570.7 \\
    \ch{C2H6O} & Etanol (l) & 1025.4 \\
    \ch{C2H6O} & Èter dimetílic (g) & 1166.9 \\
    \ch{C2H6O2} & Etilè glicol (l) & 1789.9 \\
    \midrule
    \textbf{Àcids i Èsters} & & \\
    \ch{CH2O2} & Àcid fòrmic (l) & 254.6 \\
    \ch{C2H4O2} & Àcid acètic (l) & 874.2 \\
    \ch{C2H4O2} & Formiat de metil (l) & 972.6 \\
    \ch{C3H6O2} & Acetat de metil (l) & 1592.2 \\
    \ch{C4H8O2} & Acetat d'etil (l) & 2238.1 \\
    \ch{C7H6O2} & Àcid benzoic (s) & 3226.9 \\
    \midrule
    \textbf{Compostos de Nitrogen} & & \\
    \ch{CHN} & Cianur d'hidrogen (g) & 671.5 \\
    \ch{CH3NO2} & Nitrometà (l) & 709.2 \\
    \ch{CH5N} & Metilamina (g) & 1085.6 \\
    \ch{C2H3N} & Acetonitril (l) & 1247.2 \\
    \ch{C2H5NO} & Acetamida (s) & 1184.6 \\
    \ch{C3H9N} & Trimetilamina (g) & 2443.1 \\
    \ch{C5H5N} & Piridina (l) & 2782.3 \\
    \ch{C6H7N} & Anilina (l) & 3392.8 \\
    \bottomrule
    \caption{Calor estàndard de combustió de diverses substàncies. Adaptat de la taula "Heat of Combustion" a \cite{lide_crc_2005}}
\end{longtable}

\printbibliography
\end{document}