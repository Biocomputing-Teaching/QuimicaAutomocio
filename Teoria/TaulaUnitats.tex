\documentclass{article} 
\usepackage[utf8]{inputenc}
\usepackage[catalan]{babel}
\usepackage{siunitx}
\usepackage{chemformula}
\usepackage{biblatex}
\usepackage{url}
  \let\oldurl\url
\usepackage{hyperref}
  \let\linkurl\url
  \let\url\oldurl

\usepackage{fancyhdr}
  \pagestyle{fancy}   
  
\usepackage{graphicx}
  \graphicspath{{../figures/}}%

% caixes de text
\usepackage{tcolorbox}
  \definecolor{mygreen}{RGB}{28,172,0} % color values Red, Green, Blue
  \definecolor{mylilas}{RGB}{170,55,241}
  \definecolor{LightOcean}{RGB}{81, 147, 229 }
  \definecolor{DeepOcean}{RGB}{51, 131, 229}
  \newtcolorbox{mybox}[1][]{% 
    %float, 
    %floatplacement=t,
    breakable,
    %enhanced, 
    colback=LightOcean!10, 
    colframe=DeepOcean,
    % overlay unbroken and first={%
    % \ifoddpage\coordinate (X) at ([xshift=-6mm,yshift=-6mm]frame.north east);
    %      \else\coordinate (X) at ([xshift=6mm,yshift=-6mm]frame.north west);\fi
    % \node at (X) {\includegraphics[width=8mm]{FCTE}};}
%   show bounding box,
    %notitle,
    if odd page={grow to right by=\marginparsep+\marginparwidth-15mm}{grow to left by=\marginparsep+\marginparwidth-15mm},
    toggle enlargement=evenpage,
    #1
}
  \newcounter{myc}
%%%%%%%%%%%%%%%%%%%%%%%%%%%%%%%%%%%%%%%%%
%%%%%%%%%%%%%%%%%%%%%%%%%%%%%%%%%%%%%%%%%
% lecturer or student text
% in principle the lecturer text includes some examples to be done in the c lass
\newboolean{LECT}
\setboolean{LECT}{false}
%%%%%%%%%%%%%%%%%%%%%%%%%%%%%%%%%%%%%%%%%
%%%%%%%%%%%%%%%%%%%%%%%%%%%%%%%%%%%%%%%%%

\newenvironment{lect}{ % 
	\definecolor{shadecolor}{rgb}{1.0,0.8,0.8} %
	\begin{shaded} %
	%\textcolor{BrickRed}
  {\bf  Resposta\\}%
} % 
{ %	
	\end{shaded}
} %

\newcommand{\lct}[1]{\ifthenelse{\boolean{LECT}}{\begin{lect} #1 \end{lect}}{}}


%environment for questions in exams
\newenvironment{qst}{ % 
    \addtocounter{myc}{1}
	\definecolor{shadecolor}{rgb}{0.9,1.0,0.8} %
	\begin{shaded} %
	\textcolor{OliveGreen}{\bf Qüestió \arabic{myc}\\}%
} % 
{ %	
	\end{shaded}
} %

%environment for exercises in the class notes
\newenvironment{exr}[1]{ % 
    \addtocounter{myc}{1}
	\definecolor{shadecolor}{rgb}{1.0,1.0,1.0} %
	\begin{shaded} %
	{\bf Exercici \arabic{myc} ({\bf #1}). }%
 } % 
 { %	
 	\end{shaded}
} %

\newcounter{EXMP}
\newenvironment{EXMP}[1][]{\definecolor{shadecolor}{rgb}{0.9,0.9,0.9}
							\begin{shaded}\refstepcounter{EXMP}\par\medskip\noindent%
							\textbf{EXEMPLE~\theEXMP. #1\\} \rmfamily}
							{\end{shaded}\medskip}

\usepackage{tikz}
\usetikzlibrary{patterns,decorations.pathmorphing}
\usetikzlibrary{decorations.markings}
\usetikzlibrary{calc,shadings,shapes.symbols}
\usetikzlibrary{arrows.meta}
\tikzset{>=latex}

\colorlet{mydarkblue}{blue!50!black}
\colorlet{myblue}{blue!30}
\colorlet{mydarkred}{red!60!black}
\colorlet{myred}{red!30}
\colorlet{mydarkgreen}{green!60!black}
\colorlet{mygreen}{green!30}
\colorlet{mydarkorange}{yellow!40!red}
\colorlet{myorange}{yellow!80!red}
\colorlet{myyellow}{yellow!80}
\colorlet{mygrey}{black!15}
\colorlet{mydarkgrey}{black!50}

\tikzstyle{piston}=[blue!40!black,top color=blue!40!black!30,bottom color=blue!40!black!30,
                    middle color=blue!50!black!15,shading angle=0]
\tikzstyle{walldark}=[blue!25!black,top color=blue!20!black!12,bottom color=blue!20!black!20,shading angle=-30]
\tikzstyle{wall}=[blue!20!black,top color=blue!35!black!6,bottom color=blue!25!black!12,shading angle=30]

% GAS MOLECULE with vector
\tikzset{
  gasparticle/.pic={
    \tikzset{/gasparticle/.cd,#1}
    \draw[-{Latex[length=4,width=3]},green!60!black,thick] (0,0) -- (\vec);
    \node[circle,fill,inner sep=1.5,ball color=black!80!blue] at (0,0) {};
  }
  /gasparticle/.search also={/tikz},
  /gasparticle/.cd,
  vec/.store in=\vec, vec={90:0.5},
}

\usepackage{pgfplots}
\pgfplotsset{compat=1.18}


% diferencials

\newcommand{\diff}{\mathop{}\!\mathrm{d}}
% http://tex.stackexchange.com/a/253108/4427
\newcommand{\dbar}{{\mkern2mu\mathchar'26\mkern-11mu \mathrm{d}}}
\newcommand{\idiff}{\mathop{}\!\dbar}
\usepackage{chemfig,chemmacros,chemnum}
\chemsetup[reactions]{
    before-tag = R,
    tag-open = [ , tag-close = ]
}
\renewcommand*\printatom[1]{\ensuremath{\mathsf{#1}}}
\usepackage{chemformula}
\addbibresource{../common/QuimAutom.bib}
% Definició d'unitats personalitzades per al sistema imperial

\usepackage{booktabs}
\usepackage{longtable}
\fancyhead[L]{GEA 2024-2025}
\fancyhead[R]{Formulari i Taules de Química General}
\fancyfoot[L]{\includegraphics[width=0.2\textwidth]{FCTE}}
\fancyfoot[C]{\today}
\fancyfoot[R]{\thepage}

\title{Formulari i Taules de Química General}
\author{Jordi Villà i Freixa}
\date{\today}
\begin{document}
\maketitle
\tableofcontents
\section{Formulari de Química General}

\subsection{Constants}
\begin{longtable}{ll}
    \caption{Constants rellevants per a aquest curs}\\
\toprule
\bfseries Constant & \bfseries Valor\\
\midrule
Número d'Avogadro & \SI{6.022e23}{\per\mole} \\
Càrrega d'un electró & \SI{1.602e-19}{\coulomb} \\
Massa d'un electró & \SI{9.109e-31}{\kilo\gram} \\
Massa d'un protó & \SI{1.673e-27}{\kilo\gram} \\
Massa d'un neutró & \SI{1.675e-27}{\kilo\gram} \\
Constant de Planck & \SI{6.626e-34}{\joule\second} \\
Constant de Boltzmann & \SI{1.381e-23}{\joule\per\kelvin} \\
Constant dels gasos & \SI{8.314}{\joule\per\kelvin\per\mole} \\
Constant de Faraday & \SI{96485}{\coulomb\per\mole} \\
Constant de gravitació universal & \SI{6.674e-11}{\newton\meter\squared\per\kilo\gram\squared} \\
\bottomrule
\end{longtable}

\subsection{Fórmules}
\begin{longtable}{ll}
    \caption{Fórmules rellevants per a aquest curs}\\
\toprule
\bfseries Fórmula & \bfseries Descripció\\
\midrule
\(p = mv\) & Relació entre el moment lineal, la massa i la velocitat\\
\(KE = \frac{1}{2}mv^2\) & Energia cinètica d'un cos en moviment\\
\(P = \frac{F}{A}\) & Definició de pressió\\
\(PV = nRT\) & Llei dels gasos ideals\\
\(Q = mc\Delta T\) & Calor transferit en un procés de canvi de temperatura\\
\(Q = n\Delta H\) & Calor transferit en un procés de canvi d'entalpia\\
\bottomrule
\end{longtable}
\section{Unitats de mesura}

\begin{longtable}{llll}
    \caption{Algunes unitats del SI rellevants per a aquest curs, incloent la seva \href{https://en.wikipedia.org/wiki/Dimensional_analysis}{anàlisi dimensional}. El sistema CGS (centímetre-gram-segon) és un sistema de mesura que utilitza el centímetre, el gram i el segon com a unitats bàsiques de longitud, massa i temps respectivament.}\\
\toprule
\bfseries Magnitud & \bfseries Unitat a SI & \bfseries Símbol SI & \bfseries Dimensió\\\midrule
Longitud & metre & \si{\meter} & \(\mathsf{L}\) \\
Volum & litre & \si\litre & \(\mathsf{L^3}\) \\
Massa & kilogram & \si{\kilo\gram} & \(\mathsf{M}\) \\
Temperatura & kelvin & \si{\kelvin} & \(\mathsf{\Theta}\) \\
mol & mol & \si{\mole} & \(\mathsf{N}\) \\
temps & segon & \si\second & \(\mathsf{T}\) \\
Freqüència & hertz & \si{\hertz} & \(\mathsf{T^{-1}}\) \\
Energia & joule & \si\joule & \(\mathsf{ML^2T^{-2}}\) \\
Força & newton & \si\newton & \(\mathsf{MLT^{-2}}\) \\
Pressió & pascal & \si\pascal & \(\mathsf{ML^{-1}T^{-2}}\) \\
Potencial elèctric & volt & \si\volt & \(\mathsf{ML^2T^{-3}I^{-1}}\) \\
Potència & watt & \si\watt & \(\mathsf{ML^2T^{-3}}\) \\
\bottomrule
\label{tab:unitatsSI}
\end{longtable}

\begin{longtable}{ccc}
    \caption{Conversió d'unitats del sistema americà al Sistema Internacional (SI)}\\
    \toprule
    \textbf{Magnitud} & \textbf{Unitat (EUA)} & \textbf{Equivalència en SI} \\
    \midrule
    Volum & \SI{1}{\cubic\inch} & \SI{16.387}{\cubic\centi\meter} \\
    Volum & \SI{1}{\cubic\foot} & \SI{28.317}{\liter} \\
    Volum & \SI{1}{\gallon} (US) & \SI{3.785}{\liter} \\
    \hline
    Pressió & \SI{1}{\psi} & \SI{6.895}{\kilo\pascal} \\
    Pressió & \SI{1}{\atm} & \SI{101.325}{\kilo\pascal} \\
    Pressió & \SI{1}{\inchHg} & \SI{3.386}{\kilo\pascal} \\
    \hline
    Temperatura & \SI{1}{\fah} & $T_C=(T_{F} - 32) \times \frac{5}{9} $ \\
    \hline
    Massa & \SI{1}{\ounce} & \SI{28.35}{\gram} \\
    Massa & \SI{1}{\pound} & \SI{0.4536}{\kilo\gram} \\
    Massa & \SI{1}{\ton} (US) & \SI{907.184}{\kilo\gram} \\
    \bottomrule
    \label{tab:conversio}
\end{longtable}

\newpage
    \begin{longtable}{cc}
        \caption{Comparació de les unitats de pressió amb 1 atmosfera}\\
    \toprule
    \textbf{Unitat de Pressió} & \textbf{Pressió (en relació a 1 atm)} \\ \midrule
    Atmosfera (atm) & 1 atm \\ 
    Pascal (Pa) & \( 101325 \, \text{Pa} \) \\ 
    Kilopascal (kPa) & \( 101.325 \, \text{kPa} \) \\    
    Bar & \( 1.01325 \, \text{bar} \) \\ 
    Mil·límetre de mercuri (mmHg) & \( 760 \, \text{mmHg} \) \\ 
    Torra (Torr) & \( 760 \, \text{Torr} \) \\ 
    Pounds per square inch (psi) & \( 14.696 \, \text{psi} \) \\ 
\bottomrule
    \end{longtable}


    \begin{longtable}{cc}
        \caption{Conversió de la constant dels gasos en diferents unitats}  \\
    \toprule  
        \textbf{Valor de la constant dels gasos R} & \textbf{Unitats} \\
        \midrule
        \num{0.082} & \si{\atm\litre\per\mole\per\kelvin} \\
        \num{8.3145} & \si{m^3.Pa.K^{-1}.mol^{-1}} \\
        \num{8.3145} & \si{\joule\per\kelvin\per\mole} \\
        \num{62.363} & \si{L.Torr.K^{-1}.mol^{-1}} \\
        \num{1.9872e-3} & \si{kcal.K^{-1}.mol^{-1}} \\
        \num{8.205e-5} & \si{m^3.atm.K^{-1}.mol^{-1}} \\
        \bottomrule
        \label{tab:gas_constant}
    \end{longtable}

\newpage
\section{Dades termodinàmiques}
\begin{longtable}{lcccc}
    \caption{Calor de Fusió i Vaporització d'algunes substàncies pures (específic $\Delta H$ en J/g i Molar $\Delta H$ en kJ/mol)}\\
    \toprule
    \textbf{Substància} & \multicolumn{2}{c}{\textbf{Calor de Fusió}} & \multicolumn{2}{c}{\textbf{Calor de Vaporització}} \\
    & $\Delta H_\text{fus}$ (J/g) & $\Delta H_\text{fus}$ (kJ/mol) & $\Delta H_\text{vap}$ (J/g) & $\Delta H_\text{vap}$ (kJ/mol) \\
    \midrule\endhead
    Alumini & 321 & 8.66 & 11400 & 307.6 \\
    Benzè & 127.4 & 10.0 & 390 & 30.5 \\
    Coure & 207 & 13.2 & 5069 & 322.1 \\
    Or & 67 & 13.2 & 1578 & 310.9 \\
    Ferro & 209 & 11.7 & 6340 & 354.1 \\
    Plom & 22.4 & 4.64 & 871 & 180.5 \\
    Metà & 59 & 0.946 & 537 & 8.61 \\
    Mercuri & 11.6 & 2.33 & 295 & 5.92 \\
    Metanol & 98.8 & 3.17 & 1100 & 35.2 \\
    Nitrogen & 25.5 & 0.715 & 200 & 5.60 \\
    Sodi & 113 & 2.60 & 4237 & 97.42 \\
    Aigua & 334 & 6.02 & 2260 & 40.7 \\
    \bottomrule
\end{longtable}

La taula següent mostra els valors clau de termodinàmica per a diverses substàncies, extrets de la taula \emph{CODATA KEY VALUES FOR THERMODYNAMICS} a \cite{cox_codata_1989,lide_crc_2005}.
La taula inclou l'entalpia estàndard de formació a \qty{298.15}{\kelvin}, l'entropia a \qty{298.15}{\kelvin} i la quantitat \(H^\circ\) (\qty{298.15}{\kelvin})-\(H^\circ\) (\qty{0}{\kelvin}). Un valor de 0 a la columna \(\Delta_f H^\circ\) per a un element indica l'estat de referència per a aquest element. La pressió de l'estat estàndard és \qty{e5}{\pascal} (1 bar).


\begin{longtable}{lccc}
    \caption{Valors termodinàmics per a diverses substàncies \cite{cox_codata_1989}}
    \label{tab:thermo_values}\\ 
    \toprule
    \textbf{Substància} & \(\Delta_f H^\circ\) (298.15 K) & \(S^\circ\) (298.15 K) & \(H^\circ\) (298.15 K)–\(H^\circ\) (0) \\
    & \text{(kJ/mol)} & \text{(J/K/mol)} & \text{(kJ/mol)} \\
    \midrule\endhead
    \ch{Ar} (g) & 0 & 154.846 $\pm$ 0.003 & 6.197 $\pm$ 0.001 \\
    \ch{C} (cr, graphite) & 0 & 5.74 $\pm$ 0.10 & 1.050 $\pm$ 0.020 \\
    \ch{C} (g) & 716.68 $\pm$ 0.45 & 158.100 $\pm$ 0.003 & 6.536 $\pm$ 0.001 \\
    \ch{CO} (g) & -110.53 $\pm$ 0.17 & 197.660 $\pm$ 0.004 & 8.671 $\pm$ 0.001 \\
    \ch{CO2} (aq, undissoc.) & -413.26 $\pm$ 0.20 & 119.36 $\pm$ 0.60 & \\
    \ch{CO2} (g) & -393.51 $\pm$ 0.13 & 213.785 $\pm$ 0.010 & 9.365 $\pm$ 0.003 \\
    \ch{CO3^{2-}} (aq) & -675.23 $\pm$ 0.25 & -50.0 $\pm$ 1.0 & \\
    \ch{H2} (g) & 0 & 130.680 $\pm$ 0.003 & 8.468 $\pm$ 0.001 \\
    \ch{H2O} (g) & -241.826 $\pm$ 0.040 & 188.835 $\pm$ 0.010 & 9.905 $\pm$ 0.005 \\
    \ch{H2O} (l) & -285.830 $\pm$ 0.040 & 69.95 $\pm$ 0.03 & 13.273 $\pm$ 0.020 \\
    \ch{H2PO4^{-}} (aq) & -1302.6 $\pm$ 1.5 & 92.5 $\pm$ 1.5 & \\
    \ch{H2S} (aq, undissoc.) & -38.6 $\pm$ 1.5 & 126 $\pm$ 5 & \\
    \ch{H2S} (g) & -20.6 $\pm$ 0.5 & 205.81 $\pm$ 0.05 & 9.957 $\pm$ 0.010 \\
    \ch{N} (g) & 472.68 $\pm$ 0.40 & 153.301 $\pm$ 0.003 & 6.197 $\pm$ 0.001 \\
    \ch{NH3} (g) & -45.94 $\pm$ 0.35 & 192.77 $\pm$ 0.05 & 10.043 $\pm$ 0.010 \\
    \ch{NH4^{+}} (aq) & -133.26 $\pm$ 0.25 & 111.17 $\pm$ 0.40 & \\
    \ch{NO3^{-}} (aq) & -206.85 $\pm$ 0.40 & 146.70 $\pm$ 0.40 & \\
    \ch{N2} (g) & 0 & 191.609 $\pm$ 0.004 & 8.670 $\pm$ 0.001 \\
    \ch{S} (g) & 277.17 $\pm$ 0.15 & 167.829 $\pm$ 0.006 & 6.657 $\pm$ 0.001 \\
    \ch{SO2} (g) & -296.81 $\pm$ 0.20 & 248.223 $\pm$ 0.050 & 10.549 $\pm$ 0.010 \\
    \ch{SO4^{2-}} (aq) & -909.34 $\pm$ 0.40 & 18.50 $\pm$ 0.40 & \\
    \ch{C3H8} (g) & -104.7 $\pm$ 0.4 & 269.91 $\pm$ 0.10 & 14.66 $\pm$ 0.05 \\
    \ch{H2} (g) & 0 & 130.680 $\pm$ 0.003 & 8.468 $\pm$ 0.001 \\
    \ch{H2O} (g) & -241.826 $\pm$ 0.040 & 188.835 $\pm$ 0.010 & 9.905 $\pm$ 0.005 \\
    \ch{H2O} (l) & -285.830 $\pm$ 0.040 & 69.95 $\pm$ 0.03 & 13.273 $\pm$ 0.020 \\
    \ch{H2PO4^{-}} (aq) & -1302.6 $\pm$ 1.5 & 92.5 $\pm$ 1.5 & \\
    \ch{H2S} (aq, undissoc.) & -38.6 $\pm$ 1.5 & 126 $\pm$ 5 & \\
    \ch{H2S} (g) & -20.6 $\pm$ 0.5 & 205.81 $\pm$ 0.05 & 9.957 $\pm$ 0.010 \\
    \ch{N} (g) & 472.68 $\pm$ 0.40 & 153.301 $\pm$ 0.003 & 6.197 $\pm$ 0.001 \\
    \ch{NH3} (g) & -45.94 $\pm$ 0.35 & 192.77 $\pm$ 0.05 & 10.043 $\pm$ 0.010 \\
    \ch{NH4^{+}} (aq) & -133.26 $\pm$ 0.25 & 111.17 $\pm$ 0.40 & \\
    \ch{NO3^{-}} (aq) & -206.85 $\pm$ 0.40 & 146.70 $\pm$ 0.40 & \\
    \ch{N2} (g) & 0 & 191.609 $\pm$ 0.004 & 8.670 $\pm$ 0.001 \\
    \ch{S} (g) & 277.17 $\pm$ 0.15 & 167.829 $\pm$ 0.006 & 6.657 $\pm$ 0.001 \\
    \ch{SO2} (g) & -296.81 $\pm$ 0.20 & 248.223 $\pm$ 0.050 & 10.549 $\pm$ 0.010 \\
    \ch{SO4^{2-}} (aq) & -909.34 $\pm$ 0.40 & 18.50 $\pm$ 0.40 & \\
    \bottomrule
\end{longtable}

\subsection{Calor de Combustió}

La calor de combustió d'una substància a 25°C es pot calcular a partir de les dades d'entalpia de formació (\(\Delta_f H^\circ\)). Podem escriure la reacció general de combustió com:
\[ \ch{X + O2 -> CO2 (g) + H2O (l) + Y} \]

Per a un compost que conté només carboni, hidrogen i oxigen, la reacció és simplement:

\[
\ch{C_{a}H_{b}O_{c} + $\left({a}+\frac{b}{4}-\frac{c}{2}\right)$ O2 -> a CO2 (g) + $\frac{b}{2}$ H2O (l)}
\]
i la calor estàndard de combustió \(\Delta_c H^\circ\), que es defineix com el negatiu del canvi d'entalpia per a la reacció (és a dir, el calor alliberat en el procés de combustió), es dóna per:
\[ \Delta_c H^\circ = -a \Delta_f H^\circ (CO_2, g) - \frac{b}{2} \Delta_f H^\circ (H_2O, l) + \Delta_f H^\circ (\text{C}_a \text{H}_b \text{O}_c) \]
\[ = 393.51a + 142.915b + \Delta_f H^\circ (\text{C}_a \text{H}_b \text{O}_c) \]

Aquesta equació s'aplica si els reactius comencen en els seus estats estàndard (25°C i una atmosfera de pressió) i els productes tornen a les mateixes condicions. La mateixa equació s'aplica a un compost que conté un altre element si aquest element acaba en el seu estat de referència estàndard (per exemple, nitrogen, si el producte és \ch{N2}); en general, però, els productes exactes que contenen els altres elements han de ser coneguts per calcular el calor de combustió.

\begin{longtable}{lcc}
    \caption{Calor estàndard de combustió de diverses substàncies. Adaptat de la taula \emph{Heat of Combustion} a \cite{lide_crc_2005}}
    \label{tab:combustion}\\
    \toprule
    \textbf{Fórmula Molecular} & \textbf{Nom} & \(\Delta_c H^\circ\) (kJ/mol) \\
    \midrule\endhead
    \ch{C3H8O} & 1-Propanol (l) & 2021.3 \\
    \ch{C3H8O3} & Glicerol (l) & 1655.4 \\
    \ch{C4H10O} & Èter dietílic (l) & 2723.9 \\
    \ch{C5H12O} & 1-Pentanol (l) & 3330.9 \\
    \ch{C6H6} & Fenol (s) & 3053.5 \\
    \midrule
    \textbf{Substàncies Inorgàniques} & & \\
    \ch{C} & Carboni (grafit) & 393.5 \\
    \ch{CO} & Monòxid de carboni (g) & 283.0 \\
    \ch{H2} & Hidrogen (g) & 285.8 \\
    \ch{H3N} & Amoníac (g) & 382.8 \\
    \ch{H4N2} & Hidrazina (g) & 667.1 \\
    \ch{N2O} & Òxid nitrós (g) & 82.1 \\
    \midrule
    \textbf{Compostos de Carbonil} & & \\
    \ch{CH2O} & Formaldehid (g) & 726.1 \\
    \ch{C2H2O} & Cetè (g) & 1366.8 \\
    \ch{C2H4O} & Acetaldehid (l) & 1460.4 \\
    \ch{C3H6O} & Acetona (l) & 1189.2 \\
    \ch{C3H6O} & Propanal (l) & 1822.7 \\
    \ch{C4H8O} & 2-Butanona (l) & 2444.1 \\
    \midrule
    \textbf{Hidrocarburs} & & \\
    \ch{CH4} & Metà (g) & 890.8 \\
    \ch{C2H2} & Acetilè (g) & 1301.1 \\
    \ch{C2H4} & Etilè (g) & 1411.2 \\
    \ch{C2H6} & Età (g) & 1560.7 \\
    \ch{C3H6} & Propilè (g) & 2058.0 \\
    \ch{C3H6} & Ciclopropà (g) & 2091.3 \\
    \ch{C3H8} & Propà (g) & 2219.2 \\
    \ch{C4H6} & 1,3-Butadiè (g) & 2541.5 \\
    \ch{C4H10} & Butà (g) & 2877.6 \\
    \ch{C5H12} & Pentà (l) & 3509.0 \\
    \ch{C6H6} & Benzè (l) & 3267.6 \\
    \ch{C6H12} & Ciclohexà (l) & 3919.6 \\
    \ch{C6H14} & Hexà (l) & 4163.2 \\
    \ch{C7H8} & Toluè (l) & 3910.3 \\
    \ch{C7H16} & Heptà (l) & 4817.0 \\
    \ch{C10H8} & Naftalè (s) & 5156.3 \\
    \midrule
    \textbf{Alcohols i Èters} & & \\
    \ch{CH4O} & Metanol (l) & 570.7 \\
    \ch{C2H6O} & Etanol (l) & 1025.4 \\
    \ch{C2H6O} & Èter dimetílic (g) & 1166.9 \\
    \ch{C2H6O2} & Etilè glicol (l) & 1789.9 \\
    \midrule
    \textbf{Àcids i Èsters} & & \\
    \ch{CH2O2} & Àcid fòrmic (l) & 254.6 \\
    \ch{C2H4O2} & Àcid acètic (l) & 874.2 \\
    \ch{C2H4O2} & Formiat de metil (l) & 972.6 \\
    \ch{C3H6O2} & Acetat de metil (l) & 1592.2 \\
    \ch{C4H8O2} & Acetat d'etil (l) & 2238.1 \\
    \ch{C7H6O2} & Àcid benzoic (s) & 3226.9 \\
    \midrule
    \textbf{Compostos de Nitrogen} & & \\
    \ch{CHN} & Cianur d'hidrogen (g) & 671.5 \\
    \ch{CH3NO2} & Nitrometà (l) & 709.2 \\
    \ch{CH5N} & Metilamina (g) & 1085.6 \\
    \ch{C2H3N} & Acetonitril (l) & 1247.2 \\
    \ch{C2H5NO} & Acetamida (s) & 1184.6 \\
    \ch{C3H9N} & Trimetilamina (g) & 2443.1 \\
    \ch{C5H5N} & Piridina (l) & 2782.3 \\
    \ch{C6H7N} & Anilina (l) & 3392.8 \\
    \bottomrule
\end{longtable}


\section{Enllaços d'interès}

A part de les referències incloses en aquest document, es pot trobar més informació rellevant en les següents fonts:
\begin{itemize}
    \item Sobre els errors en les mesures i la seva propagació: \cite{lindberg_uncertainties_2000}. 
\end{itemize}
\printbibliography
\end{document}