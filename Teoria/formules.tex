\section{Constants}
\begin{longtable}{ll}
    \caption{Constants rellevants per a aquest curs}\\
\toprule
\bfseries Constant & \bfseries Valor\\
\midrule
Número d'Avogadro & \SI{6.022e23}{\per\mole} \\
Càrrega d'un electró & \SI{1.602e-19}{\coulomb} \\
Massa d'un electró & \SI{9.109e-31}{\kilo\gram} \\
Massa d'un protó & \SI{1.673e-27}{\kilo\gram} \\
Massa d'un neutró & \SI{1.675e-27}{\kilo\gram} \\
Constant de Planck & \SI{6.626e-34}{\joule\second} \\
Constant de Boltzmann & \SI{1.381e-23}{\joule\per\kelvin} \\
Constant dels gasos & \SI{8.314}{\joule\per\kelvin\per\mole} \\
Constant de Faraday & \SI{96485}{\coulomb\per\mole} \\
Constant de gravitació universal & \SI{6.674e-11}{\newton\meter\squared\per\kilo\gram\squared} \\
\bottomrule
\end{longtable}


\section{Fórmules}
\begin{longtable}{ll}
    \caption{Fórmules rellevants per a aquest curs}\\
\toprule
\bfseries Fórmula & \bfseries Descripció\\
\midrule
\(p = mv\) & Relació entre el moment lineal, la massa i la velocitat\\
\(KE = \frac{1}{2}mv^2\) & Energia cinètica d'un cos en moviment\\
\(P = \frac{F}{A}\) & Definició de pressió\\
\(PV = nRT\) & Llei dels gasos ideals\\
$\left( P + \frac{n^2 a}{V^2} \right) (V -nb)=nRT$ & Equacó de van der Waals\\
$w=-P\Delta V$& Treball exercit sobre un gas\\
\midrule
\(U = q + w\) & Primera llei de la termodinàmica\\
\(H = U + PV\) & Definició d'entalpia\\
$\diff S = \frac{\diff q_{\text{rev}}}{T}$ & Definició d'entropia\\
\(G = H - TS\) & Definició d'energia lliure de Gibbs\\
\(q_v = n\Delta U\) & Calor a volum constant\\
\(q_p = n\Delta H\) & Calor a pressió constant\\
$\Delta G = \Delta H - T\Delta S$ & Canvi d'energia lliure de Gibbs\\
\bottomrule
\end{longtable}
\vfill\null
\columnbreak
\section{Unitats de mesura}

\begin{longtable}{llll}
    \caption{Algunes unitats del SI rellevants per a aquest curs, incloent la seva \href{https://en.wikipedia.org/wiki/Dimensional_analysis}{anàlisi dimensional}. El sistema CGS (centímetre-gram-segon) és un sistema de mesura que utilitza el centímetre, el gram i el segon com a unitats bàsiques de longitud, massa i temps respectivament.}\\
\toprule
\bfseries Magnitud & \bfseries Unitat a SI & \bfseries Símbol SI & \bfseries Dimensió\\\midrule
Longitud & metre & \si{\meter} & \(\mathsf{L}\) \\
Volum & litre & \si\litre & \(\mathsf{L^3}\) \\
Massa & kilogram & \si{\kilo\gram} & \(\mathsf{M}\) \\
Temperatura & kelvin & \si{\kelvin} & \(\mathsf{\Theta}\) \\
mol & mol & \si{\mole} & \(\mathsf{N}\) \\
temps & segon & \si\second & \(\mathsf{T}\) \\
Freqüència & hertz & \si{\hertz} & \(\mathsf{T^{-1}}\) \\
Energia & joule & \si\joule & \(\mathsf{ML^2T^{-2}}\) \\
Força & newton & \si\newton & \(\mathsf{MLT^{-2}}\) \\
Pressió & pascal & \si\pascal & \(\mathsf{ML^{-1}T^{-2}}\) \\
Potencial elèctric & volt & \si\volt & \(\mathsf{ML^2T^{-3}I^{-1}}\) \\
Potència & watt & \si\watt & \(\mathsf{ML^2T^{-3}}\) \\
\bottomrule
\label{tab:unitatsSI}
\end{longtable}

\begin{longtable}{ccc}
    \caption{Conversió d'unitats del sistema americà al Sistema Internacional (SI)}\\
    \toprule
    \textbf{Magnitud} & \textbf{Unitat (EUA)} & \textbf{Equivalència en SI} \\
    \midrule
    Volum & \SI{1}{\cubic\inch} & \SI{16.387}{\cubic\centi\meter} \\
    Volum & \SI{1}{\cubic\foot} & \SI{28.317}{\liter} \\
    Volum & \SI{1}{\gallon} (US) & \SI{3.785}{\liter} \\
    \hline
    Pressió & \SI{1}{\psi} & \SI{6.895}{\kilo\pascal} \\
    Pressió & \SI{1}{\atm} & \SI{101.325}{\kilo\pascal} \\
    Pressió & \SI{1}{\inchHg} & \SI{3.386}{\kilo\pascal} \\
    \hline
    Temperatura & \SI{1}{\fah} & $T_C=(T_{F} - 32) \times \frac{5}{9} $ \\
    \hline
    Massa & \SI{1}{\ounce} & \SI{28.35}{\gram} \\
    Massa & \SI{1}{\pound} & \SI{0.4536}{\kilo\gram} \\
    Massa & \SI{1}{\ton} (US) & \SI{907.184}{\kilo\gram} \\
    \bottomrule
    \label{tab:conversio}
\end{longtable}

%\newpage
    \begin{longtable}{cc}
        \caption{Comparació de les unitats de pressió amb 1 atmosfera}\\
    \toprule
    \textbf{Unitat de Pressió} & \textbf{Pressió (en relació a 1 atm)} \\ \midrule
    Atmosfera (atm) & 1 atm \\ 
    Pascal (Pa) & \( 101325 \, \text{Pa} \) \\ 
    Kilopascal (kPa) & \( 101.325 \, \text{kPa} \) \\    
    Bar & \( 1.01325 \, \text{bar} \) \\ 
    Mil·límetre de mercuri (mmHg) & \( 760 \, \text{mmHg} \) \\ 
    Torra (Torr) & \( 760 \, \text{Torr} \) \\ 
    Pounds per square inch (psi) & \( 14.696 \, \text{psi} \) \\ 
\bottomrule
    \end{longtable}


    \begin{longtable}{cc}
        \caption{Conversió de la constant dels gasos en diferents unitats}  \\
    \toprule  
        \textbf{Valor de la constant dels gasos R} & \textbf{Unitats} \\
        \midrule
        \num{0.082} & \si{\atm\litre\per\mole\per\kelvin} \\
        \num{8.3145} & \si{m^3.Pa.K^{-1}.mol^{-1}} \\
        \num{8.3145} & \si{\joule\per\kelvin\per\mole} \\
        \num{62.363} & \si{L.Torr.K^{-1}.mol^{-1}} \\
        \num{1.9872e-3} & \si{kcal.K^{-1}.mol^{-1}} \\
        \num{8.205e-5} & \si{m^3.atm.K^{-1}.mol^{-1}} \\
        \bottomrule
        \label{tab:gas_constant}
    \end{longtable}
