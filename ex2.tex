\begin{qst}
Perquè es generen bombolles de seguida que obrim una ampolla d'aigua amb gas?
\end{qst}
\lct{
Segons la llei de Henry, la concentració de gas en un líquid és proporcional a la pressió parcial externa d'aquest gas. Quan s'embotella  aigua carbonatada es fa a una pressió superior a 1 atm. Quan obrim l'ampolla, la pressió es redueix fins a 1 atm, i per tant el \ch{CO2} no és tan soluble i escapa de la dissolució formant bombolles.
}