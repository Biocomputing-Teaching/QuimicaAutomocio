% feina a fer per al curs 2010-2011
% 1) l'exercici proposat de maximitzar una funció de dues variables és difícil i cal treballar-lo. Mirar bé el mètode d'steepest descent i el de Newton Raphson (els dos proposats a l'exercici). Mirar també d'entendre correctament el mètode d'optimitzar la funció de Davidson. Mirar una bona explicació a http://linneus20.ethz.ch:8080/1_5_3.html#SECTION00253100000000000000 Fer un dibuis que mostri el concepte de "constant norm", entès com un radi determinat al voltant d'x, un radi donat per l'stepsize. És prou entendor així
\usepackage{framed}
\usepackage{url}
\usepackage{ifthen}
\usepackage{longtable}
\usepackage{fancyvrb}
\usepackage[catalan]{babel}
\usepackage{cancel}

\usepackage[utf8]{inputenc}
\usepackage[catalan]{babel}
\usepackage{lmodern}
\usepackage{amsmath,amsthm,amsfonts,amssymb,amscd}
\usepackage{multirow,booktabs}
\usepackage[dvipsnames,table]{xcolor}
\usepackage{fullpage}
\usepackage{lastpage}
\usepackage{graphicx}
\usepackage{enumitem}
\usepackage{fancyhdr}
\usepackage{mathrsfs}
\usepackage{wrapfig}
\usepackage{setspace}
\usepackage{calc}
\usepackage{multicol}
\usepackage{gensymb}
\usepackage{listings}
\usepackage{siunitx}

\usepackage{cancel}
\usepackage[retainorgcmds]{IEEEtrantools}
\usepackage[margin=3cm]{geometry}
\usepackage{amsmath}
\newlength{\tabcont}
\setlength{\parindent}{0.0in}
\setlength{\parskip}{0.05in}
\usepackage{empheq}
\usepackage{framed}
\usepackage{mdframed}

\usepackage[most]{tcolorbox}
\usepackage{chemfig,chemmacros,chemnum}
\usepackage{chemformula}

\usepackage{tcolorbox}
\usepackage{url}
  \let\oldurl\url
\usepackage{hyperref}
  \let\linkurl\url
  \let\url\oldurl
  
\usepackage[
backend=biber,
citestyle=numeric-comp 
]{biblatex}
\addbibresource{QuimAutom.bib}


%\chemsetup[chemformula]{format=\sffamily}
\renewcommand*\printatom[1]{\ensuremath{\mathsf{#1}}}
%\setatomsep{2em}
%\setdoublesep{.6ex}
%\setbondstyle{semithick}
\colorlet{shadecolor}{orange!15}
\parindent 0in
\parskip 12pt
\geometry{margin=1in, headsep=0.25in}
\theoremstyle{definition}
\newtheorem{defn}{Definition}
\newtheorem{reg}{Rule}
\newtheorem{exer}{Exercise}
\newtheorem{note}{Note}
%\RequirePackage{mathrsfs}
%\RequirePackage[psamsfonts]{amsfonts} %for Y&Y BSR AMS fonts
\RequirePackage{amsmath,amsfonts,amsthm,amssymb}
\RequirePackage{setspace}
\RequirePackage{fancyhdr}
\RequirePackage{lastpage}
\RequirePackage{extramarks}
\RequirePackage{chngpage}
\RequirePackage{soul}
%\RequirePackage[dvipsnames]{xcolor}
%\RequirePackage{graphicx,float,wrapfig}
%\RequirePackage{pgf,tikz}
%\usetikzlibrary{arrows,automata}
%\RequirePackage{pstricks}
%\RequirePackage[text]{amsthm}
%\RequirePackage{array}
%\RequirePackage{amscd}
%\RequirePackage{array}\RequirePackage{dcolumn}
%\putfig{3.5truein}{PSfig1.3}{Peter's winnings in 40 plays of heads or tails.}{fig 1.3}

\newcommand*{\DIRFig}{../../../../talks/figures}
\newcommand{\barefig}[2]{\makebox{\includegraphics*[width=#2] {\DIRFig/#1}}}
\newcommand{\putfig}[4]
{\begin{figure}
\centerline{\barefig{#2}{#1}}
\caption{#3}
\label{#4}
\end{figure}}

\newcommand{\emx}[1]{{\em{#1}\/}}
\newcommand{\abin}{{\it ab initio}}
\newcommand{\bs}{\boldsymbol}
\newcommand{\citenum}{\cite}
\newcommand{\dGo}{\ensuremath{\Delta G_0}}
\newcommand{\dG}[2]{\ensuremath{\Delta G_{\rm #1}^{\rm #2}}}
\newcommand{\dX}[3]{\ensuremath{\Delta #1_{\rm #2}^{\rm #3}}}
\newcommand{\ddgo}[1]{\ensuremath{\Delta \Delta G_{\rm solv}^{\rm #1}}}
\newcommand{\ddgstarcat}{\ensuremath{\Delta \Delta g^{\ddagger}_{\rm cat}}}
\newcommand{\ddgstar}{\ensuremath{\Delta \dgstar}}
\newcommand{\ddgt}[2]{\ensuremath{\Delta \Delta G_{\rm solv}^{\rm #1, \rm #2}}}
\newcommand{\ddsstarprime}{\ensuremath{(\Delta \dsstar)'}}
\newcommand{\deltaepsel}{\ensuremath{\Delta \varepsilon_{\rm el}}}
\newcommand{\deltaeps}{\ensuremath{\Delta \varepsilon}}
\newcommand{\dgab}[2]{\ensuremath{\Delta g_{\rm #1}^{\rm #2}}}
\newcommand{\dga}[1]{\ensuremath{\Delta g_{\rm #1}}}
\newcommand{\dgb}[1]{\ensuremath{\Delta g^{\rm #1}}}
\newcommand{\dgcage}{\ensuremath{\Delta g_{\rm cage}}}
\newcommand{\dgcat}{\ensuremath{\Delta g_{\rm cat}}}
\newcommand{\dgsoltsatsa}{\ensuremath{\dgsol (\rm TSA)_{\rm TSA}}}
\newcommand{\dgsoltstsa}{\ensuremath{\dgsol (\rm TS)_{\rm TSA}}}
\newcommand{\dgsoltsts}{\ensuremath{\dgsol (\rm TS)_{\rm TS}}}
\newcommand{\dgsol}{\ensuremath{\Delta G_{\rm sol}}}
\newcommand{\dgstarcage}{\ensuremath{\dgstar_{\rm cage}}}
\newcommand{\dgstarcat}{\ensuremath{\dgstar_{\rm cat}}}
\newcommand{\dgstarw}{\ensuremath{\dgstar_{\rm w}}}
\newcommand{\dgstar}{\ensuremath{\Delta g^{\ddagger}}}
\newcommand{\dgw}{\ensuremath{\Delta g_{\rm w}}}
\newcommand{\dg}[2]{\ensuremath{\Delta g_{\rm #1}^{\rm #2}}}
\newcommand{\dino}{\texttt{DINO}}
\newcommand{\dsstarcageprime}{\ensuremath{(\dsstarcage)'}}
\newcommand{\dsstarcage}{\ensuremath{\dsstar_{\rm cage}}}
\newcommand{\dsstarcatprime}{\ensuremath{(\dsstarcat)'}}
\newcommand{\dsstarcat}{\ensuremath{\dsstar_{\rm cat}}}
\newcommand{\dsstarwprime}{\ensuremath{(\dsstarw)'}}
\newcommand{\dsstarw}{\ensuremath{\dsstar_{\rm w}}}
\newcommand{\dsstar}{\ensuremath{\Delta S^{\ddagger}}}
\newcommand{\eg}{{\it e.g.}}
\newcommand{\etal}{{\it et al.}}
\newcommand{\gamess}{\texttt{GAMESS}}
\newcommand{\gauss}{\texttt{GAUSSIAN} 98}     
\newcommand{\golpe}{\texttt{GOLPE}}                                             
\newcommand{\grid}{\texttt{GRID}}
\newcommand{\ie}{{\it i.e.}}
\newcommand{\ith}{{\it i}$^{\rm th}$\ }
\newcommand{\kbt}{\ensuremath{k_{\rm B} T}}
\newcommand{\kb}{\ensuremath{k_{\rm B}}} 
\newcommand{\kcage}{\ensuremath{k_{\rm cage}}}
\newcommand{\kcatkm}{\ensuremath{k_{\rm cat}/K_{\rm M}}}
\newcommand{\kcat}{\ensuremath{k_{\rm cat}}}
\newcommand{\km}{\ensuremath{{\rm\, kcal \, mol}^-1}}
\newcommand{\knon}{\ensuremath{k_{\rm non}}}
\newcommand{\kw}{\ensuremath{k_{\rm w}}}
\newcommand{\mepsim}{\texttt{MEPSIM}}
\newcommand{\mgp}[1]{\marginpar{\scriptsize{#1}}}
\newcommand{\mipsim}{\texttt{MIPSIM}}
\newcommand{\mola}{\texttt{MOLARIS}}
\newcommand{\msms}{\texttt{MSMS}}
\newcommand{\pdras}{p21$^{\rm ras}$}
\newcommand{\rgran}{\ensuremath{\mathbb{R}}}
\newcommand{\rx}[2]{\ensuremath{#1_{\rm #2}}}
\newcommand{\vs}{{\it vs.}}
\newcommand{\z}[1]{\ensuremath{\mathbf{#1}}} 
\newcommand{\composed}[2]{#1\mathbin\circ #2}
\newcommand{\wrt}[1]{{\mbox{\scriptsize w.r.t. \( #1 \)} }}
\newcommand{\polyspace}{\mathcal{P}}
\newcommand{\matspace}{\mathcal{M}}
\newcommand{\C}{\mathbb{C}}
\newcommand{\N}{\mathbb{N}}
\newcommand{\Q}{\mathbb{Q}}
\newcommand{\Z}{\mathbb{Z}}
\renewcommand{\Re}{\mathbb{R}}
\newcommand{\rtres}{\ensuremath{\Re^3}}
\newcommand{\union}{\cup}
\newcommand{\dotprod}{\cdot}
\newcommand*\pkg[1]{\textsf{#1}}

\newcommand{\trans}[1]{{#1}^{\ensuremath{\mathsf{T}}}} % transpose
\newcommand{\nbyn}[1]{\ensuremath{#1 \! \times \! #1 }}
\newcommand{\nbym}[2]{#1 \! \times \! #2 }       % Use in math mode.
\newcommand{\cat}[2]{#1\!\mathbin{\raise.6ex\hbox{\( {}^\frown \)}}\!#2}
\newcommand{\generalmatrix}[3]{ %arg1: low-case letter, arg2: rows, arg3: cols
               \left(
                  \begin{array}{cccc}
                    #1_{1,1}  &#1_{1,2}  &\ldots  &#1_{1,#2}  \\
                    #1_{2,1}  &#1_{2,2}  &\ldots  &#1_{2,#2}  \\
                              &\vdots                         \\
                    #1_{#3,1} &#1_{#3,2} &\ldots  &#1_{#3,#2}
                  \end{array}
               \right)  }
\newcommand{\colvec}[1]{\begin{pmatrix} #1 \end{pmatrix}}
\newcommand{\pr}[1]{\ensuremath{\mathrm{Pr}(#1)}}
\newcommand{\rep}[2]{ {\rm Rep}_{#2}(#1) }
\newcommand{\mapsunder}[1]{\stackrel{#1}{\longmapsto}}
\newcommand{\map}[3]{\mbox{$#1\colon #2\to #3$}}
\newcommand{\identity}{\mbox{id}}
\newcommand{\stdbasis}{{\cal E}} 
\newcommand{\sequence}[1]{ \langle#1\rangle } 
\newcommand{\spacer}{\rule[-3mm]{0mm}{8mm}}
\newcommand{\email}[1]{\url{#1}}
\newcommand{\zero}{\vec{0}}
\newcommand{\proj}[2]{\mbox{proj}_{#2}({#1}) }
%\AtBeginDocument{\newlength{\heightofcdot}
%\newlength{\widthofcdot}
%\settoheight{\heightofcdot}{$\cdot$}
%\settowidth{\widthofcdot}{$\cdot$}
%\newsavebox{\dotprodcircle}       
%\savebox{\dotprodcircle}{\includegraphics{dotprod.1}} 
%\newcommand{\dotprod}{\mathbin{\raisebox{.5\heightofcdot}{%
%          \makebox[\widthofcdot]{$\smash{\usebox{\dotprodcircle}}$}}}}}
\newcommand{\spanof}[1]{\relax [#1\relax ]} % no optional argument!
\newcommand{\set}[1]{\mbox{$\{#1\}$}} \newcommand{\suchthat}{\bigm|}
\newcommand{\deter}[1]{ \mathchoice{\left|#1\right|}{|#1|}{|#1|}{|#1|} }
\newcommand{\secuence}[1]{ \langle#1\rangle }  
\newcommand{\basis}[2]{\secuence{\vec{#1}_1,\ldots,\vec{#1}_{#2}}}



%--------linsys
%  Use as \begin{linsys}{3}
%           x &+ &3y &+ &a &= &7 \\
%           x &- &3y &+ &a &= &7
%         \end{linsys}
% Remark: TeXbook pp. 167-170 says to put a medmuskip around a +; and that's
% 4/18-ths of an em.  Why does 2/18-ths of an em work?  I don't know, but
% comparing to a regular displayed equation suggests it is right.
% (darseneau says LaTeX puts in half an \arraycolsep.)
\newenvironment{linsys}[2][m]{%
\setlength{\arraycolsep}{.1111em} % p. 170 TeXbook; a medmuskip
\begin{array}[#1]{@{}*{#2}{rc}r@{}}
}{%
\end{array}}


%\newtheorem{teorema}{Teorema}
%\newtheorem{exercici}{Exercici}
%\newtheorem{definicio}{Definici\'o}
%\newtheorem{theorem}{Theorem}
\newtheorem{exercise}{Exercise}
%\newtheorem{definition}{Definition}

\newcounter{EXMP}
\newenvironment{EXMP}[1][]{\definecolor{shadecolor}{rgb}{0.6,0.6,0.6}
							\begin{shaded}\refstepcounter{EXMP}\par\medskip\noindent%
   							\textbf{EXAMPLE~\theEXMP. #1} \rmfamily}
   							{\end{shaded}\medskip}

\newcounter{BOXT}
\newenvironment{BOXT}[1][]{\definecolor{shadecolor}{rgb}{0.8,0.8,0.8}
							\begin{shaded}\refstepcounter{BOXT}\par\medskip\noindent%
   							\textbf{BOX~\theBOXT. #1} \rmfamily}
   							{\end{shaded}\medskip}

\parskip 4mm


\usepackage{makeidx}
\makeindex


% margins
\topmargin=-0.45in      %
\evensidemargin=0in     %
\oddsidemargin=0in      %
\textwidth=6in        %
\textheight=8.5in       %
\headsep=0.25in         %

% header and footer
\pagestyle{fancy}       %
\chead{}                %
\cfoot{\bf JVF 2018}                %
\rfoot{\thepage}        %
\renewcommand\headrulewidth{0.4pt}   %
\renewcommand\footrulewidth{0.4pt}   %

%\setcounter{section}{-1}

\theoremstyle{definition}
\newtheorem{thm}{Theorem}
\newtheorem{dfn}{Definition}
\newtheorem{lem}{Lemma}
\newtheorem{prp}{Proposition}



%%%%%%%%%%%%%%%%%%%%%%%%%%%%%%%%%%%%%%%%%
%%%%%%%%%%%%%%%%%%%%%%%%%%%%%%%%%%%%%%%%%
% lecturer or student text
% in principle the lecturer text includes some examples to be done in the c lass
\newboolean{LECT}
\setboolean{LECT}{false}
\setboolean{LECT}{true}
%%%%%%%%%%%%%%%%%%%%%%%%%%%%%%%%%%%%%%%%%
%%%%%%%%%%%%%%%%%%%%%%%%%%%%%%%%%%%%%%%%%

\newenvironment{lect}{ % 
	\definecolor{shadecolor}{rgb}{1.0,0.8,0.8} %
	\begin{shaded} %
	\textcolor{BrickRed}{\bf Resultat\\}%

} % 
{ %	
	\end{shaded}
} %

\newcommand{\lct}[1]{\ifthenelse{\boolean{LECT}}{\begin{lect} #1 \end{lect}}{}}



%%%%%%%%%%%%%%%%%%%
% ANGLÈS
%%%%%%%%%%%%%%%%%%%

% \newcommand{\problemName}{}%
% \newcounter{problemCounter}%
% \newenvironment{problem}[1][Problem \arabic{problemCounter}]%
% 	{\stepcounter{problemCounter}%
% 		\renewcommand{\problemName}{#1}%
% 		\section*{\problemName}%
% 		\nobreak\extramarks{\problemName}{\problemName continued on next page\ldots}\nobreak%
% 		\nobreak\extramarks{\problemName (continued)}{\problemName continued on next page\ldots}\nobreak}%
% 	{\nobreak\extramarks{\problemName (continued)}{\problemName continued on next page\ldots}\nobreak%
% 		\nobreak\extramarks{\problemName}{}\nobreak}%

\newenvironment{example}{ % 
	\definecolor{shadecolor}{rgb}{0.8,1.0,0.8} %
	\begin{shaded} %
	\textcolor{OliveGreen}{\bf Example\\}%
} % 
{ %	
	\end{shaded}
} %


\newenvironment{introduction}{ % 
	\definecolor{shadecolor}{rgb}{1.0,1.0,0.8} %
	\begin{shaded} %
	% \textcolor{BrickRed}{\bf Introduction\\}%
} % 
{ %	
	\end{shaded}
} %


%%%%%%%%%%%%%%%%%%%
% CATALÀ
%%%%%%%%%%%%%%%%%%%
\newtheorem{teorema}{theorem}
\newenvironment{definicio}{ % 
	\definecolor{shadecolor}{rgb}{0.9,1.0,0.8} %
	\begin{shaded} %
	\textcolor{OliveGreen}{\bf Definicio\\}%
} % 
{ %	
	\end{shaded}
} %

%veure http://en.wikibooks.org/wiki/LaTeX/Advanced_Topics
\newcounter{myc}
%environment for exercises in the class notes
\newenvironment{exr}{ % 
    \addtocounter{myc}{1}
	\definecolor{shadecolor}{rgb}{0.9,1.0,0.8} %
	\begin{shaded} %
	\textcolor{OliveGreen}{\bf Exercici \arabic{myc}\\}%
} % 
{ %	
	\end{shaded}
} %
%environment for questions in exams
\newenvironment{qst}{ % 
    \addtocounter{myc}{1}
	\definecolor{shadecolor}{rgb}{0.9,1.0,0.8} %
	\begin{shaded} %
	\textcolor{OliveGreen}{\bf Qüestió \arabic{myc}\\}%
} % 
{ %	
	\end{shaded}
} %

