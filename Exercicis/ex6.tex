\begin{qst}{}
Calcular l'entalpia normal de formació de l'ió \ch{OH^-_{(aq)}} a partir de les següents calors de reacció:
\[
\begin{array}{cc}
\ch{1/2 O2_{(g)} + H2_{(g)} <-> H2O_{(l)}} & \Delta H^{\circ} = -285.9 kJ mol^{-1} \\
\ch{H2O_{(l)} <-> H^+_{(aq)} + OH^-_{(aq)}} & \Delta H^{\circ}=55.9 kJ mol^{-1}
\end{array}
\]
\end{qst}
\lct{
Només cal sumar les dues equacions per obtenir la reacció desitjada de formació dels dos ions, i sumem de la mateixa manera les calors de reacció:
\[
\begin{array}{cc}
\ch{1/2 O2_{(g)} + H2_{(g)} <-> H^+_{(aq)} + OH^-_{(aq)}} & \Delta H^{\circ} = -230.0 kJ mol^{-1} \\
\end{array}
\]
Com que la calor normal de formació de l'ió \ch{H+} és $\Delta H^{\circ}_f=0$, es dedueix que 
$\Delta H^{\circ}_f [\ch{OH^-_{(aq)}}= -230.0 kJ mol^{-1}$
}
