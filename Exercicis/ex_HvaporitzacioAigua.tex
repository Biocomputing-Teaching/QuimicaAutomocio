\begin{exr}{Entalpia de vaporització de l'aigua}
    Determineu l'entalpia de vaporització de l'aigua en condicions estàndard a partir de les següents reaccions:

\[
\ch{H2(g) + 1/2 O2(g) -> H2O(g)} \quad \Delta H^\circ = \qty{-241.8}{\kilo\joule\per\mol}
\]

\[
\ch{H2(g) + 1/2 O2(g) -> H2O(l)} \quad \Delta H^\circ = \qty{-285.8}{\kilo\joule\per\mol}
\]
\end{exr}
\lct{
L'entalpia de vaporització de l'aigua en condicions estàndard es defineix com:

\[
\ch{H2O(l) -> H2O(g)}
\]

Aquesta es pot obtenir restant les dues equacions:

\[
\Delta H_{\text{vap}}^\circ = \Delta H^\circ (\ch{H2O(g)}) - \Delta H^\circ (\ch{H2O(l)})
\]

\[
\Delta H_{\text{vap}}^\circ = \qty{-241.8}{\kilo\joule\per\mol} - (\qty{-285.8}{\kilo\joule\per\mol})
\]

\[
\Delta H_{\text{vap}}^\circ = \qty{44.0}{\kilo\joule\per\mol}
\]
}