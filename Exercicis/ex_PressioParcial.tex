\begin{exr}
    Una mostra de \ch{PCl5}, que pesa \SI{2.69}{\gram}, es va col·locar en un flascó d'\SI{1.00}{\litre} i es va evaporar completament a una temperatura de \SI{25}{\celsius}. La pressió observada a aquesta temperatura va ser \SI{1.00}{\atm}. Existeix la possibilitat que una part del \ch{PCl5} s'hagi dissociat d'acord amb l'equació:

\begin{reaction}
PCl5(g) -> PCl3(g) + Cl2(g)
\end{reaction}

Quines són les pressions parcials del \ch{PCl5}, \ch{PCl3} i \ch{Cl2} en aquestes condicions experimentals? (Adaptat de \cite{mahan_quimica_1997})
\end{exr}
\lct{
    La solució d'aquest problema implica diverses etapes. Per determinar si s'ha dissociat una part del \ch{PCl5}, calculem primerament la pressió que s'hauria observat si no s'hagués dissociat el \ch{PCl5}. Això es pot calcular a partir del nombre de mols de \ch{PCl5} utilitzats, juntament amb el volum i la temperatura del flascó. Com que el pes molecular del \ch{PCl5} és \SI{208}{\gram\per\mole}, el nombre de mols de \ch{PCl5} inicialment presents en el flascó és:

\[
n = \SI{2.69}{\gram}\cdot \frac{1\si{\mole}}{\SI{208}{\gram}} = 0.0129\si{\mole}.
\]

La pressió corresponent a aquest nombre de mols seria:

\[
P = \frac{nRT}{V} = \frac{(0.0129\si{\mole})(\SI{0.082}{\liter\atm\per\mole\per\kelvin})(\SI{523.15}{\kelvin})}{\SI{1.00}{\liter}} = \SI{0.553}{\atm}.
\]

Com que la pressió observada és superior a aquesta, s'ha de produir certa dissociació del \ch{PCl5}. Aplicant la llei de les pressions parcials, podem escriure:

\begin{equation}
P_{\ch{PCl5}} + P_{\ch{PCl3}} + P_{\ch{Cl2}} = P_t = \SI{1.00}{\atm}.
\label{eq:daltonpcl}
\end{equation}

Ara observem que:


Atès que es produeix un mol de \ch{PCl3} i un mol de \ch{Cl2} per cada mol de \ch{PCl5} dissociat,
\[
P_{\ch{Cl2}} = P_{\ch{PCl3}}, \quad P_{\ch{PCl5}} = \SI{0.553}{\atm} - P_{\ch{Cl2}}.
\]
i podem reescriure l'Equació \ref{eq:daltonpcl} com:

\[
\SI{0.553}{\atm} - P_{\ch{Cl2}} + P_{\ch{Cl2}} + P_{\ch{Cl2}} = \SI{1.00}{\atm}.
\]

Resolent, obtenim:

\[
P_{\ch{Cl2}} = \SI{0.447}{\atm},
\]

i

\[
P_{\ch{PCl3}} = \SI{0.447}{\atm}, \quad P_{\ch{PCl5}} = \SI{0.106}{\atm}.
\]
}
