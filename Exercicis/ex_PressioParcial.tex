\begin{qst}
Una mostra de \(\text{PCl}_5\) que pesa 2,69 g va ser col·locada en un flascó d'1,00 litre i evaporada completament a una temperatura de 250 °C. La pressió observada a aquesta temperatura va ser d'1,00 atm. Existeix la possibilitat que una part del \(\text{PCl}_5\) s'hagi dissociat d'acord amb l'equació:

\[
\text{PCl}_5(\text{g}) \rightleftharpoons \text{PCl}_3(\text{g}) + \text{Cl}_2(\text{g}).
\]

Quines són les pressions parcials de \(\text{PCl}_5\), \(\text{PCl}_3\) i \(\text{Cl}_2\) sota aquestes condicions experimentals?
\end{qst}
\lct{
La solució d’aquest problema implica diverses etapes. Per decidir si el \(\text{PCl}_5\) s’ha dissociat, calculem primer la pressió que s’hauria observat si no s’hagués dissociat. Això es pot calcular a partir del nombre de mols de \(\text{PCl}_5\) utilitzats, juntament amb el volum i la temperatura del flascó. Sabent que el pes molecular del \(\text{PCl}_5\) és 208, el nombre de mols de \(\text{PCl}_5\) inicialment presents en el flascó és:

\[
n = \frac{2,69}{208} = 0,0129.
\]

La pressió corresponent a aquest nombre de mols seria:

\[
P = \frac{nRT}{V} = \frac{(0,0129)(0,082)(523)}{1} = 0,553 \; \text{atm}.
\]

Com que la pressió observada és major que aquesta, ha d’haver-hi hagut certa dissociació de \(\text{PCl}_5\). Utilitzant la llei de les pressions parcials podem escriure:

\[
P_{\text{PCl}_5} + P_{\text{PCl}_3} + P_{\text{Cl}_2} = P_T = 1,00 \; \text{atm}.
\]

Ara notem que:

\[
P_{\text{Cl}_2} = P_{\text{PCl}_3}, \quad P_{\text{PCl}_5} = 0,553 - P_{\text{Cl}_2},
\]

ja que es produeix un mol de \(\text{PCl}_3\) i un mol de \(\text{Cl}_2\) cada vegada que es dissocia un mol de \(\text{PCl}_5\). Per tant, podem escriure la llei de Dalton com:

\[
0,553 - P_{\text{Cl}_2} + P_{\text{Cl}_2} + P_{\text{Cl}_2} = 1,00.
\]

\[
P_{\text{Cl}_2} = 0,447 \; \text{atm},
\]

i

\[
P_{\text{PCl}_3} = 0,447 \; \text{atm}, \quad P_{\text{PCl}_5} = 0,106 \; \text{atm}.
\]

}
