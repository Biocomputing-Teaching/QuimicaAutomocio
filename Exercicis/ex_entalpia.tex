\begin{exr}{Entalpia de reacció}
Tenint en compte aquestes energies d'enllaç:
\begin{center}
\begin{tabular}{cc}
& $E_b$ / kJ mol$^{-1}$ \\
\hline
C-O al monòxid de carboni & +1077 \\
C-O al diòxid de carboni & +805 \\
O-H & +464 \\
H-H & +436 \\
\hline
\end{tabular}
\end{center}

Calcula l'entalpia de la reacció:
\ch{CO_{(g)} + H2O_{(g)} -> CO2_{(g)} + H2_{(g)}} 
\end{exr}
\lct{
La reacció donada és:
\begin{align*}
  \ch{CO_{(g)} + H2O_{(g)} -> CO2_{(g)} + H2_{(g)}}
\end{align*}
on el \ch{CO} té un triple enllaç C$\equiv$O i el \ch{CO2} té dos dobles enllaços O=C=O (és a dir, estem parlant d'enllaços diferents amb diferents energies d'enllaç).
Per calcular l'entalpia de reacció, utilitzem les energies d'enllaç proporcionades.

\textbf{Trencament d'enllaços (requereix energia):}
\begin{itemize}
    \item Enllaç \ch{C-O} en \ch{CO}: \qty{1077}{\kJ\per\mol}
    \item Enllaç \ch{O-H} en \ch{H2O}: 2$\times$\qty{464}{\kJ\per\mol} = \qty{928}{\kJ\per\mol}
\end{itemize}

\textbf{Formació d'enllaços (allibera energia):}
\begin{itemize}
    \item Enllaços \ch{C-O} en \ch{CO2}: 2$\times$\qty{805}{\kJ\per\mol} = \qty{1610}{\kJ\per\mol}
    \item Enllaç \ch{H-H} en \ch{H2}: \qty{436}{\kJ\per\mol}
\end{itemize}

El càlcul de l'entalpia alliberada durant la reacció és (notar els signes):
\begin{align*}
    \Delta H &= \text{energia trencament} - \text{energia formació} \\
    &= (\qty{1077}{\kJ\per\mol} + \qty{928}{\kJ\per\mol}) - (\qty{1610}{\kJ\per\mol} + \qty{436}{\kJ\per\mol}) \\
    &= \qty{2005}{\kJ\per\mol} - \qty{2046}{\kJ\per\mol} \\
    &= \qty{-41}{\kJ\per\mol}
\end{align*}
}
