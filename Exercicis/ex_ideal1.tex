\begin{exr}
    Un conductor comprova la pressió dels pneumàtics pel matí aviat, quan la temperatura és de 15\si\degreeCelsius, i és de 1.3$\times$10$^5$ Pa. Al migdia la temperatura és 15 graus més elevada. Quina és la pressió dels pneumàtics ara?.
    \end{exr}

    \lct{
        Les dades són:

        \begin{itemize}
            \item Pressió inicial: \( P_1 = \SI{1.3e5}{\pascal} \)
            \item Temperatura inicial: \( T_1 = \SI{15}{\celsius} = \SI{288}{\kelvin} \)
            \item Temperatura final: \( T_2 = \SI{30}{\celsius} = \SI{303}{\kelvin} \)
            \item Suposem que el volum dels pneumàtics es manté constant.
        \end{itemize}
        
        Com que el volum no canvia, podem utilitzar la llei de Gay-Lussac per determinar la pressió final:
        
        \[
        \frac{P_1}{T_1} = \frac{P_2}{T_2}
        \]
        
        Aïllant \( P_2 \):
        
        \[
        P_2 = P_1 \times \frac{T_2}{T_1} = (\SI{1.3e5}{\pascal}) \times \frac{\SI{303}{\kelvin}}{\SI{288}{\kelvin}}=(\SI{1.3e5}{\pascal}) \times 1.0521=\SI{1.37e5}{\pascal}
        \]
          
    }