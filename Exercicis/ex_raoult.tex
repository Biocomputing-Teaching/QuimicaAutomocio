\begin{exr}
Raona l'efecte que té la no idealitat de les dissolucions segons la Figura \ref{fig:desv_raoult}a a) en el seu punt d'ebullició, i b) en un procés de destil·lació fraccionada.
\end{exr}

\begin{exr}
    Un azeòtrop positiu prové d'una desviació també positiva de la llei de Raoult. a) Dibuixa la corba de Temperatura d'ebullició vs composició per a un azeòtrop positiu basant-te en les Figures \ref{fig:PhaseDiagram}a i \ref{fig:desv_raoult}a. b) Raona el resultat de fer una destil·lació a partir de diverses composicions d'aquesta mescla. c) què succeiria en un azeòtrop negatiu?
    \end{exr}
    \begin{exr}
        Volem separar una barreja equimolar d'etanol i acetat etílic per destil·lació en productes relativament purs. La barreja forma un azeòtrop de mínim punt d'ebullició segons la Figura \ref{fig:AzeotropEX}. No obstant, la composició de l'azeòtrop és sensible a la pressió, mostrant un increment significatiu de la fracció molar de l'etanol quan incrementa la pressió, com es mostra a la Figura. Dibuixa un esquema aproximat per a la separació de les dues components de la barreja que tregui profit d'aquest fet.
        \end{exr}