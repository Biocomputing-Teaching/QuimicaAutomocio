\begin{qst}
La constant de la llei de Henry de l'\ch{O2} en aigua a 25$\degree$C és 1.27·10$^{-3}$ M atm$^{-1}$, i la fracció molar de l'\ch{O2} en l'atmosfera és 0.21. Calcula la solubilitat de l'\ch{O2} en aigua a 25$\degree$C i a pressió atmosfèrica.
\end{qst}
\lct{
Segons la llei de Dalton, la pressió parcial d'un gas en una dissolució de gasos és proporcional a la seva fracció molar: $P_A=x_A P_t$. Per tant, $P_{\ch{O2}}=0.21 \times 1 atm=0.21 atm$. 

A partir de la llei de Henry, la concentració d'oxigen dissolt en les condicions donades és 
\[
[\ch{O2}]=k P_{\ch{O2}} = 1.27 \times 10^{-3}M atm^{-1} \cdot 0.21 atm=2.7 \cdot 10^{-4}M 
\]
}
