\begin{exr}{Fòrmula molecular d'un compost gasós}
    Un compost gasós que se sap que conté només carboni, hidrogen i nitrogen es barreja amb el volum d'oxigen exactament necessari per a la seva combustió completa a \ch{CO2}, \ch{H2O} i \ch{N2}. La combustió de 9 volums de la mescla gasosa produeix 4 volums de \ch{CO2}, 6 volums de vapor d'aigua i 2 volums de \ch{N2}, tots a la mateixa temperatura i pressió.

    Quants volums d'oxigen es necessiten per a la combustió? Quina és la fórmula molecular del compost?
\end{exr}

\lct{
    Sigui la fórmula molecular del compost gasós, \ch{CxHyNz}.
    La seva combustió completa segueix l'equació general:
\[\ch{CxHyNz + $\frac{2x+y/2}{2}$ O_2 -> $x$ CO2 + $\frac{y}{2}$ H2O + $\frac{z}{2}$ N2}\]
    
    Sabem que la combustió de 9 volums de la mescla gasosa produeix:
    \begin{itemize}
        \item 4 volums de \ch{CO2} \(\Rightarrow x=4\).
        \item 6 volums de \ch{H2O} \(\Rightarrow y=12\) (perquè cada mol d'aigua conté 2 àtoms d'hidrogen).
        \item 2 volums de \ch{N2} \(\Rightarrow z=4\) (perquè cada mol de \ch{N2} prové de 2 àtoms de nitrogen).
    \end{itemize}
    
    Dividint tots els valors pel més petit, obtenim la fórmula empírica del compost: 
    $[\ch{C1H3N1}]_n$. Realment no sabem si és \ch{C1H3N1}, \ch{C2H6N2}, \ch{C4H12N4}, etc. Per tal de determinar-ho, mirem quina de les fòrmules moleculars compleix les restriccions del nombre de volums inicials i finals.  
    \begin{itemize}
        \item Suposem que la molècula sigui \ch{C4H12N4}. En aquest cas, la reacció de combustió és:
        \[
        \ch{C4H12N4 + 7 O2 -> 4 CO2 + 6 H2O + 2 N2}
        \]
        que, si agafem els coeficients directament com a nombre de volums ens diu que de 8 volums a reactius es generen 12 volums a productes. Per tant, aquesta opció no compleix les restriccions.
        \item Suposem que sigui \ch{C2H6N2}. En aquest cas, la reacció seria:
        \[ 
            \ch{C2H6N2 + $\frac{7}{2}$ O2 -> 2 CO2 + 3 H2O +  N2}
        \]
        En aquest cas, el nombre de mols a esquerra i dreta és, respectivament, 4.5 i 6 que sí compleixen la proporció de volums donada en l'enunciat. 
    \end{itemize}
    Per tant, la fórmula molecular del compost és \ch{C2H6N2}.
    
    Per tal que el nombre de volums de productes sigui 12, com ens demana l'enunciat, hem de multiplicar per 2 tota l'expressió i així trobem els volums que hem de tenir d'oxigen:
    \[
        \ch{2 C2H6N2 + 7 O2 -> 4 CO2 + 6 H2O + 2 N2}
        \]

    És a dir, que per obtenir 12 volums de productes a partir de 9 de reactius, la proporció compost:oxigen ha de ser 2:7. Necessitem, doncs, 7 volums d'oxigen i 2 de compost per fer la reacció amb les dades donades.
}
