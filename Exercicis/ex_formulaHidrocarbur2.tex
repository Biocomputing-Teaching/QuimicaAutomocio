\begin{exr}{Formula empírica d'un compost \ch{CxHyNz}}
    Durant l'anàlisi per combustió d'un compost desconegut que conté només carboni, hidrogen i nitrogen, es van mesurar 12,923 g de diòxid de carboni (\ch{CO2}) i 6,608 g d'aigua (\ch{H2O}).  
    El tractament del nitrogen amb gas \ch{H2} va donar com a resultat 2,501 g d'amoníac (\ch{NH3}).  
    La combustió completa de 11,014 g del compost va necessitar 10,573 g d'oxigen (\ch{O2}).  Quina és la fórmula empírica del compost?
\end{exr}
\lct{
Escrivim primer les equacions químiques:

\ch{CxHyNz + $\left(\frac{2x+y/2}{2}\right)$ O2 -> $x$ CO2 + $\frac{y}{2}$ H2O + $\frac{z}{2}$ N2}

\ch{N2 + 3 H2 -> 2 NH3}

L'important és veure que tot el carboni (en forma de \ch{CO2}) que apareix a la primera equació prové del contingut al compost problema, com també passa amb tot l'hidrogen (en forma de \ch{H2O}) i tot el nitrogen en forma de \ch{NH3}. Per tant, només ens cal trobar els mols generats de cadascun dels tres elements i la seva proporció en els productes serà la que existia als reactius.

Càlcul del nombre de mols de carboni
\[
12,923 \text{ g } \ch{CO2} 
\times \frac{1 \text{ mol } \ch{CO2}}{44,011 \text{ g } \ch{CO2}}
\times \frac{1 \text{ mol } C}{1 \text{ mol } \ch{CO2}}
\]

\[
= \frac{12,923}{44,011} = 0,29363 \text{ mols de } C
\]

Càlcul del nombre de mols d'hidrogen (notar que la segona reacció és a part de la primera; en la segona reacció usem un excés d'hidrogen {\em addicional} per transformar el nitrogen en un altre compost: amoníac):
\[
6,608 \text{ g } \ch{H2O} 
\times \frac{1 \text{ mol } \ch{H2O}}{18,02 \text{ g } \ch{H2O}}
\times \frac{2 \text{ mols } H}{1 \text{ mol } \ch{H2O}}
\]

\[
= \frac{6,608 \times 2}{18,02} = 0,7334 \text{ mols de } H
\]

Càlcul del nombre de mols de nitrogen:
\[
2,501 \text{ g } \ch{NH3} 
\times \frac{1 \text{ mol } \ch{NH3}}{17,04 \text{ g } \ch{NH3}}
\times \frac{1 \text{ mol } N}{1 \text{ mol } \ch{NH3}}
\]

\[
= \frac{2,501}{17,04} = 0,1468 \text{ mols de } N
\]


Dividim tots els valors entre el menor nombre de mols (0,1468):

\[
\frac{0,29363}{0,1468} = 2 \quad \text{(mol C)}
\]

\[
\frac{0,7334}{0,1468} = 5 \quad \text{(mol H)}
\]

\[
\frac{0,1468}{0,1468} = 1 \quad \text{(mol N)}
\]

La fórmula empírica del compost és \ch{C2H5N}, que pot correspondre a diversos isòmers:
\renewcommand{\arraystretch}{2}
\scriptsize
\begin{longtable}{ccc}
    \toprule
    aziridina& vinilamina& etanimina\\
    \midrule
    \chemfig{--[:240]\mcfbelow{N}{H}(-[:120])}&
    \chemfig{H_2N-[:330,,2]=[:30]}&
    \chemfig{-[:330]=[:30,,,1]NH} 
\\
\end{longtable}
\normalsize
\renewcommand{\arraystretch}{1}

}
