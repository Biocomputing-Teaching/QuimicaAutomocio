\begin{exr}{Fracció metà en una mescla}
Una barreja de metà \ch{CH4} i d'acetilè \ch{C2H2} ocupava un cert volum a una pressió total de 63 mmHg. La mostra es va cremar a \ch{CO2} i \ch{H2O}. Se'n va recollir el \ch{CO2} en el mateix volum inicial i la mateixa temperatura inicial, i es va veure que la seva pressió era de 96 mmHg. Quina era la fracció de metà a la mescla de gasos inicials?
\end{exr}
\lct{

    \textbf{OPCIÓ 1:\\}
Definim $x$ com la fracció molar de metà (\ch{CH4}) i $y$ com la fracció molar d’acetilè (\ch{C2H2}):  
        \[
        x + y = 1
        \]
   
    Les reaccions de combustió són:
    \begin{eqnarray}
        \ch{CH4 + 2 O2 -> CO2 + 2 H2O}  \label{Eq:meta} \\
        \ch{C2H2 + 5/2 O2 -> 2 CO2 + H2O}  \label{Eq:acetile}
    \end{eqnarray}
on veiem que 1 mol de \ch{CH4} produeix 1 mol de \ch{CO2} i que 1 mol de \ch{C2H2} produeix 2 mols de \ch{CO2}. Si tenim un nombre total de mols $n$, llavors:
    \begin{itemize}
        \item Mols de metà: $xn$
        \item Mols d'acetilè: $yn$
    \end{itemize}
    
    Els mols de \ch{CO2} formats són:
    \[
    n_{\ch{CO2}} = xn + 2yn
    \]
     
    Com que el volum i la temperatura es mantenen constants, segons la llei dels gasos ideals la pressió és directament proporcional als mols:
        \[
    P_{\ch{CO2}} = (xn + 2yn) \cdot \frac{P_{\text{total}}}{n}
    \]
    
    Substituint els valors donats:
        \[
    96 = (x + 2y) \cdot 63
    \]
    
   d'on
    \[
    x + 2y = \frac{32}{21}
    \]
    
   Ara ja podem resoldre el sistema:
        \begin{align}
        x + y &= 1 \\
        x + 2y &= \frac{32}{21}
    \end{align}
     i obtenim   
    \[
    x = 1 - \frac{11}{21} = \frac{10}{21}
    \]
    
    Per tant, la fracció de metà en la mescla inicial és:
        \[
    \frac{10}{21} \approx 0.476 \quad \text{o} \quad 47.6\%
    \]

    \textbf{OPCIÓ 2:\\}
Definim les pressions parcials inicials com $P_{\ch{CH4}}^i=x$ i $P_{\ch{C2H2}}^i=y$. Segons les Equacions \ref{Eq:meta} i \ref{Eq:acetile}, les pressions finals seran $P_{\ch{CH4}}^f=x$ i $P_{\ch{C2H2}}^f=2y$. Sabent que la pressió total inicial és \qty{63}{\mmHg} i que la final és \qty{96}{\mmHg}, obtenim el sistema:

\[
\systeme{x+y=63,x+2y=96}
\]
Solucionant-lo, obtenim que $x=\qty{30}{\mmHg}$ i $y=\qty{33}{\mmHg}$ 
que impliquen fraccions molars inicial de $\chi_{\ch{CH4}}^i=30/63=0.476$ i $\chi_{\ch{C2H2}}^i=33/63=0.523$.

Notar que usar la llei dels gasos ideals i, per tant, observar que podem treballar indistintament amb nombre de mols o pressions parcials sempre que la reacció involcri gasos, simplifica la resolució de l'exercici.


}
