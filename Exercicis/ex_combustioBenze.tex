\begin{exr}{Combustió del benzè}
    Si 8,20 g de \ch{C6H6} (benzè) es combinen amb oxigen en una reacció de combustió, quants grams de \ch{H2O} es produiran?
\end{exr}
\lct{
    Equació química equilibrada:
    \[
    \ch{2 C6H6 + 15 O2 -> 12 CO2 + 6 H2O}
    \]
 
    \begin{align*}
    \text{Massa molar de } \ch{C6H6} &= 6(12,01) + 6(1,008) = 78,11 \, \text{g/mol} \\
    \text{Massa molar de } \ch{H2O} &= 2(1,008) + 16,00 = 18,016 \, \text{g/mol}
    \end{align*}
 
          
        \[
        8,20 \cancel{\text{ g } \ch{C6H6}} 
        \times \frac{1 \cancel{\text{ mol } \ch{C6H6}}}{78,11\cancel{ \text{ g } \ch{C6H6}}}
        \times \frac{6 \cancel{\text{ mols } \ch{H2O}}}{2 \cancel{\text{ mols } \ch{C6H6}}}
        \times \frac{18,016 \text{ g } \ch{H2O}}{1 \cancel{\text{ mol } \ch{H2O}}} = 5,68 \text{ g } \ch{H2O}
        \]
}
