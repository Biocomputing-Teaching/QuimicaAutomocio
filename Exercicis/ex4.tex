\begin{qst}
La $T$ de congelació del benzè pur és 5.40$\degree$C. 
Quan es dissol 1.15g de naftalè en 100 g de benzè. la dissolució resultant té un punt de congelació de 4.95$\degree$C.
Si la constant de descens molal del punt de congelació del benzè és 5.12$\degree$C, quin és el pes molecular del naftalè?
\end{qst}
\lct{
La molalitat de la dissolució és 
\[
m=\frac{\Delta T}{K_f} = \frac{5.40-4.95}{5.12}=0.088
\]
A partir de la quantitat de naftalè i la molalitat:
\[
\frac{1.15 {\rm \, g \, naftalè}}{100 {\rm \, g \, dissolvent}} \cdot \frac{1000 {\rm \, g \, dissolvent}}{0.088 {\rm \, mol \, naftalè}} = 130 \frac{\rm g}{\rm mol}
\]
}
