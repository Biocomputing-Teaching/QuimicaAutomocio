

\begin{exr}
    Després de la combustió amb excés d'oxigen, 12,501 g d'un compost petroquímic van produir 38,196 g de diòxid de carboni i 18,752 g d'aigua. Una anàlisi prèvia va determinar que el compost no conté oxigen. Estableix la seva fórmula empírica.
\end{exr}
\lct{

Sabem que el compost només conté carboni i hidrogen. L'objectiu és determinar les masses d'aquests elements i trobar la seva relació molar.

Cada mol de \ch{CO2} conté 1 mol de carboni, per tant, utilitzem un factor de conversió:

\[
\text{Massa molar de } \ch{CO2} = 12,01 + 2(16,00) = 44,01 \text{ g/mol}
\]

\[
38,196 \cancel{\text{ g } \ch{CO2} }
\times \frac{1}{\cancel{44,01} \text{ g } \ch{CO2}}
\times \frac{1 \text{ mol } C}{1 \text{ mol } \ch{CO2}}
\times \frac{12,01 \text{ g } C}{1 \cancel{\text{ mol } C}}
\]

\[
\frac{38,196 \times 12,01}{44,01} = 10,426 \text{ g de } C
\]


Cada mol de \ch{H2O} conté 2 mols d'hidrogen:

\[
\text{Massa molar de } \ch{H2O} = 2(1,008) + 16,00 = 18,016 \text{ g/mol}
\]


\[
18,752 \cancel{\text{ g } \ch{H2O}} 
\times \frac{1}{\cancel{18,016} \text{ g } \ch{H2O}}
\times \frac{2 \text{ mols } H}{1 \text{ mol } \ch{H2O}}
\times \frac{1,008 \text{ g } H}{1 \cancel{\text{ mol } H}}
\]

\[
\frac{18,752 \times 2 \times 1,008}{18,016} = 2,100 \text{ g de } H
\]


\[
\text{Massa total calculada} = 10,426 \text{ g C} + 2,100 \text{ g H} = 12,526 \text{ g}
\]
Com que el valor inicial és de 12,501 g, la diferència es deu a errors d'arrodoniment.

Càlcul de la relació molar:
\[
\frac{10,426 \text{ g C}}{12,01 \text{ g/mol}} = 0,868 \text{ mols C}
\]
\[
\frac{2,100 \text{ g H}}{1,008 \text{ g/mol}} = 2,083 \text{ mols H}
\]

Determinació de la fórmula empírica: 

\[
\frac{0,868}{0,868} = 1
\]
\[
\frac{2,083}{0,868} = 2,4
\]

Per obtenir nombres enters, multipliquem per 5, i obtenim la fórmula empírica del compost: \ch{C5H12} (pentà).

}
