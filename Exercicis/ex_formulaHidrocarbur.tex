\begin{exr}{Fòrmula empírica d'un compost petroquímic}
    Després de la combustió amb excés d'oxigen, 12,501 g d'un compost petroquímic van produir 38,196 g de diòxid de carboni i 18,752 g d'aigua. Una anàlisi prèvia va determinar que el compost no conté oxigen. Estableix la seva fórmula empírica.
\end{exr}
\lct{

Sabem que el compost només conté carboni i hidrogen. L'objectiu és determinar les masses d'aquests elements i trobar la seva relació molar.

Cada mol de \ch{CO2} conté 1 mol de carboni, per tant, utilitzem un factor de conversió:

\[
\text{Massa molar de } \ch{CO2} = 12,01 + 2(16,00) = 44,01 \text{ g/mol}
\]

\[
38,196 \cancel{\text{ g } \ch{CO2} }
\times \frac{1 \cancel{\text{ mol } \ch{CO2}}}{44,01\cancel{ \text{ g } \ch{CO2}}}
\times \frac{1 \cancel{\text{ mol } C}}{1 \cancel{\text{ mol } \ch{CO2}}}
\times \frac{12,01 \text{ g } C}{1 \cancel{\text{ mol } C}}=10,426 \text{ g de } C
\]

Cada mol de \ch{H2O} conté 2 mols d'hidrogen:

\[
\text{Massa molar de } \ch{H2O} = 2(1,008) + 16,00 = 18,016 \text{ g/mol}
\]


\[
18,752 \cancel{\text{ g } \ch{H2O}} 
\times \frac{1\cancel{\text{ mol } \ch{H2O}}}{18,016\cancel{ \text{ g } \ch{H2O}}}
\times \frac{2 \cancel{\text{ mols } H}}{1 \cancel{\text{ mol } \ch{H2O}}}
\times \frac{1,008 \text{ g } H}{1 \cancel{\text{ mol } H}}
=2,100 \text{ g de } H
\]


\[
\text{Massa total calculada} = 10,426 \text{ g C} + 2,100 \text{ g H} = 12,526 \text{ g}
\]
Podem comprovar que el pes de \ch{C} i \ch{H} en els productes iguala el pes dels mateixos elements en els reactius. Com que el valor inicial és de 12,501 g, la diferència es deu a errors d'arrodoniment.

Ara ens interessa veure en quina proporció estan els mols de \ch{C} i \ch{H} en el compost inicial:
\[
\frac{10,426 \text{ g C}}{12,01 \text{ g/mol}} = 0,868 \text{ mols C}
\]
\[
\frac{2,100 \text{ g H}}{1,008 \text{ g/mol}} = 2,083 \text{ mols H}
\]

a partir d'aquests valors podem calcular la fórmula empírica, dividint per qualsevol dels dos i aleshores fent que els valors obtinguts siguin nombres enters: 

\[
\frac{0,868}{0,868} = 1
\]
\[
\frac{2,083}{0,868} = 2,4
\]

Per obtenir nombres enters, multipliquem per 5, i obtenim la fórmula empírica del compost: \ch{C5H12} (pentà). No sabem en quina forma es presentarà, però, el pentà de totes les mostrades a la taula:

\scriptsize
\begin{longtable}{ccc}
    \toprule
    \emph{n}-pentà& \emph{iso}-pentà& \emph{neo}-pentà\\
    \midrule
\definesubmol\xx{C(-[::+90]H)(-[::-90]H)}
\chemfig{H-!\xx-!\xx-!\xx-!\xx-!\xx-H}&
\chemfig{[7]H_3C-CH(-[6]CH_3)-[1]CH_2-CH_3}&
\chemfig{CH_3-C(-[2]CH_3)(-[6]CH_3)-CH_3} 
\\
\bottomrule
\end{longtable}
\normalsize
}
