    \begin{exr}
        Dalt de l'Everest, la pressió atmosfèrica és de 0,33 atm i la temperatura de 50 sota zero. Quina és la densitat de l'aire si en CN és de 1.29\si{\gram\per\deci\meter\tothe{3}}?.
        \end{exr}
    \lct{


Sabem que la densitat de l'aire en condicions normals (CN) és:

\[
\rho_{\text{CN}} = \SI{1.29}{\gram\per\deci\meter\cubed}
\]

Les condicions a dalt de l’Everest són:

\begin{itemize}
    \item Pressió atmosfèrica: \( P = \SI{0.33}{\atm} \)
    \item Temperatura: \( T = \SI{-50}{\celsius} = \SI{223}{\kelvin} \)
    \item Condicions normals (CN):
    \begin{itemize}
        \item Pressió normal: \( P_{\text{CN}} = \SI{1}{\atm} \)
        \item Temperatura normal: \( T_{\text{CN}} = \SI{273}{\kelvin} \)
    \end{itemize}
\end{itemize}

Sabem que la densitat d'un gas està relacionada amb la pressió i la temperatura segons l'expressió:

\[
\frac{\rho}{\rho_{\text{CN}}} = \frac{P}{P_{\text{CN}}} \times \frac{T_{\text{CN}}}{T}
\]

Aïllant \( \rho \):

\[
\rho = \rho_{\text{CN}} \times \frac{P}{P_{\text{CN}}} \times \frac{T_{\text{CN}}}{T}
\]

Substituïm els valors donats:

\[
\rho = (\SI{1.29}{\gram\per\deci\meter\cubed}) \times \frac{\SI{0.33}{\atm}}{\SI{1}{\atm}} \times \frac{\SI{273}{\kelvin}}{\SI{223}{\kelvin}}=\SI{0.52}{\gram\per\deci\meter\cubed}
\]

    }