\begin{exr}{Relació $\frac{C_P}{C_V}$}
Perquè hi ha diferències entre els quocients de capacitat calorífica ($C_P/C_V$) de gasos monoatòmics respecte els diatòmics? (Adona't que si un gas monoatòmic ideal, pel fet d'estar només augmentant la seva energia cinètica translacional té una $C_V=\frac{3}{2}R$, es pot entendre que per a cada component (eix) necessita $\frac{1}{2}R$)
\end{exr}
\lct{
    Els quocients de la capacitat calorífica dels gasos diatòmics són molt menors que 1,67, i hem d'esbrinar la raó d'aquestes desviacions.

    Primerament, notem que $C_V$, la capacitat calorífica deguda al moviment de translació de les molècules, és igual a $\frac{3}{2}R$, i que hi ha tres components independents de velocitat associats amb el moviment de translació. Per tant, podem inferir que cadascun dels tres moviments de translació independents contribueix amb $\frac{1}{2}R$ a la capacitat calorífica molar. Sobre aquesta base, podríem esperar que, si algun altre tipus de moviment fos accessible a les molècules de gas, hi hauria més contribucions a la capacitat molar i aquestes entrarien en unitats de $\frac{1}{2}R$.
    
   A més de tenir els tres moviments de translació, una molècula diatòmica pot rotar al voltant del seu centre de massa segons dos modes mútuament perpendiculars i independents. Assignant $\frac{1}{2}R$ com la contribució de cadascun d'aquests moviments a la capacitat calorífica, tenim:
    
    \[
    C_V = \underbrace{\frac{3}{2}R}_{\text{traslació}} + \underbrace{\frac{1}{2}R + \frac{1}{2}R}_{\text{rotació}} = \frac{5}{2}R,
    \]
    
    \[
    C_P = C_V + R = \frac{7}{2}R,
    \]
    
    \[
    \frac{C_P}{C_V} = \frac{\frac{7}{2}R}{\frac{5}{2}R} = \frac{7}{5} = 1,40.
    \]
}
