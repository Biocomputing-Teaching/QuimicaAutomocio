\begin{qst}{}
Fem una dissolució barrejant dos mols de metanol amb un mol d'etanol a una $T$ donada. Si la $P_v$ del metanol pur a aquesta $T$ és de 81 kPa, i la de l'etanol pur a la mateixa $T$ és de 45 kPa,
quina és la pressió de vapor de la barreja, assumint que és una dissolució ideal?
\end{qst}
\lct{
Hi ha 3 mols totals. Les fraccions molars són fàcilment calculables:
\[
x_{metanol}=2/3
\]
i 
\[
x_{etanol}=1/3
\]
Segons la llei de Raoult, la contribució a la pressió de vapor total de cada component ve donada pel producte de la seva fracció molar per la pressió de vapor de la substància pura:
\[
P_{metanol}=x_{metanol} \times P_{metanol}^0 = 2/3 \times 81 kPa = 54 kPa
\]
\[
P_{etanol}=x_{etanol} \times P_{etanol}^0 = 1/3 \times 45 kPa = 15 kPa
\]
Per tant, $P_{total}=P_{metanol}+P_{etanol}=54 kPa + 15 kPa = 69 kPa$
}
