\documentclass[12pt]{article}

\usepackage[utf8]{inputenc}
\usepackage[catalan]{babel}
\usepackage{chemformula}
\usepackage{siunitx}

% mates
\usepackage{cancel}


\usepackage{framed}
\newcounter{myc}
%environment for exercises in the class notes
\newenvironment{exr}{ % 
    \addtocounter{myc}{1}
	\definecolor{shadecolor}{rgb}{0.9,1.0,0.8} %
	\begin{shaded} %
	\textcolor{OliveGreen}{\bf Exercici \arabic{myc}\\}%
} % 
{ %	
	\end{shaded}
} %
\usepackage[dvipsnames,table]{xcolor}
\newenvironment{qst}{ % 
    \addtocounter{myc}{1}
	\definecolor{shadecolor}{rgb}{0.9,1.0,0.8} %
	\begin{shaded} %
	\textcolor{OliveGreen}{\bf Qüestió \arabic{myc}\\}%
} % 
{ %	
	\end{shaded}
} %

%%%%%%%%%%%%%%%%%%%%%%%%%%%%%%%%%%%%%%%%%
%%%%%%%%%%%%%%%%%%%%%%%%%%%%%%%%%%%%%%%%%
% lecturer or student text
% in principle the lecturer text includes some examples to be done in the c lass
\usepackage{ifthen}
\newboolean{LECT}
\setboolean{LECT}{false}
\setboolean{LECT}{true}
%%%%%%%%%%%%%%%%%%%%%%%%%%%%%%%%%%%%%%%%%
%%%%%%%%%%%%%%%%%%%%%%%%%%%%%%%%%%%%%%%%%

\newenvironment{lect}{ % 
	% \definecolor{shadecolor}{rgb}{1.0,0.8,0.8} %
	% \begin{shaded} %
	% \textcolor{BrickRed}{\bf Resultat\\}%

} % 
{ %	
	% \end{shaded}
} %

\newcommand{\lct}[1]{\ifthenelse{\boolean{LECT}}{\begin{lect} #1 \end{lect}}{}}


% header and footer
\usepackage{fancyhdr}
\pagestyle{fancy}       %
\lhead{\bf Química GEA-17UV}                %
\chead{}                %
\rhead{\bf Grau d'Enginyeria de l'Automoció}                %
\fancyfoot[R]{\thepage}
\lfoot{Exercicis resolts}
\cfoot{\today}                %


\title{Química Enginyeria de l'Automoció: Exercicis}
\date{Març 2018}
\author{Jordi Vill\`a i Freixa (jordi.villa@uvic.cat)}

\begin{document}


\begin{exr}{Pressió parcial \ch{PCl5} en una mescla}
    Una mostra de \ch{PCl5}, que pesa \SI{2.69}{\gram}, es va col·locar en un flascó d'\SI{1.00}{\litre} i es va evaporar completament a una temperatura de \SI{250}{\celsius}. La pressió observada a aquesta temperatura va ser \SI{1.00}{\atm}. Existeix la possibilitat que una part del \ch{PCl5} s'hagi dissociat d'acord amb l'equació:

\begin{reaction}
PCl5(g) -> PCl3(g) + Cl2(g)
\end{reaction}

Quines són les pressions parcials del \ch{PCl5}, \ch{PCl3} i \ch{Cl2} en aquestes condicions experimentals? (Adaptat de \cite{mahan_quimica_1997})
\end{exr}
\lct{
    La solució d'aquest problema implica diverses etapes. Per determinar si s'ha dissociat una part del \ch{PCl5}, calculem primerament la pressió que s'hauria observat si no s'hagués dissociat el \ch{PCl5}. Això es pot calcular a partir del nombre de mols de \ch{PCl5} utilitzats, juntament amb el volum i la temperatura del flascó. Com que el pes molecular del \ch{PCl5} és \SI{208}{\gram\per\mole}, el nombre de mols de \ch{PCl5} inicialment presents en el flascó és:

\[
n = \SI{2.69}{\gram}\cdot \frac{1\si{\mole}}{\SI{208}{\gram}} = 0.0129\si{\mole}.
\]

La pressió corresponent a aquest nombre de mols seria:

\[
P = \frac{nRT}{V} = \frac{(0.0129\si{\mole})(\SI{0.082}{\liter\atm\per\mole\per\kelvin})(\SI{523.15}{\kelvin})}{\SI{1.00}{\liter}} = \SI{0.553}{\atm}.
\]

Com que la pressió observada és superior a aquesta, s'ha de produir certa dissociació del \ch{PCl5}. Aplicant la llei de les pressions parcials, podem escriure:

\begin{equation}
P_{\ch{PCl5}} + P_{\ch{PCl3}} + P_{\ch{Cl2}} = P_t = \SI{1.00}{\atm}.
\label{eq:daltonpcl}
\end{equation}

Ara observem que:


Atès que es produeix un mol de \ch{PCl3} i un mol de \ch{Cl2} per cada mol de \ch{PCl5} dissociat,
\[
P_{\ch{Cl2}} = P_{\ch{PCl3}}, \quad P_{\ch{PCl5}} = \SI{0.553}{\atm} - P_{\ch{Cl2}}.
\]
i podem reescriure l'Equació \ref{eq:daltonpcl} com:

\[
\SI{0.553}{\atm} - P_{\ch{Cl2}} + P_{\ch{Cl2}} + P_{\ch{Cl2}} = \SI{1.00}{\atm}.
\]

Resolent, obtenim:

\[
P_{\ch{Cl2}} = \SI{0.447}{\atm},
\]

i

\[
P_{\ch{PCl3}} = \SI{0.447}{\atm}, \quad P_{\ch{PCl5}} = \SI{0.106}{\atm}.
\]
}

\begin{qst}
Fem una dissolució barrejant dos mols de metanol amb un mol d'etanol a una $T$ donada. Si la $P_v$ del metanol pur a aquesta $T$ és de 81 kPa, i la de l'etanol pur a la mateixa $T$ és de 45 kPa,
quina és la pressió de vapor de la barreja, assumint que és una dissolució ideal?
\end{qst}
\lct{
Hi ha 3 mols totals. Les fraccions molars són fàcilment calculables:
\[
x_{metanol}=2/3
\]
i 
\[
x_{etanol}=1/3
\]
Segons la llei de Raoult, la contribució a la pressió de vapor total de cada component ve donada pel producte de la seva fracció molar per la pressió de vapor de la substància pura:
\[
P_{metanol}=x_{metanol} \times P_{metanol}^0 = 2/3 \times 81 kPa = 54 kPa
\]
\[
P_{etanol}=x_{etanol} \times P_{etanol}^0 = 1/3 \times 45 kPa = 15 kPa
\]
Per tant, $P_{total}=P_{metanol}+P_{etanol}=54 kPa + 15 kPa = 69 kPa$
}

\begin{qst}
Perquè es generen bombolles de seguida que obrim una ampolla d'aigua amb gas?
\end{qst}
\lct{
Segons la llei de Henry, la concentració de gas en un líquid és proporcional a la pressió parcial externa d'aquest gas. Quan s'embotella  aigua carbonatada es fa a una pressió superior a 1 atm. Quan obrim l'ampolla, la pressió es redueix fins a 1 atm, i per tant el \ch{CO2} no és tan soluble i escapa de la dissolució formant bombolles.
}
\begin{qst}{}
La constant de la llei de Henry de l'\ch{O2} en aigua a 25\si\degreeCelsius és 1.27·10$^{-3}$ M atm$^{-1}$, i la fracció molar de l'\ch{O2} en l'atmosfera és 0.21. Calcula la solubilitat de l'\ch{O2} en aigua a 25\si\degreeCelsius i a pressió atmosfèrica.
\end{qst}
\lct{
Segons la llei de Dalton, la pressió parcial d'un gas en una dissolució de gasos és proporcional a la seva fracció molar: $P_A=x_A P_t$. Per tant, $P_{\ch{O2}}=0.21 \times 1 atm=0.21 atm$. 

A partir de la llei de Henry, la concentració d'oxigen dissolt en les condicions donades és 
\[
[\ch{O2}]=k P_{\ch{O2}} = 1.27 \times 10^{-3}M atm^{-1} \cdot 0.21 atm=2.7 \cdot 10^{-4}M 
\]
}

\begin{qst}
La $T$ de congelació del benzè pur és 5.40\si\degreeCelsius. 
Quan es dissol 1.15g de naftalè en 100 g de benzè. la dissolució resultant té un punt de congelació de 4.95\si\degreeCelsius.
Si la constant de descens molal del punt de congelació del benzè és 5.12\si\degreeCelsius, quin és el pes molecular del naftalè?
\end{qst}
\lct{
La molalitat de la dissolució és 
\[
m=\frac{\Delta T}{K_f} = \frac{5.40-4.95}{5.12}=0.088
\]
A partir de la quantitat de naftalè i la molalitat:
\[
\frac{1.15 {\rm \, g \, naftalè}}{100 {\rm \, g \, dissolvent}} \cdot \frac{1000 {\rm \, g \, dissolvent}}{0.088 {\rm \, mol \, naftalè}} = 130 \frac{\rm g}{\rm mol}
\]
}

\begin{qst}
La composició percentual, en massa, de l'aire sec al nivell del mar és, aproximadament, \ch{N2}/\ch{O2}/\ch{Ar}=75.5/23.2/1.3. 
Quina és la pressió parcial de cada component quan la pressió total és 1.20 atm?
(extret de \cite{Atkins2018}).
\end{qst}
\lct{
En 100gr d'aire tindrem 75.5, 23.2 i 1.3 gr de \ch{N2}, \ch{O2} i \ch{Ar}, respectivament. Podem calcular la seva fracció molar calculant el número de mols de cadascun i dividint pel total. Després, només cal multiplicar per la pressió corresponent i sabrem la pressió parcial de cada component:

\[n_{\ch{N2}}=75.5 \cancel{g} \cdot \frac{1 mol}{28.02 \cancel{g}}=2.69 mol\]
\[n_{\ch{O2}}=23.2 \cancel{g} \cdot \frac{1 mol}{32.00 \cancel{g}}=0.725 mol\]
\[n_{\ch{Ar}}=1.3 \cancel{g} \cdot \frac{1 mol}{39.95 \cancel{g}}=0.033 mol\]

    \begin{tabular}{cccc}
      \hline
        & \ch{N2} & \ch{O2} & \ch{Ar} \\
      \hline
Fracció molar &	0.780 & 0.210 & 0.0096\\
Pressió parcial (nivell del mar)/atm & 0.780 & 0.210 & 0.0096\\
Pressió parcial ($P_T=1.20$atm))/atm & 0.936 & 0.252 & 0.012\\
      \hline
    \end{tabular}

}

\begin{qst}{}
Calcular l'entalpia normal de formació de l'ió \ch{OH^-_{(aq)}} a partir de les següents calors de reacció:
\[
\begin{array}{cc}
\ch{1/2 O2_{(g)} + H2_{(g)} <-> H2O_{(l)}} & \Delta H^{\circ} = -285.9 kJ mol^{-1} \\
\ch{H2O_{(l)} <-> H^+_{(aq)} + OH^-_{(aq)}} & \Delta H^{\circ}=55.9 kJ mol^{-1}
\end{array}
\]
\end{qst}
\lct{
Només cal sumar les dues equacions per obtenir la reacció desitjada de formació dels dos ions, i sumem de la mateixa manera les calors de reacció:
\[
\begin{array}{cc}
\ch{1/2 O2_{(g)} + H2_{(g)} <-> H^+_{(aq)} + OH^-_{(aq)}} & \Delta H^{\circ} = -230.0 kJ mol^{-1} \\
\end{array}
\]
Com que la calor normal de formació de l'ió \ch{H+} és $\Delta H^{\circ}_f=0$, es dedueix que 
$\Delta H^{\circ}_f [\ch{OH^-_{(aq)}}= -230.0 kJ mol^{-1}$
}


\end{document}
