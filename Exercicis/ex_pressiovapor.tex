\begin{exr}{}
    Raona com canvia la $p_v$ d'una dissolució en funció de la seva concentració.
    \end{exr}
    
    \begin{exr}{}
    Determina la relació entre l'increment de pressió de vapor d'una dissolució i la fracció molar del solut.
    \end{exr}
    
    \begin{exr}{}
    La pressió de vapor de l'aigua a \qty{20}{\celsius} és \qty{17.54}{\mmHg}. Quan dissolem \qty{114}{\gram} de sucrosa en \qty{1000}{\gram} d'aigua, la pressió de vapor es redueix en \qty{0.11}{\mmHg}. Quin és el pes molecular de la sucrosa?
    \end{exr}
