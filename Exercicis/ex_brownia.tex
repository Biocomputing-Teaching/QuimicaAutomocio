\begin{exr}
Usant R, prova d'executar aquest script que mostra com simular el moviment Brownià d'una partícula en un líquid (extret de \url{http://www.phytools.org/eqg/phytools/}):
\begin{lstlisting}[language=R]
t <- 0:100  # temps de simulacio
sig2 <- 0.01
## primer, calcula un conjunt de desviacions aleatòries puntuals
x <- rnorm(n = length(t) - 1, sd = sqrt(sig2))
## després, acumula'n els resultats
x <- c(0, cumsum(x))
plot(t, x, type = "l", ylim = c(-2, 2))
\end{lstlisting}

Després, executa el següent script, que produeix 10000 simulacions diferents:

\begin{lstlisting}[language=R]
nsim <- 1000
## creo una matriu que hostatgi totes les simulacions
X <- matrix(0, nsim, length(t))
for (i in 1:nsim) X[i, ] <- c(0, cumsum(rnorm(n = length(t) - 1, sd = sqrt(sig2))))
plot(t, X[1, ], xlab = "temps", ylab = "desviacions", ylim = c(-2, 2), type = "l")
for (i in 1:nsim) lines(t, X[i, ])
\end{lstlisting}
\end{exr}