\begin{exr}{Energia interna de la combustió del propà}
Determinar la variació d'energia interna per al procés de combustió d'\qty{1}{\mol} de propà a \qty{25}{\celsius} i \qty{1}{\atm}, si la variació d'entalpia, en aquestes condicions, val \qty{-2219.8}{\kilo\joule}.
\end{exr}

\lct{
    Escrivim primer la reacció igualada
\[
\ch{C3H8(g) + 5 O2(g) -> 3 CO2(g) + 4 H2O(l)}
\]

\[
\Delta H = \qty{-2219.8}{\kilo\joule}
\]

\[
n_{\text{reactius}} = 1 + 5 = 6 ; \quad n_{\text{productes}} = 3 \quad (\text{mols de gasos}) \Rightarrow \Delta n = -3
\]

\[
\Delta U = \Delta H - \Delta n\ R\ T = \qty{-2219.8}{\kilo\joule} + \qty{3}{\mol} \times \qty{8.3}{\joule\per\mol\per\kelvin} \times \qty{298}{\kelvin}= \qty{-2214}{\kilo\joule}
\]
}