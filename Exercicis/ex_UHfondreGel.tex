\begin{exr}{Fonent gel}
Calcula l'increment d'energia i d'entalpia en fondre 1 mol de gel. Els volums molars del gel i l'aigua són 0.0196 L/mol i 0.0180 L/mol, respectivament. La calor de fusió de l'aigua és $\Delta H_f = 6.01$ kJ/mol.
\end{exr}

\lct{
L'increment d'entalpia ($\Delta H$) en fondre 1 mol de gel és simplement la calor de fusió:
\[
\Delta H = \Delta H_f = 6.01 \text{ kJ/mol}
\]

Els volums molars del gel i l'aigua són 0.0196 L/mol i 0.0180 L/mol, respectivament. L'increment d'energia interna ($\Delta U$) es pot calcular utilitzant la relació entre entalpia i energia interna:
\[
\Delta H = \Delta U + P\Delta V
\]
On $P$ és la pressió i $\Delta V$ és el canvi de volum. El canvi de volum $\Delta V$ es pot calcular com:
\[
\Delta V = V_{\text{líquid}} - V_{\text{sòlid}} = 0.0180 \text{ L/mol} - 0.0196 \text{ L/mol} = -0.0016 \text{ L/mol}
\]

Convertint el canvi de volum a metres cúbics:
\[
\Delta V = -0.0016 \text{ L/mol} \times \frac{1 \text{ m}^3}{1000 \text{ L}} = -1.6 \times 10^{-6} \text{ m}^3/\text{mol}
\]

Assumint que la pressió és 1 atm (101.3 kPa):
\[
P\Delta V = 101.3 \text{ kPa} \times (-1.6 \times 10^{-6} \text{ m}^3/\text{mol}) = -0.000162 \text{ kJ/mol}
\]

Així doncs, l'increment d'energia interna és:
\[
\Delta U = \Delta H - P\Delta V = 6.01 \text{ kJ/mol} - (-0.000162 \text{ kJ/mol}) = 6.010162 \text{ kJ/mol}
\]

Per tant, l'increment d'energia interna en fondre 1 mol de gel és aproximadament 6.01 kJ/mol, no significativament diferent de l'increment d'entalpia.
}
