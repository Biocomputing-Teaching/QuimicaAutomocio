\begin{exr}
Quants nodes té la funció $\psi(2s)$? i la $\psi(3s)$? i la $\psi(2p)$? 
\end{exr}
\begin{exr}
    A partir de la densitat de probabilitat podem preguntar-nos coses com on és el màxim de probabilitat (solucionant  $\frac{d D(r)}{dr}=0$) o bé calculant el valor promig de la distància de l'electró al nucli segons $<r>_{n,l}=\int_0^{\infty} r D(r)dr$. Mostra que $<r>_{2s}=\frac{6a_0}{Z}$ i $<r>_{2p}=\frac{5a_0}{Z}$ (veure Figura \ref{fig:D2s2p}).%Al tanto, no puc posar un marginnote dins d'un exr \marginnote{https://chemistry.stackexchange.com/questions/15208/difference-between-actual-position-of-electron-and-radial-distribution-probabili}
    \end{exr}
    \begin{exr}
        L'àtom de sodi es comporta de forma similar a l'àtom d'hidrogen pel que fa a la seva facilitat de "donar" un electró. Ho pots explicar en base a les densitats de probabilitat explicades a l'apartat anterior? Pensa en la llei de Coulomb i l'efecte pantalla dels electrons interiors.
        \end{exr}
        \begin{exr}
            Escriu la configuració electrònica de l'argó i del potassi. Perquè després d'omplir els orbitals 3p no omplim els orbitals 3d? Com raones que els metalls de transició de les  darreres columnes de la taula periòdica tinguin típicament valències de +2?
            \end{exr}
            \begin{exr}
                Pots explicar les dades de la Figura \ref{fig:AfinitatElectronica} en base a la configuració electrònica dels elements representats?
                \end{exr}