\begin{exr}{}
S'afegeix ió \ch{Ag+} a una dissolució que conté \ch{Cl-} i \ch{I-}, ambdós a una concentració de 0.01 M. Què precipita abans, \ch{AgCl} i \ch{AgI}. Quina és la concentració d'ions \ch{Ag+} quan la primera sal comença a precipitar? I quina és la concentració de l'anió del primer precipitat quan la segona sal comença a precipitar?
\end{exr}
