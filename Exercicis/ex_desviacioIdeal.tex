\begin{exr}
Perquè \ch{CO2} i \ch{O2} 
tenen una desviació negativa respecte al comportament del gas ideal a pressions i temperatures moderades, mentres que l'He i el 
\ch{H2} 
presenten una deviació positiva en les mateixes condicions?
\end{exr}
\lct{
    Els gasos \ch{CO2} i \ch{O2} presenten una desviació negativa respecte al comportament ideal perquè tenen interaccions intermoleculars atractives significatives. Aquestes forces atractives fan que, a pressions i temperatures moderades, les molècules s'acostin més del que prediu l'equació del gas ideal, reduint així el volum efectiu i fent que el factor de compressibilitat \( z = \frac{PV}{RT} \) sigui menor que 1.

D'altra banda, els gasos com l'heli (\ch{He}) i l'hidrogen (\ch{H2}) presenten una desviació positiva perquè tenen interaccions intermoleculars molt febles i, a mesura que augmenta la pressió, dominen els efectes de repulsió a causa del volum finit de les molècules. Això fa que el gas ocupi un volum lleugerament superior al que prediu el model ideal, fent que \( z > 1 \) en aquestes condicions.
}