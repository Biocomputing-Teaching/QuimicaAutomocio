\begin{exr}{Comparativa TCG per a \ch{H2} i \ch{He}}
Es prepara una mescla de gasos d'hidrogen (\ch{H2}) i heli (\ch{He}) tal que les molècules de cada gas produeixin el mateix nombre de col·lisions amb la paret per unitat de temps. Determinem quin gas té la concentració més alta.
\end{exr}
\lct{
\textbf{Consideració com a gasos ideals}

L'energia cinètica translacional d'un mol de gas és 
\[\left<E_c\right>=N_0 \frac{m <c^2>}{2}=\frac{3}{2} RT\]
on $M=N_0m$ és la massa molecular del gas en \si{\kilogram\per\mole}. 

Per tant, la velocitat quadràtica mitjana és:
\begin{equation}
    c_{\text{rms}} = \sqrt{\frac{3RT}{M}}
\end{equation}
Com que la taxa de col·lisions amb la paret és proporcional a $n v_{\text{rms}}$, imposem la condició d’igualtat:
\begin{equation}
    n_{\ch{H}} \cdot \sqrt{\frac{3RT}{M_{\ch{H}}}} = n_{\ch{He}} \cdot \sqrt{\frac{3RT}{M_{\ch{He}}}}
\end{equation}
Substituint masses moleculars $M_{\ch{H}} = 2$ g/mol i $M_{\ch{He}} = 4$ g/mol:
\begin{equation}
    n_{\ch{H}} \cdot \sqrt{\frac{1}{2}} = n_{\ch{He}} \cdot \sqrt{\frac{1}{4}}
\end{equation}
\begin{equation}
    n_{\ch{H}} \cdot \frac{1}{\sqrt{2}} = n_{\ch{He}} \cdot \frac{1}{2}
\end{equation}
Resolent per $n_{\ch{H}}$:
\begin{equation}
    n_{\ch{H}} = \frac{n_{\ch{He}}}{\sqrt{2}}
\end{equation}
Per tant, la concentració de $\ch{H2}$ ha de ser més alta que la de $\ch{He}$.
\\[10pt]
\textbf{Consideració com a gasos no ideals}

Si considerem gasos reals, hem de corregir la velocitat mitjana tenint en compte el factor de compressibilitat $Z$:
\begin{equation}
    v_{\text{rms}} = \sqrt{\frac{3ZRT}{M}}
\end{equation}
A pressions altes, $Z_{\ch{H2}} > Z_{\ch{He}}$ per les interaccions intermoleculars més fortes d’hidrogen, la qual cosa redueix la seva velocitat i altera la relació de concentracions calculada abans. 

Amb l’equació de van der Waals:
\begin{equation}
    \left( P + \frac{a}{V^2} \right) (V - b) = RT
\end{equation}
On $a_{\ch{H}} > a_{\ch{He}}$, la densitat efectiva de $\ch{H2}$ serà menor que en el cas ideal, cosa que, novament, fa necessari afegir més partícules de \ch{H2} que d'\ch{He} per a assolir la mateixa taxa de col·lisions.
}
