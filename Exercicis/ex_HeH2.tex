\begin{exr}{Comparativa TCG per a \ch{H2} i \ch{He}}
Es prepara una mescla de gasos d'hidrogen (\ch{H2}) i heli (\ch{He}) tal que les molècules de cada gas produeixin el mateix nombre de col·lisions amb la paret per unitat de temps. Determinem quin gas té la concentració més alta.
\end{exr}
\lct{
\textbf{Consideració com a gasos ideals}

L'energia cinètica translacional d'un mol de gas és 
\[\left<E_c\right>=N_0 \frac{m <c^2>}{2}=\frac{3}{2} RT\]
on $M=N_0m$ és la massa molecular del gas en \si{\kilogram\per\mole}. 

Per tant, la velocitat quadràtica mitjana és:

\begin{equation}
    c_{\text{rms}} = \sqrt{\frac{3RT}{M}}
\end{equation}
Com que la taxa de col·lisions amb la paret és proporcional a $n v_{\text{rms}}$, imposem la condició d'igualtat:
\begin{equation}
    n_{\ch{H2}} \cdot \sqrt{\frac{3RT}{M_{\ch{H2}}}} = n_{\ch{He}} \cdot \sqrt{\frac{3RT}{M_{\ch{He}}}}
    \label{Eq:igualtat}
\end{equation}

A l'Eq. \ref{Eq:igualtat} hem usat que el nombre de col·lisions és proporcional al producte del nombre de molècules per la velocitat promig a la que es mouen. Per a entendre-ho, imaginem un cas simple de tres pilotes que es mouen a 10 m/s en línia recta i fan rebots entre dues parets d'una habitació. Si l'habitació fa 10 metres de llarg, en 10 segons cada pilota haurà tocat les parets 10 cops. Per tant, el nombre de xocs haurà estat 30.
Si enlloc de 3 pilotes en tinguéssim 10 que es mouen a 3 metres per segon, haurien tocat les parets també 30 cops (cada pilota, en 10 segons, hauria recorregut 30 metres, i per tant hauria xocat 3 cops contra les parets; com que tenim 10 pilotes, el nombre total de xocs és 30).

Substituint masses moleculars $M_{\ch{H2}} = 2$ g/mol i $M_{\ch{He}} = 4$ g/mol a l'Eq. \ref{Eq:igualtat}:
\begin{equation}
    n_{\ch{H2}} \cdot \sqrt{\frac{1}{2}} = n_{\ch{He}} \cdot \sqrt{\frac{1}{4}}
\end{equation}
\begin{equation}
    n_{\ch{H2}} \cdot \frac{1}{\sqrt{2}} = n_{\ch{He}} \cdot \frac{1}{2}
\end{equation}
Resolent per $n_{\ch{H2}}$:
\begin{equation}
    n_{\ch{H2}} = \frac{n_{\ch{He}}}{\sqrt{2}}
\end{equation}
Per tant, la concentració de $\ch{H2}$ ha de ser més baixa que la de $\ch{He}$.
És a dir, si volem igualar les vegades que xoquen contra les parets d'un volum les molècules d'He i d'H2, hem de plantejar l'expressió d'igualtat de l'Eq. \ref{Eq:igualtat} on la partícula amb més massa, pel fet d'anar més lenta, necessitarà més partícules en moviment, és a dir, més concentració.

%\\[10pt]
\textbf{Consideració com a gasos no ideals}

Si considerem gasos reals, hem de corregir la velocitat mitjana tenint en compte el factor de compressibilitat $Z$:
\begin{equation}
    v_{\text{rms}} = \sqrt{\frac{3ZRT}{M}}
\end{equation}
Ara, l'Eq. \ref{Eq:igualtat} es transforma en:
\[
    n_{\ch{H2}} \cdot \sqrt{\frac{3Z_{\ch{H2}}RT}{M_{\ch{H2}}}} = n_{\ch{He}} \cdot \sqrt{\frac{3Z_{\ch{He}}RT}{M_{\ch{He}}}}
\]
d'on
\[
    \frac{n_{\ch{H2}}}{n_{\ch{He}}}=\sqrt{\frac{\frac{3Z_{\ch{He}}RT}{M_{\ch{He}}}}{\frac{3Z_{\ch{H2}}RT}{M_{\ch{H2}}}}}
    =\sqrt{\frac{Z_{\ch{He}}M_{\ch{H2}}}{Z_{\ch{H2}}M_{\ch{He}}}}
\]
A pressions altes, $Z_{\ch{H2}} > Z_{\ch{He}}$ per les interaccions intermoleculars més fortes d'hidrogen (veure la Fig. Gasos-\ref{Gasos-fig:FactorCompress}), la qual cosa encara reforça més la diferència entre les concentracions de les dues expècies químiques. 
}

