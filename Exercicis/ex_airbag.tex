\begin{exr}{Airbag}
    Els coixins de seguretat ({\em airbag}) dels cotxes s'inflen mitjançant una sèrie de reaccions químiques ràpides que produeixen gas en menys de \qty{0.04}{\second}. En les seves primeres versions, la reacció es basava en la descomposició de \ch{NaN3} (extremadament tòxic), seguida de dues reaccions addicionals per neutralitzar els subproductes perillosos. Les equacions químiques d'aquest procés són:  
 
    \begin{reactions}
        2 NaN3 &-> 2 Na + 3 N2 (g) "\label{reac:nan3}"\\
        10 Na + 2 KNO3 &-> K2O + 5 Na2O + N2 (g) "\label{reac:na}"\\
        K2O + Na2O + 2 SiO2 &-> K2SiO3 + Na2SiO3
    \end{reactions}

Un coixí de seguretat de conductor té un volum aproximat de \qty{65}{\liter} i la pressió final dins del coixí és de \qty{1.35}{\atm}. La temperatura dins del coixí just després de la reacció és \qty{300}{\celsius} (\qty{573}{\kelvin}). Suposem que s'utilitzen \qty{65}{\gram} de \ch{NaN3}.  

\begin{enumerate}
    \item Quina quantitat de nitrogen gas (\ch{N2}) es genera en mols només en la primera reacció?
    \item Quin volum ocuparà aquest gas dins del coixí de seguretat segons la llei dels gasos ideals? És suficient aquest volum per inflar completament el coixí de seguretat?
    \item Si considerem també la segona reacció, que genera més nitrogen gas, com afectaria això el volum total de gas produït?
    \item Quan el gas s'expandeix a l'exterior a través dels orificis del coixí, la seva pressió baixa de \qty{1.35}{\atm} a \qty{1.00}{\atm}. Quin percentatge de reducció de temperatura es produeix durant aquesta expansió?
\end{enumerate}
(Adaptat de \cite{bowers_understanding_2014}).
\end{exr}

\lct{
     \begin{enumerate}
    \item La quantitat de nitrogen gas (\ch{N2}) generada a R\ref{reac:nan3} ve donada per la descomposició de \ch{NaN3}:  
    \begin{equation*}
        \ch{2 NaN3 -> 2 Na + 3 N2 (g)}
    \end{equation*}

    Primer, calculem el nombre de mols de \ch{NaN3} disponibles:  
    \begin{equation}
        n_{\ch{NaN3}} = \frac{\qty{65}{\gram} \, \ch{NaN3}}{\qty{65.019}{\gram\per\mol} \,\ch{NaN3}} = \qty{1.00}{\mol} \,\ch{NaN3}
    \end{equation}
    
    De l'estequiometria de la reacció, per cada \qty{2}{\mol} de \ch{NaN3}, es formen \qty{3}{\mol} de \ch{N2}:  
    
    \begin{equation}
        n_{\ch{N2}} = \qty{1.00}{\mol} \, \ch{NaN3} \times \frac{3}{2} = \qty{1.50}{\mol} \, \ch{N2}
    \end{equation}
    
    \item Per a calcular el volum ocupat pel gas,  segons la llei dels gasos ideals:
    \begin{equation}
        V = \frac{nRT}{P}
    \end{equation}
    on:
    \begin{itemize}
        \item $n = \qty{1.50}{\mol}$
        \item $R = \qty{0.0821}{\liter\atm\per\mol\per\kelvin}$
        \item $T = \qty{573}{\kelvin}$
        \item $P = \qty{1.35}{\atm}$
    \end{itemize}
    
    \begin{equation}
        V = \frac{\qty{1.50}{\mol} \times \qty{0.0821}{\liter\atm\per\mol\per\kelvin} \times \qty{573}{\kelvin}}{\qty{1.35}{\atm}}= \qty{52.3}{\liter}
    \end{equation}
      
    El volum necessari per inflar el coixí de seguretat és d'uns \qty{65}{\liter}. Atès que només la primera reacció genera \qty{52.3}{\liter}, sembla que no és suficient. No obstant això, la segona reacció també genera gas \ch{N2}, augmentant el volum total.  
    
    \item Calculem ara la contribució de la reacció R\ref{reac:na}.
    Cada \qty{10}{\mol} de Na reacciona per produir \qty{1}{\mol} de \ch{N2}. És fàcil veure que 1 mol de \ch{NaN3} a la primera reacció va generar \qty{1.00}{\mol} de Na. Per tant, la segona reacció produeix:  
    \begin{equation}
        n_{\ch{N2,2}} = \qty{1.00}{\mol} \, \ch{Na} \times \frac{1}{10} = \qty{0.10}{\mol}\, \ch{N2}
    \end{equation}
    
    Afegint aquest nitrogen al total:  
    \begin{equation}
        n_{\ch{N2,\text{total}}} = \qty{1.50}{\mol} + \qty{0.10}{\mol} = \qty{1.60}{\mol}
    \end{equation}
    
    El nou volum total serà:  
    \begin{equation}
        V_{\text{total}} = \frac{\qty{1.60}{\mol} \times \qty{0.0821}{\liter\atm\per\mol\per\kelvin} \times \qty{573}{\kelvin}}{\qty{1.35}{\atm}} = \qty{55.7}{\liter}
    \end{equation}
    
    Aquest volum segueix estant per sota del mínim requerit (\qty{65}{\liter}), però cal recordar que les reaccions són fortament exotèrmiques, la qual cosa elevarà la temperatura i, en conseqüència, augmentarà el volum de gas.
    
    \item Refredament del gas en expandir-se fora del coixí:
    
    Segons la llei de Gay-Lussac:
    
    \begin{equation}
        \frac{P_1}{T_1} = \frac{P_2}{T_2}
    \end{equation}
    
    On:
    \begin{itemize}
        \item $P_1 = \qty{1.35}{\atm}$, $T_1 = \qty{573}{\kelvin}$
        \item $P_2 = \qty{1.00}{\atm}$, $T_2$ és la temperatura final
    \end{itemize}
    
    \begin{equation}
        T_2 = T_1 \times \frac{P_2}{P_1}
 = \qty{573}{\kelvin} \times \frac{\qty{1.00}{\atm}}{\qty{1.35}{\atm}}
= \qty{424}{\kelvin}
    \end{equation}
    
    El percentatge de reducció de temperatura és:
    
    \begin{equation}
        \frac{T_1 - T_2}{T_1} \times 100 = \frac{\qty{573}{\kelvin} - \qty{424}{\kelvin}}{\qty{573}{\kelvin}} \times 100 = 25.9\%
    \end{equation}
    
    Així, la temperatura del gas disminueix aproximadament un \qty{26}{\percent} quan s'expandeix fora del coixí de seguretat, ajudant a evitar cremades als passatgers.
\end{enumerate}
}
