\begin{exr}
    Calcular el volum molar d'un gas ideal a condicions normals (1 atm i 0\si\degreeCelsius).
    \end{exr}

\lct{

    
    \subsection*{Dades donades}
    Les condicions normals (CN) per a un gas ideal són:
    
    \begin{itemize}
        \item Pressió: \( P = \SI{1}{\atm} \)
        \item Temperatura: \( T = \SI{0}{\celsius} = \SI{273.15}{\kelvin} \)
        \item Constant dels gasos: \( R = \SI{0.0821}{\liter\atm\per\mole\per\kelvin} \)
    \end{itemize}
    
    \subsection*{Aplicació de l'equació dels gasos ideals}
    L'equació dels gasos ideals és:
    
    \[
    PV = nRT
    \]
    
    Aïllem el volum molar \( V_m \), considerant \( n = 1 \) mol:
    
    \[
    V_m = \frac{RT}{P}
    \]
    
    \subsection*{Càlcul del volum molar}
    Substituïm les dades:
    
    \[
    V_m = \frac{(\SI{0.0821}{\liter\atm\per\mole\per\kelvin}) \times (\SI{273.15}{\kelvin})}{\SI{1}{\atm}}
    \]
    
    \[
    V_m = \frac{\SI{22.414}{\liter\atm\per\mole}}{\SI{1}{\atm}}
    \]
    
    \[
    V_m \approx \SI{22.4}{\liter\per\mole}
    \]
    
    \subsection*{Conclusió}
    El volum molar d’un gas ideal en **condicions normals** és **\(\SI{22.4}{\liter\per\mole}\)**.
    
    

}