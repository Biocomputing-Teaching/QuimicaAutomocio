\begin{exr}
Càlcul de $\Delta U$ per a la combustió del grafit a \ch{CO} (gas) en condicions estàndard (\qty{298}{\kelvin} i \qty{1}{\atm}), si l'entalpia de combustió del grafit a \ch{CO} ($\Delta H$): \qty{-110.5}{\kilo\joule\per\mole}. El grafit té un volum molar de \qty{0.0053}{\litre\per\mole}.
\end{exr}

\lct{
La reacció de combustió del grafit a \ch{CO} (gas) es pot escriure com:
\[
\ch{C (grafit)} + \frac{1}{2}\ch{O2} \rightarrow \ch{CO (gas)}
\]

Per calcular el canvi d'energia interna ($\Delta U$) per a aquesta reacció, utilitzarem la relació entre $\Delta U$ i $\Delta H$ (entalpia de reacció):
\[
\Delta U = \Delta H - \Delta(PV)= \Delta H - \Delta n_g RT
\]

On:
\begin{itemize}
    \item $\Delta H$ és l'entalpia de combustió del grafit a \ch{CO}.
    \item $\Delta n_g$ és el canvi en el nombre de mols de gasos.
    \item $R$ és la constant dels gasos ideals (\qty{8.314}{\joule\per\mole\per\kelvin}).
    \item $T$ és la temperatura en Kelvin.
\end{itemize}

Per a la reacció de combustió del grafit a \ch{CO}:
\[
\Delta n_g = n_{\text{productes}} - n_{\text{reactius}} = 1 - \frac{1}{2} = \frac{1}{2}
\]
Un mol de gas a condicions estàndard ocupa un volum de \qty{22.4}{\litre}. Per tant, el canvi de 11.2 litres de gas a \qty{298}{\kelvin} fa que la desaparicció del grafit (\qty{0.0053}{\litre\per\mole}) sigui menyspreable.

Així doncs, $\Delta U$ es calcula com:
\[
\Delta U = \Delta H - \frac{1}{2} RT
\]

L'entalpia de combustió del grafit a \ch{CO} ($\Delta H$) és aproximadament \qty{-110.5}{\kilo\joule\per\mole}. Agafant la temperatura de \qty{298}{\kelvin}:
\begin{align*}
\Delta U &= \qty{-110.5}{\kilo\joule\per\mole} - \frac{1}{2} \cdot \qty{8.314}{\joule\per\mole\per\kelvin} \cdot \qty{298}{\kelvin} \times \qty{e-3}{\kilo\joule\per\joule} \\
&= \qty{-110.5}{\kilo\joule\per\mole} - \qty{1.239}{\kilo\joule\per\mole} \\
&= \qty{-111.739}{\kilo\joule\per\mole}
\end{align*}
}