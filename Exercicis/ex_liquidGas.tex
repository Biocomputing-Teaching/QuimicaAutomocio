\begin{exr}{}
És possible que un líquid arribi a estar sobreescalfat: temperatura superior a la d'ecullició per a aquella pressió però encara estat líquid, la qual cosa succeeix quan és molt pur i no hi ha partícules de pols.
Com aconseguiries que no es produeixi aquest sobreecalfament?
\end{exr} 
\begin{exr}{}
    Què ens produirà una cremada més gran: una massa $m$ d'\ch{H2O}(g) a 100 graus o la mateixa quantitat d'aigua líquida a la mateixa temperatura?
    \end{exr}
    
    \begin{exr}{}
    En un recipient hi ha aigua líquida. Es conecta el frecipient a una bomba de buit i es va abaixant la pressió sobre el líquid. Si la temperatura és de 60 graus, a quina pressió bullirà l'aigua?
    \end{exr}
    
    \begin{exr}{}
    Perquè a la Taula \ref{tab:pv} no apareix la $p_v$ de l'\ch{He}, \ch{H_2} i \ch{CH4}?
    \end{exr}
