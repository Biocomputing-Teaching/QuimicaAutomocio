\begin{exr}{Entalpia de reacció}
Fent servir les dades de la taula d'energies d'enllaç, estima la calor alliberada a pressió constant en la reacció:
\begin{align*}
  \ch{H2 \gas{} + Cl2 \gas{} + C(grafit) -> CH3Cl \gas{}}
\end{align*}
si la calor de vaporització del grafit a àtoms de carboni és de 170.9 kcal mol$^{-1}$.
\end{exr}
  \lct{
    La reacció donada és:
    \begin{align*}
      \ch{H2 \gas{} + Cl2 \gas{} + C(grafit) -> CH3Cl \gas{}}
    \end{align*}
    
    Per calcular l'entalpia de reacció, utilitzem les energies d'enllaç de la taula corresponent i la calor de vaporització del grafit.
    
    \textbf{Trencament d'enllaços (requereix energia):}
    \begin{itemize}
        \item \ch{H-H}: \qty{104.2}{\kcal\per\mol}
        \item \ch{Cl-Cl}: \qty{57.8}{\kcal\per\mol}
        \item \ch{C (grafit -> àtoms)}: \qty{170.9}{\kcal\per\mol}
    \end{itemize}
    \textbf{Formació d'enllaços (allibera energia):}
    \begin{itemize}
        \item 3 enllaços \ch{C-H}: \(3 \times \qty{98.7}{\kcal\per\mol} = \qty{296.1}{\kcal\per\mol} \)
        \item 1 enllaç \ch{C-Cl}: \qty{80}{\kcal\per\mol}
    \end{itemize}
    
    El càlcul de l'entalpia de reacció és:
    \begin{align*}
        \Delta H &= \text{energia trencament} - \text{energia formació} \\
        &= (\qty{104.2}{\kcal\per\mol} + \qty{57.8}{\kcal\per\mol} + \qty{170.9}{\kcal\per\mol}) \\
        &\quad - (\qty{296.1}{\kcal\per\mol} + \qty{80}{\kcal\per\mol}) \\
        &= \qty{332.9}{\kcal\per\mol} - \qty{376.1}{\kcal\per\mol} \\
        &= \qty{-43.2}{\kcal\per\mol}
    \end{align*}
        }
    