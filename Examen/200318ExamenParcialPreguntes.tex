\documentclass[11pt]{article}
% SILENCE THE WARNINGS!
%\usepackage{silence}

% tufte-book ja inclou el paquet geometry, i per tant només
% cal canviar alguns paràmetres amb \geometry
% \geometry{a4paper, top=25mm, bottom=30mm, inner=20mm, outer=70mm}
% \setlength{\marginparwidth}{50mm}  % Adjust margin for sidenotes
% %\geometry{margin=3cm,headsep=0.25in}
%\geometry{showframe}% for debugging purposes -- displays the margins
% The units package provides nice, non-stacked fractions and better spacing
% for units.
%\usepackage{units}
%\usepackage{todonotes}

\usepackage[backend=bibtex,style=numeric]{biblatex}  %backend=biber is 'better'

\usepackage{framed}
\usepackage{ifthen}
\usepackage{longtable}
\usepackage{fancyvrb}
\fvset{fontsize=\normalsize}
%\usepackage{cancel}

\usepackage[utf8]{inputenc}
\usepackage[catalan]{babel}
\usepackage{lmodern}
\usepackage{amsmath,amsthm,amsfonts,amssymb,amscd}

\usepackage{multirow,booktabs}
\usepackage[dvipsnames,table]{xcolor}
%\usepackage{fullpage}
\usepackage{lastpage}
\usepackage{graphicx}
%\setkeys{Gin}{width=\linewidth,totalheight=\textheight,keepaspectratio}
\graphicspath{{../figures/}}
\usepackage{enumitem}
\usepackage{mathrsfs}
\usepackage{wrapfig}
\usepackage{setspace}
\usepackage{calc}
\usepackage{multicol}
\usepackage{gensymb}




\usepackage{cancel}
\usepackage[retainorgcmds]{IEEEtrantools}

%\newlength{\tabcont}
% \setlength{\parindent}{0.0in}
% \setlength{\parskip}{0.05in}
%\usepackage{empheq}
% es recomana que mdframed es carregui després de xcolor
\usepackage[framemethod=TikZ]{mdframed}
\mdfdefinestyle{caixa}{leftmargin=1cm,innerrightmargin=0.5cm, linecolor=blue}

\usepackage{changepage}






  
%\chemsetup[chemformula]{format=\sffamily}

%\setatomsep{2em}
%\setdoublesep{.6ex}
%\setbondstyle{semithick}
\colorlet{shadecolor}{orange!15}
\parindent 0in
\parskip 12pt


\theoremstyle{definition}
\newtheorem{defn}{Definition}
\newtheorem{reg}{Rule}
\newtheorem{exer}{Exercise}
\newtheorem{note}{Note}
%\RequirePackage{mathrsfs}
%\RequirePackage[psamsfonts]{amsfonts} %for Y&Y BSR AMS fonts
\RequirePackage{amsmath,amsfonts,amsthm,amssymb}
\RequirePackage{setspace}
\RequirePackage{fancyhdr}
\RequirePackage{lastpage}
\RequirePackage{extramarks}
%\RequirePackage{chngpage}
\RequirePackage{soul}
%\RequirePackage{graphicx,float,wrapfig}
%\RequirePackage{pgf,tikz}
%\usetikzlibrary{arrows,automata}
%\RequirePackage{pstricks}
%\RequirePackage[text]{amsthm}
%\RequirePackage{array}
%\RequirePackage{amscd}
%\RequirePackage{array}\RequirePackage{dcolumn}

\newcommand{\emx}[1]{{\em{#1}\/}}
\newcommand{\abin}{{\it ab initio}}
\newcommand{\bs}{\boldsymbol}
%\newcommand{\citepnum}{\citep}
\newcommand{\dGo}{\ensuremath{\Delta G_0}}
\newcommand{\dG}[2]{\ensuremath{\Delta G_{\rm #1}^{\rm #2}}}
\newcommand{\dX}[3]{\ensuremath{\Delta #1_{\rm #2}^{\rm #3}}}
\newcommand{\ddgo}[1]{\ensuremath{\Delta \Delta G_{\rm solv}^{\rm #1}}}
\newcommand{\ddgstarcat}{\ensuremath{\Delta \Delta g^{\ddagger}_{\rm cat}}}
\newcommand{\ddgstar}{\ensuremath{\Delta \dgstar}}
\newcommand{\ddgt}[2]{\ensuremath{\Delta \Delta G_{\rm solv}^{\rm #1, \rm #2}}}
\newcommand{\ddsstarprime}{\ensuremath{(\Delta \dsstar)'}}
\newcommand{\deltaepsel}{\ensuremath{\Delta \varepsilon_{\rm el}}}
\newcommand{\deltaeps}{\ensuremath{\Delta \varepsilon}}
\newcommand{\dgab}[2]{\ensuremath{\Delta g_{\rm #1}^{\rm #2}}}
\newcommand{\dga}[1]{\ensuremath{\Delta g_{\rm #1}}}
\newcommand{\dgb}[1]{\ensuremath{\Delta g^{\rm #1}}}
\newcommand{\dgcage}{\ensuremath{\Delta g_{\rm cage}}}
\newcommand{\dgcat}{\ensuremath{\Delta g_{\rm cat}}}
\newcommand{\dgsoltsatsa}{\ensuremath{\dgsol (\rm TSA)_{\rm TSA}}}
\newcommand{\dgsoltstsa}{\ensuremath{\dgsol (\rm TS)_{\rm TSA}}}
\newcommand{\dgsoltsts}{\ensuremath{\dgsol (\rm TS)_{\rm TS}}}
\newcommand{\dgsol}{\ensuremath{\Delta G_{\rm sol}}}
\newcommand{\dgstarcage}{\ensuremath{\dgstar_{\rm cage}}}
\newcommand{\dgstarcat}{\ensuremath{\dgstar_{\rm cat}}}
\newcommand{\dgstarw}{\ensuremath{\dgstar_{\rm w}}}
\newcommand{\dgstar}{\ensuremath{\Delta g^{\ddagger}}}
\newcommand{\dgw}{\ensuremath{\Delta g_{\rm w}}}
\newcommand{\dg}[2]{\ensuremath{\Delta g_{\rm #1}^{\rm #2}}}
\newcommand{\dino}{\texttt{DINO}}
\newcommand{\dsstarcageprime}{\ensuremath{(\dsstarcage)'}}
\newcommand{\dsstarcage}{\ensuremath{\dsstar_{\rm cage}}}
\newcommand{\dsstarcatprime}{\ensuremath{(\dsstarcat)'}}
\newcommand{\dsstarcat}{\ensuremath{\dsstar_{\rm cat}}}
\newcommand{\dsstarwprime}{\ensuremath{(\dsstarw)'}}
\newcommand{\dsstarw}{\ensuremath{\dsstar_{\rm w}}}
\newcommand{\dsstar}{\ensuremath{\Delta S^{\ddagger}}}
\newcommand{\eg}{{\it e.g.}}
\newcommand{\etal}{{\it et al.}}
\newcommand{\gamess}{\texttt{GAMESS}}
\newcommand{\gauss}{\texttt{GAUSSIAN} 98}     
\newcommand{\golpe}{\texttt{GOLPE}}                                             
\newcommand{\grid}{\texttt{GRID}}
\newcommand{\ie}{{\it i.e.}}
\newcommand{\ith}{{\it i}$^{\rm th}$\ }
\newcommand{\kbt}{\ensuremath{k_{\rm B} T}}
\newcommand{\kb}{\ensuremath{k_{\rm B}}} 
\newcommand{\kcage}{\ensuremath{k_{\rm cage}}}
\newcommand{\kcatkm}{\ensuremath{k_{\rm cat}/K_{\rm M}}}
\newcommand{\kcat}{\ensuremath{k_{\rm cat}}}
\newcommand{\km}{\ensuremath{{\rm\, kcal \, mol}^-1}}
\newcommand{\knon}{\ensuremath{k_{\rm non}}}
\newcommand{\kw}{\ensuremath{k_{\rm w}}}
\newcommand{\mepsim}{\texttt{MEPSIM}}
\newcommand{\mgp}[1]{\marginpar{\scriptsize{#1}}}
\newcommand{\mipsim}{\texttt{MIPSIM}}
\newcommand{\mola}{\texttt{MOLARIS}}
\newcommand{\msms}{\texttt{MSMS}}
\newcommand{\pdras}{p21$^{\rm ras}$}
\newcommand{\rgran}{\ensuremath{\mathbb{R}}}
\newcommand{\rx}[2]{\ensuremath{#1_{\rm #2}}}
\newcommand{\vs}{{\it vs.}}
\newcommand{\z}[1]{\ensuremath{\mathbf{#1}}} 
\newcommand{\composed}[2]{#1\mathbin\circ #2}
\newcommand{\wrt}[1]{{\mbox{\scriptsize w.r.t. \( #1 \)} }}
\newcommand{\polyspace}{\mathcal{P}}
\newcommand{\matspace}{\mathcal{M}}
\newcommand{\C}{\mathbb{C}}
\newcommand{\N}{\mathbb{N}}
\newcommand{\Q}{\mathbb{Q}}
\newcommand{\Z}{\mathbb{Z}}
\renewcommand{\Re}{\mathbb{R}}
\newcommand{\rtres}{\ensuremath{\Re^3}}
\newcommand{\union}{\cup}
\newcommand{\dotprod}{\cdot}
%\newcommand*\pkg[1]{\textsf{#1}}

\newcommand{\trans}[1]{{#1}^{\ensuremath{\mathsf{T}}}} % transpose
\newcommand{\nbyn}[1]{\ensuremath{#1 \! \times \! #1 }}
\newcommand{\nbym}[2]{#1 \! \times \! #2 }       % Use in math mode.
\newcommand{\cat}[2]{#1\!\mathbin{\raise.6ex\hbox{\( {}^\frown \)}}\!#2}
\newcommand{\generalmatrix}[3]{ %arg1: low-case letter, arg2: rows, arg3: cols
               \left(
                  \begin{array}{cccc}
                    #1_{1,1}  &#1_{1,2}  &\ldots  &#1_{1,#2}  \\
                    #1_{2,1}  &#1_{2,2}  &\ldots  &#1_{2,#2}  \\
                              &\vdots                         \\
                    #1_{#3,1} &#1_{#3,2} &\ldots  &#1_{#3,#2}
                  \end{array}
               \right)  }
\newcommand{\colvec}[1]{\begin{pmatrix} #1 \end{pmatrix}}
\newcommand{\pr}[1]{\ensuremath{\mathrm{Pr}(#1)}}
\newcommand{\rep}[2]{ {\rm Rep}_{#2}(#1) }
\newcommand{\mapsunder}[1]{\stackrel{#1}{\longmapsto}}
\newcommand{\map}[3]{\mbox{$#1\colon #2\to #3$}}
\newcommand{\identity}{\mbox{id}}
\newcommand{\stdbasis}{{\cal E}} 
\newcommand{\sequence}[1]{ \langle#1\rangle } 
\newcommand{\spacer}{\rule[-3mm]{0mm}{8mm}}
\newcommand{\email}[1]{\url{#1}}
\newcommand{\zero}{\vec{0}}
\newcommand{\proj}[2]{\mbox{proj}_{#2}({#1}) }
%\AtBeginDocument{\newlength{\heightofcdot}
%\newlength{\widthofcdot}
%\settoheight{\heightofcdot}{$\cdot$}
%\settowidth{\widthofcdot}{$\cdot$}
%\newsavebox{\dotprodcircle}       
%\savebox{\dotprodcircle}{\includegraphics{dotprod.1}} 
%\newcommand{\dotprod}{\mathbin{\raisebox{.5\heightofcdot}{%
%          \makebox[\widthofcdot]{$\smash{\usebox{\dotprodcircle}}$}}}}}
\newcommand{\spanof}[1]{\relax [#1\relax ]} % no optional argument!
\newcommand{\set}[1]{\mbox{$\{#1\}$}} \newcommand{\suchthat}{\bigm|}
\newcommand{\deter}[1]{ \mathchoice{\left|#1\right|}{|#1|}{|#1|}{|#1|} }
\newcommand{\secuence}[1]{ \langle#1\rangle }  
\newcommand{\basis}[2]{\secuence{\vec{#1}_1,\ldots,\vec{#1}_{#2}}}



%--------linsys
%  Use as \begin{linsys}{3}
%           x &+ &3y &+ &a &= &7 \\
%           x &- &3y &+ &a &= &7
%         \end{linsys}
% Remark: TeXbook pp. 167-170 says to put a medmuskip around a +; and that's
% 4/18-ths of an em.  Why does 2/18-ths of an em work?  I don't know, but
% comparing to a regular displayed equation suggests it is right.
% (darseneau says LaTeX puts in half an \arraycolsep.)
\newenvironment{linsys}[2][m]{%
\setlength{\arraycolsep}{.1111em} % p. 170 TeXbook; a medmuskip
\begin{array}[#1]{@{}*{#2}{rc}r@{}}
}{%
\end{array}}


%\newtheorem{teorema}{Teorema}
%\newtheorem{exercici}{Exercici}
%\newtheorem{definicio}{Definici\'o}
%\newtheorem{theorem}{Theorem}
\newtheorem{exercise}{Exercise}
%\newtheorem{definition}{Definition}



\parskip 4mm


\usepackage{makeidx}
\makeindex




%\setcounter{section}{-1}

\theoremstyle{definition}
\newtheorem{thm}{Theorem}
\newtheorem{dfn}{Definition}
\newtheorem{lem}{Lemma}
\newtheorem{prp}{Proposition}





%%%%%%%%%%%%%%%%%%%
% ANGLÈS
%%%%%%%%%%%%%%%%%%%

% \newcommand{\problemName}{}%
% \newcounter{problemCounter}%
% \newenvironment{problem}[1][Problem \arabic{problemCounter}]%
% 	{\stepcounter{problemCounter}%
% 		\renewcommand{\problemName}{#1}%
% 		\section*{\problemName}%
% 		\nobreak\extramarks{\problemName}{\problemName continued on next page\ldots}\nobreak%
% 		\nobreak\extramarks{\problemName (continued)}{\problemName continued on next page\ldots}\nobreak}%
% 	{\nobreak\extramarks{\problemName (continued)}{\problemName continued on next page\ldots}\nobreak%
% 		\nobreak\extramarks{\problemName}{}\nobreak}%

\newenvironment{example}{ % 
	\definecolor{shadecolor}{rgb}{0.8,1.0,0.8} %
	\begin{shaded} %
	\textcolor{OliveGreen}{\bf Example\\}%
} % 
{ %	
	\end{shaded}
} %


\newenvironment{introduction}{ % 
	\definecolor{shadecolor}{rgb}{1.0,1.0,0.8} %
	\begin{shaded} %
	% \textcolor{BrickRed}{\bf Introduction\\}%
} % 
{ %	
	\end{shaded}
} %


%%%%%%%%%%%%%%%%%%%
% CATALÀ
%%%%%%%%%%%%%%%%%%%
\newtheorem{teorema}{theorem}
\newenvironment{definicio}{ % 
	\definecolor{shadecolor}{rgb}{0.9,1.0,0.8} %
	\begin{shaded} %
	\textcolor{OliveGreen}{\bf Definicio\\}%
} % 
{ %	
	\end{shaded}
} %

%veure http://en.wikibooks.org/wiki/LaTeX/Advanced_Topics

\newcommand{\doccmd}[1]{\texttt{\textbackslash#1}}% command name -- adds backslash automatically
\newcommand{\docopt}[1]{\ensuremath{\langle}\textrm{\textit{#1}}\ensuremath{\rangle}}% optional command argument
\newcommand{\docarg}[1]{\textrm{\textit{#1}}}% (required) command argument
\newenvironment{docspec}{\begin{quote}\noindent}{\end{quote}}% command specification environment
\newcommand{\docenv}[1]{\textsf{#1}}% environment name
\newcommand{\docpkg}[1]{\texttt{#1}}% package name
\newcommand{\doccls}[1]{\texttt{#1}}% document class name
\newcommand{\docclsopt}[1]{\texttt{#1}}% document class option name
\newcommand{\logos}{%
\begin{figure}
\includegraphics{FCTE}
\end{figure}
}

% margins
% \topmargin=-0.45in      %
% \evensidemargin=0in     %
% \oddsidemargin=0in      %
% \textwidth=6in        %
% \textheight=8.5in       %
% \headsep=0.25in         %

% header and footer
\pagestyle{fancy}       %
\chead{}                %
\makeatletter
\fancyfoot[R]{%
   % We want italics
   \itshape
   % The chapter number only if it's greater than 0
   \ifnum\value{chapter}>0 \@chapapp\ \thechapter. \fi
   % The chapter title
   \leftmark}
\makeatother

%\lfoot{\includegraphics[trim=-5cm 0 0 -3cm,width=0.4\textwidth]{FCTE}}      
\lfoot{\raisebox{-0.5cm}[0pt][0pt]{\includegraphics[width=3cm]{FCTE}}} 

\cfoot{}        %
\renewcommand\headrulewidth{0.4pt}   %
\renewcommand\footrulewidth{0.4pt}   %

% Essential Formatting
   
%\usepackage{epsfig,amsmath,amsthm,amssymb}
\usepackage[questions]{urmathtest_cat}[2001/05/12]
%\usepackage[answersheet]{urmathtest}[2001/05/12]
%\usepackage[answers]{urmathtest}[2001/05/12]
\usepackage{graphicx}

%% For use with pdflatex
%\pdfpagewidth\paperwidth
%\pdfpageheight\paperheight

% Basic User Defs

%\def\ds{\displaystyle}

%\newcommand{\ansbox}[1]
%{\work{
%  \pos\hfill \framebox[#1][l]{SCORE:\rule[-.3in]{0in}{.7in}}
%}{}}
%
%\newcommand{\ansrectangle}
%{\work{
%  \pos\hfill \framebox[6in][l]{SCORE:\rule[-.3in]{0in}{.7in}}
%}{}}

% Beginning of the Document
\cfoot{\bf Examen Parcial Química GEA-17UV}

\begin{document}
\examtitle{Enginyeria de l'Automoció}{Examen Parcial Química GEA-17UV}{20 de Març de 2018}
\studentinfo
\instructions{
  \textbf{Professor: Jordi Villà i Freixa}
  

  \begin{itemize}
  \item
    \textbf{No es permet l'ús d'ordinador. Només calculadora, apunts de classe i full d'exercicis resolts del campus virtual}
  \item
%   \textbf{Please show all your work if you need to.
%           You may use back pages if necessary.
%           You may not receive full credit for
%           a correct answer if there is no work shown.}
   \textbf{Desenvolupa tot el teu argumentari de forma clara.
           No usis més espai del proveït.}
%  \item
%    \textbf{Please put your \underline{simplified}
%            final answers in the spaces provided.}
  \end{itemize}
}
\finishfirstpage

% Problems Start Here % ----------------------------------------------------- %
%%%%%%%%%%%%%%%%%%%%%%%%%%%%%%%%%%%%%%%%
\problem{20}
{Un parell de preguntes per escalfar: 
\begin{enumerate}[label=\emph{\alph*})]
\item En un submarí nuclear a 2000m de profunditat, el cuiner està fent un arròs blanc perquè alguns tenen una mica de mal de panxa. El cas és que el capità del vaixell els ha imposat un examen de química per aconseguir ascendir (de categoria, és clar); podries dir-li al cuiner si li cal escalfar l'aigua a més o menys temperatura per fer l'arròs que quan en fa al seu Vilassar de Mar natal?
\item Un cop els mariners a terra, surten a pre-celebrar el previst aprovat fent un aperitiu al bar Espinaler, convidats pel cuiner que hi té bons contactes. El torpeder, encara estabornit per la tensió de l'examen, es queda mirant com, en obrir l'ampolla d'aigua amb gas afegit que ha demanat, es formen unes precioses bombolles. El capità, que és un setciències i no calla ni sota l'aigua (ni fora d'ella, pel que sembla), pregunta qui vol pujar nota explicant aquest fenòmen. Ajudes el torpeder a donar una bona resposta? 
\end{enumerate}
}
{
\vfill
\newpage
}
{
\begin{enumerate}[label=\emph{\alph*})]
\item Dins d'un submarí, la pressió és la mateixa que a nivell del mar, o gairebé, ja que les parets del vaixell estan construïdes de tal manera que puguin suportar la pressió exterior sense que es noti dins. Per tant, haurà d'escalfar l'aigua a la mateixa temperatura que si fos al nivell del mar: 100$\degree$C.
\item Segons la llei de Henry, la concentració de gas en un líquid és proporcional a la pressió parcial externa d'aquest gas: $C=kP$. Quan s'embotella  aigua carbonatada es fa a una pressió superior a 1 atm per tal que no escapi el \ch{CO2} que s'hi afegeix a partir de les substàncies carbonatades que conté. Quan obrim l'ampolla, la pressió es redueix fins a 1 atm, i per tant el \ch{CO2} no és tan soluble i escapa de la dissolució formant bombolles. 
\end{enumerate}
}

%%%%%%%%%%%%%%%%%%%%%%%%%%%%%%%%%%%%%%%%
%%%%%%%%%%%%%%%%%%%%%%%%%%%%%%%%%%%%%%%%
\problem{30}
{Imaginem un gas que està dins d'un contenidor A de volum desconegut a la pressió de 2 atm. Obrim la vàlvula i el gas es pot expandir cap a un altre contenidor B que té un volum d'1 l. Un cop assolit l'equilibri mesurem la pressió i és de 380 mm Hg. Si el procés és isoterm ($T=$cnt), calcula el volum del contenidor A.
\begin{center}
\includegraphics[scale=0.7]{../figures/gas.png}
\end{center}}
{
%\vspace{22cm}
\vfill
\newpage
}
{
Les dades que ens donen són:

\[P_A=2atm=1520 \,mmHg\]
\[V_A=?\]
\[P_A=380 \,mmHg\]
\[V_B=1l\]

En tractar-se d'un procés isoterm, podem aplicar la llei de Boyle, segons la qual $\Delta (PV)=0$ a temperatura constant.
Per tant:
\[\Delta (PV)=0\]
\[P_i V_i = P_f V_f\]
\[1520  \,mmHg \cdot V_A = 380  \,mmHg \cdot (V_A + 1 l)\]
\[V_A = 380/1140 l = 0.\overline{3} l\]
}

%%%%%%%%%%%%%%%%%%%%%%%%%%%%%%%%%%%%%%%%
\problem{30}
{Si l'aire conté, aproximadament, un 21\% de volum d'oxigen per un 79\% de nitrogen, quin volum ha de tenir un recipient tancat per assegurar la combustió total de 2 gr de propà (\ch{C3H8}) a 1 atm i 25$\degree$C? Assumeix comportament ideal de tots els gasos.
}
{
\vfill
\newpage
}
{
La reacció de combustió del propà serà:

\ch{C3H8 + 5 O2 -> 3 CO2 + 4 H2O} 

Primer avaluem quant oxigen necessitem:

\[2 g \ch{C3H8} \cdot \frac{1 mol \ch{C3H8}}{44 g \ch{C3H8}} \cdot \frac{5 mol \ch{O2}}{1 mol \ch{C3H8}}=0.23 mol \ch{O2}\]

a 1 atm i 25$\degree$C aquests mols ocupen:

\[PV=nRT\]
\[ V = 0.23mol \ch{O2} \cdot 0.082 \frac{atm \, l}{mol \, K} \cdot 298 K = 5.6l \ch{O2}\]

Si aquest volum representa el 21\% del volum de l'aire en aquestes condicions, $5.6l=\frac{21}{100}V$, necessitarem un recipient de $V=26.7l$
}

%%%%%%%%%%%%%%%%%%%%%%%%%%%%%%%%%%%%%%%%
\problem{20}
{Quina és la pressió total d'una barreja de 6 mols de benzè (\ch{C6H6}) i 9 de toluè (\ch{C7H8}) si la pressió de vapor del benzè és de 95.1 mm Hg i la del toluè és 28.4 mm Hg a 25$\degree$C. 
}
{
\vfill

}
{
La llei de Raolut estableix que la pressió parcial d'un determinat component volàtil en una dissolució serà proporcional a la seva fracció molar i la seva pressió de vapor quan és pur. Per tant, la pressió total d'una dissolució de dos components volàtils es pot calcular segons 
\[
P_T=P_1+P_2=x_1P_1^0+x_2P_2^0
\]
Per tant:
\[
P_T=P_{\ch{C6H6}}+P_{\ch{C7H8}}=x_{\ch{C6H6}}P_{\ch{C6H6}}^0+x_{\ch{C7H8}}P_{\ch{C7H8}}^0=\frac{6}{15}95.1 mm Hg+\frac{9}{15}28.4 mm Hg=55.1mmHg
\]
}

% Problems End Here % ------------------------------------------------------- %

\problemsdone
\end{document}

% End of the Document
