\documentclass[11pt]{article}
% SILENCE THE WARNINGS!
%\usepackage{silence}

% tufte-book ja inclou el paquet geometry, i per tant només
% cal canviar alguns paràmetres amb \geometry
% \geometry{a4paper, top=25mm, bottom=30mm, inner=20mm, outer=70mm}
% \setlength{\marginparwidth}{50mm}  % Adjust margin for sidenotes
% %\geometry{margin=3cm,headsep=0.25in}
%\geometry{showframe}% for debugging purposes -- displays the margins
% The units package provides nice, non-stacked fractions and better spacing
% for units.
%\usepackage{units}
%\usepackage{todonotes}

\usepackage[backend=bibtex,style=numeric]{biblatex}  %backend=biber is 'better'

\usepackage{framed}
\usepackage{ifthen}
\usepackage{longtable}
\usepackage{fancyvrb}
\fvset{fontsize=\normalsize}
%\usepackage{cancel}

\usepackage[utf8]{inputenc}
\usepackage[catalan]{babel}
\usepackage{lmodern}
\usepackage{amsmath,amsthm,amsfonts,amssymb,amscd}

\usepackage{multirow,booktabs}
\usepackage[dvipsnames,table]{xcolor}
%\usepackage{fullpage}
\usepackage{lastpage}
\usepackage{graphicx}
%\setkeys{Gin}{width=\linewidth,totalheight=\textheight,keepaspectratio}
\graphicspath{{../figures/}}
\usepackage{enumitem}
\usepackage{mathrsfs}
\usepackage{wrapfig}
\usepackage{setspace}
\usepackage{calc}
\usepackage{multicol}
\usepackage{gensymb}




\usepackage{cancel}
\usepackage[retainorgcmds]{IEEEtrantools}

%\newlength{\tabcont}
% \setlength{\parindent}{0.0in}
% \setlength{\parskip}{0.05in}
%\usepackage{empheq}
% es recomana que mdframed es carregui després de xcolor
\usepackage[framemethod=TikZ]{mdframed}
\mdfdefinestyle{caixa}{leftmargin=1cm,innerrightmargin=0.5cm, linecolor=blue}

\usepackage{changepage}






  
%\chemsetup[chemformula]{format=\sffamily}

%\setatomsep{2em}
%\setdoublesep{.6ex}
%\setbondstyle{semithick}
\colorlet{shadecolor}{orange!15}
\parindent 0in
\parskip 12pt


\theoremstyle{definition}
\newtheorem{defn}{Definition}
\newtheorem{reg}{Rule}
\newtheorem{exer}{Exercise}
\newtheorem{note}{Note}
%\RequirePackage{mathrsfs}
%\RequirePackage[psamsfonts]{amsfonts} %for Y&Y BSR AMS fonts
\RequirePackage{amsmath,amsfonts,amsthm,amssymb}
\RequirePackage{setspace}
\RequirePackage{fancyhdr}
\RequirePackage{lastpage}
\RequirePackage{extramarks}
%\RequirePackage{chngpage}
\RequirePackage{soul}
%\RequirePackage{graphicx,float,wrapfig}
%\RequirePackage{pgf,tikz}
%\usetikzlibrary{arrows,automata}
%\RequirePackage{pstricks}
%\RequirePackage[text]{amsthm}
%\RequirePackage{array}
%\RequirePackage{amscd}
%\RequirePackage{array}\RequirePackage{dcolumn}

\newcommand{\emx}[1]{{\em{#1}\/}}
\newcommand{\abin}{{\it ab initio}}
\newcommand{\bs}{\boldsymbol}
%\newcommand{\citepnum}{\citep}
\newcommand{\dGo}{\ensuremath{\Delta G_0}}
\newcommand{\dG}[2]{\ensuremath{\Delta G_{\rm #1}^{\rm #2}}}
\newcommand{\dX}[3]{\ensuremath{\Delta #1_{\rm #2}^{\rm #3}}}
\newcommand{\ddgo}[1]{\ensuremath{\Delta \Delta G_{\rm solv}^{\rm #1}}}
\newcommand{\ddgstarcat}{\ensuremath{\Delta \Delta g^{\ddagger}_{\rm cat}}}
\newcommand{\ddgstar}{\ensuremath{\Delta \dgstar}}
\newcommand{\ddgt}[2]{\ensuremath{\Delta \Delta G_{\rm solv}^{\rm #1, \rm #2}}}
\newcommand{\ddsstarprime}{\ensuremath{(\Delta \dsstar)'}}
\newcommand{\deltaepsel}{\ensuremath{\Delta \varepsilon_{\rm el}}}
\newcommand{\deltaeps}{\ensuremath{\Delta \varepsilon}}
\newcommand{\dgab}[2]{\ensuremath{\Delta g_{\rm #1}^{\rm #2}}}
\newcommand{\dga}[1]{\ensuremath{\Delta g_{\rm #1}}}
\newcommand{\dgb}[1]{\ensuremath{\Delta g^{\rm #1}}}
\newcommand{\dgcage}{\ensuremath{\Delta g_{\rm cage}}}
\newcommand{\dgcat}{\ensuremath{\Delta g_{\rm cat}}}
\newcommand{\dgsoltsatsa}{\ensuremath{\dgsol (\rm TSA)_{\rm TSA}}}
\newcommand{\dgsoltstsa}{\ensuremath{\dgsol (\rm TS)_{\rm TSA}}}
\newcommand{\dgsoltsts}{\ensuremath{\dgsol (\rm TS)_{\rm TS}}}
\newcommand{\dgsol}{\ensuremath{\Delta G_{\rm sol}}}
\newcommand{\dgstarcage}{\ensuremath{\dgstar_{\rm cage}}}
\newcommand{\dgstarcat}{\ensuremath{\dgstar_{\rm cat}}}
\newcommand{\dgstarw}{\ensuremath{\dgstar_{\rm w}}}
\newcommand{\dgstar}{\ensuremath{\Delta g^{\ddagger}}}
\newcommand{\dgw}{\ensuremath{\Delta g_{\rm w}}}
\newcommand{\dg}[2]{\ensuremath{\Delta g_{\rm #1}^{\rm #2}}}
\newcommand{\dino}{\texttt{DINO}}
\newcommand{\dsstarcageprime}{\ensuremath{(\dsstarcage)'}}
\newcommand{\dsstarcage}{\ensuremath{\dsstar_{\rm cage}}}
\newcommand{\dsstarcatprime}{\ensuremath{(\dsstarcat)'}}
\newcommand{\dsstarcat}{\ensuremath{\dsstar_{\rm cat}}}
\newcommand{\dsstarwprime}{\ensuremath{(\dsstarw)'}}
\newcommand{\dsstarw}{\ensuremath{\dsstar_{\rm w}}}
\newcommand{\dsstar}{\ensuremath{\Delta S^{\ddagger}}}
\newcommand{\eg}{{\it e.g.}}
\newcommand{\etal}{{\it et al.}}
\newcommand{\gamess}{\texttt{GAMESS}}
\newcommand{\gauss}{\texttt{GAUSSIAN} 98}     
\newcommand{\golpe}{\texttt{GOLPE}}                                             
\newcommand{\grid}{\texttt{GRID}}
\newcommand{\ie}{{\it i.e.}}
\newcommand{\ith}{{\it i}$^{\rm th}$\ }
\newcommand{\kbt}{\ensuremath{k_{\rm B} T}}
\newcommand{\kb}{\ensuremath{k_{\rm B}}} 
\newcommand{\kcage}{\ensuremath{k_{\rm cage}}}
\newcommand{\kcatkm}{\ensuremath{k_{\rm cat}/K_{\rm M}}}
\newcommand{\kcat}{\ensuremath{k_{\rm cat}}}
\newcommand{\km}{\ensuremath{{\rm\, kcal \, mol}^-1}}
\newcommand{\knon}{\ensuremath{k_{\rm non}}}
\newcommand{\kw}{\ensuremath{k_{\rm w}}}
\newcommand{\mepsim}{\texttt{MEPSIM}}
\newcommand{\mgp}[1]{\marginpar{\scriptsize{#1}}}
\newcommand{\mipsim}{\texttt{MIPSIM}}
\newcommand{\mola}{\texttt{MOLARIS}}
\newcommand{\msms}{\texttt{MSMS}}
\newcommand{\pdras}{p21$^{\rm ras}$}
\newcommand{\rgran}{\ensuremath{\mathbb{R}}}
\newcommand{\rx}[2]{\ensuremath{#1_{\rm #2}}}
\newcommand{\vs}{{\it vs.}}
\newcommand{\z}[1]{\ensuremath{\mathbf{#1}}} 
\newcommand{\composed}[2]{#1\mathbin\circ #2}
\newcommand{\wrt}[1]{{\mbox{\scriptsize w.r.t. \( #1 \)} }}
\newcommand{\polyspace}{\mathcal{P}}
\newcommand{\matspace}{\mathcal{M}}
\newcommand{\C}{\mathbb{C}}
\newcommand{\N}{\mathbb{N}}
\newcommand{\Q}{\mathbb{Q}}
\newcommand{\Z}{\mathbb{Z}}
\renewcommand{\Re}{\mathbb{R}}
\newcommand{\rtres}{\ensuremath{\Re^3}}
\newcommand{\union}{\cup}
\newcommand{\dotprod}{\cdot}
%\newcommand*\pkg[1]{\textsf{#1}}

\newcommand{\trans}[1]{{#1}^{\ensuremath{\mathsf{T}}}} % transpose
\newcommand{\nbyn}[1]{\ensuremath{#1 \! \times \! #1 }}
\newcommand{\nbym}[2]{#1 \! \times \! #2 }       % Use in math mode.
\newcommand{\cat}[2]{#1\!\mathbin{\raise.6ex\hbox{\( {}^\frown \)}}\!#2}
\newcommand{\generalmatrix}[3]{ %arg1: low-case letter, arg2: rows, arg3: cols
               \left(
                  \begin{array}{cccc}
                    #1_{1,1}  &#1_{1,2}  &\ldots  &#1_{1,#2}  \\
                    #1_{2,1}  &#1_{2,2}  &\ldots  &#1_{2,#2}  \\
                              &\vdots                         \\
                    #1_{#3,1} &#1_{#3,2} &\ldots  &#1_{#3,#2}
                  \end{array}
               \right)  }
\newcommand{\colvec}[1]{\begin{pmatrix} #1 \end{pmatrix}}
\newcommand{\pr}[1]{\ensuremath{\mathrm{Pr}(#1)}}
\newcommand{\rep}[2]{ {\rm Rep}_{#2}(#1) }
\newcommand{\mapsunder}[1]{\stackrel{#1}{\longmapsto}}
\newcommand{\map}[3]{\mbox{$#1\colon #2\to #3$}}
\newcommand{\identity}{\mbox{id}}
\newcommand{\stdbasis}{{\cal E}} 
\newcommand{\sequence}[1]{ \langle#1\rangle } 
\newcommand{\spacer}{\rule[-3mm]{0mm}{8mm}}
\newcommand{\email}[1]{\url{#1}}
\newcommand{\zero}{\vec{0}}
\newcommand{\proj}[2]{\mbox{proj}_{#2}({#1}) }
%\AtBeginDocument{\newlength{\heightofcdot}
%\newlength{\widthofcdot}
%\settoheight{\heightofcdot}{$\cdot$}
%\settowidth{\widthofcdot}{$\cdot$}
%\newsavebox{\dotprodcircle}       
%\savebox{\dotprodcircle}{\includegraphics{dotprod.1}} 
%\newcommand{\dotprod}{\mathbin{\raisebox{.5\heightofcdot}{%
%          \makebox[\widthofcdot]{$\smash{\usebox{\dotprodcircle}}$}}}}}
\newcommand{\spanof}[1]{\relax [#1\relax ]} % no optional argument!
\newcommand{\set}[1]{\mbox{$\{#1\}$}} \newcommand{\suchthat}{\bigm|}
\newcommand{\deter}[1]{ \mathchoice{\left|#1\right|}{|#1|}{|#1|}{|#1|} }
\newcommand{\secuence}[1]{ \langle#1\rangle }  
\newcommand{\basis}[2]{\secuence{\vec{#1}_1,\ldots,\vec{#1}_{#2}}}



%--------linsys
%  Use as \begin{linsys}{3}
%           x &+ &3y &+ &a &= &7 \\
%           x &- &3y &+ &a &= &7
%         \end{linsys}
% Remark: TeXbook pp. 167-170 says to put a medmuskip around a +; and that's
% 4/18-ths of an em.  Why does 2/18-ths of an em work?  I don't know, but
% comparing to a regular displayed equation suggests it is right.
% (darseneau says LaTeX puts in half an \arraycolsep.)
\newenvironment{linsys}[2][m]{%
\setlength{\arraycolsep}{.1111em} % p. 170 TeXbook; a medmuskip
\begin{array}[#1]{@{}*{#2}{rc}r@{}}
}{%
\end{array}}


%\newtheorem{teorema}{Teorema}
%\newtheorem{exercici}{Exercici}
%\newtheorem{definicio}{Definici\'o}
%\newtheorem{theorem}{Theorem}
\newtheorem{exercise}{Exercise}
%\newtheorem{definition}{Definition}



\parskip 4mm


\usepackage{makeidx}
\makeindex




%\setcounter{section}{-1}

\theoremstyle{definition}
\newtheorem{thm}{Theorem}
\newtheorem{dfn}{Definition}
\newtheorem{lem}{Lemma}
\newtheorem{prp}{Proposition}





%%%%%%%%%%%%%%%%%%%
% ANGLÈS
%%%%%%%%%%%%%%%%%%%

% \newcommand{\problemName}{}%
% \newcounter{problemCounter}%
% \newenvironment{problem}[1][Problem \arabic{problemCounter}]%
% 	{\stepcounter{problemCounter}%
% 		\renewcommand{\problemName}{#1}%
% 		\section*{\problemName}%
% 		\nobreak\extramarks{\problemName}{\problemName continued on next page\ldots}\nobreak%
% 		\nobreak\extramarks{\problemName (continued)}{\problemName continued on next page\ldots}\nobreak}%
% 	{\nobreak\extramarks{\problemName (continued)}{\problemName continued on next page\ldots}\nobreak%
% 		\nobreak\extramarks{\problemName}{}\nobreak}%

\newenvironment{example}{ % 
	\definecolor{shadecolor}{rgb}{0.8,1.0,0.8} %
	\begin{shaded} %
	\textcolor{OliveGreen}{\bf Example\\}%
} % 
{ %	
	\end{shaded}
} %


\newenvironment{introduction}{ % 
	\definecolor{shadecolor}{rgb}{1.0,1.0,0.8} %
	\begin{shaded} %
	% \textcolor{BrickRed}{\bf Introduction\\}%
} % 
{ %	
	\end{shaded}
} %


%%%%%%%%%%%%%%%%%%%
% CATALÀ
%%%%%%%%%%%%%%%%%%%
\newtheorem{teorema}{theorem}
\newenvironment{definicio}{ % 
	\definecolor{shadecolor}{rgb}{0.9,1.0,0.8} %
	\begin{shaded} %
	\textcolor{OliveGreen}{\bf Definicio\\}%
} % 
{ %	
	\end{shaded}
} %

%veure http://en.wikibooks.org/wiki/LaTeX/Advanced_Topics

\newcommand{\doccmd}[1]{\texttt{\textbackslash#1}}% command name -- adds backslash automatically
\newcommand{\docopt}[1]{\ensuremath{\langle}\textrm{\textit{#1}}\ensuremath{\rangle}}% optional command argument
\newcommand{\docarg}[1]{\textrm{\textit{#1}}}% (required) command argument
\newenvironment{docspec}{\begin{quote}\noindent}{\end{quote}}% command specification environment
\newcommand{\docenv}[1]{\textsf{#1}}% environment name
\newcommand{\docpkg}[1]{\texttt{#1}}% package name
\newcommand{\doccls}[1]{\texttt{#1}}% document class name
\newcommand{\docclsopt}[1]{\texttt{#1}}% document class option name
\newcommand{\logos}{%
\begin{figure}
\includegraphics{FCTE}
\end{figure}
}

% margins
% \topmargin=-0.45in      %
% \evensidemargin=0in     %
% \oddsidemargin=0in      %
% \textwidth=6in        %
% \textheight=8.5in       %
% \headsep=0.25in         %

% header and footer
\pagestyle{fancy}       %
\chead{}                %
\makeatletter
\fancyfoot[R]{%
   % We want italics
   \itshape
   % The chapter number only if it's greater than 0
   \ifnum\value{chapter}>0 \@chapapp\ \thechapter. \fi
   % The chapter title
   \leftmark}
\makeatother

%\lfoot{\includegraphics[trim=-5cm 0 0 -3cm,width=0.4\textwidth]{FCTE}}      
\lfoot{\raisebox{-0.5cm}[0pt][0pt]{\includegraphics[width=3cm]{FCTE}}} 

\cfoot{}        %
\renewcommand\headrulewidth{0.4pt}   %
\renewcommand\footrulewidth{0.4pt}   %

% Essential Formatting
   
%\usepackage{epsfig,amsmath,amsthm,amssymb}
\usepackage[TYPE]{urmathtest_cat}[2001/05/12]
%\usepackage[answersheet]{urmathtest}[2001/05/12]
%\usepackage[answers]{urmathtest}[2001/05/12]
\usepackage{graphicx}

%% For use with pdflatex
%\pdfpagewidth\paperwidth
%\pdfpageheight\paperheight

% Basic User Defs

%\def\ds{\displaystyle}

%\newcommand{\ansbox}[1]
%{\work{
%  \pos\hfill \framebox[#1][l]{SCORE:\rule[-.3in]{0in}{.7in}}
%}{}}
%
%\newcommand{\ansrectangle}
%{\work{
%  \pos\hfill \framebox[6in][l]{SCORE:\rule[-.3in]{0in}{.7in}}
%}{}}

% Beginning of the Document
\cfoot{\bf Examen Parcial Química GEA-17UV}

\begin{document}
\examtitle{Enginyeria de l'Automoció}{Examen Parcial+Final Química GEA-17UV}{14 de Maig de 2018}
\studentinfo
\instructions{
  \textbf{Professor: Jordi Villà i Freixa}
  

  \begin{itemize}
  \item
    \textbf{No es permet l'ús d'ordinador. Només calculadora, apunts de classe i full d'exercicis resolts del campus virtual. Els dos primers exercicis es tindran en compte per a la nota del segon parcial. Els tres darrers exercicis es tindran en compte per a la nota de l'examen final.}
  \item
%   \textbf{Please show all your work if you need to.
%           You may use back pages if necessary.
%           You may not receive full credit for
%           a correct answer if there is no work shown.}
   \textbf{Desenvolupa tot el teu argumentari de forma clara.
           No usis més espai del proveït.}
%  \item
%    \textbf{Please put your \underline{simplified}
%            final answers in the spaces provided.}
  \end{itemize}
}
\finishfirstpage

% Problems Start Here % ----------------------------------------------------- %
%%%%%%%%%%%%%%%%%%%%%%%%%%%%%%%%%%%%%%%%
\problem{50}
{A partir de la configuració electrònica dels elements N, O i F, representa l'energia dels orbitals atòmics i dels orbitals moleculars que formaran les corresponents molècules diatòmiques \ch{N2}, \ch{O2} i \ch{F2}. Quina té major energia d'enllaç?
}
{
\vfill
\newpage
}
{
Comencem per representar la configuració electrònica dels tres elements
\begin{center}
\includegraphics[scale=0.5]{ConfElectNOF.png}
\end{center}
En fer les respectives molècules diatòmiques, només cal que ens fixem en els electrons de la capa de valència ($n=2$), ja que els $1s$ estaran molt mes interns i no participaran significativament a l'enllaç.
En el cas del F2 podem usar directament la configuració representada perquè només hi ha un electró disponible per a l'enllaç, que acaba essent ti pus $\sigma$ ($\sigma_{2p_z}$):
\begin{center}
\includegraphics[scale=0.4]{F2.png}
\end{center}
En el cas de,l'O i el N, en canvi, hem de repartir els electrons $2p$ de manera que ens permetin fer accessibles tots els electrons $2p$:
\begin{center}
\includegraphics[scale=0.5]{ConfElectNOF2.png}
\end{center}
A partir d'aquestes configuracions electròniques podem construir ela diagrames d'orbitals moleculars de les molècules \ch{N2} i \ch{O2} fent:
\begin{center}
\includegraphics[scale=0.55]{N2.png}
\includegraphics[scale=0.4]{O2.png}
\end{center}
Veiem que en el cas del \ch{N2} tots els electrons de valència es troben en orbitals enllaçants, formant un enllaç triple ($\sigma_{2p_z}$, $\pi_{2p_x}$, i $\pi_{2p_y}$). En canvi, en el cas de l'\ch{O2} hi ha dos electrons en orbitals antienllaçants que afebleixen l'enllaç respecte el que passava a \ch{N2}. 

Així, per ordre de força d'enllaç, $\ch{N2} > \ch{O2} > \ch{F2}$.
}

%%%%%%%%%%%%%%%%%%%%%%%%%%%%%%%%%%%%%%%%
%%%%%%%%%%%%%%%%%%%%%%%%%%%%%%%%%%%%%%%%
\problem{50}
{Ordena aquests metalls de forma raonada segons la seva capacitat reductora: Ca, Na, Ba, K, Ag.}
{
%\vspace{22cm}
\vfill
\newpage
}
{
En general, el potencial reductor (la capacitat de donar electrons) creix cap a l'esquerra i cap avall de la taula periòdica.
Seguint aquest raonament, podem dir que de ben segur Ag serà el menys reductor (el que tindrà menys capacitat de cedir electrons a un agent oxidant). En els altres casos les diferències són més subtils, però podem dir que el poder reductor del \ch{Ba} serà major que el del \ch{Ca} i que el del \ch{K} serà major que el del \ch{Na}. Així mateix, el poder reductor del \ch{Ca} és menor que el del \ch{K}. 

Si mirem una taula de potencials de reducció, podem comprovar que aquest raonament és correcte i que és complicat dir massa més sense dades:

\begin{tabular}{cc}
Reacció de reducció & $\varepsilon^{\circ}$ / V \\
\hline
\ch{Na+ _{(aq)} + e- -> Na_{(s)}} & -2.714 \\
\ch{K+ _{(aq)} + e- -> K_{(s)}} & -2.925 \\
\ch{Ca^{2+} _{(aq)} + 2 e- -> Ca_{(s)}} & -2.87 \\
\ch{Ag+ _{(aq)} + e- -> Ag_{(s)}} & +0.7994 \\
\ch{Ba^{2+} _{(aq)} + 2 e- -> Ba_{(s)}} & -2.9 \\
\end{tabular}

Per tant, pel que fa als seus potencials de reducció (la seva afinitat per acceptar electrons):
$
\varepsilon^{\circ} (\ch{Ag+}) >
\varepsilon^{\circ} (\ch{Na+}) >
\varepsilon^{\circ} (\ch{Ca^{2+}}) >
\varepsilon^{\circ} (\ch{Ba^{2+}}) >
\varepsilon^{\circ} (\ch{K+})$
Però és més pràctic analitzar els primers potencials d'ionització (PI) de cada element (l'energia necessària per arrencar-los un \ch{e-}):

\begin{tabular}{cc}
Reacció de reducció & PI / kJ mol$^{-1}$ \\
\hline
\ch{Na -> Na+ + e-} & 495.8 \\
\ch{K -> K+ + e-} & 418.8 \\
\ch{Ca -> Ca+ + e-} & 598.8 \\
\ch{Ag -> Ag+ + e-} & 731.0 \\
\ch{Ba -> Ba+ + e-} & 502.9 \\
\end{tabular}

i, per tant, $PI (\ch{Ag}) > PI (\ch{Ca}) > PI (\ch{Ba}) > PI (\ch{Na}) > PI (\ch{K})$. Això vol dir, per exemple, que Ag és el que té menys tendència a cedir electrons (el menys reductor) i en canvi el K té una molt més gran facilitat per a fer-ho (el més reductor), tot comprovant les afirmacions fetes més amunt. 
}

%%%%%%%%%%%%%%%%%%%%%%%%%%%%%%%%%%%%%%%%
\problem{30}
{Calcula l'energia de malla, $U$, del clorur de calci (l'energia necessària per formar \ch{CaCl2} a partir dels ions en fase gas) sabent:
\begin{itemize}
\item l'entalpia d'atomització del calci
\[\ch{Ca_{(s)} -> Ca_{(g)}} \quad \Delta H_a^{\circ}=178\; \mathrm{kJ \, mol}^{-1}\]
\item les entalpies d'ionització del calci: 
\[\ch{Ca_{(g)} -> Ca+ _{(g)} + e-} \quad \Delta H_{ei1}^{\circ}=590 \; \mathrm{kJ \, mol}^{-1}\]
\[\ch{Ca+ _{(g)} -> Ca^{2+} _{(g)} + e-} \quad \Delta H_{ei2}^{\circ}=1145 \; \mathrm{kJ \, mol}^{-1}\]
\item l'entalpia d'atomització del clor
\[\ch{1/2 Cl2_{(s)} -> Cl_{(g)}} \quad \Delta H_a^{\circ}=\frac{1}{2} \Delta H_{\ch{Cl-Cl}}^{\circ}=121 \; \mathrm{kJ \, mol}^{-1}\]
\item L'afinitat electrònica del clor
\[\ch{Cl_{(g)} + e-  -> Cl-_{(g)}} \quad \Delta H_{ae}^{\circ}=-364 \; \mathrm{kJ \, mol}^{-1}\]
\item l'entalpia de formació del \ch{CaCl2_{(s)}}, $\Delta H_f^{\circ}=-796$ kJ mol$^{-1}$.
\end{itemize}
}
{
\vfill
\newpage
}
{
Seguint el cicle de Born Haber, es pot veure com
\[U=\Delta H_f^{\circ} (\ch{CaCl2}) - \Delta H_a^{\circ} (Ca) - \Delta H_{\ch{Cl-Cl}}^{\circ} - (\Delta H_{ei1}^{\circ}+\Delta H_{ei2}^{\circ})-2 \times \Delta H_{ae}^{\circ}\]
\[U=-796-178-242-(590+1145)-2\times (-364))=-2223\mathrm{kJ \, mol}^{-1}\]
}

%%%%%%%%%%%%%%%%%%%%%%%%%%%%%%%%%%%%%%%%
\problem{40}
{Considera un mol d'un gas ideal dins d'un cilindre tancat amb un pistó, ocupant un volum de 10l i a una temperatura de 80$^{\circ}$C. Després d'un procés isoterm, el gas ocupa un volum de 15l.

Calcula $w$ i $q$ per a cadascun d'aquests dos processos (tingues presenta que l'energia interna només depèn de la temperatura):
\begin{itemize}
\item si es produeix l'expansió tot alliberant de cop el pistó fins al nou volum, contra la pressió atmosfèrica
\item si el procés té lloc de forma reversible.
\end{itemize}

Si, en el primer cas, l'expansió s'hagués produït contra el buit, quins valors de $w$ i $q$ hauríem aconseguit?

(La constant dels gasos en unitats del SI és 8.31 J mol$^{-1}$ K$^{-1}$)
}
{
\vfill
\newpage
}
{
Si l'energia interna només depèn de $T$, $\Delta U=0$ per a un procés isoterm. Per tant, només caldrà tenir en compte que $w=-q$ per a tots dos processos. Així, si som capaços de calcular el treball ja tindrem el problema resolt.

{\bf Cas 1: expansió lliure contra la pressió atmosfèrica (constant)}

En aquest cas, el treball es pot calcular segons
\[w = -P_{ext} \Delta V\]
\[w = -101325 \, \mathrm{Pa} \times 5 \cdot 10^{-3} \, \mathrm{m}^3 = 506.6 J = -q\]

{\bf Cas 2: expansió reversible}

En aquest cas, l'expansió es produeix reversiblement, i per tant la pressió no és sobtadament l'atmosfèrica, sinó que va variant progressivament i, per tant, no és contsnat. Hem d'aplicar l'expressió del treball i integrar-la:
\[w= -\int_{V_1}^{V_2} p dV \]
si el gas és ideal:
\[w= -\int_{V_1}^{V_2} \frac{nRT}{V} dV = -nRT \int_{V_1}^{V_2} \frac{1}{V} dV = -nRT \log_e \frac{V_f}{V_i} \]
Substituint
\[w=-8.31 \log_e \frac{15}{10} = 1189.4 J = -q\]

Finalment, si l'expansió s'hagués produït segons el primer cas, però contra el buit, $w=q=0$. 

}

%%%%%%%%%%%%%%%%%%%%%%%%%%%%%%%%%%%%%%%%%
\problem{30}
{
Tenint en compte que 
\[\ch{Fe^{3+} + e- ->  Fe^{2+}} \quad \varepsilon^{\circ} = 0.77 \, \mathrm{V} \] 
i
\[\ch{I2 + e- ->  2 I-} \quad \varepsilon^{\circ} = 0.54 \, \mathrm{V} \]
si tinc una pila formada per un electrode d'ions \ch{Fe^{2+}}, \ch{Fe^{3+}} i un electrode amb \ch{I2} i \ch{I-}, quina és la direcció espontània de la reacció?
}
{
\vfill

}
{
Si volem que una pila sigui espontània, és el mateix que dir que el seu potencial sigui positiu. Per tant, la reacció es donarà enla direcció de reduir els ions fèrrics i oxidar l'ió iodur.
\[\ch{2 Fe^{3+} + 2 I- ->  2 Fe^{2+} + I2} \quad \varepsilon^{\circ} = 0.23 \, \mathrm{V} \]
}


% Problems End Here % ------------------------------------------------------- %

\problemsdone
\end{document}

% End of the Document
