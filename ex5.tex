\begin{qst}
La composició percentual, en massa, de l'aire sec al nivell del mar és, aproximadament, \ch{N2}/\ch{O2}/\ch{Ar}=75.5/23.2/1.3. 
Quina és la pressió parcial de cada component quan la pressió total és 1.20 atm?
(extret de \cite{Atkins2018}).
\end{qst}
\lct{
En 100gr d'aire tindrem 75.5, 23.2 i 1.3 gr de \ch{N2}, \ch{O2} i \ch{Ar}, respectivament. Podem calcular la seva fracció molar calculant el número de mols de cadascun i dividint pel total. Després, només cal multiplicar per la pressió corresponent i sabrem la pressió parcial de cada component:

\[n_{\ch{N2}}=75.5 \cancel{g} \cdot \frac{1 mol}{28.02 \cancel{g}}=2.69 mol\]
\[n_{\ch{O2}}=23.2 \cancel{g} \cdot \frac{1 mol}{32.00 \cancel{g}}=0.725 mol\]
\[n_{\ch{Ar}}=1.3 \cancel{g} \cdot \frac{1 mol}{39.95 \cancel{g}}=0.033 mol\]

    \begin{tabular}{cccc}
      \hline
        & \ch{N2} & \ch{O2} & \ch{Ar} \\
      \hline
Fracció molar &	0.780 & 0.210 & 0.0096\\
Pressió parcial (nivell del mar)/atm & 0.780 & 0.210 & 0.0096\\
Pressió parcial ($P_T=1.20$atm))/atm & 0.936 & 0.252 & 0.012\\
      \hline
    \end{tabular}

}
