\usepackage{chemfig,chemmacros,chemnum}
\chemsetup[reactions]{
    before-tag = R,
    tag-open = [ , tag-close = ]
}
\chemsetup{
  formula = chemformula , % or mhchem
  modules = thermodynamics
}
\setchemfig{scale=0.3}
\renewcommand*\printatom[1]{\ensuremath{\mathsf{#1}}}
\usepackage{chemformula}

\usepackage{siunitx}
\sisetup{output-decimal-marker = {,}}

% Definició d'unitats personalitzades per al sistema imperial
\DeclareSIUnit\inch{in}
\DeclareSIUnit\foot{ft}
\DeclareSIUnit\atm{atm}
\DeclareSIUnit\oz{oz}
\DeclareSIUnit\ounce{oz}
\DeclareSIUnit\pound{lb}
\DeclareSIUnit\ton{t}
\DeclareSIUnit\cal{cal}
\DeclareSIUnit\kcal{kcal}

\DeclareSIUnit\cubicinch{\inch\cubed}
\DeclareSIUnit\cubicfoot{\foot\cubed}
\DeclareSIUnit\gallon{gal}
\DeclareSIUnit\psi{psi}
\DeclareSIUnit\inchHg{inHg}
\DeclareSIUnit\dyn{dynes}
\DeclareSIUnit\torr{Torr}
\DeclareSIUnit\Torr{Torr}
\DeclareSIUnit\inch{in}
\DeclareSIUnit\foot{ft}
\DeclareSIUnit\atm{atm}
\DeclareSIUnit\oz{oz}
\DeclareSIUnit\ounce{oz}
\DeclareSIUnit\pound{lb}
\DeclareSIUnit\ton{t}
\DeclareSIUnit\fah{F}
\DeclareSIUnit\cycle{cicle}
\DeclareSIUnit\hp{hp}
\DeclareSIUnit\rpm{rpm}
\DeclareSIUnit{\degreeFahrenheit}{\unit{\degree}F}
\newcommand{\degC}{\degreeCelsius}
\newcommand{\degF}{\degreeFahrenheit}


\usepackage{../common/mol2chemfig}
