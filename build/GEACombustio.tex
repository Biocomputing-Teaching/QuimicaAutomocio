\chapter{Combustió}
 
%\newpage

\section{El motor de combustió interna}

Un motor de combustió interna (IC) és un conjunt d'elements mecànics que permeten obtenir energia mecànica a partir de l'estat tèrmic d'un fluid de treball generat en el seu propi interior mitjançant un procés de combustió.
 
Els motors de combustió interna, ja siguin alternatius o de reacció, són les principals fonts d'energia en el transport terrestre, marítim i aeri gràcies a la seva elevada potència específica. Aquests motors només competeixen amb els motors elèctrics en algunes aplicacions del transport ferroviari i, de manera creixent, en vehicles elèctrics purs o en configuració híbrida\cite{de_antonio_motores_2015}. 

En un motor de combusti\'o interna s'introdueix aire i combustible. En els motors d'encesa per espurna, la mescla d'aire i combustible es preparava antigament en el carburador i es condu\"ia al cilindre. Ara es realitza per mitj\`a d'injectors, cosa que permet un estalvi de combustible i un millor aprofitament d'aquest. En els motors d'encesa per compressi\'o (Diesel), la mescla es realitza directament dins del cilindre, on el combustible s'injecta despr\'es d'haver-hi introdu\"it i comprimit l'aire. Cada cilindre del motor t\'e una v\`alvula d'admissi\'o i una d'escapament, que s'obren i tanquen en el moment oport\'u per permetre l'entrada i sortida de gasos. Els motors típics tenen entre 3 i 12 cilindres, i la pot\`encia es pot augmentar afegint m\'es cilindres.

La paret de la cambra de combustió està formada per una camisa de ferro o alumini, i està inserida en un bloc de ferro o acer.

La mescla comprimida a la cambra de combusti\'o es transforma, per efecte de la combusti\'o, en vapor d'aigua (\ch{H2O}), di\`oxid de carboni (\ch{CO2}) i nitrogen (\ch{N2}). El nitrogen, un gas inert contingut a l'aire, no interv\'e en la combusti\'o. El vapor d'aigua produ\"it en la combusti\'o es mant\'e i es comporta com un gas permanent.

Entre els altres productes de la combusti\'o es troben altres gasos com: mon\`oxid de carboni (\ch{CO}), hidrogen (\ch{H2}), metà (\ch{CH4}) i oxigen (\ch{O2}), quan la combusti\'o \`es incompleta. La quantitat d'oxigen que participa en el proc\'es dep\`en directament de l'exc\'es d'aire introdu\"it respecte al necessari per a la combusti\'o.

En conseq\"u\`encia, el fluid de treball est\`a format inicialment per l'aire i el combustible i, despr\'es, pel conjunt de gasos produ\"its durant la combusti\'o. Com \`es evident, la seva composici\'o qu\'imica varia durant el cicle de treball.


\subsection{El motor de quatre temps}

    Un motor de quatre temps és aquell que necessita quatre recorreguts del pistó, dues voltes completes del cigonyal, per completar el seu cicle termodinàmic (veure animació a \url{https://www.grc.nasa.gov/www/k-12/airplane/engopt.html}).

    \newif\ifspark
\tikzset{tangent of circles/.style args={% https://tex.stackexchange.com/a/464143/194703
    at #1 and #2 with radii #3 and #4}{insert path={%
    let \p1=($(#2)-(#1)$),\n1={atan2(\y1,\x1)},\n2={veclen(\y1,\x1)*1pt/1cm},
    \n3={atan2(#4-#3,\n2)}
     in ($(#1)+(\n3+\n1+90:#3)$) coordinate(aux1) -- 
     ($(#2)+(\n3+\n1+90:#4)$) coordinate(aux2)}},
     pics/engine/.style={code={
  \tikzset{engine/.cd,#1}
  \draw[fill=gray!20] (0,0) -- (-0.8,-0.4) coordinate[pos=0.4] (p1)
  coordinate[pos=0.8] (p2) |- (-1,-3)[rounded corners=1mm] |- (-1.2,0) [sharp corners]
  -- (-1.2,0.7) coordinate[pos=0.2] (p3)
  coordinate[pos=0.8] (p4) -- (-0.9,0.85) -- (-0.6,0.7) -- (0,0.4) -- (0.6,0.7)
  -- (0.9,0.85)-- (1.2,0.7) -- (1.2,0)coordinate[pos=0.2] (p6)
  coordinate[pos=0.8] (p5) {[rounded corners=1mm] -- (1,0)}
  [sharp corners] -- (1,-3)
  -| (0.8,-0.4) -- cycle coordinate[pos=0.2] (p8)
  coordinate[pos=0.6] (p7);
  \draw[engine/left exhaust] (p1) to[bend right=18] (p4) -- (p3) to[bend left=18] (p2) -- cycle;
  \draw[engine/right exhaust] (p7) to[bend left=18] (p6) -- (p5) to[bend right=18] (p8) -- cycle;
  \draw[fill=gray!50] (0,-4) circle[radius=5mm];
  \pgfmathsetmacro{\pistonpos}{-4+0.4*sin(\pgfkeysvalueof{/tikz/engine/rod angle})
  +sqrt(1.5*1.5-pow(0.4*cos(\pgfkeysvalueof{/tikz/engine/rod angle}),2))}
  \path (0,-4) + (\pgfkeysvalueof{/tikz/engine/rod angle}:0.4) coordinate (p9)
   (0,\pistonpos) coordinate (p10);
  \draw[fill=gray!15] (p9) circle [radius=2mm] -- (p10) circle [radius=1mm];
  \path[tangent of circles={at p10 and p9 with radii 0.1 and 0.2}]
  (aux1) coordinate (aux3) (aux2) coordinate (aux4); 
  \path[tangent of circles={at p9 and p10 with radii 0.2 and 0.1}];
  \path[fill=gray!15] (aux1) -- (aux2) -- (aux3) -- (aux4);
  \draw  (aux1) -- (aux2)  (aux3) -- (aux4);
  \path[fill=gray!45] (p9) circle [radius=1.2mm];
  \begin{scope}
   \clip (-0.8,\pistonpos)   rectangle ++ (1.6,1);
   \draw[left color=gray!60,right color=gray!50,middle color=gray!10] (-0.8,\pistonpos) 
  rectangle ++ (2,1);
  \end{scope}
  \draw[left color=\pgfkeysvalueof{/tikz/engine/interior color}!80,
  right color=\pgfkeysvalueof{/tikz/engine/interior color}!50,
  middle color=white] 
  (-0.8,\pistonpos+1) --  (-0.8,-0.4)  -- (0,0)--  (0.8,-0.4) |- cycle;
  \draw[thin,fill=gray!30] (-0.42,-0.5) 
   ++ ({90+atan(1/2)}:0.25*\pgfkeysvalueof{/tikz/engine/left valve}) 
   -- ++ ({90+atan(1/2)}:1.9) -- ++ ({atan(1/2)}:0.1)
   -- ++ ({-90+atan(1/2)}:1.9) -- ++({atan(1/2)}:0.3)
   -- ++ ({-90+atan(1/2)}:0.1) -- ++({atan(1/2)+180}:0.7)
   -- ++ ({90+atan(1/2)}:0.1) -- cycle;
  \draw[thin,fill=gray!30] (0.42,-0.5) 
   ++ ({90-atan(1/2)}:0.25*\pgfkeysvalueof{/tikz/engine/right valve}) 
   -- ++ ({90-atan(1/2)}:1.9) -- ++ ({180-atan(1/2)}:0.1)
   -- ++ ({-90-atan(1/2)}:1.9) -- ++({180-atan(1/2)}:0.3)
   -- ++ ({-90-atan(1/2)}:0.1) -- ++({-atan(1/2)}:0.7)
   -- ++ ({90-atan(1/2)}:0.1) -- cycle;
  \draw[left color=gray!60,right color=gray!50,middle color=gray!10]
   (-0.1,-0.2) rectangle (0.1,1);   
  \ifspark
  \begin{scope}
   \clip (-1.8,-0.2) rectangle (1.8,\pistonpos+1.1);
   \path (0,-0.2) node[starburst, inner color=yellow, outer color=red,minimum size=1cm]{};
  \end{scope}
  \fi 
 }},engine/.cd,left valve/.initial=1,right valve/.initial=1,
 left exhaust/.style={fill=gray!50},
 right exhaust/.style={fill=gray!50},
 rod angle/.initial=30,interior color/.initial=white,
 spark/.is if=spark}
 \begin{center}
 \scalebox{0.8}{
\begin{tikzpicture}[] 
 \path (0,0) pic{engine={left valve=0,rod angle=-40,
  left exhaust/.style={fill=gray!10}}}
 (3.2,0) pic{engine={rod angle=-170,interior color=yellow}}
 (6.4,0) pic{engine={rod angle=105,interior color=orange,spark}}
 (9.6,0) pic{engine={rod angle=-80,interior color=red}}
 (12.8,0) pic{engine={right valve=0,rod angle=-170,interior color=purple,
    right exhaust/.style={fill=purple!30}}};
\end{tikzpicture}
 }
\end{center}

\begin{itemize}

\item{Primer pas o admissió}
En aquesta etapa, quan el pistó baixa des del Punt Mort superior (PMS o, en anglès, top dead center, TDC) al Punt Mort Inferior (PMI o, en anglès bottom dead center, BDC), permet que el nou combustible entri per la vàlvula d'injecció. Mentre s'obre aquesta vàlvula, la d'escapament es manté tancada.

\item{Segon pas o compressió}
Al final de l'execució anterior, el gas dins del cilindre es comprimeix per mitjà del moviment ascendent del pistó, de manera que la vàlvula d'injecció es tanca per la pressió.

\item{Tercer pas o explosió/expansió}
Després del temps de compressió, quan el pistó torna a la posició superior, s'obté la pressió màxima dins del cilindre. En el nostre cas, tenim un motor dièsel, per la qual cosa el combustible s'injecta polvoritzat i es crema per mitjà de la pressió i la temperatura dins del cilindre. Aleshores, l'expansió del gas fa que el pistó es mogui de nou cap avall; és en aquest moment quan es crea el treball de tot el procés. El treball d'expansió obtingut és aproximadament cinc vegades el treball de compressió necessari.

\item Quart pas o escapament
En aquest últim pas, el moviment superior del pistó fa que els gasos de combustió surtin a través de la vàlvula d'escapament. Quan el pistó està a la part superior, la vàlvula d'escapament es tanca i la injecció s'obre perquè tot el procés es torni a iniciar.
\end{itemize}

El cigonyal completa dues voltes (720 graus) per cada cicle de quatre temps. Així, el motor de quatre temps necessita dues voltes completes del cigonyal per completar el seu cicle termodinàmic.

Molts dels comportaments del motor es poden descriure mitjan\c{c}ant els conceptes de les lleis dels gasos. Per exemple, segons la llei de Boyle, quan augmenta el volum de la cambra de combusti\'o durant l'aspiraci\'o, la pressi\'o disminueix i permet que l'aire entri al cilindre. Durant la compressi\'o, el gas s'escalfa i augmenta la pressi\'o. L'expansi\'o dels gasos calents, descrita per la llei de Charles, \`es el mecanisme pel qual es captura l'energia de la combusti\'o i es converteix en energia mec\`anica per impulsar el vehicle\cite{bowers_understanding_2014}.

    \subsection{Fases del Cicle Otto ideal}

    La Figure \ref{fig:otto} mostra els processos termodinàmics que es donen en el cicle Otto\cite{morales_caracterizacion_nodate}:
\begin{enumerate}
    \item 0-1 Aspiraci\'o (proc\'es isoc\`oric): 
    La v\'alvula d'admissi\'o s'obre i s'aspira una c\`arrega d'aire i combustible a una pressi\'o te\`oricament igual a l'atmosf\`erica, provocant el descens del pist\'o. La v\'alvula d'escapament roman tancada. L'injector de fuel  genera un aerosol de combustible, en forma d'una fina boira de gotes minúscules, que es barreja amb l'aire aspirat.
    
    \item 1-2 Compressi\'o (proc\'es adiab\`atic):
    No existeix intercanvi de calor entre el gas i les parets del cilindre. La v\'alvula d'admissi\'o i la d'escapament estan tancades i el pist\'o comen\c{c}a a pujar, comprimint la mescla que es vaporitza.
    
    \item 2-3 Combusti\'o (proc\'es isoc\`oric):
    Ambdues v\'alvules romanen tancades. Quan el pist\'o arriba a la part superior del seu recorregut, el gas comprimit s'inflama per l'espurna de la bugia. La combusti\'o de tota la massa gasosa \`es instant\`ania, per la qual cosa el volum no variar\`a i la pressi\'o augmentar\`a r\`apidament. Això és degut a que la reacció genera molts més mols de gas que els inicials, i la temperatura augmenta enormement degut a la reacció química.
    
    \item 3-4 Expansi\'o (proc\'es adiab\`atic): 
    El gas inflamat empeny el pist\'o. Durant l'expansi\'o, no hi ha intercanvi de calor i, en augmentar el volum, la pressi\'o tamb\'e augmenta.
    
    \item 4-1 Escapament (proc\'es isoc\`oric)
    Quan el pist\'o es troba en l'extrem inferior del seu recorregut, la v\'alvula d'admissi\'o roman tancada i s'obre la d'escapament, disminuint r\`apidament la pressi\'o sense variar el volum interior. Despr\'es, mantenint la pressi\'o igual a l'atmosf\`erica, el volum disminueix.
\end{enumerate}
    
    
        \begin{figure}
            \centering
            \scalebox{0.8}{
    \begin{tikzpicture}[annotate/.style 2 args={postaction={decorate,decoration={markings,
        mark=at position 0 with {\node[circle,inner sep=1.5pt,draw,fill=white,#1]{};},
        mark=at position 0.5 with {\arrow[>=stealth,line width=1.5pt]{>};
        \node at (0,0.4) {#2};}}}}]
         \draw[stealth-stealth] (0,5) node[below left]{$p$} |- (5,0) node[below left]{$V$};
         \begin{scope}[thick]
          \draw[annotate={label=below right:1,alias=1}{$\dbar Q=0$}] plot[variable=\x,domain=4:1.5] (\x,{5/(\x+3)});
          \draw[annotate={label=below left:2,alias=2}{}] (1.5,5/4.5) -- (1.5,15/4.5);
          \draw[annotate={label=above left:3,alias=3}{$\dbar Q=0$}] plot[variable=\x,domain=1.5:4] (\x,{15/(\x+3)});
          \draw[annotate={label=above right:4,alias=4}{}] (4,15/7) -- (1);  
         \end{scope} 
         \path (2) -- (3) coordinate[pos=0.5] (23) (1) -- (4) coordinate[pos=0.5] (14);
         \draw[stealth-] ([xshift=-2mm]23) -- ++ (-1,0) node[midway,above]{$\Delta Q_h$};
         \draw[-stealth] ([xshift=2mm]14) -- ++ (1,0) node[midway,above]{$\Delta Q_c$};
         \draw[dashed] (1) -- (1|-0,0) node[below] {$V_1$};
         \draw[dashed] (2) -- (2|-0,0) node[below] {$V_2$};
        \end{tikzpicture}
        }
        \includegraphics[scale=0.9]{../figures/Otto-real.png}
        \caption{La termodinàmica del cicle d'Otto. A l'esquerra, la situació ideal, on els processos d'expansió i compressió són adiabàtics, mentre que els de combustió i escapament són isocòrics. A la dreta, un esquema del cicle real.}
        \label{fig:otto}
    \end{figure}

    \subsection{Cicle Otto Real}

    El procés Otto real (Figura \ref{fig:otto}) s'allunya \href{http://tesla.us.es/wiki/index.php/Ciclo_Otto}{de forma significativa} de l'ideal. 



    \begin{enumerate}
        \item 0-1 Aspiraci\'o: 
        La pressi\'o del gas durant l'aspiraci\'o \'es inferior a la pressi\'o atmosf\`erica, per tant, el tancament de la v\'alvula d'admissi\'o es produeix despr\'es que el pist\'o arriba a l'extrem inferior de la seva carrera. Aix\`o prolonga el per\'iode d'admissi\'o i permet l'entrada de la m\`axima quantitat de mescla d'aire i combustible al cilindre.
        
        \item 1-2 Compressi\'o:
        El gas cedeix calor al cilindre, cosa que fa que es refredi i adquireixi menys pressi\'o.
        
        \item 2-3 Combusti\'o:
        La combusti\'o no \`es instant\`ania i el volum de la mescla varia mentre es propaga la inflamaci\'o. Per obtenir un m\`axim treball, \'es essencial triar el moment adequat per a l'encesa. La xispa ha de saltar abans que el pist\'o hagi finalitzat la carrera de compressi\'o, cosa que augmenta considerablement la pressi\'o assolida despr\'es de la combusti\'o i, per tant, el treball guanyat.
        
        \item 3-4 Expansi\'o: 
        L'augment de temperatura dins del cilindre durant la combusti\'o fa que, durant l'expansi\'o, els gasos cedeixin calor al cilindre i es refredin, resultant en una pressi\'o menor. Per tant, es tracta d'un procés no adiabàtic.
        
        \item 4-1 Escapament:
        En realitat, l'escapament no es produeix instant\`aniament. Els gasos encara tenen una pressi\'o superior a l'atmosf\`erica en aquest per\'iode. Per aix\`o, la v\'alvula d'escapament s'obre abans que el pist\'o arribi a l'extrem inferior del seu recorregut, permetent que la pressi\'o del gas disminueixi a mesura que el pist\'o acaba la seva carrera descendent. Quan el pist\'o realitza la seva carrera ascendent, troba davant seu gasos ja gaireb\'e totalment expandits. A m\'es, la v\'alvula d'admissi\'o s'obre abans que el pist\'o arribi a l'extrem superior del seu recorregut, generant una certa depressi\'o en el cilindre que afavoreix una aspiraci\'o m\'es en\`ergica.
    \end{enumerate}



\subsection{Cicle Diesel}

El motor Diesel \`es un motor de combusti\'o interna basat en el cicle Otto, per\`o amb la difer\`encia que el combustible s'injecta despr\'es de la compressi\'o de l'aire. 

Durant l'aspiraci\'o, entra nom\'es aire en el cilindre. En la compressi\'o, l'aire s'escalfa i, quan el pist\'o arriba al punt mort superior, s'injecta el di\`esel. Un motor diesel presenta uns factors de compressió molt més elevats que un motor Otto, i per tant, la temperatura de l'aire comprimit és molt més alta. Això permet que el dièsel s'encengui per la pressió i la temperatura de l'aire comprimit, sense necessitat d'una espurna. Finalment, l'escapament funciona de manera similar al motor d'encesa per espurna. 

Aquest motor permet una major efici\`encia t\'ermica i t\'e avantatges econ\`omics en diverses aplicacions. Tot i aix\`o, presenta dificultats t\`ecniques en sistemes d'injecci\'o i combusti\'o. Per garantir una combusti\'o neta i eficient, el proc\'es es realitza en mil·lisegons. Els motors Diesel usen ratios combustible/aire molt més baixos, amb la qual cosa la combustió és més completa.





    \section{Reaccions de combustió}

    Per definici\'o, una reacci\'o de combusti\'o \`es qualsevol reacci\'o entre un material i un oxidant [t\'ipicament \ch{O2 (g)}] que allibera energia en forma de calor. Les reaccions qu\'imiques alteren els tipus d'enlla\c{c}os i les posicions relatives dels \`atoms dins de les mol\'ecules. Els materials inicials s'anomenen reactius, i els materials finals despr\'es de la reordenaci\'o s'anomenen productes. En una reacci\'o de combusti\'o, el material inicial no oxidant s'anomena combustible i pot ser una varietat de compostos qu\'imics\cite{bowers_understanding_2014}.

Normalment, la combusti\'o es presenta en qu\'imica general i org\`anica com la reacci\'o dels combustibles hidrocarbonats amb l'oxigen per produir di\`oxid de carboni i aigua:
\begin{equation}
\ch{CH4 + 2 O2 -> CO2 + 2 H2O}
\end{equation}

No obstant aix\`o, els combustibles org\`anics contenen m\'es elements que nom\'es carboni i hidrogen, i produeixen altres gasos a banda del di\`oxid de carboni i l'aigua. Els motors de combusti\'o interna tamb\'e generen combustibles hidrocarbonats no cremats i els anomenats NOx t\'ermics, gasos amb la f\'ormula \ch{NO_x} que es formen quan el nitrogen atmosf\`eric es torna molt calent i reacciona amb l'oxigen atmosf\`eric. Aquests gasos contribueixen a les emissions dels motors i es redueixen mitjan\c{c}ant tecnologies d'emissions que es tractaran més endavant.


\subsection{Destil·lació del petroli}

La majoria de motors de combustió interna de gasolina i dièsel estan dissenyats per utilitzar fraccions específiques d'hidrocarburs obtingudes del petroli cru. El petroli és una barreja complexa de compostos orgànics provinents de la descomposició de microorganismes marins enterrats. Només els components més lleugers i volàtils són adequats com a combustible per a vehicles\cite{bowers_understanding_2014}.

Aquests components se separen del petroli mitjançant destil·lació, un procés on el líquid s'escalfa fins a bullir, i els vapors es refreden i es condensen en un recipient. En la destil·lació industrial, això es fa en una torre de destil·lació, un cilindre metàl·lic on els diferents components del petroli es condensen a diferents alçades segons el seu punt d'ebullició. Els compostos més lleugers surten per la part superior com a vapor, mentre que els més pesats es condensen més avall (Figura \ref{fig:torredestillacio}).

\begin{figure}
    \centering
    \includegraphics[width=\textwidth]{TorreDestillacio.jpg}
    \caption{Destil·lació fraccionada del petroli\cite{noauthor_38_2015}.}
    \label{fig:torredestillacio}
\end{figure}


El dièsel es destil·la entre 200°C i 350°C i conté hidrocarburs amb entre 8 i 21 àtoms de carboni. La gasolina, més volàtil, conté alcans lieals (parafines), alcans cíclics (naftalens) i alquens (olefines) de 4 a 12 carbonis i es destil·la a temperatures més baixes, pel fet de ser més volàtil. Tant la gasolina com el dièsel inclouen additius químics per millorar la seva estabilitat i resistència a la compressió. Aquests additius solen ser substàncies orgàniques contenir nitrogen, fòsfor i oxigen, i també compostos aromàtics (anells de carboni amb enllaços híbrids).

Per simplificar, l'anàlisi de la combustió es centrarà en la gasolina i un dels seus principals components, l'octà, tot i que el mateix principi s'aplica a altres combustibles.

\subsection{L'índex d'octà}

    Què significa el número d'octà de la benzina o per què alguns cotxes necessiten gasolina premium? El número d'octà mesura la resistència del combustible a la ignició espontània quan es comprimeix.

    L'octà, o n-octà, és un hidrocarbur de la família dels alquans amb fórmula molecular \ch{C8H18}. És un líquid incolor, inodor i inflamable. És un component important de la gasolina, ja que té una estructura lineal que li permet tenir una alta resistència a la detonació. Això fa que sigui un combustible ideal per a motors d'alta compressió.

En un motor de combustió interna, el combustible ha de cremar quan s'encén la bugia. Si la compressió fa que es detoni abans d'horaes poden danyar components com vàlvules i pistons. Això es coneix com a picat de biela o preignició.

El número d'octà es determina en un laboratori, cremant el combustible en un motor amb ràtio de compressió variable fins que es detecta el picat. A partir d'això, es compara amb una barreja de \href{https://www.ebi.ac.uk/chebi/searchId.do?printerFriendlyView=true&chebiId=62805&structureView=applet}{isooctà} i heptà amb la mateixa resistència a la detonació. El número d'octà indica el percentatge d'isooctà en aquesta barreja equivalent. Per exemple, un combustible amb un número d'octà de 90 té la mateixa resistència a la preignició que una barreja del 90\% d'isooctà i 10\% d'heptà (veure Taula \ref{tab:octa}).

És important saber que aquest número no indica la quantitat real d'octà en la benzina. Hi ha altres compostos amb més resistència a la detonació que poden donar valors superiors a 100. En resum, com més alta sigui la ràtio de compressió del motor, més alt ha de ser el número d'octà per evitar problemes de preignició.  
\
\begin{table}[h!]
    \centering
    \caption{Taula de compostos amb les seves fórmules condensades i índex d'octà (Adaptada de \cite{noauthor_38_2015}). }
    \renewcommand{\arraystretch}{1.5}
    \scriptsize
    \begin{tabular}{p{1cm}cc|p{1cm}cc}
        \toprule
        \textbf{Nom} & \textbf{Fórmula } & \textbf{Índex} & \textbf{Nom} & \textbf{Fórmula } & \textbf{Índex} \\
        \midrule
        n-heptà & \ch{CH3-(CH2)5-CH3} & 0 &
        o-xilè & \chemfig{[:-30]**6(--(-CH3)-(-CH3)--(-[,,,,,draw=none])-)} & 107 \\
        n-hexà & \ch{CH3-(CH2)4-CH3} & 25 & 
        etanol & \ch{CH3CH2OH} & 108 \\
        n-pentà & \ch{CH3-(CH2)3-CH3} & 62 &
        t-butil alcohol & \ch{(CH3)3COH} & 113 \\
        isooctà & \ch{(CH3)3CCH2CH(CH3)2} & 100 &
        p-xilè & \chemfig{[:-30]**6((-CH3)---(-CH3)---)} & 116 \\
        benzè & \chemfig{[:-30]**6(------)} & 106 &
        metil terc-butil èter & \ch{H3COC(CH3)3} & 116 \\
        metanol & \ch{CH3OH} & 107 &
        toluè & \chemfig{[:-60]*6(-=-=(-CH3)-=)} & 118 \\
        \bottomrule
    \end{tabular}
    \normalsize
    \label{tab:octa}
\end{table}


Molts compostos actuals tenen un índex d'octà superior a 100, cosa que els fa millors combustibles que l'isooctà pur. A més, s'han desenvolupat agents anticolp, també anomenats potenciadors d'octà. Durant molts anys, un dels més utilitzats va ser el tetraetilplom \ch{(C2H5)4Pb}, que a una concentració d'aproximadament \qty{11.36}{\gram\per\litre} augmentava l'índex d'octà en 10-15 punts. No obstant això, des de 1975, els compostos de plom han estat progressivament eliminats com a additius de la gasolina a causa de la seva elevada toxicitat.

Per substituir-los, es van desenvolupar altres potenciadors com el metil terc-butil èter (MTBE), que té un índex d'octà elevat i causa poca corrosió al motor i al sistema de combustible. Tanmateix, les fuites de gasolina amb MTBE des de dipòsits subterranis han contaminat aigües subterrànies en algunes zones, fet que ha portat a restriccions o prohibicions del seu ús. Com a alternativa, s'està promovent l'ús d'etanol, que es pot obtenir de fonts renovables com el blat de moro, la canya de sucre o les gramínies.

\section{Termodinàmica de la combustió}

Les reaccions químiques poden ser, a nivell del calor que intercanvien amb l'entorn:
\begin{description}
\item[exotèrmiques] si desprenen calor i, per tant, l'energia dels productes és més baixa que la dels reactius; o bé
\item[endotèrmiques] si l'absorbeixen i els productes acaben tenint més energia que els reactius.
\end{description}

En moltes ocasions mirem d'obtenir treball a partir de la calor produïda en una reacció, com succeix, per exemple, en un procés de combustió o en les reaccions electroquímiques que fan funcionar motors de combustió o elèctrics. La calor generada per la combustió d'una quantitat de combustible s'anomena calor de combustió, i es mesura en \si{\joule\per\mole}. Aquesta quantitat varia segons el combustible i la seva composició.

La termodinàmica estudia les relacions entre energia, calor i treball.
En aquest capítol treballarem al voltant de la termoquímica, la termodinàmica associada a les reaccions químiques, no només a la combustió

\begin{mybox}[title={Sistemes, estats i funcions d'estat}]
    Anomenem \emph{sistema} a aquella part de l'\emph{univers} que volem tractar en algun càlcul o experiment. 
Per exemple, un sistema pot ser un cilindre en un motor de combustió o bé una bateria elèctrica.
\begin{center}
\includegraphics[scale=0.5]{SistEntornUnivers.png}
\end{center}
 
L'\textit{estat del sistema} es caracteritza  per unes determinades \textit{variables d'estat} ($P$, $V$, $T$, $E$, $H$,...), magnituds físiques macroscòpiques mesurables. La termodinàmica estudia els \textit{estats d'equilibri} dels sistemes, en els quals les variables d'estat són idèntiques en totes les seves parts i invariables en el temps:
\begin{enumerate}
\item En els \textit{canvis d'estat} d'un sistema, les variables d'estat només depenen de l'estat inicial i final, i no del camí utilitzat. Així, per exemple, el treball $w$ no és funció d'estat, mentre que l'energia $E$ sí que ho és.
\item En fixar els valors d'algunes d'elles, una equació d'estat determina automàticament el valor de les altres. Així, per exemple, en un gas ideal, si coneixem $P$, $V$ i $T$, podem determinar $E$, $H$, $S$, etc.
\end{enumerate}

Els canvis d'estat poden ser 
\begin{description}
\item[reversibles] quan les funcions d'estat varian de manera infinitessimal, mantenint el sistema constantment en l'equilibri (l'expansió d'un gas contra una pressió que difereix només $dP$ de la pressió interna, per exemple);
\item[irreversibles] en qualsevol altre situació (un procés de combustió, l'expansió d'un gas contra el buit, etc).
\end{description}
\end{mybox}

\subsection{Treball}

El treball realitzat per una força en desplaçar un cos entre dues posicions es calcula segons:
\[
w=\int_{x_1}^{x_2} \mathbf{F} \cdot \mathbf{dx}
\]
Tenint en compte que $P=\frac{F}{A}$, és fàcil veure que, en el cas d'un pistó que exerceixi una pressió externa sobre un gas 
\begin{center}
\includegraphics[scale=0.6]{pisto_dw.png}
\end{center}
tenim
\[
dw=-F_{ext}dx = -P_{ext} A dx = -P_{ext} dV
\]
i, per tant,
\[
w=-\int_{V_1}^{V_2} P_{ext} dV
\]
\begin{EXMP}[Treball en una expansió isobàrica]

Considerem un gas ideal que s'expandeix isobàricament (a pressió constant) des d'un volum inicial $V_1$ fins a un volum final $V_2$. El treball realitzat pel gas durant aquesta expansió es pot calcular com:

\[
w = -P_{\text{ext}} \Delta V = -P_{\text{ext}} (V_2 - V_1)
\]

On $P_{\text{ext}}$ és la pressió externa constant. Si la pressió està en \si{\pascal} i el volum en \si{\meter\cubed}, el treball es mesura en \si{\joule}.

Per exemple, si un gas s'expandeix des de \qty{1}{\meter\cubed} fins a \qty{2}{\meter\cubed} a una pressió constant de \qty{100}{\kilo\pascal}, el treball realitzat pel gas és:

\[
w = -\qty{100}{\kilo\pascal} \times (\qty{2}{\meter\cubed} - \qty{1}{\meter\cubed}) = -\qty{100}{\kilo\pascal} \times \qty{1}{\meter\cubed} = -\qty{100}{\kilo\joule}
\]

El signe negatiu indica que el treball és realitzat pel sistema (el gas) sobre l'entorn.
\end{EXMP}



\subsection{Calor}

La calor $q$ és una magnitud macroscòpica que representa l'efecte d'infinitud de treballs microscòpics deguts als moviments de les partícules d'un sistema.
Com el treball, no és una funció d'estat, ja que depèn del camí que utilitzem per transferir-lo.
La calor es medeix en calories o Joules.\marginnote{Definim com caloria la quantitat de calor necessària per escalfar 1 gr d'aigua \qty{1}{\degreeCelsius}. Per tant, la capacitat calorífica de l'aigua és $C_p=\qty{1}{\cal\per\gram\per\degreeCelsius}$. En realitat, això només és cert per a una temperatura donada, ja que la capacitat calorífica depèn lleugerament de la temperatura de partida. En el cas de l'aigua, la caloria es defineix per al pas de 14.5\unit{\degreeCelsius} a 15.5\unit{\degreeCelsius}. La quantitat de treball necessària per produir aquesta calor es va determinar per Mayer y Joule el s. XIX com \qty{1}{\cal}=\qty{4.1860}{\joule}. En química usem més sovint les Capacitats calorífiques molars, $C_m$,  que indiquen la quantitat de calor necessària per escalfar un mol d'una substància 1\unit{\degreeCelsius}.}

La quantitat de calor necessària per incrementar la temperatura un determinat valor d'\qty{1}{\mole} de substància és
\[
q=nC_m\Delta T
\]
Si aquesta expressió la usem per explicar un procés infinitessimal obtenim
\[
C_m=\frac{1}{n}\frac{dq}{dT}
\]
I com que la capacitat calorífica es pot obtenir a $V=\text{cnt}$ o a $P=\text{cnt}$, podem calcular
\[
q_v=\int_{T_1}^{T_2} n C_{v,m} dT
\]
i
\[
q_p=\int_{T_1}^{T_2} n C_{p,m} dT
\]

\begin{EXMP}[Calor en un procés isocòric]

Considerem ara un gas ideal que s'escalfa isocòricament (a volum constant) des d'una temperatura inicial $T_1$ fins a una temperatura final $T_2$. La calor transferida al gas durant aquest procés es pot calcular com:

\[
q_v = n C_{v,m} \Delta T = n C_{v,m} (T_2 - T_1)
\]

On $n$ és el nombre de mols de gas, $C_{v,m}$ és la capacitat calorífica molar a volum constant, i $\Delta T$ és el canvi de temperatura. Si la capacitat calorífica està en \si{\joule\per\mole\per\kelvin} i la temperatura en \si{\kelvin}, la calor es mesura en \si{\joule}.
\end{EXMP}

\subsection{Primera llei de la termodinàmica}

La primera llei de la termodinàmica estableix que l'energia no es pot crear ni destruir, sinó que es pot transformar d'una forma a una altra. Això es pot expressar com:
\begin{equation}
    \Delta U = q + w
\end{equation}
on $\Delta U$ és la variació d'{\bf energia interna}, $q$ és la calor transferida al sistema i $w$ és el treball realitzat sobre el sistema. $U$ és una {\bf funció d'estat}, ja que el seu increment $\Delta U$ depèn només de l'estat inicial i final, no del camí seguit per arribar-hi. És per això que l'escrivim en majúscules, a diferència de la calor i el treball, que són funcions de camí.

Imaginem una reacció que es dona a pressió constant. En aquest cas, la calor transferida al sistema és la calor de combustió, i el treball realitzat és el treball de compressió. Així, la primera llei de la termodinàmica es pot reescriure com:
\begin{equation}
    \Delta U = q - P \Delta V
\end{equation}
on $P$ és la pressió i $\Delta V$ és el canvi de volum.
En forma integral, si la pressió no fos constant, això es pot expressar com:
\begin{equation}
    \Delta U = q - \int_{V_1}^{V_2} P \diff V
\end{equation}

Si la reacció estudiada fos a volum constant, és a dir, en un recipient tancat, el treball de compressió seria zero i la primera llei es reduiria a:
\begin{equation}
    \Delta U = q_v
\end{equation}

Per tant, per mesurar la calor de combustió d'un combustible, es pot utilitzar un calorímetre a volum constant, on tota l'energia alliberada per la reacció es converteix en calor. Les reaccions que desprenen calor s'anomenen \textbf{exotèrmiques}, mentre que les que l'absorbeixen s'anomenen \textbf{endotèrmiques}. Si el sistema absorbeix calor, la variació d'energia interna serà positiva, i si la despren, serà negativa.

Normalment, però, les reaccions químiques succeeixen a pressió constant, i per tant, la calor de combustió es mesura a pressió constant. Això es pot fer amb un calorímetre a pressió constant, on la calor de combustió es converteix en treball de compressió. En aquest cas, ens convé més usar una altra funció d'estat, l'{\bf entalpia}, que es defineix com:
\begin{equation}
    H = U + PV
\end{equation}
i la calor de combustió es pot expressar com:
\begin{equation}
    \Delta H = \Delta U + \Delta (PV) = q+ w + \Delta (PV) \end{equation}
    cal notar que a pressió constant, $w = -P \Delta V$, i $\Delta (PV)=P\Delta V$. Per tant,
    \begin{equation}
    \Delta H = q_p
\end{equation}

Novament, per a un procés exotèrmic a pressió constant, la variació d'entalpia serà negativa, ja que el sistema allibera calor. Això és el que succeeix en una reacció de combustió.

Cal notar que $\Delta H$ i $\Delta U$ són funcions d'estat, però no són iguals, ja que $H$ inclou el treball de compressió. No obstant això, en processos en solució, el treball de compressió és negligible i $\Delta U \approx \Delta H$.

\subsection{Increment d'entalpia estàndard}

L'increment d'entalpia estàndard d'una reacció, $\Delta H^\circ$, és la variació d'entalpia que es produeix quan els reactius en els seus estats estàndard es converteixen en productes en els seus estats estàndard. Els estats estàndard es defineixen a una pressió d'1 bar i una temperatura específica, generalment 298.15 K (25°C). Aquesta magnitud és molt útil per calcular la calor alliberada o absorbida en una reacció química, ja que permet comparar diferents reaccions en condicions similars. Per exemple, l'increment d'entalpia estàndard de la combustió del metà és:

\begin{equation}
\ch{CH4(g) + 2 O2(g) -> CO2(g) + 2 H2O(l)} \quad \Delta H^\circ = -890.3 \, \si{\kilo\joule\per\mole}
\end{equation}

Aquesta equació indica que la combustió d'un mol de metà allibera 890.3 kJ d'energia en forma de calor. Els valors d'increment d'entalpia estàndard per a moltes reaccions es poden trobar en taules termodinàmiques\cite{lide_crc_2005} (algunes estàn recollides en la \href{https://biocomputing-teaching.github.io/WebQuimicaAutomocio/pdf/TaulaUnitats.pdf}{taula de paràmetres termodinàmics} del curs).

\subsection{Llei de Hess}

La llei de Hess estableix que el canvi d'entalpia d'una reacció química és independent del camí seguit per arribar als productes finals, depenent només dels estats inicial i final. Això permet calcular l'entalpia de reaccions complexes a partir de reaccions més senzilles. Per exemple, considerem la combustió del propà (\ch{C3H8}):

\begin{equation}
\ch{C3H8(g) + 5 O2(g) -> 3 CO2(g) + 4 H2O(l)}
\end{equation}

Podem descompondre aquesta reacció en passos més simples basats en les entalpies de formació (veure \href{https://biocomputing-teaching.github.io/WebQuimicaAutomocio/pdf/TaulaUnitats.pdf}{taula de paràmetres termodinàmics}):
\begin{align}
\ch{C3H8(g) -> 3 C(s) + 4 H2(g)} & \quad \Delta H_1^\circ = 104.7 \, \si{\kilo\joule\per\mole} \\
\ch{C(s) + O2(g) -> CO2(g)} & \quad \Delta H_2^\circ = -393.5 \, \si{\kilo\joule\per\mole} \\
\ch{H2(g) + 1/2 O2(g) -> H2O(l)} & \quad \Delta H_3^\circ = -285.8 \, \si{\kilo\joule\per\mole}
\end{align}

L'entalpia total de la reacció de combustió es pot calcular sumant les entalpies dels passos individuals:
\begin{equation}
\Delta H^\circ = \Delta H_1^\circ + 3 \Delta H_2^\circ + 4 \Delta H_3^\circ
\end{equation}

Substituint els valors:
\begin{equation}
\Delta H^\circ = 104.7 + 3(-393.5) + 4(-285.8) = 104.7 - 1180.5 - 1143.2 = -2219 \, \si{\kilo\joule\per\mole}
\end{equation}

Així, la llei de Hess ens permet determinar l'entalpia de reaccions complexes utilitzant dades d'entalpia de reaccions més simples.

L'entalpia estàndard de formació, $\Delta H_f^\circ$ és la variació d'entalpia que es produeix quan un mol d'una substància es forma a partir dels seus elements en els seus estats estàndard. L'entalpia d'una reacció es pot calcular a partir de les entalpies de formació estàndard dels reactius i productes utilitzant la següent fórmula:

\begin{equation}
\Delta H^\circ_{\text{reacció}} = \sum \Delta H^\circ_f (\text{productes}) - \sum \Delta H^\circ_f (\text{reactius})
\end{equation}

\subsection{Capacitat calorífica}   

Com hem vist més amunt, la capacitat calorífica es pot expressar com:
\begin{equation}
    C = \frac{q}{\Delta T}
\end{equation}

La capacitat calorífica a pressió constant es denota com $C_p$ i a volum constant com $C_v$. La diferència entre ambdues és el treball de compressió, i es pot expressar, en el cas dels gasos, com:
\begin{equation}
    C_p - C_v = R
\end{equation}

En el cas de líquids i sòlids, les dues capacitats calorífiques són pràcticament iguals, ja que el treball de compressió és negligible. En el cas dels gasos, la capacitat calorífica a pressió constant és lleugerament més gran que a volum constant, ja que el treball de compressió és positiu. 

\begin{table}[h!]
    \caption{Capacitats calorífiques (\si{\cal\per\mole\per\kelvin}) de diverses substàncies a 298 K i a pressió constant\cite{mahan_quimica_1997}.}
    \centering
    \renewcommand{\arraystretch}{1.5}
    \begin{tabular}{ccc|ccc}
        \toprule
        Substància & Fórmula & $C_p$  & Substància & Fórmula & $C_p$  \\
        \midrule
        Monòxid de carboni & \ch{CO} & 6.97 & Metà & \ch{CH4} & 8.53 \\
        Oxigen & \ch{O2} & 7.05 & Nitrogen & \ch{N2} & 6.97 \\
        Diòxid de carboni & \ch{CO2} & 8.96 & Hidrogen & \ch{H2} & 6.88 \\
        Aigua (vapor) & \ch{H2O(g)} & 8.02 & Etanol & \ch{C2H5OH} & 26.9 \\
        Propà & \ch{C3H8} & 17.6 & Butà & \ch{C4H10} & 23.5 \\
        \bottomrule
    \end{tabular}

    \label{tab:capacitats_calorifiques}
\end{table} 

\begin{mybox}[title=Relació entre la capacitat calorífica a pressió constant i a volum constant] 
Per deduir aquesta relació, considerem la primera llei de la termodinàmica:
\[
    \Delta U = q - P \Delta V
\]

A volum constant, el treball de compressió és zero ($\Delta V = 0$), i per tant. 
\[
    \Delta U = q_v = C_v \Delta T
\]

A pressió constant, la calor afegida al sistema es descompon en l'increment d'energia interna i el treball de compressió. 
\[
    q_p = \Delta U + P \Delta V = C_p \Delta T
\]

Utilitzant l'equació d'estat dels gasos ideals, $P \Delta V = nR \Delta T$, i per a $n = 1$ podem escriure:
\[
    C_p \Delta T = C_v \Delta T + R \Delta T
\]
 i, per tant:
\[
    C_p - C_v = R
\]

\end{mybox}

La capacitat calorífica es pot expressar en forma diferencial com:

\begin{equation}
    C_v = \left( \frac{\partial U}{\partial T} \right)_V = \left( \frac{\partial q_v}{\partial T} \right)_V
\end{equation}

\begin{equation}
    C_p = \left( \frac{\partial H}{\partial T} \right)_P = \left( \frac{\partial q_p}{\partial T} \right)_P
\end{equation}

\subsection{Dependència de l'entalpia amb la temperatura}

L'entalpia d'una substància depèn de la temperatura. Imaginem que volem calcular l'entalpia d'una reacció a una temperatura $T_2$ a partir de l'entalpia a una temperatura $T_1$. Com que l'entalpia és una funció d'estat, podem calcular la variació d'entalpia entre $T_1$ i $T_2$ seguint aquest camí:

\begin{center}
    \schemestart
      \ch{a A + b B}
      \arrow{->[\state{H}^{}_1$(T_2)$]}[,1.5]
      \ch{c C + d D}
      \arrow{<-[\state{H}^{prod}$(T_2\to T_1)$]}[-90,2]
      \ch{c C + d D}
      \arrow{<-[\state{H}^{}_2$(T_1)$]}[180,1.5]
      \ch{a A + b B}  
      \arrow(@c4--@c1){->[\state{H}^{react}$(T_1\to T_2)$]}
    \schemestop
    \end{center}

Podem obtenir la variació d'entalpia com (recordem que la variació d'entalpia és precisament la variació de calor a pressió constant):

\begin{equation}
    \Delta H_2 = \Delta H_1 + \int_{T_1}^{T_2} C_p(\text{productes})\, dT - \int_{T_1}^{T_2} C_p(\text{reactius})\, dT.
\end{equation}

Si definim la diferència de calor específica:
\begin{equation}
    \Delta C_p = C_p(\text{productes}) - C_p(\text{reactius}),
\end{equation}
les integrals es poden combinar en:
\begin{equation}
    \Delta H_2 = \Delta H_1 + \int_{T_1}^{T_2} \Delta C_p dT.
\end{equation}

En el cas que $\Delta C_p$ sigui constant, la integral es resol com:
\begin{equation}
    \Delta H_2 = \Delta H_1 + \Delta C_p (T_2 - T_1).
    \label{eq:entalpia_temperatura}
\end{equation}

\begin{EXMP}[Càlcul de l'entalpia a una altra temperatura]
    Per exemple, donada la reacció:
    \begin{center}
    \ch{CO + $\frac{1}{2}$ O2 -> CO2}
    \end{center}
amb $\Delta H_{298} = \qty{-67.640}{\cal}$, calculem $\Delta H^\circ$ a \qty{398}{\kelvin}. De la Taula \ref{tab:capacitats_calorifiques}, tenim:   
\begin{align*}
    C_p(\ch{CO}) &= \qty{6.97}{\cal\per\mole\per\kelvin},\\
    C_p(\ch{O2}) &= \qty{7.05}{\cal\per\mole\per\kelvin},\\
    C_p(\ch{CO2}) &= \qty{8.96}{\cal\per\mole\per\kelvin}
\end{align*}
Substituïm a la fórmula \ref{eq:entalpia_temperatura}:
\begin{align*}
    \Delta C_p &= 8.96 - 6.97 - \frac{7.05}{2} = \qty{-1.53}{\cal\per\mole\per\kelvin},\\
    \Delta H_{398} &= \Delta H_{298} - \Delta C_p (\qty{398}{\kelvin} - \qty{298}{\kelvin})\\
    &= \qty{-67.640}{\cal} - (\qty{-1.53}{\cal\per\mole\per\kelvin} \times \qty{100}{\kelvin})\\
    &= \qty{-67.793}{\cal} = \qty{-283.6}{\kilo\joule\per\mole}.    
\end{align*}
\end{EXMP}
 
